%======== Abschnitt 1.4 ============
%Fixpunkts�tze
\index{Fixpunkts�tze|see{Satz}}

\begin{defi}
Seien $(X,d_1)$ und $(Y,d_2)$ zwei (pseudo-) metrische R�ume und $f \colon X \nach Y$.\\
(i) $f$ hei�t \highl[Stetigkeit!lipschitzstetig (metr. Raum)]{lipschitzsetig} \gdwdef $\exists L \geq 0$, sodass $\forall x,y \in X$ gilt: $d_2(f(x),f(y)) \leq L d_1(x,y)$. Dabei hei�t $L$ \highl[Lipschitzkonstante|see{lipschitzstetig}]{Lipschitzkonstante}.\\
(ii) $f$ hei�t \highl[Kontraktion, strenge|see{lipschitzstetig}]{strenge Kontraktion} \gdwdef $f$ ist lipschitzstetig mit $0 \leq L < 1$.
\end{defi}
\medskip

\begin{lemmas}
Ist eine Abbildung $f \colon X \nach Y$ zwischen zwei (pseudo-) metrischen R�umen lipschitzstetig, so ist $f$ auch gleichm��ig stetig.
\end{lemmas}
\begin{proof}
Offensichtlich.
\end{proof}
\medskip

\begin{bem}
Die Umkehrung kann gelten, muss aber nicht, wie folgendes Beispiel zeigt.
\end{bem}
\medskip

\begin{bsp}
$X=[0,1] \subset \R$. Dann ist $f(x)=\sqrt{x}$ gleichm��ig stetig, aber nicht lipschitzstetig bez�glich der Betragsmetrik ($d(x,y)=\abs{y-x}$), bez�glich $d(x,y)=\sqrt{\abs{y-x}}$ jedoch schon.
\end{bsp}
\medskip

\begin{satz}[Banach'scher Fixpunktsatz, 1922]\index{Satz!Banach'scher Fixpunktsatz}
Sei $(X,d)$ ein vollst�ndiger metrischer Raum und $f \colon X \nach X$ eine strenge Kontraktion ($0 \leq L < 1$). Dann besitzt $f$ genau einen Fixpunkt, d.h. $\exists ! x^*\in X$ mit $f(x^*)=x^*$. Dieser kann als Grenzwert einer Folge $(x_j)_{j=0}^{\infty} \subset X$ mit $x_{j+1}=f(x_j)$ und beliebigem Startwert $x_0 \in X$ gewonnen werden. Es gilt ferner f�r alle $j\in \N$: $d(x_j, x^*) \leq \frac{L^j}{1-L} d(x_1,x_0)$.
\end{satz}
\begin{proof}
Siehe Vorlesung.
\end{proof}
\medskip

\begin{bem}
Die Bedingung $L<1$ ist wesentlich, wie folgendes Beispiel zeigt:
\end{bem}
\medskip

\begin{bsp}
W�hle $(X,d)=(\C, \abs{\cdot})$ (vollst�ndig!). Sei $M \subset \C \setminus \{0\}$ ein abgeschlossener Kreisring (z.B.: $M = \{ z \in \C \sep \frac{1}{2} \leq \abs{z} \leq 1 \}$). Dann ist $(M, \abs{\cdot })$ ein vollst�ndiger metrischer Raum. Sei die Abbildung $f$ nun die Drehung um den Winkel $\alpha$ (mit $0 < \alpha < 2\pi$). F�r alle $z,w \in M$ gilt: $\abs{f(z)-f(w)} = 1 \cdot \abs{z-w}$, d.h. $L=1$. Man sieht leicht, dass $f$ keinen Fixpunkt in $M$ besitzt, da $0 \notin M$.
\end{bsp}
\medskip

\begin{folg}
Sei $(X,d)$ ein vollst�ndiger metrischer Raum, $A \subset X$ eine abgeschlossene, nichtleere Teilmenge und $f \colon A \nach A$ eine strenge Kontraktion. Dann besitzt $f$ genau einen Fixpunkt $x^*$ in $A$.
\end{folg}
\begin{proof}
Siehe Vorlesung.
\end{proof}
\medskip

\begin{bsp}[nichtlineare Integralgleichung als Fixpunktproblem]
Gegeben sei die Gleichung
\begin{gather*}
f(t) - \left( \int_0^t \frac{1}{2} f(\tau) d\tau \right)^2 = 1, \quad t \in [0,1]
\end{gather*}
Gibt es daf�r eine L�sung $f \in C([0,1])$? Es ist bereits bekannt, dass der metrische Raum $(C([0,1]), d_{\infty})$ mit $d_{\infty}(f,g)=\sup_{t \in [0,1]} \abs{f(t)-g(t)}$ vollst�ndig ist.\\
Ein Fixpunkt der Abbildung $T \colon C([0,1]) \nach C([0,1])$, mit $(Tf)(t):=1+\left( \int_0^t \frac{1}{2} f(\tau) d\tau \right)^2$ w�re L�sung der obigen Gleichung. $T$ ist jedoch keine Kontraktion auf ganz $C([0,1])$. Definiere daher die Menge $A:=\{ f \in C([a,b]) \sep 1 \leq f(t) \leq 1+t, \quad t \in [0,1] \}$ und zeige, dass $T \colon A \nach A$. Es gilt:
\begin{align*}
1 \leq (Tf)(t) & \leq 1 + \left( \int_0^t \frac{1}{2} (1 + \tau) d\tau \right)^2 = 1 + \left( \frac{t}{2} + \frac{t^2}{4} \right)^2 \leq 1 + \left( \frac{t}{2} + \frac{t}{4} \right)^2 \\
               & \leq 1+t^2 \leq 1+t
\end{align*}
Zeige dar�ber hinaus, dass $A$ abgeschlossen ist:\\
Dazu sei $(f_j)_{j=1}^{\infty} \subset A$ mit $f_j \nach g \in C([0,1])$ gleichm��ig (d.h. bez�glich $d_{\infty}$). Dann ist $f_j(t) \nach g(t)$ f�r alle $t \in [0,1]$ und wegen $1 \leq f_j(t) \leq 1+t$ gilt f�r jedes feste $t$ auch $1 \leq g(t) \leq 1+t$. Damit ist $g \in A$.\\
Es bleibt zu zeigen, dass es sich bei $T$ um eine strenge Kontraktion handelt:\\
Es gilt f�r beliebige $f,g \in A$ und $t \in [0,1]$
\begin{align*}
\abs{(Tf)(t)-(Tg)(t)} & = \abs{\left( \int_0^t \frac{1}{2} f(\tau) d\tau \right)^2 - \left( \int_0^t \frac{1}{2} g(\tau) d\tau \right)^2} \\
                  & = \abs{\int_0^t \frac{1}{2} f(\tau) d\tau - \int_0^t \frac{1}{2} g(\tau) d\tau} \cdot \abs{\int_0^t \frac{1}{2} f(\tau) d\tau + \int_0^t \frac{1}{2} g(\tau) d\tau}\\
                  & \leq \abs{\int_0^t \frac{f(\tau) - g(\tau)}{2} d\tau} \cdot \abs{\int_0^t \frac{1}{2} (1+\tau) d\tau + \int_0^t \frac{1}{2} (1+\tau) d\tau} \\
                  & \leq t \cdot \frac{1}{2} d_\infty(f,g) \cdot \left( t + \frac{t^2}{2} \right) \\
                  & \leq 1 \cdot \frac{1}{2} d_\infty(f,g) \cdot \frac{3}{2} \\
                  & = \frac{3}{4} \cdot d_\infty(f,g)
\end{align*}
Damit existiert nach dem Banach'schen Fixpunktsatz genau ein Fixpunkt von $T$ in $A$, also ist die Integralgleichung l�sbar.
\end{bsp}
\medskip

\begin{folg}
Sei $(X,d)$ ein vollst�ndiger metrischer Raum, $f \colon X \nach X$ lipschitzstetig und erst die Funktion $f^{j_0}(\cdot)=f(f(\ldots(\cdot)))$ eine strenge Kontraktion. Dann besitzt $f$ genau einen Fixpunkt $x^* \in X$, welcher auch wieder iterativ gewonnen werden kann.
\end{folg}
\begin{proof}
Siehe Vorlesung.
\end{proof}
\medskip

\begin{satz}[Weissinger'scher Fixpunktsatz]\index{Satz!Weissinger'scher Fixpunktsatz}
Sei $(X,d)$ ein vollst�ndiger metrischer Raum und $F \colon X \nach X$ eine Abbildung. Weiter exisiteren $a_n \geq 0$ mit $\sum_{n=1}^{\infty} a_n <\infty$, sodass f�r alle $x,y \in X$ und $n \in \N$ gilt: $d(F^n(x),F^n(y)) \leq a_n d(x,y)$. Dann besitzt $F$ genau einen Fixpunkt $x^* \in X$, welcher iterativ als Grenzwert der Folge $(F^n(x_0))_{n=1}^{\infty}$ mit beliebigem Startpunkt $x_0 \in X$ ermittelt werden kann. Es gilt die Fehlerabsch�tzung $d(x^*, F^n(x_0)) \leq \left( \sum_{j=n}^{\infty} a_j \right) d(F(x_0),x_0)$.
\end{satz}
\begin{proof}
Siehe �bungsserie 4 / Aufgabe 3.
\end{proof}
\medskip

%========== Ende =============
