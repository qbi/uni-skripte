%========== Abschnitt 2.2 =============
%Kompakte Mengen in speziellen Banachr�umen

\begin{lemma}[Lemma von Riesz]\index{Lemma!von Riesz}
Sei $E$ ein (nichtleerer) abgeschlossener echter Unterraum eines Banachraums $(X, \norm{\cdot})$. Dann exisitert f�r jedes $\epsilon >0$ ein $x_\epsilon \in X \setminus E$ mit $\norm{x_\epsilon } = 1$ und $\inf_{y \in E} \norm{x_\epsilon - y} \geq 1-\epsilon$.
\begin{proof}
Siehe Vorlesung.
\end{proof}
\end{lemma}
\medskip

\begin{bem}
Abgeschlossen ist hier im topologischen Sinne zu sehen. Z.B. ist $\Q$ ein echter Unterraum von $\R$, aber nicht abgeschlossen.
\end{bem}
\medskip

\begin{satz}
Ein Banachraum $(X, \norm{\cdot})$ ist endlich-dimensional \gdw Jede beschr�nkte Menge in X ist pr�kompakt.
\begin{proof}
Siehe Vorlesung.
\end{proof}
\end{satz}
\medskip

\begin{bem}
Vergleiche dazu kompakte / pr�kompakte Mengen in metrischen R�umen (Abschnitt 1.1): Folgenkompaktheit $\cong$ �berdeckungskompaktheit, $\epsilon$-Netz, usw.
\end{bem}
\medskip

\begin{satz}
F�r $1 \leq p < \infty$ ist $M \subset l_p$ pr�kompakt \gdw \\
(i) $M$ ist in $l_p$ beschr�nkt und\\
(ii) $\forall \epsilon >0 \, \exists k_0 (\epsilon)$ sodass $\forall x=(\xi_k)_{k=1}^{\infty} \in M$ gilt $\sum_{k=k_0}^{\infty} | \xi_k |^p < \epsilon$ (gleichm��ge Restsummenabsch�tzung).
\begin{proof}
Siehe �bungsserie 5.
\end{proof}
\end{satz}
\medskip

\begin{bem}
Ist $M$ endlich-dimensinal und beschr�nkt, so auch pr�kompakt.
\end{bem}
\medskip

\begin{defi}
Eine Familie stetiger Funktionen $\{ f_\alpha \sep \alpha \in A\} \subset C(\overline{\Omega})$ hei�t \highl[Stetigkeit!gleichgradig]{gleichgradig stetig} \gdwdef $\forall \epsilon >0 \, \exists \delta(\epsilon) > 0$, sodass f�r alle $x, x' \in \overline{\Omega}$ mit $\norm{ x-x'} < \delta$ und alle $\alpha$ der Indexmenge $A$ gilt $|f_\alpha (x) - f_\alpha (x')| < \epsilon$.
\end{defi}
\medskip

\begin{bem}
Es sind dann alle Funktionen $f_\alpha$ gleichm��ig stetig und $\delta$ h�ngt nicht von der konkreten Funktion der Familie ab.\\
Wir betrachten $C(\overline{\Omega})$ (stetige Funktionen $f \colon \overline{\Omega} \nach \R$ oder $\C$) im Folgenden stets mit der Supremumsnorm.
\end{bem}
\medskip

\begin{satz}[Arzela-Ascoli]\index{Satz!von Arzela-Ascoli}
Sei $\Omega \subset \R^n$ offen und beschr�nkt. Dann ist\\
$\{ f_\alpha \sep \alpha \in A\} \subset C(\overline{\Omega})$ pr�kompakt \gdw $\{ f_\alpha \sep \alpha \in A\}$ ist beschr�nkt und gleichgradig stetig.
\begin{proof}
Siehe Vorlesung.
\end{proof}
\end{satz}
\medskip

\begin{bem}
In metrischen R�umen ist jede pr�kompakte Menge stets beschr�nkt.
\end{bem}
\medskip

%========== Ende =============
