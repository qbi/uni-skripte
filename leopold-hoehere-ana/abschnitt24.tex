%======== Abschnitt 2.4 ============
%Lineare Funktionale

In diesem Abschnitt sei $X$ ein normierter Raum und $Y=\C$ oder $\R$. Man betrachtet nun $\L(X, \C)$ bzw. $\L(X, \R)$ - d.h. lineare und beschr�nkte Operatoren $T \colon X \nach \C$.\\
Da $\C$ Banachraum ist, folgt, dass auch $\L(X, \C)$ zum Banachraum wird.
\medskip

\begin{defis}
\highl[$X'$|see{Dualraum}]{$X'$} $:= \L(X, \C)$ hei�t Raum der linearen und stetigen (beschr�nkten) \highl[Funktionale �ber $X$|see{Operator}]{Funktionale �ber $X$} oder auch \highl[Raum!Dualraum $X'$ von $X$]{Dualraum von $X$}. \highl[$x'$|see{Funktional}]{Elemente aus $X'$} werden Funktionale genannt und mit \highl[Operator!Funktional $x'$]{$x'$} bezeichnet.\\
Man schreibt auch $\norm{ x' | X' } = \sup_{\norm{x|X}=1}{|x'(x)|} = \sup_{\norm{x|X}=1}{|<x',x>|}$
\end{defis}
\medskip

Wie sieht $X'$ in konkreten F�llen aus?
\medskip

\begin{bsp}
$(\C^n, \norm{\cdot}_2)' \cong (\C^n, \norm{\cdot}_2)$
\end{bsp}
\medskip

\begin{satz}
(i) Sei $1 \leq p < \infty$ und $\frac{1}{p}+\frac{1}{q}=1$ ($\frac{1}{\infty}=0$ im Fall $p=1, q=\infty$). Dann ist die Abbildung $T \colon l_q \nach (l_p)'$ mit $(Ty)(x):=\sum_{k=1}^{\infty} \eta_k \xi_k$, wobei $x=(\xi_k)_{k=1}^{\infty} \in l_p$ und $y=(\eta_k)_{k=1}^{\infty} \in l_q$, ein isometrischer Isomorphismus.\\
(ii) Die selbe Abbildungsvorschrift vermittelt einen isometrischen Isomorphismus zwischen $l_1$ und $(c_0)'$
\end{satz}
\begin{proof}
Siehe Vorlesung.
\end{proof}
\medskip

\begin{satz}
Sei $1 \leq p < \infty$ und $\frac{1}{p}+\frac{1}{q}=1$. Ferner sei $(\Omega, \Sigma, \mu)$ ein $\sigma$-endlicher Ma�raum. Dann definiert $T \colon L_q(\Omega, \Sigma, \mu) \nach (L_p(\Omega, \Sigma, \mu))'$ mit $(Tg)(f):=\int_{\Omega} (fg)d\mu$ einen isometrischen Isomorphismus.
\end{satz}
\begin{proof}
Siehe \cite[II 2.4]{werner}.
\end{proof}
\medskip

\begin{bsp}
F�r $f \in L_p([a,b])$ mit Lebesgue-Ma� und $0<p<1$ ist $\norm{f|L_p}=\left( \int_a^b |f(t)|^p dt\right)^{\frac{1}{p}} < \infty$. $L_p$ ist in diesem Fall ein Quasi-Banachraum, da es sich nur um eine Quasi-Norm handelt (Dreiecksungleichung mit einer Konstanten $>1$).\\
Es gilt $(L_p([a,b]))'=\{ 0 \}$.
\end{bsp}
\begin{proof}
Siehe Vorlesung.
\end{proof}
\medskip

\begin{bem}
Wie "`reichhaltig"' ist $X'$ also? Kann man durch $X'$ Elemente aus $X$ "`trennen"' - d.h. existiert f�r $x,y \in X$ mit $x \neq y$ ein $x' \in X'$ mit $x'(x) \neq x'(y)$ bzw. $x'(x-y) \neq 0$?\\
Das h�ngt mit der Fortsetzbarkeit von linearen Funktionalen zusammen:\\ \\
Sei $E \subset X$ ein echter linearer Teilraum eines Vektorraums $X$, sowie $l \colon E \nach \C$ ein lineares Funktional auf $E$. Kann dieses auf ganz $X$ fortgesetzt werden?\\
$\exists y_1 \in X \setminus E$. Damit ist $E \cup \{y_1\}$ linear unabh�ngig, denn sonst w�re $y_1 \in E$.\\
$x \in \spn{E \cup \{y_1\}}$ besitzt die eindeutige Darstellung $x=z+\alpha y_1$ mit $z \in E$ und $\alpha \in \C$.\\
Definiere $l_1(x):=l(y)+\alpha c_1$ mit $c_1=const.$ beliebig fixiert.\\
Setzt man $E_1:=\spn{E \cup \{y_1\}}$ so folgt $l_1 \colon E_1 \nach \C$ ist eine lineare Fortsetzung von $l$. F�hrt man diesen Prozess itterativ weiter so erh�lt man $\bigcup_{j=1}^{\infty} E_j$. Ist dies bereits der gesamte Raum $X$ (was in der Regel nicht der Fall ist) so ist man fertig. Der Fortsetzungsprozess ist f�r R�ume diesen "`Typs"' also so m�glich.
\end{bem}
\medskip

\begin{defi}
Eine Abbildung $p \colon X \nach \R$ hei�t \highl[sublineare Abbildung]{sublinear} \gdwdef f�r beliebige $x,y \in X$ gilt:
\begin{align*}
p(x+y)\leq p(x)+p(y) & \quad \text{(subadditiv) und} \\
p(\alpha x) = \alpha p(x) \quad \forall \alpha > 0 & \quad \text{(positiv-homogen)}
\end{align*}
\end{defi}
\medskip

\begin{bsp}
(i) Normen und Halbnormen \\
(ii) Jede lineare Abbildung von $X$ nach $\R$ auf reellem Vektorraum\\
(iii) $x=(\xi_k)_{k=1}^{\infty} \in l_\infty$ (reelle Folgen) und $p(x):= \limsup_k \xi_k$\\
(iv) $x=(\xi_k)_{k=1}^{\infty} \in l_\infty$ (komplexe Folgen) und $p(x):= \limsup_k \Re(\xi_k)$ (oder $\Im(\xi_k)$)
\end{bsp}
\medskip

\begin{satz}[Satz von Hahn-Banach, Variante: Lineare Algebra]\index{Satz!von Hahn-Banach (lin. Alg.)}
Sei $X$ ein reeller Vektorraum und p eine sublineare Abbildung auf $X$. Ferner sei auf einem linearen Teilraum $E$ von $X$ ein lineares Funktional $l \colon E \nach \R$ mit $l(x) \leq p(x) \quad \forall x \in E$ gegeben. Dann existiert eine Fortsetzung $L$ von $l$ auf ganz $X$ mit $L(x) \leq p(x) \quad \forall x \in X$.
\end{satz}
\begin{proof}
Siehe Vorlesung.
\end{proof}
\medskip

\begin{bem}
Eine solche Fortsetzung ist im Allgemeinen nicht eindeutig bestimmt.
\end{bem}
\medskip

\begin{satzs}[Satz von Hahn-Banach, Variante: Lineare Algebra, komplex]\index{Satz!von Hahn-Banach (lin. Alg., komplex)}
Sei $X$ ein reeller oder komplexer Vektorraum und $p$ ein reellwertiges, subadditves und betragshomogenes Funktional auf $X$ ($p(\alpha x) = |\alpha| p(x), \alpha \in \K$). Au�erdem sei auf einem linearen Unterraum $E \subset X$ ein lineares Funktional $l \colon E \nach \C$ mit $|l(x)| \leq p(x) \quad \forall x \in E$ gegeben. Dann hat $l$ eine lineare Fortsetzung $L$ auf ganz $X$ mit $|L(x)| \leq p(x) \quad \forall x \in X$.
\end{satzs}
\begin{proof}
Siehe Vorlesung.
\end{proof}
\medskip

\begin{satz}[Hahn-Banach-Theorem]\index{Satz!von Hahn-Banach}
Sei $l$ ein stetiges, lineares Funktional auf einem linearen Teilraum $E$ eines normierten Raumes $X$. Dann existiert eine stetige lineare Fortsetzung $L \colon X \nach \C$ von $l$ auf $X$ mit der gleichen Norm, d.h. $\norm{ l | E'} = \sup_{\norm{x} \leq 1} |l(x)| = \sup_{\norm{x}} |L(x)| = \norm{ L(x) | X'}$.
\end{satz}
\begin{proof}
Siehe Vorlesung.
\end{proof}
\medskip

\begin{folg}
Ist $X$ ein normierter Raum und $x_0 \in X \setminus \{0\}$. Dann existiert ein lineares und stetiges Funktional $L \in X'$ mit $\norm{ L | X' } = 1$ und $L(x_0)=\norm{x_0}$.
\end{folg}
\begin{proof}
Siehe Vorlesung.
\end{proof}
\medskip

\begin{bem}
Variante: $\norm{ \widetilde{L} | X' } = \frac{1}{\norm{x_0}}$ und $\widetilde{L}(x_0)=1$.
\end{bem}
\medskip

\begin{bem}
Damit kann man Elemente $x$ und $y$ trennen, denn es existiert ein Funktional $L$, welches auf $x_0:=y-x$ nicht Null ist. Da $L$ linear ist folgt $L(x) \neq L(y)$.
\end{bem}
\medskip

\begin{folg}
Ist $X$ ein normierter Raum, so gilt f�r jedes $x \in X$ stets
\begin{gather*}
\norm{x} = \sup_{\norm{ L | X' } \leq 1} |L(x)| = \sup_{L \in X' \setminus \{ \0 \}} \frac{|L(x)|}{\norm{ L | X'}}
\end{gather*}
\end{folg}
\begin{proof}
Siehe Vorlesung.
\end{proof}
\medskip

\begin{folg}
Gilt f�r ein Element $x$ eines normierten Raumes $X$ stets $L(x)=0$ f�r jedes Funktional $L \in X'$, so ist $x = 0$.
\end{folg}
\begin{proof}
klar.
\end{proof}
\medskip

\begin{bem}
Ist $X$ ein normierter Raum, so ist $X'=\L(X,\C)$ der Raum der linearen und stetigen Funktionale �ber $X$ (Dualraum) - ein Banachraum, da $\R$ bzw. $\C$ Banachr�ume sind. Man kann nun �ber $X'$ abermals lineare, stetige Funktionale definieren.
\end{bem}
\medskip

\begin{defi}
F�r einen normierten Raum $X$ wird \highl[$X''$|see{Bidualraum}]{$X''$} $:=(X')'=\L(X', \C)$ der \highl[Raum!Bidualraum $X''$ von $X$]{Bidualraum von $X$} genannt.
\end{defi}
\medskip

\begin{lemmas}
Ist $x \in X$ und $\K=\R$ oder $\C$, so ist $i_x \colon X' \nach \K$, definiert durch $i_x (x'):=x'(x)$ f�r $x' \in X'$, ein lineares und stetiges Funktional �ber $X'$, d.h. $i_x \in X''$.
\end{lemmas}
\begin{proof}
Siehe Vorlesung.
\end{proof}
\medskip

\begin{bem}
Es gilt wegen $\norm{x} = \sup_{\norm{x'| X'} \leq 1} |x'(x)|$ (vgl. Folgerung 2.4.2) sogar $\norm{i_x |X''}=\norm{x}$.
\end{bem}
\medskip

\begin{folgs}
$J \colon X \nach X''$ mit $J(x):=i_x$ ist eine lineare, stetige und normerhaltende (nicht notwendig surjektive) Abbildung, die jedem $x$ das lineare Funktional $i_x \colon X' \nach \K$ zuordnet.
\end{folgs}
\begin{proof}
Linearit�t klar: $J(x+y)(x') = i_{x+y}(x') = x'(x+y)=x'(x)+x'(y)=i_x(x') + i_y(x') = J(x)(x')+J(y)(x') \quad \forall x' \in X'$, damit folgt $J(x+y)=J(x)+J(y)$. \\
Beschr�nktheit (Stetigkeit) und Normerhaltung folgen unmittelbar aus obiger Bemerkung: $\norm{ J(x) | X''} = \norm{ i_x | X''} = \norm{ x } \quad \forall x \in X$.
\end{proof}
\medskip

\begin{bem}
Man nennt $J$ die kanonische Einbettung von $X$ in $X''$.\\
Insbesondere ist $\norm{ J | \L(X,X'')} = 1$.
\end{bem}
\medskip

\begin{lemmas}
Ist $X$ ein normierter Raum so folgt:\\
(i) $X''=\L(X',\C)$ ist Banachraum.\\
(ii) $J(X)$ (das Bild von $X$ unter der Abbildung $J$) ist ein linearer Teilraum von $X''$.\\
(iii) Ist X �berdies vollst�ndig (Banachraum) so ist $J(X)$ abgeschlossen (im topol. Sinne) und damit auch Banachraum.
\end{lemmas}
\begin{proof}
(i) Wegen Satz 2.3.2, da $\C$ vollst�ndig ist.\\
(ii) Siehe obige Folgerung und Lemma aus Abschnitt 2.3.\\
(iii) W�hle eine beliebige Cauchy-Folge $(x''_k)_{k=1}^{\infty}$ in $J(X) \subset X''$. Dann existiert wegen (i) ein Grenzelement $\widehat{x''} \in X''$. Die Linearit�t und Normerhaltung von $J$ liefern ferner, dass auch die Urbild-Folge $(x_k)_{k=1}^{\infty} \subset X$ eine Cauchy-Folge ist:\\ $\norm{x_m-x_n|X}=\norm{J(x_m-x_n)|X''}=\norm{J(x_m)-J(x_n)|X''}=\norm{x''_m-x''_n|X''}<\epsilon$.\\
Ist $X$ Banachraum, so existiert auch dort ein Grenzelement $x \in X$, dessen Bild $J(x)=x''$ offensichtlich in $J(X)$ liegt.\\
Aus der Stetigkeit von $J$ folgt, dass $x''_k =J(x_k)$ gegen $J(x)=x''\in J(X)$ konvergiert. Da der Grenzwert eindeutig bestimmt ist folgt $\widehat{x''}=x''\in X''$. Damit ist $J(X)$ abgeschlossen und daher als abgeschlossene Teilmenge eines Banachraums auch selbst Banachraum.
\end{proof}
\medskip

\begin{bem}
Damit kann man die Vervollst�ndigungsproblematik in normierten R�umen behandeln:
\end{bem}
\medskip

\begin{folgs}
Jeder normierte Raum ist isometrisch isomorph zu einem dichten Unterraum eines Banachraumes.
\end{folgs}
\begin{proof}
Siehe Vorlesung.
\end{proof}
\medskip

\begin{bsp}[$c_0'' \cong l_\infty$]
$c_0$ mit der Supremumsnorm $\norm{\cdot}_{\infty}$ ist ein Banachraum. Fr�here �berlegungen zeigten, dass $(c_0)' \cong l_1$ und $(l_1)' \cong l_{\infty}$ gilt. Damit folgt unmittelbar $(c_0)'' \cong l_\infty$.\\
Die Abbildung $J_{c_0} \colon c_0 \nach l_{\infty}$ sieht dann mit $x=(\xi_k)_{k=1}^{\infty} \in c_0$ und $y=(\eta_k)_{k=1}^{\infty} \in l_1 \cong (c_0)'$ so aus: $J_{c_0}(x)(y)=\sum_{k=1}^{\infty} \xi_k \eta_k$. Damit ist $c_0$ in $l_{\infty}$ eingebettet, wobei $J_{c_0}$ nicht surjektiv ist.
\end{bsp}
\medskip

\begin{bsp}[$l_p'' \cong l_p$]
F�r $1<p<\infty$ gilt $(l_p)' \cong l_q$ mit $\frac{1}{p} + \frac{1}{q}=1$ (siehe oben). Dann ist offensichtlich $1<q<\infty$ und damit nach gleichem Argument $(l_p')'=(l_p)'' \cong (l_q)' \cong l_p$. Daher ist $l_p'' \cong l_p$.\\
D.h. mit $x=(\xi_k)_{k=1}^{\infty} \in l_p$ und $y=(\eta_k)_{k=1}^{\infty} \in l_q \cong (l_p)'$ ist $J_{l_p}(x)(y)=\sum_{k=1}^{\infty} \xi_k \eta_k$.
\end{bsp}
\medskip

\begin{defis}
Ist $X$ Banachraum und die Abbildung $J$ surjektiv so hei�t $X$ \highl[Raum!reflexiv (Banachraum)]{reflexiv} oder selbstbez�glich, d.h. es gilt $X'' \cong X$.
\end{defis}
\medskip

\begin{bem}
Mit Hilfe des Dualraums $X'$ eines normierten Raumes $(X, \norm{\cdot})$ kann man eine neue, schache Konvergenz definieren: Eine Folge konvergiert genau dann schwach in X, wenn alle linearen, stetigen Funktionale angewendet auf die Folge gegen das entsprechende Funktional angewendet auf den schwachen Grenzwert konvergieren.
\end{bem}
\medskip

\begin{defis}
Ist $(x_j)_{j=1}^{\infty}$ eine Folge in einem normierten Raum $(X, \norm{\cdot})$ und $x\in X$, so definiert man:\\
\highl[$x_j \stackrel{\sigma}{\longrightarrow} x$|see{schwach konvergente Folge}]{$x_j \stackrel{\sigma}{\longrightarrow} x$} (oder $\sigma-\lim_{j \nach \infty} x_j = x$ oder \highl[$\wlim{j \nach \infty} x_j$|see{schwach konvergente Folge}]{$\wlim{j \nach \infty} x_j$} $= x$) \gdwdef $\forall x'\in X'=\L(X,\C)$ gilt $x'(x_j) \nach x'(x)$ f�r $j \nach \infty$.\\
In diesem Falle nennt man die Folge $(x_j)_{j=1}^{\infty}$ \highl[Folge!schwach konvergente Folge (norm. Raum)]{schwach konvergent} in $X$.
\end{defis}
\medskip

\begin{lemmas}
Aus der Konvergenz einer Folge folgt stets auch die schwache Konvergenz.
\end{lemmas}
\begin{proof}
Siehe Vorlesung.
\end{proof}
\medskip

\begin{bem}
Die Umkehrung ist im Allgemeinen falsch.
\end{bem}
\medskip

\begin{bsp}[Schwach-konvergente Folge, die nicht konvergiert]
Sei $1 < p < \infty$. Man betrachte die Folge der Einheitsvektoren $(e_j)_{j=1}^{\infty} \subset l_p$ mit \\ $e_j=(0,\ldots,0,1,0,\ldots)=(\xi_k^{(j)})_{k=1}^{\infty}$ und w�hle ferner ein $x'=(\eta_k)_{k=1}^{\infty} \in (l_p)' \cong l_q$. Dann gilt (siehe fr�her): $x'(e_j)=\sum_{k=1}^{\infty} \eta_k \xi_k^{(j)}=\eta_j$. Damit ist $\lim_{j \nach \infty} x'(e_j)=\lim_{j \nach \infty} \eta_j=0$, wegen $\left( \sum_{k=1}^{\infty} |\eta_k|^q \right)^{\frac{1}{q}}<\infty$. Also gilt $\wlim{j \nach \infty} e_j = 0$, aber $\norm{e_j}=1$, d.h. es existiert kein $\lim_{j \nach \infty} e_j$.
\end{bsp}
\medskip

\begin{bem}
Der schwache Grenzwert ist stets eindeutig bestimmt.
\end{bem}
\medskip

\begin{lemmas}
$(f_j)_{j=1}^{\infty} \subset C([0,1])$ konvergiert schwach gegen $0 \in C([0,1])$ \gdw $(f_j)_{j=1}^{\infty}$ konvergiert punktweise gegen die Nullfunktion: $\lim_{j \nach \infty} f_j(t)=0 \quad \forall t\in [0,1]$
\end{lemmas}
\begin{proof}
Entf�llt.
\end{proof}
\medskip

%========== Ende =============
