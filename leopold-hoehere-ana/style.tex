%============== Abschnitt 0 ==================
% Header / Style Paket

% scrreprt trifft am Besten die Bed�rfnisse eines Skripts, das ganze wird
% zweiseitig (twoside), d.h. es wird zwischen linker und rechter Seite
% unterschieden, und wir verwenden zwischen den Abs�tzen einen Abstand
% von einer Zeile (parskip) und daf�r keinen Absatzeinzug,
% wobei die letzte Zeile eines Absatzes zu min. 1/4 leer ist.
% bibtotoc -> literaturverzeichnis ins inhaltsverzeichnis

\documentclass[parskip*,german,bibtotoc,idxtotoc]{scrreprt}

\usepackage{ifthen}
\usepackage{makeidx}
% \usepackage[final]{graphicx}  % F�r Grafiken

% Farbanpassung f�r verweise etc
\usepackage{color}
\usepackage[draft=false,colorlinks,bookmarksnumbered,linkcolor=blue, citecolor=blue]{hyperref}

\usepackage[latin1]{inputenc}
\usepackage{ngerman}
% \usepackage{nicefrac}
% \usepackage{tabularx}

\usepackage{lmodern}	        	% Latin Modern
% \usepackage{type1ec}           % cm-super
\usepackage[T1]{fontenc}        % T1-Schriften notwendig f�r PDFs
\usepackage{textcomp}           % wird ben�tigt, damit der \textbullet
                                % f�r itemize in lmodern gefunden wird.

\usepackage[leqno]{amsmath}     % leqno: gleichungsnummern links statt rechts
                                % intlimits: grenzen �ber dem integral 
\usepackage[all,warning]{onlyamsmath}  % warnt bei Verwendung von nicht
                                       % amsmath-Umgebungen z.\,B. $$ formel $$
\usepackage{amssymb}     % wird f�r \R, \C, usw. gebraucht
\usepackage{fixmath}     % ISO-konforme griech. Buchstaben

\usepackage[thmmarks,hyperref]{ntheorem} % f�r die Theorem-Umgebungen
                                         % (satz, defini, bemerk)
\usepackage{xspace}      % wird weiter unten gebraucht
\usepackage{slashbox}    % f�r schr�ge Striche links oben in der
                         % Tabelle; s. texdoc slashbox

\usepackage{paralist}    % besseres enumerate und itemize und neue
                         % compactenum/compactitem; s. texdoc paralist
% \usepackage{ifpdf}       % Erkennung, ob PDF generiert wird; n�tzlich zur
                         % Unterscheidung bei Grafiken \input{XYZ.pdf_t}
\usepackage{ellipsis}    % Korrektur f�r \dots
\usepackage{fixltx2e}

\usepackage[final]{microtype} % Verbesserung der Typographie

% Damit auch die Zeichen im Mathemode in �berschriften fett sind
% <news:lzfyyvx3pt.fsf@tfkp12.physik.uni-erlangen.de>
\addtokomafont{sectioning}{\boldmath}

% nach dem Theoremkopf wird ein Zeilenumbruch eingef�gt, die Schrift des
% K�rpers ist normal und der Kopf wird fett gesetzt
\theoremstyle{break}
\theorembodyfont{\normalfont}
\theoremheaderfont{\normalfont\bfseries}
\theoremnumbering{arabic}
\theoremseparator{:}

% Die folgenden Umgebungen werden einzeln nummeriert und am Ende jedes
% Kapitels zur�ckgesetzt
\newtheorem{satz}{\underline{Satz}}[section]
\newtheorem{lemma}{\underline{Lemma}}[section]
\newtheorem{folg}{\underline{Folgerung}}[section]
\newtheorem{defi}{\underline{Definition}}[section]

%% Die folgenden Theoremumgebungen bekommen keine Nummer
\theoremstyle{nonumberbreak}
\newtheorem{bem}{\underline{Bemerkung}}
\newtheorem{bsp}{\underline{Beispiel}}
\newtheorem{defis}{\underline{Definition}}
\newtheorem{lemmas}{\underline{Lemma}}
\newtheorem{folgs}{\underline{Folgerung}}
\newtheorem{satzs}{\underline{Satz}}


\theoremheaderfont{\scshape}
\theorembodyfont{\normalfont}
\theoremsymbol{\ensuremath{_\blacksquare}} % Das Zeichen am Ende eines Beweises
\newtheorem{proof}{Beweis:}

% Hier die Definition, wie \autoref die Umgebungen nennen soll, die mit \newtheorem definiert wurden
\newcommand*{\satzautorefname}{Satz}
\newcommand*{\folgautorefname}{Folgerung}
\newcommand*{\lemmaautorefname}{Lemma}


% Zwischen Unter- und Unterunterabschnitten sollte nicht unterschieden werden.
\renewcommand*{\subsectionautorefname}{Abschnitt}
\renewcommand*{\subsubsectionautorefname}{Abschnitt}


% Kopf + Fu�zeilen
\usepackage[headsepline,automark]{scrpage2}
\automark[section]{chapter}
\clearscrheadfoot
\ihead{\leftmark \, $\rightarrow$ \rightmark}
\cofoot[\pagemark]{\pagemark}


%--------- Eigene Befehl-Definitionen:
\newcommand*{\R}{\mathbb{R}}      % reelle Zahlen
\newcommand*{\C}{\mathbb{C}}      % komplexe Zahlen
\newcommand*{\N}{\mathbb{N}}      % nat�rliche Zahlen
\newcommand*{\Q}{\mathbb{Q}}      % gebrochene Zahlen
\newcommand*{\Z}{\mathbb{Z}}      % ganze Zahlen
\newcommand*{\K}{\mathbb{K}}      % allgemeiner K�rper
\renewcommand*{\L}{\mathcal{L}}   % geschwungen L (lin.+stet. Operatoren)
\newcommand*{\E}{\mathcal{E}}     % geschwungen E - Funktionenraum
\renewcommand*{\S}{\mathcal{S}}   % geschwungen S - Schwartzraum der schnell fallenden Fkt'n
\newcommand*{\D}{\mathcal{D}}     % geschwungen E - Funktionenraum
\newcommand*{\U}{\mathfrak{U}}    % Fraktur U - Nullumgebungsbasis
\newcommand*{\0}{\mathcal{O}}     % Nulloperator
\newcommand*{\komp}{\mathcal{K}}  % Menge der Kompakten operatoren
\newcommand*{\Pot}{\mathcal{P}}   % Potenzmenge geschwungen P
\newcommand*{\leer}{\emptyset}    % leere menge
\newcommand*{\nach}{\rightarrow}  % f von X \nach Y
\newcommand*{\sep}{\, | \,}       % trennungs-strich in menge: { x \in X \sep x=ay }
\newcommand*{\wlim}[1]{\text{weak-}\lim_{#1}} %schwacher Grenzwert: \wlim{k \nach \infty} -> weak-lim_{k \nach \infty}
\newcommand*{\esssup}[1]{\text{ess-}\sup_{#1}} %schwacher Grenzwert: \wlim{k \nach \infty} -> weak-lim_{k \nach \infty}
\newcommand*{\distr}[2]{\langle #1, #2 \rangle} % Distribution <f,g>


% mathematische operatoren definieren
\DeclareMathOperator{\support}{supp} % Support / Tr�ger einer Funktion \supp{f} -> supp(f)
\newcommand{\supp}[1]{{\support(#1)}}

\DeclareMathOperator{\spann}{Span} % Spann einer Menge M: \spn{M} -> Span(M)
\newcommand{\spn}[1]{{\spann(#1)}}

\DeclareMathOperator{\ima}{Im} % Bild eines Operator T: \im{T} -> Im(T)
\newcommand{\im}[1]{{\ima(#1)}}

\DeclareMathOperator{\kernel}{Ker}     % Kern eines Operator T: \ker{T} -> Ker(T)
\renewcommand*{\ker}[1]{\kernel(#1)}  

\DeclareMathOperator{\inter}{int} % Inneres einer Menge: \inner{M} -> int(M)
\newcommand{\inner}[1]{{\inter(#1)}}
%---------


% Index umbenennen in Stichwortverzeichnis
\renewcommand{\indexname}{Stichwortverzeichnis}

% Hervorheben und indizieren, Aufruf: \highl[wort im Index]{wort im text} oder erstes argument incl klammern weglassen, falls gleich
\newcommand*{\highl}[2][\empty]{\textbf{\boldmath{#2}} \ifthenelse{\equal{#1}{\empty}}{\index{#2}}{\index{#1}}}
% Beispiel: diese \highl{Ausarbeitung} erhebt keinen \highl[An!-spruch]{Anspruch} \highl[An!-auf|see {Ausarbe}]{auf} irgendwas

% "siehe" im Index durch Pfeil nach rechts ersetzen
\renewcommand{\seename}{$\rightarrow$}


% Definition f�r Xindy f�r die Trennung der einzelnen Abschnitte im
% Index. siehe auch die Datei indexstyle.xdy
\newcommand*{\indexsection}{\minisec}

\newcommand*{\gdw}{\ifthenelse{\boolean{mmode}}%
			       {\mspace{8mu}\Leftrightarrow\mspace{8mu}}%
			       {$\Leftrightarrow$\xspace}}

\newcommand*{\gdwdef}{\ifthenelse{\boolean{mmode}}%
			       {\mspace{8mu}:\Leftrightarrow\mspace{8mu}}%
			       {$:\Leftrightarrow$\xspace}}


% Um sicherzustellen, dass jeder Betrag-/jede Norm links und rechts die
% Striche bekommt, sind diese Befehle da. Damit kann man nicht die
% rechten Striche vergessen und es wird etwas �bersichtlicher. (Vorschlag
% ist aus amsldoc) \abs[\big]{\abs{a}-\abs{b}} \leq \abs{a+b}
\newcommand*{\abs}[2][]{#1\lvert#2#1\rvert}
\newcommand*{\norm}[2][]{#1\lVert#2#1\rVert}

% Das original Epsilon sieht nicht so toll aus
\renewcommand*{\epsilon}{\varepsilon}
% und mancheinem gef�llt auch das Phi nicht
\renewcommand*{\phi}{\varphi}