%======== Abschnitt 4.2 ============
%Lineare stetige Funktionale

\begin{lemma}
Die Halbnormenfamilie $P$ erzeuge die lokalkonvexe Topologie $\tau$ auf einem Vektorraum $X$. Dann gilt:
\begin{align*}
\text{(i) }   & \text{F�r die Halbnorm } q \colon X \nach [0, \infty) \text{ mit } q \notin P \text{ sind �quivalent:} \\
						  & \text{ 1.) } q \text{ ist stetig} \\
						  & \text{ 2.) } q \text{ ist stetig in } 0 \\
						  & \text{ 3.) } \{ x \in X \sep q(x) \leq 1 \} \text{ ist Nullumgebung} \\
\text{(ii) }  & \text{Alle } p \in P \text{ sind stetig} \\
\text{(iii) } & \text{Eine Halbnorm } q \text { ist stetig} \gdw \exists M \geq 0 \text{ und } \exists F \subset P \text { endlich mit} \\
							& q(x) \leq M \cdot \max_{p \in F} p(x) \quad \forall x \in X
\end{align*}
\end{lemma}
\begin{proof}
Siehe Vorlesung.
\end{proof}
\medskip

\begin{satz}
Seien $(X, \tau_P)$ und $(Y, \tau_Q)$ zwei lokalkonvexe R�ume, erzeugt durch die Halbnormenfamilien $P$ und $Q$, sowie $T \colon X \nach Y$ ein linearer Operator. Dann sind �quivalent:
\begin{align*}
\text{(i) }   & T \text{ ist stetig} \\
\text{(ii) }  & T \text{ ist stetig in } 0 \\
\text{(iii) } & \forall q \in Q \text{ existiert eine endliche Teilmenge } F \subset P \text{ und ein } M \geq 0 \text{ mit} \\
              & q(Tx) \leq M \cdot \max_{p \in F} p(x) \quad \forall x \in X
\end{align*}
\end{satz}
\begin{proof}
Entf�llt.
\end{proof}
\medskip

\begin{folgs}
F�r einen lokalkonvexen Raum $(X, \tau_P)$ und $l \colon (X,\tau_P) \nach \C$ gilt:
\begin{gather*}
l \text{ linear und stetig} \gdw \exists F \subset P \text{ und } M \geq 0 \text { mit } |l(x)| \leq M \cdot \max_{p \in F} p(x) \quad \forall x \in X
\end{gather*}
\end{folgs}
\begin{proof}
Klar nach Satz 4.2.1. $(Y,\tau_Q)=\C$ mit $Q=\{ |\cdot| \}$.
\end{proof}
\medskip

\begin{bem}
Im Abschnitt 1.2 bzw. 1.3 wurden Topologien verglichen. Es wurden topologische R�ume $(X,\tau_1)$ und $(X,\tau_2)$ betrachtet und es wurde definiert:\\ $\tau_2$ feiner als $\tau_1$ \gdwdef $\tau_1 \subset \tau_2$. �quivalent dazu war, dass die identische Abbildung $id \colon (X,\tau_2)\nach (X,\tau_1)$ stetig ist.\\
Weiter ist bekannt, dass f�r jede Topologie gilt $\{\leer, X\}=\tau_{in}\subset \tau \subset \tau_{dis}=\Pot(X)$.\\
Ist eine Topologie $\tau_2$ feiner als $\tau_1$, so hat $\tau_2$
\begin{align*}
& \text{mehr offene Mengen} \\
& \text{mehr abgeschlossene Mengen} \\
& \text{weniger kompakte Mengen} \\
& \text{mehr stetige Funktionale, d.h. Abbildungen $X \nach \K$ (f�r $\K=\R$ oder $\C$)} \\
& \text{weniger konvergente Folgen}
\end{align*}
\end{bem}
\medskip

\begin{bsp}[Feinheit lokalkonvexer Topologien]
(i) Betrachte $X=C^b(\R^n)$ (stetige und beschr�nkte Funktionen) mit der Topologie der punktweisen Konvergenz $\tau$ und der Topologie der gleichm��igen Konvergenz auf Kompakta $\sigma$ (vergleiche dazu Beispiel (i) und (ii) in 4.1).\\
Dann ist $\sigma$ feiner als $\tau$, denn $id \colon (X,\sigma)\nach (X,\tau)$ ist stetig. Nach Satz 4.2.1 (iv) ist $q(Tx)\leq M \cdot \max_{p\in F} p(x)$, denn $p_t(x)\leq 1\cdot p_{\{t\}}(x)$.\\
\\
(ii) Ist $X$ ein normierter Raum, so ist die Normtopologie stets feiner als die schwache Topologie $\sigma(X,X')$.
\end{bsp}
\medskip

\begin{defi}
Die Menge der linearen stetigen Funktionale auf einem lokalkonvexen Raum $(X,\tau)$ hei�t \highl[Raum!Dualraum $(X_\tau)'$]{Dualraum von $X$} und wird mit $X'$ oder auch \highl[$(X_\tau)'$|see{Dualraum}]{$(X_\tau)'$} bezeichnet.
\end{defi}
\medskip

\begin{bem}
Der Dualraum $X'$ sei ausgestattet mit der schwach* Topologie $\sigma(X',X)$, d.h. die durch die Halbnormen
\begin{gather*}
p_x(x'):=|x'(x)|,\quad \forall x' \in X'
\end{gather*}
f�r $x\in X$ erzeugte lokalkonvexe Topologie (vergleiche dazu Beispiel (ix) in 4.1). Das entspricht der punktweisen Konvergenz im Dualraum, d.h.
\begin{gather*}
(x_j')_{j=1}^\infty \subset X' \text{ konvergiert gegen } x'\in X' \gdw x_j'(x) \nach x'(x),\quad \forall x' \in X'
\end{gather*}
\end{bem}
\medskip

\begin{satz}[Fortsetzungssatz von Hahn-Banach]\index{Satz!Fortsetzungssatz von Hahn-Banach}
Sei $X$ ein lokalkonvexer Raum, $U\subset X$ ein linearer Teilraum, sowie $l$ ein lineares und stetiges Funktional �ber $U$ (d.h. $l\in U'$). Dann existiert eine Fortsetzung $L$ von $l$ mit $L\in X'$.
\end{satz}
\begin{proof}
Siehe Vorlesung (verwende die Folgerung nach Satz 4.2.1 und den Satz von Hahn-Banach aus 2.4).
\end{proof}
\medskip

\begin{bem}
Analog zur Folgerung 1 bzw. der danach folgenden Bemerkung in Abschnitt 2.4 l�sst sich aus dem Satz von Hahn-Banach die Trennungseigenschaft des Dualraumes ableiten:
\end{bem}
\medskip

\begin{satz}[Trennungssatz]\index{Satz!Trennungssatz}
Sei $X$ ein lokalkonvexer Hausdorffraum. Dann trennt $X'$ die Punkte von $X$, d.h. f�r $x\neq y$ existiert $x' \in X'$ mit $x'(x) \neq x'(y)$.
\end{satz}
\begin{proof}
Entf�llt.
\end{proof}
\medskip

%========== Ende =============
