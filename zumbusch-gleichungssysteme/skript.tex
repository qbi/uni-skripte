\documentclass[halfparskip,pointednumbers,smallheadings,headsepline,footsepline]{scrreprt}
\usepackage[ngerman]{babel}
\usepackage[T1]{fontenc}
\usepackage[latin1]{inputenc}
\usepackage{amsmath,amsthm,times,amssymb,xspace,url,svn}

\theoremstyle{definition}

\newtheorem{defi}{Definition}


%%%%%%%%%%%%%%%%%%%%%%%%%%%%%%%%%%%%%%%%%%%%%%% MATHE
%%%%%%%%%%%%%%%%%%%%%%%%%%%%%%%%%%%%%%%%%%%%%%%%%%%%%

%%%%%%%%%%%%%%%%%%%%%%%%%%%%%%%%%%%%%%%%%%%%%%%%%%%%%%%%%%%%%%%%%%%%%%%%%%%%%%%
%%% Mengensymbole                                                           %%%
%%%%%%%%%%%%%%%%%%%%%%%%%%%%%%%%%%%%%%%%%%%%%%%%%%%%%%%%%%%%%%%%%%%%%%%%%%%%%%%
\newcommand{\bN}{\ensuremath{\mathbb{N}}\xspace}
\newcommand{\bQ}{\ensuremath{\mathbb{Q}}\xspace}
\newcommand{\bR}{\ensuremath{\mathbb{R}}\xspace}
\newcommand{\bC}{\ensuremath{\mathbb{C}}\xspace}
\newcommand{\bH}{\ensuremath{\mathbb{H}}\xspace}
\newcommand{\bF}{\ensuremath{\mathbb{F}}\xspace}
\newcommand{\bZ}{\ensuremath{\mathbb{Z}}\xspace}

%%%%%%%%%%%%%%%%%%%%%%%%%%%%%%%%%%%%%%%%%%%%%%%%%%%%%%%%%%%%%%%%%%%%%%%%%%%%%%%
%%% Transponieren                                                           %%%
%%%%%%%%%%%%%%%%%%%%%%%%%%%%%%%%%%%%%%%%%%%%%%%%%%%%%%%%%%%%%%%%%%%%%%%%%%%%%%%
\newcommand{\tr}{^\top}

%%%%%%%%%%%%%%%%%%%%%%%%%%%%%%%%%%%%%%%%%%%%%%%%%%%%%%%%%%%%%%%%%%%%%%%%%%%%%%%
%%% Wahrscheinlichkeitstheorie                                              %%%
%%%%%%%%%%%%%%%%%%%%%%%%%%%%%%%%%%%%%%%%%%%%%%%%%%%%%%%%%%%%%%%%%%%%%%%%%%%%%%%
\newcommand{\wP}{\ensuremath{\mathbf{P}}\xspace}

%%%%%%%%%%%%%%%%%%%%%%%%%%%%%%%%%%%%%%%%%%%%%%%%%%%%%%%%%%%%%%%%%%%%%%%%%%%%%%%
%%% cal- und frak- Alphabete in Kurzform                                    %%%
%%%%%%%%%%%%%%%%%%%%%%%%%%%%%%%%%%%%%%%%%%%%%%%%%%%%%%%%%%%%%%%%%%%%%%%%%%%%%%%

\newcommand{\fA}{\ensuremath{\mathfrak{A}}\xspace}
\newcommand{\fB}{\ensuremath{\mathfrak{B}}\xspace}
\newcommand{\fC}{\ensuremath{\mathfrak{C}}\xspace}
\newcommand{\fD}{\ensuremath{\mathfrak{D}}\xspace}
\newcommand{\fE}{\ensuremath{\mathfrak{E}}\xspace}
\newcommand{\fF}{\ensuremath{\mathfrak{F}}\xspace}
\newcommand{\fG}{\ensuremath{\mathfrak{G}}\xspace}
\newcommand{\fH}{\ensuremath{\mathfrak{H}}\xspace}
\newcommand{\fI}{\ensuremath{\mathfrak{I}}\xspace}
\newcommand{\fJ}{\ensuremath{\mathfrak{J}}\xspace}
\newcommand{\fK}{\ensuremath{\mathfrak{K}}\xspace}
\newcommand{\fL}{\ensuremath{\mathfrak{L}}\xspace}
\newcommand{\fM}{\ensuremath{\mathfrak{M}}\xspace}
\newcommand{\fN}{\ensuremath{\mathfrak{N}}\xspace}
\newcommand{\fO}{\ensuremath{\mathfrak{O}}\xspace}
\newcommand{\fP}{\ensuremath{\mathfrak{P}}\xspace}
\newcommand{\fQ}{\ensuremath{\mathfrak{Q}}\xspace}
\newcommand{\fR}{\ensuremath{\mathfrak{R}}\xspace}
\newcommand{\fS}{\ensuremath{\mathfrak{S}}\xspace}
\newcommand{\fT}{\ensuremath{\mathfrak{T}}\xspace}
\newcommand{\fU}{\ensuremath{\mathfrak{U}}\xspace}
\newcommand{\fV}{\ensuremath{\mathfrak{V}}\xspace}
\newcommand{\fW}{\ensuremath{\mathfrak{W}}\xspace}
\newcommand{\fX}{\ensuremath{\mathfrak{X}}\xspace}
\newcommand{\fY}{\ensuremath{\mathfrak{Y}}\xspace}
\newcommand{\fZ}{\ensuremath{\mathfrak{Z}}\xspace}

\newcommand{\cA}{\ensuremath{\mathcal{A}}\xspace}
\newcommand{\cB}{\ensuremath{\mathcal{B}}\xspace}
\newcommand{\cC}{\ensuremath{\mathcal{C}}\xspace}
\newcommand{\cD}{\ensuremath{\mathcal{D}}\xspace}
\newcommand{\cE}{\ensuremath{\mathcal{E}}\xspace}
\newcommand{\cF}{\ensuremath{\mathcal{F}}\xspace}
\newcommand{\cG}{\ensuremath{\mathcal{G}}\xspace}
\newcommand{\cH}{\ensuremath{\mathcal{H}}\xspace}
\newcommand{\cI}{\ensuremath{\mathcal{I}}\xspace}
\newcommand{\cJ}{\ensuremath{\mathcal{J}}\xspace}
\newcommand{\cK}{\ensuremath{\mathcal{K}}\xspace}
\newcommand{\cL}{\ensuremath{\mathcal{L}}\xspace}
\newcommand{\cM}{\ensuremath{\mathcal{M}}\xspace}
\newcommand{\cN}{\ensuremath{\mathcal{N}}\xspace}
\newcommand{\cO}{\ensuremath{\mathcal{O}}\xspace}
\newcommand{\cP}{\ensuremath{\mathcal{P}}\xspace}
\newcommand{\cQ}{\ensuremath{\mathcal{Q}}\xspace}
\newcommand{\cR}{\ensuremath{\mathcal{R}}\xspace}
\newcommand{\cS}{\ensuremath{\mathcal{S}}\xspace}
\newcommand{\cT}{\ensuremath{\mathcal{T}}\xspace}
\newcommand{\cU}{\ensuremath{\mathcal{U}}\xspace}
\newcommand{\cV}{\ensuremath{\mathcal{V}}\xspace}
\newcommand{\cW}{\ensuremath{\mathcal{W}}\xspace}
\newcommand{\cX}{\ensuremath{\mathcal{X}}\xspace}
\newcommand{\cY}{\ensuremath{\mathcal{Y}}\xspace}
\newcommand{\cZ}{\ensuremath{\mathcal{Z}}\xspace}

%%%%%%%%%%%%%%%%%%%%%%%%%%%%%%%%%%%%%%%%%%%%%%%%%%%%%%%%%%%%%%%%%%%%%%%%%%%%%%%
%%% Deutsche Operatoren                                                     %%%
%%%%%%%%%%%%%%%%%%%%%%%%%%%%%%%%%%%%%%%%%%%%%%%%%%%%%%%%%%%%%%%%%%%%%%%%%%%%%%%

\DeclareMathOperator{\Id}{Id}
\DeclareMathOperator{\kgV}{kgV}
\DeclareMathOperator{\ggT}{ggT}
\DeclareMathOperator{\Bild}{Bild}
\DeclareMathOperator{\Kern}{Kern}
\DeclareMathOperator{\Bahn}{Bahn}
\DeclareMathOperator{\Stab}{Stab}
\DeclareMathOperator{\Var}{Var}

%%%%%%%%%%%%%%%%%%%%%%%%%%%%%%%%%%%%%%%%%%%%%%%%%%%%%%%%%%%%%%%%%%%%%%%%%%%%%%%
%%% im Deutschen �bliche griech. Buchstaben                                 %%%
%%%%%%%%%%%%%%%%%%%%%%%%%%%%%%%%%%%%%%%%%%%%%%%%%%%%%%%%%%%%%%%%%%%%%%%%%%%%%%%
\let\epsilon\varepsilon
\let\phi\varphi
\let\rho\varrho
\let\kappa\varkappa

%%%%%%%%%%%%%%%%%%%%%%%%%%%%%%%%%%%%%%%%%%%%%%%%%%%%%%%%%%%%%%%%%%%%%%%%%%%%%%%
%%% Mengenlehre                                                             %%%
%%%%%%%%%%%%%%%%%%%%%%%%%%%%%%%%%%%%%%%%%%%%%%%%%%%%%%%%%%%%%%%%%%%%%%%%%%%%%%%
\let\emptyset\varnothing
\let\subset\subseteq
\let\subsetneq\subsetneqq

%%%%%%%%%%%%%%%%%%%%%%%%%%%%%%%%%%%%%%%%%%%%%%%%%%%%%%%%%%%%%%%%%%%%%%%%%%%%%%%
%%% kurze Matrixbefehle mit ensuremath, kleine Matrizen                     %%%
%%%%%%%%%%%%%%%%%%%%%%%%%%%%%%%%%%%%%%%%%%%%%%%%%%%%%%%%%%%%%%%%%%%%%%%%%%%%%%%

\newcommand{\pmat}[1]{\ensuremath{\begin{pmatrix} #1 \end{pmatrix}}\xspace}
\newcommand{\bmat}[1]{\ensuremath{\begin{bmatrix} #1 \end{bmatrix}}\xspace}
\newcommand{\vmat}[1]{\ensuremath{\begin{vmatrix} #1 \end{vmatrix}}\xspace}
\newcommand{\Bmat}[1]{\ensuremath{\begin{Bmatrix} #1 \end{Bmatrix}}\xspace}
\newcommand{\Vmat}[1]{\ensuremath{\begin{Vmatrix} #1 \end{Vmatrix}}\xspace}
\newcommand{\kpmat}[1]{\ensuremath{\left(\begin{smallmatrix} #1 \end{smallmatrix}\right)}\xspace}
\newcommand{\kbmat}[1]{\ensuremath{\left[\begin{smallmatrix} #1 \end{smallmatrix}\right]}\xspace}
\newcommand{\kBmat}[1]{\ensuremath{\left\{\begin{smallmatrix} #1 \end{smallmatrix}\right\}}\xspace}
\newcommand{\kvmat}[1]{\ensuremath{\left|\begin{smallmatrix} #1 \end{smallmatrix}\right|}\xspace}
\newcommand{\kVmat}[1]{\ensuremath{\left\|\begin{smallmatrix} #1 \end{smallmatrix}\right\|}\xspace}


%%%%%%%%%%%%%%%%%%%%%%%%%%%%%%%%%%%%%%%%%%%%%%%%%%%%%
%%%%%%%%%%%%%%%%%%%%%%%%%%%%%%%%%%%%%%%%%%%%%%%%%%%%%








\SVN $LastChangedRevision: 1688 $
\SVN $LastChangedDate: 2008-10-21 17:11:22 +0200 (Di, 21 Okt 2008) $

\begin{document}
\pagestyle{headings}
\title{Schnelle iterative L�ser gro�er schwach besetzter linearer Gleichungssysteme}
\author{Professor Dr. G. W. Zumbusch\\FSU Jena}
\date{}
\maketitle

\chapter*{Vorwort}

  Dieses Dokument wurde als Skript f�r die auf der Titelseite genannte
  Vorlesung erstellt und wird jetzt im Rahmen des Projekts "`Vorlesungsskripte
  der Fakult�t f�r Mathematik und Informatik"' weiter betreut. Das Dokument
  wurde nach bestem Wissen und Gewissen angefertigt. Dennoch garantiert weder
  der auf der Titelseite genannte Dozent, die Personen, die an dem Dokument
  mitgewirkt haben, noch die Mitglieder des Projekts f�r dessen
  Fehlerfreiheit. F�r etwaige Fehler und dessen Folgen wird von keiner der
  genannten Personen eine Haftung �bernommen. Es steht jeder Person frei,
  dieses Dokument zu lesen, zu ver�ndern oder auf anderen Medien verf�gbar zu
  machen, solange ein Verweis auf die Internetadresse des Projekts
  \url{http://uni-skripte.lug-jena.de/} enthalten ist.

  Diese Ausgabe tr�gt die Versionsnummer~\SVNLastChangedRevision{} und ist vom
  \SVNDate{}. Eine neue Ausgabe k�nnte auf der Webseite des Projekts
  verf�gbar sein.

  Jeder ist dazu aufgerufen, Verbesserungen, Erweiterungen und
  Fehlerkorrekturen f�r das Skript einzureichen bzw. zu melden oder diese
  selbst einzupflegen -- einfach eine E-Mail an die Mailingliste
  \texttt{<uni-skripte@lug-jena.de>} senden. Weitere Informationen sind
  unter der oben genannten Internetadresse verf�gbar.

  Hiermit m�chten wir allen Personen, die an diesem Skript mitgewirkt haben,
  vielmals danken:
  \begin{itemize}
   \item Ivo Hedtke \texttt{<hedtke@math.uni-jena.de>} (2008/09)
  \end{itemize}


\tableofcontents

\chapter*{Literatur}

\begin{enumerate}
    \item \textsc{W. Hackbusch}: Iterative L�sung gro�er schwachbesetzter Gleichungssysteme. 1991.
	\item \textsc{A. Meister}: Numerik linearer Gleichungssysteme. Eine Einf�hrung in moderne Verfahren. 2007.
\end{enumerate}

\chapter{Grundbegriffe}

Parallele Algorithmen und Parallelisierung von Iterationsverfahren

\section{�berblick}

\paragraph{Warum?} \begin{itemize}
    \item Verf�gbarkeit von Parallelrechnern (Dual-Core, \dots)
\item Physikalische Grenzen
\item Parallelit�t in vielen Anwendungen
\end{itemize}

\subsection{Modellproblem: Skalarprodukt von Vektoren}

Seien $x,y \in \bR^n$. Wir wollen berechnen: \[S = \langle x,y \rangle = \sum_{i=0}^{n-1}x_i y_i.\]

\paragraph{Parallelit�t?}
\begin{enumerate}
    \item Berechnung der $x_i y_i$ $\forall i$ unabh�ngig
\item Sei $P$ die Anzahl der Prozessoren ($n \gg P$). Die Indizes $\{0,\dots, n-1\}$ werden auf die Prozessoren verteilt: \[I_p \subset \{0,\dots,n-1\}.\] Jeder Prozessor berechnet eine Teilsumme: \[S_p = \sum_{i \in I_p} x_i y_i.\]
\item Gesamtsumme:
\begin{enumerate}
    \item sequentiell: $S = \sum_{p=0}^{P-1} S_p$ oder
\item parallel: (Beispiel $P=8)$\begin{align*}
S &= \underbrace{S_0 + S_1}_{S_{01}} + \underbrace{S_2 + S_3}_{S_{23}} + \underbrace{S_4 + S_5}_{S_{45}} + \underbrace{S_6 + S_7}_{S_{67}}\\
&= \underbrace{S_{01} + S_{23}}_{S_{0123}} + \underbrace{S_{45} + S_{67}}_{S_{4567}}\\
&= \underbrace{S_{0123} + S_{4567}}_{S_{01234567}}\\
&= S_{01234567} = S
\end{align*}
\end{enumerate}
Statt $7$ Schritten im sequentiellen Verfahren werden hier nur $3$ ben�tigt.
\end{enumerate}

Gesamtaufwand sequentiell: $\cO(n)$.

Aufwand in jedem parallelen Schritt:
\begin{enumerate}
    \item $\frac{n}{P}$
\item $\frac{n}{P}$
\item $\log P$
\end{enumerate}

Gesamtaufwand im parallelen Verfahren: $\cO\left(\frac{n}{P} + \log P\right)$. F�r die Extremf�lle haben wir:
\begin{itemize}
    \item $n \to \infty$: $\cO\left(\frac{n}{P}\right)$, eine Beschleunigung um den Faktor $P$
\item $n \approx P$: $\cO(\log N)$
\end{itemize}

\subsection{Kommunizierende sequentielle Prozesse}

\begin{defi}[Sequentieller Prozess]
Abstraktion, Ausf�hren eines sequentiellen Programms. Jederzeit klarer Zustand. Genau ein Befehlsz�hler (PC, Programm Counter) und Variablen (Speicher, Register).
\end{defi}

\begin{defi}[Paralleles Programm]
Interagierende sequentielle Prozesse. Sinnvoll auf mehreren Prozessoren. Gegebenenfalls zyklisches Umschalten.
\end{defi}

\newpage\minisec{Vereinfachtes Muster eines parallelen Programms}

\begin{verbatim}
{
   globale Variablen

   thread<name>[Parameter]{
      lokale Variablen
      Anweisungen
   }

   thread<name>[...]{
      ...
   }

   ...
}
\end{verbatim}

\minisec{Modell mit statischen Prozessen}

Starte alle.  Paralleles Programm terminiert, wenn alle Prozesse terminieren.

�blicherweise unterscheidet man zwischen:
\begin{itemize}
    \item Prozessen: Eigener Adressraum
\item Threads: Gemeinsamer Adressraum, billiger
\end{itemize}

\end{document}
