\RequirePackage[l2tabu,orthodox]{nag}
\documentclass[halfparskip,pointednumbers,smallheadings,headsepline,footsepline]{scrreprt}
\usepackage[ngerman]{babel}
\usepackage[T1]{fontenc}
\usepackage[latin1]{inputenc}
\usepackage{amsmath,amsthm,amssymb,xspace,url,svn,color,fixmath,pifont,hyperref,booktabs,tikz}
\usetikzlibrary{calc}

\theoremstyle{definition}

\newtheorem{defi}{Definition}
\theoremstyle{remark}
\newtheorem*{bsp}{Beispiel}

\newcommand*{\bR}{\ensuremath{\mathbb{R}}\xspace}
\newcommand*{\cO}{\ensuremath{\mathcal{O}}\xspace}

\let\epsilon\varepsilon
\let\phi\varphi
\let\rho\varrho
\let\kappa\varkappa

\let\emptyset\varnothing
\let\subset\subseteq
\let\subsetneq\subsetneqq

\newcommand*{\pmat}[1]{\ensuremath{\begin{pmatrix} #1 \end{pmatrix}}\xspace}
\newcommand*{\bmat}[1]{\ensuremath{\begin{bmatrix} #1 \end{bmatrix}}\xspace}
\newcommand*{\vmat}[1]{\ensuremath{\begin{vmatrix} #1 \end{vmatrix}}\xspace}
\newcommand*{\Bmat}[1]{\ensuremath{\begin{Bmatrix} #1 \end{Bmatrix}}\xspace}
\newcommand*{\Vmat}[1]{\ensuremath{\begin{Vmatrix} #1 \end{Vmatrix}}\xspace}
\newcommand*{\kpmat}[1]{\ensuremath{\left(\begin{smallmatrix} #1 \end{smallmatrix}\right)}\xspace}
\newcommand*{\kbmat}[1]{\ensuremath{\left[\begin{smallmatrix} #1 \end{smallmatrix}\right]}\xspace}
\newcommand*{\kBmat}[1]{\ensuremath{\left\{\begin{smallmatrix} #1 \end{smallmatrix}\right\}}\xspace}
\newcommand*{\kvmat}[1]{\ensuremath{\left|\begin{smallmatrix} #1 \end{smallmatrix}\right|}\xspace}
\newcommand*{\kVmat}[1]{\ensuremath{\left\|\begin{smallmatrix} #1 \end{smallmatrix}\right\|}\xspace}

\SVN $Revision$
\SVN $Date$
\SVN $LastChangedRevision$
\SVN $LastChangedDate$

\begin{document}
\pagestyle{headings}
\title{Schnelle iterative L�ser gro�er schwach besetzter linearer Gleichungssysteme}
\author{Professor\,Dr.\,G. W. Zumbusch\\FSU Jena}
\date{}
\maketitle

\chapter*{Vorwort}

  Dieses Dokument wurde als Skript f�r die auf der Titelseite genannte
  Vorlesung erstellt und wird jetzt im Rahmen des Projekts "`Vorlesungsskripte
  der Fakult�t f�r Mathematik und Informatik"' weiter betreut. Das Dokument
  wurde nach bestem Wissen und Gewissen angefertigt. Dennoch garantiert weder
  der auf der Titelseite genannte Dozent, die Personen, die an dem Dokument
  mitgewirkt haben, noch die Mitglieder des Projekts f�r dessen
  Fehlerfreiheit. F�r etwaige Fehler und dessen Folgen wird von keiner der
  genannten Personen eine Haftung �bernommen. Es steht jeder Person frei,
  dieses Dokument zu lesen, zu ver�ndern oder auf anderen Medien verf�gbar zu
  machen, solange ein Verweis auf die Internetadresse des Projekts
  \url{http://uni-skripte.lug-jena.de/} enthalten ist.

  Diese Ausgabe tr�gt die Versionsnummer~\SVNRevision{} und ist vom
  \SVNDate{}. Eine neue Ausgabe k�nnte auf der Webseite des Projekts
  verf�gbar sein.

  Jeder ist dazu aufgerufen, Verbesserungen, Erweiterungen und
  Fehlerkorrekturen f�r das Skript einzureichen bzw. zu melden oder diese
  selbst einzupflegen -- einfach eine E-Mail an die Mailingliste
  \texttt{<uni-skripte@lug-jena.de>} senden. Weitere Informationen sind
  unter der oben genannten Internetadresse verf�gbar.

  Hiermit m�chten wir allen Personen, die an diesem Skript mitgewirkt haben,
  vielmals danken:
  \begin{itemize}
   \item Ivo Hedtke \texttt{<hedtke@math.uni-jena.de>} (2008/09)
  \end{itemize}


\tableofcontents

\chapter*{Literatur}

\begin{enumerate}
    \item \textsc{W. Hackbusch}: Iterative L�sung gro�er schwachbesetzter Gleichungssysteme. 1991.
	\item \textsc{A. Meister}: Numerik linearer Gleichungssysteme. Eine Einf�hrung in moderne Verfahren. 2007.
\end{enumerate}

\chapter{Grundbegriffe}

Parallele Algorithmen und Parallelisierung von Iterationsverfahren

\paragraph{Warum?} \begin{itemize}
    \item Verf�gbarkeit von Parallelrechnern (Dual-Core, \dots)
\item Physikalische Grenzen
\item Parallelit�t in vielen Anwendungen
\end{itemize}

\section{Modellproblem: Skalarprodukt von Vektoren}

Seien $x,y \in \bR^n$. Wir wollen berechnen: \[S = \langle x,y \rangle = \sum_{i=0}^{n-1}x_i y_i.\]

\paragraph{Parallelit�t?}
\begin{enumerate}
    \item Berechnung der $x_i y_i$ $\forall i$ unabh�ngig
\item Sei $P$ die Anzahl der Prozessoren ($n \gg P$). Die Indizes $\{0,\dots, n-1\}$ werden auf die Prozessoren verteilt: \[I_p \subset \{0,\dots,n-1\}.\] Jeder Prozessor berechnet eine Teilsumme: \[S_p = \sum_{i \in I_p} x_i y_i.\]
\item Gesamtsumme:
\begin{enumerate}
    \item sequentiell: $S = \sum_{p=0}^{P-1} S_p$ oder
\item parallel: (Beispiel $P=8)$\begin{align*}
S &= \underbrace{S_0 + S_1}_{S_{01}} + \underbrace{S_2 + S_3}_{S_{23}} + \underbrace{S_4 + S_5}_{S_{45}} + \underbrace{S_6 + S_7}_{S_{67}}\\
&= \underbrace{S_{01} + S_{23}}_{S_{0123}} + \underbrace{S_{45} + S_{67}}_{S_{4567}}\\
&= \underbrace{S_{0123} + S_{4567}}_{S_{01234567}}\\
&= S_{01234567} = S
\end{align*}
\end{enumerate}
Statt $7$ Schritten im sequentiellen Verfahren werden hier nur $3$ ben�tigt.
\end{enumerate}

Gesamtaufwand sequentiell: $\cO(n)$.

Aufwand in jedem parallelen Schritt:
\begin{enumerate}
    \item $\frac{n}{P}$
\item $\frac{n}{P}$
\item $\log P$
\end{enumerate}

Gesamtaufwand im parallelen Verfahren: $\cO\left(\frac{n}{P} + \log P\right)$. F�r die Extremf�lle haben wir:
\begin{itemize}
    \item $n \to \infty$: $\cO\left(\frac{n}{P}\right)$, eine Beschleunigung um den Faktor $P$
\item $n \approx P$: $\cO(\log N)$
\end{itemize}

\section{Kommunizierende sequentielle Prozesse}

\begin{defi}[Sequentieller Prozess]
Abstraktion, Ausf�hren eines sequentiellen Programms. Jederzeit klarer Zustand. Genau ein Befehlsz�hler (PC, Program Counter) und Variablen (Speicher, Register).
\end{defi}

\begin{defi}[Paralleles Programm]
Interagierende sequentielle Prozesse. Sinnvoll auf mehreren Prozessoren. Gegebenenfalls zyklisches Umschalten.
\end{defi}

\newpage\minisec{Vereinfachtes Muster eines parallelen Programms}

\begin{verbatim}
{
   globale Variablen

   thread<name>[Parameter]{
      lokale Variablen
      Anweisungen
   }

   thread<name>[...]{
      ...
   }

   ...
}
\end{verbatim}

\minisec{Modell mit statischen Prozessen}

Starte alle.  Paralleles Programm terminiert, wenn alle Prozesse terminieren.

�blicherweise unterscheidet man zwischen:
\begin{itemize}
    \item Prozessen: Eigener Adressraum
\item Threads: Gemeinsamer Adressraum, billiger
\end{itemize}

\minisec{Darstellung des Skalarproduktes in verschiedenen Modellen}

\begin{bsp} Skalarprodukt mit $2$ Prozessen\\
\verb|{ N=8; real X[N], Y[N], s=0;|\\
\verb|   thread| $\pi_\mathtt{1}$ \verb|{ real t=0;|\\
\verb|      for i=0 ...| $\mathtt{\left(\frac{N}{2} -1\right)}$\\
\verb|         t = t + X[i]*Y[i];|\\
\verb|      s = s + t;}|\\
\verb|   thread| $\pi_\mathtt{2}$ \verb|{ real t=0;|\\
\verb|      for i=| $\mathtt{\frac{N}{2}}$ \verb|... N|\\
\verb|         t = t + X[i]*Y[i];|\\
\verb|      s = s + t;}|\\
\verb|}|

Es fehlt die Eingabe von \texttt{X} und \texttt{Y}. Ein Thema hier ist \textit{paralleles Lesen}. Fehler k�nnen beim \textit{parallelen Schreiben} auftreten. Was ist das Ergebnis beim parallelen Schreiben?

Problem ist die Zeile "`\verb|s=s+t|"'.  
\end{bsp}

\section{Der kritische Abschnitt (Detailbetrachtung von \texttt{s=s+t})}

(Die Zahlen in den Kreisen kennzeichnen Programmzeilen, auf die wir uns sp�ter wieder beziehen werden.)

\begin{tabbing}
\ding{202}\verb|  |\= \verb|ADDIERE  R1, R2 NACH R3| XXXXXXXXXXX \= \ding{202}\verb|  |\= \verb|ADDIERE  R1, R2 NACH R3|\kill
\verb|THREAD| $\pi_\mathtt{1}$ \> \> \verb|THREAD| $\pi_\mathtt{2}$\\
\ding{202}\>\verb|LADE     s      IN   R1| \> \ding{204}\>\verb|LADE     s      IN   R1|\\
\>\verb|LADE     t      IN   R2| \> \>\verb|LADE     t      IN   R2|\\
\>\verb|ADDIERE  R1, R2 NACH R3| \>\>\verb|ADDIERE  R1, R2 NACH R3|\\
\ding{203}\>\verb|SCHREIBE R3     NACH s| \> \ding{205}\>\verb|SCHREIBE R3     NACH s|\\
\end{tabbing}





Innerhalb eines Prozesses wird die Reihenfolge der Programmzeilen eingehalten.

CSP: Hintereinanderausf�hrung in jedem Thread.
\begin{itemize}
    \item In Thread $\pi_1$ wird Zeile 1 zuerst und Zeile 4 zuletzt ausgef�hrt
\item Analog in Thread $\pi_2$
\end{itemize}

Aber wie sieht die relative Reihenfolge aus, wenn die Threads nicht Zeile f�r Zeile parallel laufen?

Beispiele:
\begin{center}
\begin{tabular}{ll}
\toprule
\textbf{Reihenfolge} & \textbf{Ergebnis}\\ \midrule
\ding{202} \ding{203} \ding{204} \ding{205} & $s=t_1 + t_2$\\
\ding{204} \ding{205} \ding{202} \ding{203} & $s= t_2 + t_2$ auch korrekt\\
\ding{202} \ding{204} \ding{203} \ding{205} & $s = t_2$ falsch\\
\ding{202} \ding{204} \ding{205} \ding{203} & $s=t_1$ falsch\\
\ding{204} \ding{202} \ding{203} \ding{205} & $s=t_2$ falsch\\
\ding{204} \ding{202} \ding{205} \ding{203} & $s=t_1$ falsch\\\bottomrule
\end{tabular}
\end{center}


Ben�tigen einen Mechanismus, sodass die beiden Prozessoren aufeinander R�cksicht nehmen.

\minisec{L�sung des Problems "`Kritische Abschnitte"' (critical region)}
Mit gegenseitigem Ausschluss (mutual exclusion = "`MUTEX"')
Schreibweise in CSP: Anweisung in eckige Klammern setzen: \verb|[s=s+t]|. Die Reihenfolge (auch im Konfliktfall) ist noch immer unbestimmt, diese brauchen wir auch nicht. Aber im Konfliktfall wird diese Operation nacheinander abgearbeitet.

Es wird so ausgef�hrt: [\ding{202} \ding{203}] [\ding{204} \ding{205}] oder [\ding{204} \ding{205}] [\ding{202} \ding{203}]

\section{Parametrisierte Form (single program -- multiple data, = "`SPMD"')}

H�ufig eine variable Zahl von Prozessen. Wir wollen eine Aufteilung gleichartiger Operationen. Wir wollen ein Programm f�r alle.

\begin{bsp}
Skalarprodukt mit $P=$\texttt{Pr} Prozessoren

\verb|{ int N, Pr; real X[N], Y[N], s=0;|\\
\verb|   thread| $\pi$ \verb|(int p) {|\\
\verb|      real t = 0;|\\
\verb|      for i=| $\mathtt{N\cdot \frac{p}{Pr}}$ \verb|...| $\mathtt{\left( N \cdot \frac{p+1}{Pr} - 1 \right)}$\\
\verb|         t = t + X[i]*Y[i];|\\
\verb|      [s = s + t;]}   // Kritischer Abschnitt|\\
\verb|}|

Jeder Prozess wird meinem anderen $p$ gestartet. Das Programm macht daraus die Verteilung der Operationen.

F�r diese Implementierung haben einen Aufwand von $\cO \left( \frac{N}{P} + P\right)$
\end{bsp}

\section{Verbesserung der Effizienz}

Der kritische Abschnitt l�uft sequentiell ab.

Ziel ist es, den kritischen Abschnitt baumartig auszuf�hren, damit ein logarithmischer Aufwand entsteht. Auch ein Ziel ist es, dass der kritische Abschnitt nicht mehr kritisch ist.

\begin{figure}[t!]
\centering \begin{tikzpicture}
\draw[->] (-2,0) -- (-1,0) (-1.5,0) -- (-1.5,-6.5);
\draw (-1.5,-3) node[rotate=90,above] {Schritte};
\draw (-1.5,-1) node[right] {1.};
\draw (-1.5,-3) node[right] {2.};
\draw (-1.5,-5) node[right] {3.};
\draw (0,0) node[right] (a) {$S_0$};
\draw (1,0) node[right] (b) {$S_1$};
\draw (2,0) node[right] (c) {$S_2$};
\draw (3,0) node[right] (d) {$S_3$};
\draw (4,0) node[right] (e) {$S_4$};
\draw (5,0) node[right] (f) {$S_5$};
\draw (6,0) node[right] (g) {$S_6$};
\draw (7,0) node[right] (h) {$S_7$};
\draw (0,-2) node[right] (aa) {$S_0 + S_1$};
\draw (2,-2) node[right] (bb) {$S_2 + S_3$};
\draw (4,-2) node[right] (cc) {$S_4 + S_5$};
\draw (6,-2) node[right] (dd) {$S_6 + S_7$};
\draw (0,-4) node[right] (aaa) {$S_0 + S_1 + S_2 + S_3$};
\draw (4,-4) node[right] (bbb) {$S_4 + S_5 + S_6 + S_7$};
\draw (0,-6) node[right] (aaaa) {$S_0 + S_1 + S_2 + S_3+S_4 + S_5 + S_6 + S_7$};
\draw[->] (a.south west) -- (aa.north west);
\draw[->] (b.south west) -- ($(aa.north west)+(0.1,0)$);
\draw[->] (c.south west) -- (bb.north west);
\draw[->] (d.south west) -- ($(bb.north west)+(0.1,0)$);
\draw[->] (e.south west) -- (cc.north west);
\draw[->] (f.south west) -- ($(cc.north west)+(0.1,0)$);
\draw[->] (g.south west) -- (dd.north west);
\draw[->] (h.south west) -- ($(dd.north west)+(0.1,0)$);
\draw[->] (aa.south west) -- (aaa.north west);
\draw[->] (bb.south west) -- ($(aaa.north west)+(0.1,0)$);
\draw[->] (cc.south west) -- (bbb.north west);
\draw[->] (dd.south west) -- ($(bbb.north west)+(0.1,0)$);
\draw[->] (aaa.south west) -- (aaaa.north west);
\draw[->] (bbb.south west) -- ($(aaaa.north west)+(0.1,0)$);
\draw (0,-.5) node[right,fill=black!20] {\tiny \sffamily 000};
\draw (1,-.5) node[right,fill=black!20] {\tiny \sffamily 001};
\draw (2,-.5) node[right,fill=black!20] {\tiny \sffamily 010};
\draw (3,-.5) node[right,fill=black!20] {\tiny \sffamily 011};
\draw (4,-.5) node[right,fill=black!20] {\tiny \sffamily 100};
\draw (5,-.5) node[right,fill=black!20] {\tiny \sffamily 101};
\draw (6,-.5) node[right,fill=black!20] {\tiny \sffamily 110};
\draw (7,-.5) node[right,fill=black!20] {\tiny \sffamily 110};
\draw (0,-2.5) node[right,fill=black!20] {\tiny \sffamily 000};
\draw (2,-2.5) node[right,fill=black!20] {\tiny \sffamily 010};
\draw (4,-2.5) node[right,fill=black!20] {\tiny \sffamily 100};
\draw (6,-2.5) node[right,fill=black!20] {\tiny \sffamily 110};
\draw (0,-4.5) node[right,fill=black!20] {\tiny \sffamily 000};
\draw (4,-4.5) node[right,fill=black!20] {\tiny \sffamily 100};
\draw (0,-6.5) node[right,fill=black!20] {\tiny \sffamily 000};
\end{tikzpicture}
\caption{Schema zu "`Skalarprodukt Baumkombination"'. \newline \tiny Die Pfeile kennzeichnen Warteabh�ngigkeiten zwischen Threads. Die Zahlen in den grauen K�sten sind die Bin�rdarstellungen der Zahl $p$ des aktivin Threads $p$.}

\vspace{5mm}

\hrule
\end{figure}

Wir brauchen eine globale Variable f�r Zwischenergebnisse \verb|S[p]| und f�r die Terminierung $\verb|flag[p]|.$

Sei \texttt{Pr}$=P= 2^d$. Im Schritt $i \in \{0, \dots, d-1\}$ addiert
\begin{center}
Thread "`$l=\dots \underbrace{0\dots\dots 0}_{i+1 \text{ Nullen} }$"'
\end{center}
mit
\begin{center}
Thread "`$r=\dots 1 \underbrace{0\dots\dots 0}_{i \text{ Nullen}}$"'
\end{center}
(die vorderen Zifferen -- die "`\dots"' -- m�ssen gleich sein), der danach terminiert.

\minisec{Skalarprodukt Baumkombination}
\verb|{ int N, d, Pr=| $\mathtt{2^d}$ \verb|; real X[N], y[N], s[Pr]=0;|\\
\verb|  boolean flag[Pr] = false; // Initialisierung vor Start des|\\
\verb|                           // ersten Threads|\\
\verb|   thread | $\pi$ \verb|(int p){|\\
\verb|      for i=| $\mathtt{N\cdot\frac{p}{Pr}}$ \verb|...| $\mathtt{\left( N\cdot\frac{p+1}{Pr} - 1\right)}$\\
\verb|         S[p] = S[p] + X[i]*Y[i];|\\
\verb|      for i = 0,...,d-1|\\
\verb|         l = (p >> (i+1)) << (i+1); // l�sche (i+1) Bits|\\
\verb|         r = l +| $\mathtt{2^i}$ \verb|;|\\
\verb|         if (p==r)|\\
\verb|            flag[p]=true; end;|\\
\verb|         if (p==l)|\\
\verb|            while (flag[r] == false) // BUSY WAIT|\\
\verb|               S[p] = S[p] + S[r];}|\\
\verb|}|

Ergebnis liegt am Ende in \verb|S[0]|. Alternativ zum Busy Wait kann man Synchronisation des OS verwenden.




\end{document}
