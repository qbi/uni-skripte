\renewcommand{\lecdate}{13.01.2015}

\begin{Fakt}[Schwacher Approximationssatz]
 Seien $|\cdot|_1,\ldots,|\cdot|_n$ paarweise inäquivalente Bewertungen eines Körpers $K$ und $a_1,\ldots,a_n\in K$. Dann existiert für alle $\eps>0$ ein $a\in K$, s.d. $|a-a_i|_i<\eps$.
 D.h. die Diagonale $K\rightarrow K\times \ldots \times K$: $a\mapsto (a,\ldots,a)$ ist dicht.
\end{Fakt}

\begin{Beweis}
 Es existiert $x\in K$ mit $|x|_1<1$, $|x|_n\geq 1$ und ein $x'$ mit $|x'|_1\geq 1$, $|x'|_n<1$. Also gilt für $y=x'/x$: $|y|_1>1$, $|y|_n<1$. Induktion über $n$:
 Es existiert $z\in K:$ $|z|_1>1, |z|_2<1, \ldots, |z|_n<1$. IA: $n=2$ siehe oben. Sei also $|z|_1>1, |z|_2<1, \ldots, |z|_{n-1}<1$. Ist $|z|_n\leq 1$, so ist $z^my$ gut für $m>>1$:
 $|z^my|=|z|^m|y|>1$, $|z|_j^m|y|_j<1$ für $m>>1$ und $2\leq j\leq n-1$, $|z|_n^m|y|_n<1$ wegen $|y|_n<1$. 
 
 Der heikle Fall ist $|z|_n>1$: Wir betrachten $t_m=\frac{z^m}{1+z^m}$, dann gilt $t_m\rightarrow 1$ in $|\cdot|_1$: $|t_m-1|_1=\frac{1}{|1+z^m|_1}\leq \frac{1}{|z|_1^m-1}\rightarrow 0$.
 
 Analog für $|\cdot|_n$: $|t_m-1|_n\leq \frac{1}{|z|_n^m-1}\rightarrow 0$ \folge $t_m\rightarrow 1$ in $|\cdot|_n$.
 
 Für $2\leq j\leq n-1$ gilt: $|t_m|_j=\frac{|z|_j^m}{|1+z^m|_j}\leq \frac{|z|_j^m}{1-|z|_j^m}\rightarrow 0$. Also ist $(t_m)$ eine Nullfolge für diese $|\cdot|_j$. Aus $t_m\rightarrow 1$ in $|\cdot|_n$ folgt $|t_m|_n\rightarrow 1$: $||t_m|_n-1|\leq |t_m-1|_n \rightarrow 0$. Also leistet $t_my$ das Gewünschte.
 
 Sei nun $z\in K$, s.d. $|z|_1>1$, $|z|_j<1$ für $j=2,\ldots,n$. Dann gilt \[ \frac{z^m}{1+z^m}\rightarrow\begin{cases}
                                                                                                           1 & \text{ für } |\cdot|_1\\
                                                                                                           0 & \text{ für } |\cdot|_2, \ldots, |\cdot|_n
                                                                                                          \end{cases}.
\]
Es folgt $\forall \eta>0\exists w\in K\kreuz$ mit $|w-1|_1<\eta$, $|w|_j<\eta$ für alle $j=2,\ldots,n$. Oder: $\forall j\forall \eta>0\exists w_j\in K\kreuz$ mit $|w_j-1|<\eta$, $|w_j|_i<\eta$ für alle $i\neq j$.

Setze $a=\sum_{j=1}^n a_j w_j$. $a-a_j=\sum a_i w_i + a_j(w_j-1)$. \folge $|a-a_j|_j\leq \sum_{i\neq j} |a_i|_j |w_i|_j + |a_j|_j|w_j-1|_j\leq c\cdot\eta$. $\eta=\sum_{i\neq j} |a_i|_j$ 
\end{Beweis}

\begin{Bemerkung}
 \begin{enumerate}
  \item Das ist eine Verallgemeinerung des chinesischen Restsatzes (Details später).
  \item Starker Approximationssatz im Fall $\IQ$: Ist $\nu_0$ eine fixierte Bewertung von $\IQ$, so kann man $a$ so wählern, dass $|a|_\nu\leq 1$ für alle $\nu\neq\nu_0$ .
 \end{enumerate}
\end{Bemerkung}

\begin{Definition}
 Eine Bewertung heißt \highl{nichtarchimedisch}, wenn $|n|\leq 1$ für alle $n\in\IN$, sonst \highl{archimedisch}.
\end{Definition}

\begin{Beispiel}
 Der übliche Absolutbetrag auf $\IQ$ ist archimedisch, die $p$-adische Bewertungen sind nichtarchimedisch: $|a+b|_p\leq \max(|a|_p,|b|_p)$.
\end{Beispiel}

\begin{Fakt}
 $|\cdot|$ ist nichtarchimedisch \gdw $|x+y|\leq \max(|x|,|y|)$ für alle $x,y\in K$.
\end{Fakt}

\begin{Beweis}
 Aus der ultrametrischen Ungleichung folgt $|n|\leq \max(|1|,\ldots,|1|)$, $|1|=1$, also $|n|\leq 1$ für alle $n\in \IN$. Sei nun $n\leq 1$ für alle $n\in\IN$. Seien $x,y\in K$, $|x|\leq |y|$.
 \[ |x+y|^n\leq \sum_{r\neq 0}^n |\binom{n}{r}||x|^r|y^{n-r}|\leq \sum |x|^r|y^{n-r}|\leq (n+1)|y|^r\]
 \folge $|x+y|\leq (n+1)^{1/n}|y|$ für alle $n\in\IN$ \folge $|x+y|\leq |y|$. Also $|x+y|\leq \max(|x|,|y|)$.
\end{Beweis}

\begin{Satz}[OSTROWSKI]
 Jede (nichttriviale) Bewertung von $\IQ$ ist äquivalent zu $|\cdot|_\infty$ (Absolutbetrag) oder zu einer der $p$-adischen Bewertungen, $p\in\IP$: $|\cdot|_p$.
\end{Satz}

\begin{Beweis}
 Sei $|\cdot|$ nichtarchimedische Bewertung von $\IQ$. Dann ist also $|n|\leq 1$ für alle $n\leq IN$. Wäre $|n|=1$ für alle $n\in\IN\oN$, so würde dies auch für alle rationalen Zahlen außer $0$ folgen. Also $|\cdot|$= triviale Bewertung. Also existiert eine Primzahl $p\in\IP$ mit $|p|<1$. Sei $I=\{a\in\IZ: |a|<1\}$ - das ist ein Ideal in $\IZ$ und $p\IZ\subset I\subsetneq \IZ$, also $I=p\IZ$. Für $m\in\IZ$ gilt: $m=p^{\ord_p(m)}\cdot m_0$, $\ggT(p,m_0)=1$. Also $|m|=|p|^r|m_0|=|p|^r$, $r=\ord_p(m)$. Das verallgemeinert sich sofort auf rationale Zahlern $\neq 0$. Also $|\alpha|=c^{\ord_p(\alpha)}$, $0<c=|p|<1$. D.h. $|\cdot|=|\cdot|_p$.
 
 Sei $|\cdot|$ archimedisch, $m,n\in\IN$ beide $>1$, $m=a_0+a_1n+\ldots+a_rn^r$, $0\leq a_j<n$, $a_r\neq 0$. Dann ist $n^r\leq m$, $|a_j|\leq n$, also $|m|\leq \sum_{j=0}^r|a_j||n|^j\leq n(r+1)\max(1,|n|)^r$.
 
 Nun ist $r\log n\leq \log m$, also $r\leq \frac{\log m}{\log n}$, und damit $|m|\leq n(1+\frac{\log m}{\log n})\max(1,|n|)^{\log m/\log n}$.
 
 Ersetze $m$ durch $m^k$, das gibt: $|m|\leq n^{1/k}(1+k\frac{\log m}{\log n})^{1/k} \max(1,|n|)^{\log m/\log n}$. Für $k\rightarrow\infty$ folgt $|m|\leq \max(1,|n|)^{\log m/\log n}$. Die Bewertung ist archimedisch, also ex. $m\in \IN$ mit $|m|>1$. $|0|=0$, $|1|=1$ \folge $m\geq 2$. Somit $1<\max(1,|n|)^{\log m/\log n}$ \folge $|n\geq 1|$ für alle $n\geq 2$. Es folgt $|m|\leq |n|^{\log m/\log n}$ oder $|m|^{1/\log m}\leq |n|^{1/\log n}$ für alle $m,n\geq 2$. Somit $|m|^{1/\log m}=|n|^{1/\log n}$ für alle $m,n\geq 2$. $|m|^{1/\log m}=:c=e^s$, $c>1$ \folge $s>0$. Also $|m|=e^{s\log m}=|m|_\infty^s$. Das gilt auch für $m=0,1$ und folgt dann für $\IZ$ und $\IQ\kreuz$.
\end{Beweis}

\begin{Fakt}
Sei $K$ eine Körper mit Bewertung $|\cdot|$, dann existiert ein Körper $\hat K$, eine Einbettung $K\rightarrow \hat K$, eine Fortsetzung $|\cdot|\hat{}$ von $|\cdot|$ auf $\hat K$ mit 
\begin{enumerate}
 \item $K$ ist dicht in $\hat K$.
 \item $\hat K$ ist vollständig.
\end{enumerate}
Darüber hinaus ist $\hat K$ bis auf Isometrie eindeutig bestimmt: Ist $\hat L$, $\|\cdot\|\hat{}$ weiterer solcher Körper, so existiert $K$-Isomorphismus $\sigma: \hat K\rightarrow \hat L$ mit $\|\sigma x\|\hat{}=|x|\hat{}$ für alle $x\in \hat K$.
\end{Fakt}
