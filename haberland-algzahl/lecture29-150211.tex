\renewcommand{\lecdate}{11.02.2015}

\begin{Beispiel}[d'apr´es WEIL, A. Exercise dyadiques Inv. Math. 27 (1974), 1-22]
 Wir konstruieren eine $S_4$-Erweiterung von $\IQ_2$.
 \begin{enumerate}[1)]
  \item Sei $E$ der ZFK des Polynoms $X^3-2$. Es ist irreduzibel über $\IQ_2$, die $3$ Wurzeln sind $\pi,\zeta\pi,\zeta^2\pi$ mit $\pi$ primitive 3. Einheitswurzel, $\pi^3=2$.
  Also ist $E=\IQ_2(\zeta,\pi)$. $[\IQ_2(\pi):\IQ_2]=3$, voll verzweigt. $[\IQ_2(\zeta):\IQ_2]=2$, unverzweigt ($2$ prim zu $3$). Also hat $E/\IQ_2$ die Galois-Gruppe $S_3$, $e(E/\IQ_2)=3$, $f(E/\IQ_2)=2$.
  
  Die Trägeheitsgruppe ist $A_3$, max. unverzweigte Erweiterung ist $\IQ_2(\zeta)/\IQ_2$. $V=\{ \Id, (1,2)(3,4), (1,3)(2,4), (1,4)(2,3)\}\subset S_4$, $V$ ist Normalteiler und $S_4/V=S_3$. Wir suchen also zwei geeignete quadratische Erweiterungen von $E$.
  \item Seinen $\alpha=\pi-1$, $\beta=\zeta\pi-1$, $\gamma=\zeta^2\pi-1$. Das sind Einheiten in $E$, sie sind konjugiert unter der Galoisgruppe. Es gilt $\alpha\beta\gamma=(\pi-1)(\pi-\zeta)(\pi-\zeta^2)=\pi^3-1=1$.
  Keine der drei Zahlen $\alpha,\beta,\gamma$ ist Quadrat in $E$:
  Sei $u=\zeta(1+\pi x)$, $u^2=\zeta^2(1+2\pi x+ \pi^2 x^2)$, also $u^2\equiv \zeta^2\mod{\pi^2}$. Aber $\alpha=\pi-1\equiv \pi+1 \mod{\pi^2}$, also ist $\alpha\neq u^2$. Damit sind $\beta$, $\gamma$ keine Quadrate. Wegen $\alpha\beta\gamma=1$ folgt $\alpha/\beta$, $\beta/\gamma$, $\gamma/\alpha$ sind auch keine Quadrate: $\alpha/\beta=v^2$ \folge $\gamma=\frac{1}{\alpha\beta}=\frac{1}{(\beta v)^2}$ \lightning.
  Somit ist $K=E(\sqrt{\alpha}, \sqrt{\beta}, \sqrt{gamma})/E$ von jeweils $2$ der $3$ Wurzeln erzeugt und ist Galoiserweiterung mit Galoisgruppe $V=C_2\times C_2$ (diese hat $3$ 2-elementige Untergruppen). Die quadratischen Zwischenkörper sind also $E(\sqrt \alpha)$, $E(\sqrt\beta)$, $E(\sqrt\gamma)$ (nach Galoistheorie).
  
  Wir zeigen, dass $K/\IQ_2$ normal ist. Betrachte 
  \begin{align*}
   f(X)&=(X^2-\alpha)(X^2-\beta)(X^2-\gamma)\\
   &=X^6-(\alpha\beta\gamma)X^4+(\alpha\beta+\alpha\gamma+\beta\gamma)X^2-1\\
  \alpha+\beta+\gamma= (1+\zeta+\zeta^2)\pi-3=-3\\
  \alpha\beta+\alpha\gamma+\beta\gamma=\ldots=3
  \end{align*}
  
  Also $f(X)=X^6+3X^4+3X^2-1$. $K$ enthält alle 6 Wurzeln $\pm \sqrt\alpha$, $\pm \sqrt\beta$, $\pm \sqrt\gamma$ von $f$, also den ZFK von $f$ über $\IQ_2$. Andererseits liegen im ZFK von $f$ die Zahlen $\sqrt{\alpha}$, $\sqrt{\beta}$, $\sqrt{\gamma}$, also auch $\pi=1-(\sqrt\alpha)^2$, $\zeta\pi$, $\zeta^2\pi$, also auch $\zeta$.
  
  D.h. $K$ ist der ZFK von $f$ über $\IQ_2$, also normal. Die Galoisgruppe $G=\Gal(K/\IQ_2)$ hat $24$ Elemente, besitzt Normalteiler $V$, die Faktorgruppe ist $S_3$. Also Gruppenerweiterung 
  \[1\rightarrow V\rightarrow G \rightarrow S_3 \rightarrow 1.\]
  Das wird klassifiziert durch $H^2(S_3,V)$ (Homologie).
  
  Seien $\sigma,\tau\in\Gal(E/\IQ_2)$, s.d.
  \begin{align*}
   \sigma:& \pi\mapsto \zeta\pi, \zeta\mapsto\zeta\\
   \tau: & \pi\mapsto \pi, \zeta\mapsto\zeta^2
  \end{align*}
  Diese erzeugen $\Gal(E/\IQ_2)$. Ist $\rho\in\Gal(E/\IQ_2)$, so besitzt $\rho$ genau 4 Liftungen nach $G=\Gal(K/\IQ_2)$. Solch ein $\rho$ permutiert $\alpha,\beta,\gamma$ die Liftungen sehen so auch:
  \begin{align*}
   \tilde\rho: & \sqrt{\alpha} \mapsto \pm\sqrt{\rho\alpha}\\
   &\sqrt{\beta} \mapsto \pm\sqrt{\rho\beta}\\
   &\sqrt{\gamma} \mapsto \pm\sqrt{\rho\gamma}
  \end{align*}
 Wir wählen links die Wurzeln so auch, dass $\sqrt\alpha\sqrt\beta\sqrt\gamma=1$ gilt. Dann treten rechts nur 4 der 8 Vorzeichen-Kombinationen auf.
 
 Betrachte $\theta=\sqrt\alpha+\sqrt{\beta}+\sqrt\gamma\in K$. $\theta$ besitzt unter $G=\Gal(K/\IQ_2)$ genau $4$ Konjugierte.
 \begin{align*}
  \theta_1&=\theta=\sqrt\alpha+\sqrt{\beta}+\sqrt{\gamma},\\
  \theta_2&=\sqrt\alpha-\sqrt{\beta}-\sqrt{\gamma},\\
  \theta_3&=-\sqrt\alpha+\sqrt{\beta}-\sqrt{\gamma},\\
  \theta_4&=-\sqrt\alpha-\sqrt{\beta}+\sqrt{\gamma}.
 \end{align*}
 Also ist $\theta$ Nullstelle eines Polynoms vom Grad $4$ über $\IQ_2$: $g(X)=(X-\theta_1)(X-\theta_2)(X-\theta_3)(X-\theta_4)$. Wir berechnen $g(X)$:
 \begin{enumerate}[i)]
  \item $\theta_1+\theta_2+\theta_3+\theta_4=0$
  \item $\theta_1\theta_2+\theta_1\theta_3+\theta_1\theta_4+\theta_2\theta_3+\theta_2\theta_4+\theta_3\theta_4=6$ (knallhart ausrechnen)
  \item Koeffizient bei $X^1$: $8$ (rechnen)
  \item $\theta_1\theta_2\theta_3\theta_4=-3$ (rechnen)
 \end{enumerate}

 Somit gilt $g(X)=X^4+6X^2-8X-3$. Sei $L\subset K$ der ZFK von $g(X)$. Aus $\theta_1+\theta_4=2\sqrt{\gamma}$, $\theta_1+\theta_2=2\sqrt{\alpha}$, $\theta_1+\theta_3=2\sqrt{\beta}$ folgt $K\subset L$, mithin $L=K$.
 Es folgt $\Gal(K/\IQ_2)=S_4$.
 
 \item Wir zeigen $K/E$ ist voll verzweigt. $K/E$ besitzt $3$ quadratische Zwischenkörper: $E(\sqrt\alpha)$, $E(\sqrt\beta)$, $E(\sqrt\gamma)$. Sie sind konjugiert unter $\Gal(K/E)$, also gleichzeitig verzweigt oder unverzweigt. $E$ besitzt genau eine unverzweigte quadratische Erweiterung, also sind alle drei verzweigt. Es folgt $e(K/E)=4$, $f(K/E)=1$ und $e(K/\IQ_2)=12$ und $f(K/\IQ_2)=2$.
 
 Maximale unverzweigte Erweiterung ist $\IQ_2(\zeta)$, maximale zahmverzweigte ist $E/\IQ_2$, Trägeheitsgruppe ist $A_4$, erste Verzweigungsgruppe ist $V = $ KLEINsche Vierergruppe.
 \end{enumerate}

\end{Beispiel}
