\renewcommand{\lecdate}{14.01.2015}

\begin{Beweis}
$C$ sei der Ring der CAUCHY-Folgen aus $K$. $\Fm$ das Ideal der Nullfolgen, $\hat K:=C/\Fm$. Das ist ein Körper, da $\Fm$ maximales Ideal ist: $(x_n)\notin\Fm \folge |x_n|\geq\eta>0$ für alle $n\geq n_0$. 
\[ y_n:=\begin{cases}
         1/x_n	& n\geq n\\
         1 	& n<n_0	
        \end{cases},
\]
Dann ist $(x_ny_n)\equiv 1\mod{\Fm}$. $K$ bettet sich in $\hat K$ durch die konstanten Folgen ein. Wir setzen $|\cdot|$ fort auf $\hat K$ durch $|(x_n)|:=\lim_{n\rightarrow\infty}|x_n|$.
$(|x_n|)$ ist reelle Fundamentalfolge, wegen $||x_m|-|x_n||\leq |x_m-x_n|$ also konvergent. Ist $(y_n)$ Nullfolge, so ist $\lim_{n\rightarrow\infty} |x_n+y_n|=\lim_{n\rightarrow\infty} |x_n|$. Also wohldefinierte Abbildung: \[ |\cdot|\hat{}:\hat K\rightarrow \IR_+\kreuz=[0,\infty).\]
Die 3 Eigenschaften von Bewertungen sind schnell überprüft. $|(x_n)+\Fm|=0\folge\lim |x_n|=0 \folge x_n\rightarrow 0 \folge (x_n)\in\Fm\folge (x_n)+\Fm=\Fm$.

\begin{enumerate}
 \item $K$ ist dicht in $\hat K$.
 
 Sei $x\in \hat K, (x_n)$ Repräsentant in $C$. Sei $y_n$ die Konstante Folge $(x_n)$. Dann ist \[ |x-y_n|\hat{}=\lim_{k\rightarrow\infty} |x_k-x_n|\leq \eps\]
für alle $n\geq n_0$. Also $y_n\rightarrow x$ in $\hat K$.
\item $\hat K$ ist vollständig.

Sei $(x_n)$ FUndamentalfolge in $\hat K$. Nach (i) existiert $y_n\in K$, s.d. $|x_n-y_n|\hat{}\leq\frac{1}{n}$ (Wir identifizieren $y_n$ mit der konstanten Folge $y_n$).
Es gilt \[ |y_m-y_n|\hat{}\leq |y_m-x_m|\hat{}+|x_m-x_n|\hat{}+|x_n-y_n|\hat{}\leq \frac{1}{m}+|x_m-x_n|\hat{}+\frac{1}{n}.\]
\folge $(y_n)$ ist Fundamentalfolge aus $K$. Sie definiert also ein $x\in \hat K$, Nach (i) gilt $\lim_{n \rightarrow \infty}y_n=x$. Also
\[ |x_n-x|\hat{}\leq |x_n-y_n|\hat{} + |y_n-x|\hat{} \rightarrow 0.\]
Somit gilt $\lim_{n\rightarrow \infty}x_n=x$
\item Die Eindeutigkeit

Wir haben $\hat L \overset{j}{\leftarrow} K \overset{i}{\rightarrow} \hat K$ mit Einbettungen $i,j$.
Sei $x\in \hat K$ und $x=\lim_{n\rightarrow \infty} i(y_n)$, $y_n\in K$. $(y_n)$ ist CAUCHY-Folge in $K$, also ist $j(y_n)$ eine solche in $\hat L$. Sei $\sigma(x)=\lim_{n\rightarrow\infty}j(y_n)$. Ist $(z_n)$ eine weitere Folge aus $K$ mit $x=\lim i(z_n)$, so ist $(y_n-z_n)$ Nullfolge, also auch $(j(y_n)-j(z_n))$ Nullfolge in $\hat L$. Somit ist $\sigma:\hat K\rightarrow\hat L$ wohldefiniert. Wegen $|x|\hat{}=\lim |y_n|$ und $\|\sigma\|\hat{}=\lim |y_n|$ ist $\|\sigma x\|\hat{}=|x|\hat{}$, also ist $\sigma$ eine Isometrie. Das Bild $\sigma(\hat K)\subset \hat L$ enthält die dichte Teilmenge $j(K)$. Sei $x\in \hat L$, $x_n\in\sigma(\hat K)$, $x_n\rightarrow x$ \folge $(x_n)$ Fundamentalfolge \folge $x\in\sigma(\hat K)$, wegen der Vollständigkeit von $\sigma(\hat K)$. Also ist $\sigma(\hat K)$ abgeschlossen in $\hat L$.
 \end{enumerate}
 \end{Beweis}
 
 \subsection{Die $p$-adischen Zahlen}
 
 \begin{Definition}
  Sei $p\in\IP$, dann heißt die Vervollständigung $\IQ_p$ von $\IQ$ bezüglich der $p$-adischen Bewertung der \highl{Körper der $p$-adischen Zahlen}. Die Teilmenge
  \[\IZ_p=\{ x\in\IQ_p : |x|_p\leq 1\} \]
  heißt \highl{Ring der ganzen $p$-adischen Zahlen}.
 \end{Definition}
 
 \begin{Fakt}
  \begin{enumerate}
   \item $\IZ_p$ ist ein dBR.
   \item Jedes $x\in\IZ_p$ besitzt eine eindeutige Darstellung $x=\sum_{j=0}^{\infty} a_jp^j$, $0\leq a_j<p$, $a_j\in\IN$.
   \item Jedes $x\in\IQ_p$ besitzt eine eindeutige Darstellung $x=\sum_{j>>-\infty} a_jp^p$, $0\leq a_j<p$, $a_j\in\IN$.
  \end{enumerate}
 \end{Fakt}
 
 \begin{Beweis}
 \begin{enumerate}
  \item Es gilt $|x+y|_p\leq \max(|x|_p,|y|_p)$, also ist $\IZ_p$ ein Ring. Offensichtlich ist $\IZ_p\kreuz=\{u\in\IZ_p : |u|_p=1\}$ und nach (ii) ist $\IZ_p/p\IZ=\IF_p$. Also ist $(p)=p\IZ_p$ maximales Ideal. Jedes andere Ideal $\neq \IZ_p$ ist in diesem enthalten: Es enthält keine Einheiten. Somit ist $\IZ_p$ ein lokaler Ring. Wir zeigen, dass jedes Ideal $\neq (0), \IZ_p$ die Form $(p^n)=p^n\IZ_p$ hat.
  
  Sei $I\subset\IZ_p$ solch ein Ideal und $x\in I$ habe maximale Bewertung $(|p|_p=\frac{1}{p})$. $x=p^m\cdot u$, $u\in\IZ_p\kreuz$, wieder nach (ii), also $p^m\in I$. Jedes $y\in I$ erfüllt $|y|_p\leq |x|_p$, also $y=ax$ mit $a\in\IZ_p$ \folge $I=(p^m)$. Wir haben gezeigt: $\IZ_p$ ist dBR.
  \item Aus $\sum a_jp^j=\sum b_jp^j$ und $a_0=b_0,\ldots,a_{r-1}=b_{r-1}$, $a_r\neq b_r$ folgt $a_r+$ (Norm<1) $= b_r + $(Norm<1) \folge $|a_r-b_r|<1$ \lightning  
  
  Alle solche Reihen konvergieren: $|\sum_{M}^Na_jp^j|\leq p^{-M}$.
  \begin{Lemma}
   Sei $x\in\IQ$ mit $|x|_p\leq 1$ (d.h. der Nenner ist prim zu $p$). Dann gilt: $\forall n\in\IN\exists! m\in\{0,1,\ldots,p^n-1\}:|x-m|_p\leq\frac{1}{p^n}$.
  \end{Lemma}
  \begin{Beweis}
    Sei $x=\frac{a}{b}$, $\ggT(a,b)=1$, $b>0$, $p\nmid b$.
  
    Also $\ggT(b,p^n)=1$ \folge $r,s\in\IZ$: $rb+sp^n=1$. Sei $m':=ar$, es folgt $|x-m'|_p=|\frac{a}{b}-ar|_p=|a|_p|\frac{1}{b}-r|_p=|a|_p|1-br|_p$ ($|b|_p=1$) $=|a|_p|sp^n|_p\leq\frac{1}{p^n}$.
    Das selbe gilt für alle $m'+p^nt$, $t\in\IZ$, wg. der ultrametrischen Ungleichung.
    
    Eindeutigkeit: $|x-m_1|_p<\frac{1}{p^n}$, $|x-m_2|_p<\frac{1}{p^n}$ \folge $|m_1-m_2|_p<\frac{1}{p^n}$ \folge $m_2-m_1$ ist durch $p^n$ teilbar. $0\leq m_1,m_2<p^n$ \folge $m_1=m_2$.
    Das war das Lemma.
  \end{Beweis}
  Sei $x\in\IZ_p$, $(\alpha_n)$ Fundamentalfolge aus $x$. $\forall k\geq 1\exists N(k)\forall m,n\geq N(k): |\alpha_m-\alpha_n|_p<\frac{1}{p^k}$.
  
  Man darf voraussetzen, dass die Folge der $N(k)$ streng monoton wächst. Es gilt $|\alpha_n|_p\leq\max\{|\alpha_n-\alpha_m|_p,|\alpha_m|_p\}\leq\max\{\frac{1}{p},|\alpha_m|_p\}$ für alle $m,n\geq N(1)$.
  Nun ist $\lim_{n\rightarrow\infty} |\alpha_m|_p=|x|_p\leq 1$, also gilt $|\alpha_n|_p\leq 1$ \forall $n\geq N(1)$. Wir wählen für jedes $\alpha_{N(k)}$ das $\beta_k\in\{0,1,\ldots,p^k-1\}$ aus dem Lemma. Also $|\alpha_{N(k)-\beta_K}|_p\leq \frac{1}{p^k}$. Dann ist $(\beta_k)\in x$, d.h. $\lim\beta_k=x$ und $|\beta_{k+1}-\beta_k|\leq \max(|\alpha_{N(k+1)}-\alpha_{N(k)}|_p,|\alpha_{N(k+1)}-\beta_{k+1}|_p,|\alpha_{N(k)}-\beta_k|_p)\leq \frac{1}{p^k}$. D.h. $\beta_{k+1}-\beta_k$ ist durch $p^k$ teilbar. Wegen $0\leq \beta_k<p^k$, $0\leq \beta_{k+1}<p^{k+1}$ folgt $-p^k<\beta_{k+1}-\beta_k<p^{k+1}$, also $\beta_{k+1}\geq \beta_k$ und $\beta_{k+1}=\beta_k+a_kp^k$, $0\leq a_k<p$. Somit $\beta_k=\sum_{j=0}^k a_jp^j$, $k\rightarrow\infty$ gibt $x=\sum_{j=0}^\infty a_jp^j$.
  \item folgt sofort: $x\in\IQ_p\kreuz$ \folge $p^Nx\in\IZ_p$ für geeignete $N\in\IN$.
\end{enumerate}  
\end{Beweis}
  
  \begin{Folgerung}
   \begin{enumerate}
    \item $x\in\IQ_p\kreuz$ \folge $x=\frac{a_{-N}}{p^N}+\ldots+\frac{a_{-1}}{p}+a_0+a_1p+\ldots$, $a_{-N}\neq 0$ ($0\leq a_j<p$) \folge $|x|_p=p^N$.
    \item $\IZ_p$ ist integer: ist klar, denn es liegt in einem Körper - alternativ: $xy=0$ \folge $|xy|_p=0$ \folge $|x|_p|y|_p=0$ \folge $|x|_p=0$ oder $|y|_p=0$ \folge $x=0$ oder $y=0$.
    \item $\card \IZ_p=\card\IR$
   \end{enumerate}
  \end{Folgerung}

   \begin{Beispiel}
    $p=13$: $-\frac{1}{12}=\frac{1}{1-13}=\sum_{n=0}^{\infty}(13)^n$
   \end{Beispiel}





