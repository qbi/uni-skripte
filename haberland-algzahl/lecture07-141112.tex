%!TEX root = Algebraische_Zahlentheorie.tex

\renewcommand{\lecdate}{12.11.14}

\begin{Beispiel}
 \[\Id_\IZ = \prod_{p\in\IP}\langle p\rangle\]
\end{Beispiel}

\begin{Beispiel}
 $K=\IQ(\sqrt{-5})$, $\cO_K=\IZ[-5]$, denn $-5=3\mod{8}$. Es gilt $21=3\cdot 7=(4+\sqrt{-5})(4-\sqrt{-5})$. Wir haben gesehen, dass dies zwei Faktorisierungen in irreduzible Faktoren sind.
 \begin{align*}
 \Fp_1&=(3)+(4+\sqrt{-5})=(3,4+\sqrt{-5})\\
 &=\{(a+b\omega)3+(c+d\omega)(4+\omega), a,b,c,d\in\IZ\} && \omega=\sqrt{-5}\\
 \Fp_2&=(3,4-\omega)\\
 \Fp_3&=(7,4+\omega)\\
 \Fp_4&=(7,4-\omega)
 \end{align*}
 
 Basis von $\cO_K$ ist $1,\omega$. Wir bilden $I=\IZ\cdot 3+\IZ\cdot (4+\omega)\subset \Fp_1$. Es gilt 
 \begin{align*}
    \omega\cdot 3&=3(4+\omega)-4\cdot 3\in I\\
    \omega\cdot (4+\omega)&=4(4+\omega)-7\cdot 3\in I\\
 \end{align*}
 \folge $(a+b\omega)3+(c+d\omega)(4+\omega)\in I$ \folge $\Fp_1\subset I$.
 
 Also ist $3$, $4+\omega$ eine Basis von $\Fp_1$. Der Index von $\Fp_1$ in $\cO_K$ ist daher \[ \left|\det\begin{pmatrix} 3 & 4\\ 0 & 1 \end{pmatrix} \right|=3\]
 Somit ist $\cO_K/\Fp_1=\IF_3$ \folge $\Fp_1$ maximal. Es folgt $\cO_K/\Fp_2=\IF_3$.
 
 Wir zeigen $7$, $4+\omega$ ist $\IZ$-Basis von $\Fp_3$:
  \begin{align*}
    \omega\cdot 7&=7(4+\omega)-4\cdot 7\in I\\
    \omega\cdot (4+\omega)&=4(4+\omega)-3\cdot 7\in I\\
 \end{align*}
 
 Also $(\cO_K : \Fp_3)=\left|\det\begin{pmatrix}
                                  7 & 4\\ 0& 1
                                 \end{pmatrix}\right|=7$ \folge $\cO_K/\Fp_3=\cO_K/\Fp_4=\IF_7$

                  
 Wir zeigen $\Fp_1\Fp_2=(3)=3\cdot \cO_K$: Jedenfalls ist $\Fp_1\Fp_2\subset(3)$, denn es gilt:
 \begin{align*}
  (3x+(4+\omega)y)(3z+(4-\omega)w)&=9xz+3(4-\omega)xw+3(4+\omega)yz+21yw
 \end{align*}
Dies ist Vielfaches von $3$, also auch jede Summe solcher Elemente. Nun schauen wir uns spezielle Elemente an:
\begin{align*}
 r:=&(3-(4+\omega))(3-(4-\omega))=(-1-\omega)(-1+\omega)\\
 =&(1-\omega^2)=6\\
 s:=&(6-(4+\omega))(6-(4-\omega))=(2+\omega)(2-\omega)\\
  =&4+5=9
\end{align*}

$r,s\in\Fp_1\cdot\Fp_2$, $s-r=3$ \folge $3\in\Fp_1\Fp_2$ \folge $(3)\subset \Fp_1\Fp_2$.

Analog erhält man $\Fp_3\Fp_4=(7)$, $\Fp_1\Fp_3=(4+\omega)$ und $\Fp_2\Fp_4=(4-\omega)$. Wir zeigen noch $\Fp_1\Fp_3=(4+\omega)$:

\begin{align*}
 (3x+(4+\omega)y)(7z+(4+\omega)w)&= 21xz+7(4+\omega)yz+3(4+\omega)xw+(4+\omega)^2yw\\
\end{align*}
Wegen $21=(4+\omega)(4-\omega)$ folgt $\Fp_1\Fp_3\subset (4+\omega)$. Spezielle Elemente:
\begin{align*}
 r:=&(3-(4+\omega))(7-(4+\omega))=(-1-\omega)(3-\omega)\\
 =&-3+\omega-3\omega-5= -8-2\omega = -2(4+\omega)\\
 s:=&(6-(4+\omega))(7-2(4+\omega))=(2-\omega)(-1-2\omega)\\
 =& -2-4\omega + \omega -10 = -12-3\omega = -3(4+\omega)
\end{align*}
\folge $r-s=4+\omega\in \Fp_1\Fp_3$ \folge $(4+\omega)\subset \Fp_1\Fp_3.$
\end{Beispiel}

\begin{Fakt}
Sei $A$ DED-Ring. Dann gilt: \[ A\text{ faktoriell } \gdw A\text{ Hauptidealring }\] 
\end{Fakt}

\begin{Beweis}
 Ist $A$ HIR, so folgt $A$ faktoriell auch ohne DEDEKIND-Eigenschaft (Algebra 1).
 Sei nun $A$ faktorieller DEDEKIND-Ring und $\Fp\subset A$ Primideal. Sei $x\in\Fp$, $x\neq 0$, dann gilt
 \[ x=\pi_1\ldots\pi_n\,\,\mbox{ mit irred. El. }\pi_i.\]
 Dann existiert ein $\pi_j$ mit $\pi_j\in\Fp$, also $(\pi_j)\subset\Fp$.
 $(\pi)$ ist Primideal für irreduzibles $\pi$: $x,y\in A$, $xy\in(\pi)$ \folge $xy=\pi z$\folge $\pi\mid xy$ \folge $\pi\mid x$ oder $\pi\mid y$ (denn $A$ faktoriell).
 Also $(\pi_j)$ prim, $\neq (0)$ und  somit $\Fp=(\pi_j)$.
\end{Beweis}

\subsection{Die Idealklassengruppe}

Sei $K$ algebraischer Zahlkörper. Wir haben einen kanonischen Homomorphismus
\[ K\kreuz \rightarrow \Id_{K}=\Id_{\cO_K},\,\, x\mapsto x\cO_K.\]

\begin{Fakt}
 Der Kern dieses HM ist die Einheitengruppe $E_K$ (oder $U_K$) $=\cO_K\kreuz$ von $\cO_K$.
\end{Fakt}

\begin{Beweis}
 Ist $\eps\in \cO_K$ Einheit, so ist $(\eps)=(1)=\cO_K$. Sei $\alpha\in K\kreuz$ und $(\alpha)=(1)$. D.h. es ex. $\beta\in \cO_K$ mit $\alpha\beta=1$. Nun ist auch $(\alpha\inv)=(1)$, also ex. $\gamma\in \cO_K$ mit $\alpha\inv\gamma=1$\folge $\alpha\in \cO_K$, aus $\alpha\beta=1$ folgt $\alpha\in \cO_K\kreuz$. 
\end{Beweis}

\begin{Beispiel}
 Sei $K=\IQ(\sqrt 2)\supset \cO_K=\IZ[\sqrt 2]$ und $\eps=1+\sqrt 2$. Wegen $(1+\sqrt{2})(1-\sqrt{-2})=-1$ folgt $\eps$ ist Einheit. $\eps$ hat unendliche Ordnung, wegen Betrag $>1$. \folge unendliche Einheitengruppe.
\end{Beispiel}


\begin{Definition}
 Der Cokern\footnote{Für einen Homomorphismus $f:A\rightarrow B$ ist $\Coker f=B/\Img f$. Damit erhält man die exakte Sequenz $0\rightarrow\Ker f\rightarrow A\overset{f}{\rightarrow} B\rightarrow\Coker f\rightarrow 0$.} dieses HM heißt \highl{Idealklassengruppe} von $K$. Bezeichnung: $\Cl_K$. Also \[\Cl_K=\frac{\text{gebr. Ideale}}{\text{gebr. Hauptideale}} \]
\end{Definition}

\begin{Bemerkung}
 $\Cl_K$ misst also die Abweichung von $\cO_K$ davon HIR zu sein. $\Cl_K$ ist abelsche Gruppe. $\Cl_K$ is die wichtigste und mysteriöseste Invariante eines algebraischen Zahlenkörpers.
\end{Bemerkung}


\begin{Beispiel}
 \begin{enumerate}
  \item $\IZ,\IZ[i]$ sind HIR, also ist für $\IQ$, $\IQ(i)$ jeweils $\Cl_K=1$.
  \item $\IZ[\sqrt{-5}]$ ist nicht faktoriell, also kein HIR. Also $\Cl_K\neq 1$.
  
  Wir zeigen $\Fp_1=(3,4+\sqrt{-5})$ ist kein Hauptideal. Indirekt: $\Fp_1=(a+b\sqrt{-5})$ \folge Jedes Element aus $\Fp_1$ hat durch $a^2+5b^2$ teilbare Norm. Also teilt $a^2+5b^2$ die Norm $\N(3)=9$ und $\N(4+\sqrt{-5})=21$. Also teilt $a^2+5b^2$ die Zahl $3=\ggT(21,9)$. \folge $a=\pm 1$, $b=0$ \folge $\Fp_1=\cO_K$ $\lightning$.
 \end{enumerate}
\end{Beispiel}
 
 \begin{Definition}
  Sei $K$ algebraischer Zahlkörper und $\Fa\subset \cO_K$ Ideal, $\Fa\neq(0)$. Wir definieren die \highl{Absolutnorm} $\IN\Fa$ durch \[\IN(\Fa):=(\cO_K:\Fa)=\card \cO_K/\Fa.\]
 \end{Definition}
 
 \begin{Fakt}[Chinesischer Restsatz]
  Sei $A$ kommutativer Ring mit $1$, $\Fa_1,\ldots,\Fa_n$ paarweise verschiedene Primideale, $\Fa_i+\Fa_j=A$ für alle $i\neq j$.
  Dann ist der kanonische HM
  \[A\rightarrow A/\Fa_1 \oplus A/\Fa_2 \oplus \ldots \oplus A/\Fa_n \]
  surjektiv, der Kern ist $\Fa_1\cap\ldots\cap \Fa_n=\Fa_1\ldots\Fa_n$.
 \end{Fakt}

\begin{Bemerkung}
 Das ist ein Resultat der kommutativen Algebra, keine Zahlentheorie.
\end{Bemerkung}

\begin{Beweis}
 Für zwei Ideale $\Fa,\Fb$, gilt sicher $\Fa\Fb\subset\Fa,\Fb$, also $\Fa\Fb\subset\Fa\cap\Fb$.
 Sei $\Fa+\Fb=A$, dann folgt $\exists x\in \Fa, y\in\Fb$: $x+y=1$. Ist $z\in \Fa\cap\Fb$, so ist $z=z(x+y)=zx+zy\in\Fa\Fb$.
 Daher $\Fa\Fb=\Fa\cap\Fb$. Das nehmen wir als Induktionsanfang und verallgemeinern: $\Fa_i+\Fa_n=A$ \folge $\exists x_i\in\Fa_i, y_i\in\Fa_n$ mit $x_i+y_i=1$, für $i=1,\ldots,n-1$. Sei $\Fb=\Fa_1\cap\ldots\cap\Fa_{n-1}=\Fa_1\ldots\Fa_{n-1}$.
 \[\prod_{i=1}^{n-1} x_i = \prod_{i=1}^{n-1}(1-y_i)\equiv 1\mod{\Fa_n} \]
 Also ist $\Fb+\Fa_n=A$: $\prod_{i=1}^{n-1} x_i\in\Fb$, $\prod_{i=1}^{n-1}  x_i=1+y$ mit $y\in\Fa_n \folge 1=\prod_i x_i-y$. Nach Induktionsanfang folgt $\Fb\cap \Fa_n=\Fa_1\cap\ldots\cap\Fa_n=\Fb\cdot\Fa_n=\Fa_1\cdot\ldots\cdot\Fa_n$.
 
 Zur Surjektivität: Fixiere $i$: $\exists x_j\in\Fa_i$, $y_j\in\Fa_j$, $j\neq i$ mit $x_j+y_j=1$. Sei $x=\prod_{j\neq i} (1-x_j)\in A$. $x$ liegt in jedem $\Fa_j$ für $j\neq i$ und ist $\equiv 1 \mod{\Fa_i}$.
 Also hat $x$ das Bild $(0,\ldots,0,\underset{\substack{\uparrow\\i\text{-te}}}{1},0,\ldots,0)$.
\end{Beweis}

\begin{Bemerkung}
 Für DED-Ringe haben wir scheinbar zwei Definitionen für relativ prim (d.h. teilerfremd):
 \begin{itemize}
  \item $\Fa+\Fb=A$
  \item $\Fa,\Fb$ haben keine gemeinsamen Primidealteiler.
 \end{itemize}
 Wir zeigen deren Äquivalenz.
\end{Bemerkung}

\begin{Beweis}
\begin{enumerate}
\item  $\Fp\mid\Fa$, $\Fp\mid\Fb$ \folge $\Fa\subset\Fp$, $\Fb\subset \Fp$ \folge $\Fa+\Fb\subset\Fp$ \folge $\Fa+\Fb\subsetneq A$.
 
 Also $\Fa+\Fb=A$ \folge $\Fa,\Fb$ haben keine gemeinsamen Primteiler.
 \item Ist $\Fa+\Fb\subsetneq A$, so existiert maximales Ideal $\Fp$ mit $\Fa+\Fb\subset\Fp$. Es folgt $\Fa\subset\Fp$, $\Fb\subset\Fp$ \folge $\Fp\mid\Fa,\Fb$. 
 
 Also: Haben $\Fa$ und $\Fb$ keine gemeinsamen Primidealteiler, so folgt $\Fa+\Fb=A$.
 
 Wir haben benutzt: $\Fp\mid \Fa$ \gdw $\Fa\subset \Fp$, was wir jetzt zeigen: $\Fp \mid \Fa$ \folge $\exists \Fb\subset A$ mit $\Fa=\Fp\cdot \Fb$ \folge $\Fa\subset \Fp$. Ist $\Fa\subset\Fp$, so ist $\Fp\inv\Fa\subset A$, d.h. $\Fp\inv\Fa=\Fb$ Ideal in $A$.  Multiplikation mit $\Fp$ gibt $\Fa=\Fp\Fb$.
\end{enumerate}
\end{Beweis}