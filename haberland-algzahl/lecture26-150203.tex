\renewcommand{\lecdate}{03.02.2015}

\begin{Fakt}
Sei $L/K$ endliche Erweiterung von ZK, $\FP|\Fp$. Dann gilt:
\begin{enumerate}
 \item $e(L_\FP/K_\Fp)=e(\FP/\Fp)$
 \item $f(L_\FP/K_\Fp)=f(\FP/\Fp)$
 \item $\cO_K/\Fp=\cO_{K_\Fp}/\Fm_\Fp$
\end{enumerate}
\end{Fakt}

\begin{Beweis}
 $\cO=S\inv\cO_K$ ist dBR, sein maximales Ideal ist $\Fm=(\pi)$, $S\inv\Fp=\Fm$, QK ist $K$.
 $p=u\pi^e$, $u\in\cO\kreuz$, $e=e(\Fp/p)$. Wir zeigen: $S\inv\cO_K/S\inv\Fp (=\cO/\Fm)=\cO_K/\Fp$:
 $\cO_K\rightarrow S\inv\cO_K \rightarrow S\inv\cO_K/S\inv\Fp$ hat den Kern $\cO_K\cap S\inv\Fp=\Fp$ (siehe KapI). Wir zeigen die Surjektivität:
 
 $x\in S\inv\cO_K$ \folge $x=y/s$, $y\in\cO_K$, $s\in\cO_K\setminus \Fp$. $\exists t\in \cO_K\setminus\Fp$, s.d. $st\equiv 1\mod{\Fp}$ \folge $y/s\equiv yt\mod{\Fp}$ \folge $x\equiv yt\mod{\Fp}$. Also ist $x$ Bild von $yt\in\cO_K \mod{S\inv\Fp}$. Es folgt $f(\Fp/p)=[\cO/\Fm:\IF_p].$ Nach Turmformel folgt analoge Formel für $f(\FP/\Fp)$.
 
 Beim Komplettieren ändert sich im nichtarchimedischen Fall die Wertegruppe der Bewertung nicht: Sie bleibt $\{ p^{m/e} : m\in\IZ\}\cup\{ 0\}$ wie auf $K$. Somit folgt aus $p=u\cdot \pi^{K_\Fp/\IQ_p}$, dass $e(K_\Fp/\IQ_p)=e(\Fp/p)$. $p\cO_K=\Fp^{e(\Fp/p)}\cdot\Fa$, $\Fp\nmid\Fa$.
 
 Es bleibt $\cO/\Fm=\cO_\Fp/\Fm_\Fp$. Wir haben \[ \cO\rightarrow \cO_\Fp \rightarrow \cO_\Fp/\Fm_\Fp\]
 hat Kern $\cO\cap\Fm_\Fp=\Fm$.
 
 $\cO_\Fp \ni x = a_0+a_1\pi+a_2\pi^2+\ldots$ ($a_j$ Repräsentantensystem von $\cO_K$ modulo $\Fp$)
 
 $m_\Fp=(\pi)$. Also $x\equiv a_0\mod{\Fm_\Fp}$.
\end{Beweis}

\begin{Bemerkung}
 Sei $K$ algebraischer ZK, $M_K$ Menge der Äquivalenzklassen nichttrivialer Bewertungen, $M_K=M_K^\infty\cup M_K^0$.
 \begin{itemize}
  \item $M_K^\infty = $ archimedische Bewertungen
  \item $M_K^0=M_K^f=$ nichtarchimedische Bewertungen
  \item $M_K^0 = Spec \cO_K$ (fast)
 \end{itemize}
 Wir hatten sogenannte geometrische Abbildung \[ j: K\rightarrow \prod_{v\in M_K^\infty} K_v\]
 mit $K_v = $ Komplettierung von $K$ an der Stelle $v$. 
 
 CHEVALLEY ~ 1930
\begin{enumerate}
 \item Problem: $\prod_{v\in\M_K} K_v$ ist zu groß - nicht lokalkompakt.
 \item $\sum_{v\in\M_K} K_v$ zu klein - $K$ passt nicht diagonal hinein.
 \item Sei $S\subset M_K$ endlich, $M_K^\infty\subset S$. Die $S$-Adele sind
 \[ \IA_S=\prod_{v\in S} K_v \times \prod_{v\in M_K\setminus S} \cO_{K_v}\]
 \highl{Adele}
 
\end{enumerate}
Nun gilt $\IA_K$ ist LCA (locally compact abelian) und $K$ bettet sich diagonal ein.

Adele = adjoint elements

Die Einheiten $\IA_K\kreuz$ heißen \highl{Idele} = ideal elements ($J_K:=\IA_K\kreuz$).
\end{Bemerkung}


\subsection{Unverzweigte Erweiterungen}

\begin{Definition}
 Eine endliche Erweiterung von lokalen ZK $L/K$ heißt \highl{unverzweigt} genau dann, wenn $e(L/K)=1$.
\end{Definition}

\begin{Bemerkung}
 \begin{enumerate}
  \item Sind $l,k$ die Restklassenkörper von $L,K$, so gilt im unverzweigten Fall $[L:K]=[l:k]$.
  \item Ist $L/K$ endliche Erweiterung von Zahlkörpern, so sind nur endlich viele der lokalen Erweiterungen $L_\FP/K_\Fp$ verzweigt.
 \end{enumerate}
\end{Bemerkung}

\begin{Fakt}
 $L/K/\IQ_p$, $L/K$ unverzweigt \gdw $L=K(\zeta_n)$ mit $\zeta_n$ primitive $n$-te Einheitswurzel, $\ggT(p,n)=1$.
\end{Fakt}

\begin{Beweis}
 \begin{enumerate}
  \item Sei $L/K$ unverzweigt, dann ist $l=k(\theta)$, $\theta\in\overline{\IF_p}$ primitive $n$-te Einheitswurzel mit $p\nmid n$.
  Sei $g=\Irr(X,\theta,k)\in k[X]$, $g\mid X^n-1$.  Nach HENSELs Lemma existiert eine Wurzel $\zeta\in\cO_L$ von $X^n-1$ mit $\zeta\equiv \theta \mod{\Fm_L}$. Dann ist $\zeta$ primitive $n$-te Einheitswurzel in $L$
  Sei $f=\Irr(X,\zeta,K)\in\cO_K[X]$ (VIETAischer Wurzelsatz). Sei $\overline{f}$ die Reduktion von $f$ $\mod{\Fm_K}$, dann ist $\overline f(\theta)=0$ (wg. $\zeta\mapsto \theta$). Also teilt $g$ das Polynom $\overline f$ in $k[X]$. Nun ist $[K(\zeta):K]=\deg f$, $[k(\theta):k]=[l:k]=[L:K]$ (wegen Unverzweigtheit) $=\deg g$. Also $\deg f \leq \deg g$. Aber $g\mid \overline{f}$. Damit $\overline f = g$ und $L=K(\zeta)$.
  \item Sei $L=K(\zeta_n)$ mit $zeta_n = $ primitive $n$-te Einheitswurzel, $p\nmid n$. Wir haben zu zeigen: $L/K$ ist unverzweigt. 
  Dann ist $l=\cO_L/\Fm_L$, $k=\cO_K/\Fm_K$. $l\supset k(\overline\zeta_n)$, $\overline{\zeta}_n=$ Bild von $\zeta_n$ in $l$ ($\zeta_n$ ist ganz).
  Die Abbildung $\mu_n(L)\rightarrow \mu_n(l): \zeta\mapsto \overline{\zeta}=\zeta\mod{\Fm_L}$ ist nach HENSELs Lemma Isomorphismus (Eindeutigkeit). Sei $f=\Irr(X,\zeta_n,K)$, $\overline g=\Irr(X,\overline\zeta_n,k)$. Dann gilt $\overline{g}\mid\overline{f}$.
  
  Sei $\overline{f}=\overline{g}\cdot \overline h$ in $k[X]$, die Polynome $\overline g, \overline h$ sind teilerfremd, d.h. $X^n-1\in k[X]$ keine mehrfachen Nullstellen hat. Nach HENSELs Lemma-Variante folgt $f=gh$ mit $g\mapsto \overline{g}$, $h\mapsto \overline{h}$, $g,h\in\cO_K[X]$. Aber $f$ ist irreduzibel \folge $h=\overline{h}=1$. Somit $[L:K]=\deg f=\deg \overline{g}=[k(\overline\zeta_n):k]\leq [l:k]\leq [L:K] \folge [l:k]=[L:K]\folge L/K$ unverzweigt.
 \end{enumerate}
\end{Beweis}

