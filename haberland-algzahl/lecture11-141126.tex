\renewcommand{\lecdate}{26.11.2014}

\subsubsection{Reell quadratische ZK}

Sei $d>1$, sqf, $c_K=\frac{2!}{2^2}=\frac{1}{2}$, $K=\IQ(\sqrt{d})$, $r=2$, $s=0$.

$c_K\sqrt{|d_K|}<2 \gdw d_K<4^2=16$ Also $\Cl_K=1$ für $d_K=5,8,12,13$.

\subsubsection{Kreisteilungskörper}

$K=\IQ(\zeta_p)$, $p\in\IP$, $p>2$, $d_K=(-1)^{\frac{p-1}{2}}p^{p-2}$
\[ c_K=\frac{(p-1)!}{(p-1)^{p-1}}\left(\frac{4}{\pi} \right)^{\frac{p-1}{2}}\]

$\Cl_K=1$ für $p=3$, $p=5$ (ÜA)

Bisher verstehen wir die Idealklassengruppen der Kreisteilungskörper sehr schlecht. (S. LANG Cyclotomic fields I,II)

\subsection{Die Einheitengruppe}%1.8

Wir wollen $E_K=U_K=\cO_K\kreuz$ studieren. Das ist abelsche Gruppe. Diese kann unendlich sein: $K=\IQ(\sqrt{2})$, $\cO_K=\IZ+\IZ\sqrt 2$, $\eps:=1+\sqrt 2$. Dann ist $\eps$ eine Einheit. $\eps(1-\sqrt 2)=1$.
Dann ist $\eps^n\uparrow\infty$.

\begin{Bemerkung} $(E_K)_{\tor}$ besteht genau aus den Einheitswurzeln $\mu(K)$ in $K$. Ihre Anzahl ist endlich, denn unter $j:K\rightarrow K_{\IR}$ ist $j(E_K)$ diskret (wegen $j(O_K)=$Gitter) und jede Komponente von $j(\zeta)$ ($\zeta\in\mu(K)$) hat Abssolutbetrag $1$. Eine diskrete beschränkte Menge ist endlich. 
\end{Bemerkung}

\begin{Beispiel}
 \begin{itemize}
  \item Einheiten in imaginärquadratischen ZK
  
  $K=\IQ(\sqrt{-d})$, $d>0$, sqf.
  \begin{enumerate}
   \item $-d\equiv 2,3\mod{4}$ \folge $\cO_K=\IZ+\IZ\sqrt{-d}$
   
   $\eps\in\cO_K$ Einheit \gdw $\N\eps=\pm 1$:
   \[\eps\cdot\eta=1 \folge \N\eps\cdot\N\eta=1, \mbox{ wobei } \N\eps,\N\eta\in\IZ \]
   Ist $\N\eps=1$, so gilt $\eps\cdot\underbrace{\prod_{\sigma\neq\Id}\sigma\eps}_{=\N\eps=\eta}=1$
   
   $\eta\in K$ ist ganz, also aus $\cO_K$. Somit $\eps\in E_K$. Sei $\eps=a+b\sqrt{-d}$, $\N\eps=(a+b\sqrt{-d})(a-b\sqrt{-d})=a^2+db^2\geq 0$. $\N\eps=1=a^2+db^2$ hat Lösungen $(\pm1,0)$ außer $d=1$, dann $4$ Lösungen.
   
   \item $-d\equiv 1\mod{4}$ \folge $\cO_K=\IZ+\IZ\omega$ mit $\omega=\frac{1+\sqrt{-d}}{2}$.
   \begin{align*}
    \N(a+b\omega)&=(a+b\tfrac{1+\sqrt{-d}}{2})(a+b\tfrac{1-\sqrt{-d}}{2})\\
    &=a^2+ab+b^2\tfrac{1+d}{4}
   \end{align*}
   
   \folge $4\N(a+b\omega)=(2a+b)^2+db^2=\pm 4$, $-4$ geht nicht, also $(2a+b)^2+db^2= 4$. Ist $d>4$, so nur triviale Lösungen: $b=0$, $a=\pm1$.
   
   $d=3$: $(2a+b)^2+3b^2=4$. Alle $\IZ$-Lösungen sind $(\pm1,0),(0,\pm1),(1,-1),(-1,1)$, also 6 Lösungen.
   \[ 2a+b=\pm 1 \folge \mu(\IQ(\sqrt{-3}))=\mu_6\]
  \end{enumerate}
  
  
  \begin{Definition}
   \[K_\IR\kreuz=\{ x\in K_\IR : \mbox{ alle Komponenten sind }\neq 0\}\]
   
   \[ l: K_\IR\kreuz\rightarrow \IR^{r+s}: (x_{\sigma_1,\ldots,x_{\sigma_r},y_{\tau_1},\ldots,y_{\tau_s}})\mapsto (\log|x_{\sigma_1}|,\ldots,\log|x_{\sigma_r}|, \log|y_{\tau_1}|^2,\ldots,\log|y_{\tau_s}|^2)\]
   
   Wobei $\log=\ln$. Oder einheitlich: Für $\rho\in\Hom(K,\IC)$ sei 
   \[ \N_\rho:=\begin{cases}
                1& \rho\mbox{ reell}\\
                2& \rho\mbox{ nicht reell}
               \end{cases}
\]

Dann ist $l(x)=(\log|x_\rho|^{\N_\rho})_{\rho\in S}$, $S\subset\Hom(K,\IC)$ aus den Paaren komplex-konjugiertert wird eines ausgewählt
  \end{Definition}
 
  Es gilt $\sum_{\rho\in S}\log|x_\rho|^{\N_\rho}=\log\prod_{\rho\in S} |x_\rho|^{\N_\rho}$. Für $\alpha\in K\kreuz$ gilt $l(j(\alpha))=\log |\N_\IQ^K(\alpha)|$. (Deshalb die $\N_\rho$)
  
  Wir setzen noch $\lambda=l\circ j$, $\lambda: K\kreuz \rightarrow \IR^{r+s}$. Das ist ein Homomorphismus:
  \[ \lambda(\alpha\beta)=l(j(\alpha\beta))=l(j(\alpha)j(\beta))=\lambda(\alpha)+\lambda(\beta)\]
  \end{itemize}
\end{Beispiel}

\begin{Fakt}
 \begin{enumerate}
  \item Der Kern von $\lambda:E_K\rightarrow \IR^{r+s}$ ist $\mu(K)=$ die Gruppe der Einheitswurzeln in $K$.
  \item Das Bild von $E_K$ ist diskret und liegt in der Hyperebende $H=\{x\in\IR^{r+s} : \sum x_\rho = 0\}$
 \end{enumerate}
\end{Fakt}

\begin{Beweis}
 \begin{enumerate}
  \item Für $\zeta\in\mu(K)$ ist $\zeta^N=1$ \folge $|\sigma\zeta|=1$. $\forall\sigma\in S$, also $\log|\sigma\zeta|=0$ \folge $\zeta\in\Ker\lambda$. Sei $\lambda(\alpha)=0$, also $\log|\sigma\alpha|=0 \forall \sigma \in S$ \folge $\log|\sigma\alpha|=0 \forall \sigma\in\Hom(K,\IC)$. Somit $|\sigma\alpha|=1 \forall \sigma\in\Hom$. Somit liegt $j(\alpha)$ in beschränktem Gebiet in $K_\IR$. Dasselbe gilt für alle Potenzen $\alpha^N$ von $\alpha$. $j(\cO_K)$ ist diskret, also ist die Menge $\{j(\alpha^N): N\in\IN\oN \}$ endlich.
  $\exists m>n>0$ sd. $j(\alpha^m)=j(\alpha^n)=j(\alpha^{m-n})=1$. \folge $\alpha^{m-n}$ ist Eigenwert \folge $\alpha$ auch.
  \item Für $\eps\in E_K$ gilt $N_\IQ^K(\eps)=\pm 1=\prod_{\sigma\in\Hom(K,\IC)} \sigma(\eps)$. Also 
  \[ \sum_{\sigma\in\Hom(K,\IC)} \log |\sigma\eps| = \sum_{\rho\in S} \log|\rho\eps|^{\N_\rho}=0.\]
  Somit liegt $\lambda(\eps)$ in $H$.
  
  Diskretheit: Sei $\Omega_c=\{(x_\rho)\in\IR^{r+s} : |x_\rho|\leq c \}$. Dann ist $l\inv\Omega_c=\{(y_\rho)\in K_\IR : e^{-c}<|y_\rho|^{\N_\rho}<e^c \}$. $l\inv\Omega_c\cap j(\cO_K)$ ist endlich, da $l\inv\Omega_c$ beschränkt und $j(\cO_K)$ diskret ist. Also erst recht $\Omega_c\cap\lambda(E_K)$ endlich.
 \end{enumerate}
\end{Beweis}

\begin{Folgerung}
 $E_K$ ist endlich erzeugt und $\rk E_K\leq r+s-1$.
\end{Folgerung}

\begin{Beweis}
 $\lambda(E_K)$ ist diskrete Untergruppe in $H$, also Gitter in der linearen Hülle von $\lambda(E_K)$.
\end{Beweis}

\begin{Fakt}
 Sei $K$ reellquadratisch, dann hat $E_K$ den Rang $1$. Es existiert also Einheit $\eps_K$, s.d. $E_K=\mu_2\times \langle \eps_K\rangle$.
 $\eps_K$ ist eindeutig bestimmt durch $\eps_K>1$. Sie heißt \highl{Fundamentaleinheit}.
\end{Fakt}
 
 \begin{Lemma}[DIRICHLET]
  Sei $x\in\IR$, $N>1$ natürliche Zahl. Dann existieren $a,q\in\IZ$, $0<q\leq N$ mit $|x-\frac{a}{q}|<\frac{1}{Nq}$
 \end{Lemma}

 \begin{Beweis}
 Unterteile $[0,1)$ in $N$ gleichlange Intervalle. Unter den Zahlen $qx$, $0\leq q\leq N$ gibt es zwei verschiedene, welche $\mod{1}$ im selben Intervall liegen. 
 Also \[ |q_1x-q_2x-a|\leq \frac{1}{N}\hspace{1cm} a\in\IZ, 0\leq q_1,q_2\leq N, q_1\neq q_2.\]
 Setze $q=|q_1-q_2|$, dann gilt $0<q\leq N$ und
 \[|qx\pm a|<\frac{1}{N}  \folge |x\pm\frac{a}{q}|<\frac{1}{qN}\]
\end{Beweis}

\begin{Beweis}[des Fakts]
$K:=\IQ(\sqrt d)$, $d>1$ sqf, $K_\IR=\IR\oplus\IR$. $j(a+b\sqrt{d})=(a_b\sqrt{d},a-b\sqrt{d})$. Sei $N>1$, es existieren $q,a\in\IZ$, s.d.
\[|a-q\sqrt{d}|<\frac{1}{N}\hspace{1cm} 0<q\leq N.\]
Das ist Streifen in $K_\IR$

\begin{center}
 \definecolor{zzttqq}{rgb}{0.27,0.27,0.27}
\definecolor{qqqqff}{rgb}{0.33,0.33,0.33}
\begin{tikzpicture}[line cap=round,line join=round,>=triangle 45,x=0.5cm,y=0.5cm]
\draw[->,color=black] (-3,0) -- (3,0);
\draw[->,color=black] (0,-3.5) -- (0,3.5);
\clip(-3,-3.5) rectangle (3,3.5);
\fill[pattern color=zzttqq,fill=zzttqq,pattern=north east lines] (-0.5,-4) -- (0.5,-4) -- (0.5,4) -- (-0.5,4) -- cycle;
\draw[smooth,samples=100,domain=-3.0:3.0] plot(\x,{1/(\x)});
\draw [color=zzttqq] (-0.5,-4)-- (0.5,-4);
\draw [color=zzttqq] (0.5,-4)-- (0.5,4);
\draw [color=zzttqq] (0.5,4)-- (-0.5,4);
\draw [color=zzttqq] (-0.5,4)-- (-0.5,-4);
\begin{scriptsize}
\fill [color=qqqqff] (-0.5,-4) circle (1.5pt);
\fill [color=qqqqff] (0.5,-4) circle (1.5pt);
\fill [color=qqqqff] (0.5,4) circle (1.5pt);
\fill [color=qqqqff] (-0.5,4) circle (1.5pt);
\draw[color=qqqqff] (2,1.5) node {$xy=1$};
\end{scriptsize}
\end{tikzpicture}
\end{center}

Es folgt $|a+q\sqrt d|\leq |a-q\sqrt{d}|+2q\sqrt d< \frac{1}{N} + 2N\sqrt{d}$. Also ist $|\N(a+q\sqrt d)|<1+2\sqrt{d}$. Das ist also eine Schranke, die unabhängig von $a,q,N$ ist. In solch einem Streifen zu $N$ gibt es also stets Punkte aus $j(\cO_K)$, deren Norm unabhängig von $N$ beschränkten Betrag hat. Dann sind auch die Absolutnormen $\IN((a+q\sqrt{d}))$ beschränkt.

Es gibt nur endliche viele ganze Ideale mit beschränkter Absolutnorm. Also existieren Ideale $(a+q\sqrt d), (b+r\sqrt d)$ welche gleich sind, aber $(a,q)\neq (b,r)$.
Dann gilt $a+q\sqrt{d}=\eps(b+r\sqrt{d})$ mit Einheit $\eps$. Diese kann nicht $-1$ sein, wegen $q,r$ positiv. D.h. $\eps\neq 1, \neq -1$. Nun Fallunterscheidung:
\begin{align*}
 1<\eps &\folge  e_K=\eps\\
 0<\eps<1 &\folge  e_K=\frac{1}{\eps}\\
 -1<\eps<1 &\folge  e_K=-\frac{1}{\eps}\\
 \eps<-1 &\folge  e_K=-\eps
\end{align*}
\end{Beweis}

\begin{Bemerkung}
 Es folgt $E_K=\mu_2 \times$ unendliche zyklische Gruppe.
\end{Bemerkung}

\begin{Satz}[\glqq Satz 6\grqq{} DIRICHLETscher Einheitensatz]
 $\lambda(E_K)$ ist ein Gitter in $H$, d.h. $\rk E_K=r+s-1$. Also $\cO_K\kreuz \cong \mu(K)\times\IZ^{r+s-1}$
\end{Satz}

\begin{Definition}
 Jede Menge von Einheiten $\eps_1,\ldots,\eps_{r+s-1}$, welche $E_K$ modulo Torsion erzeugt, heißt \highl{System von Fundamentaleinheiten}.
\end{Definition}





