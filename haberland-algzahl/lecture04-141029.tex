%!TEX root = Algebraische_Zahlentheorie.tex

\renewcommand{\lecdate}{29.10.14}

\subsection{Die Ringe ganzer algebraischer Zahlen I - additive Struktur}

$L/K$ endlich, separabel (z.B. mit Charakteristik $0$). $L$ ist als $K$-VR endlich dimensional, $x\in L$ verursacht durch Multiplikation eine $K$-lineare Abbildung
\[A_x : L\rightarrow L : \, y\mapsto xy\]
$\Tr_K^L(x):=\Tr A_x$, $\N_K^L(x)=\det(A_x)$.

Ersatz für Galois-Gruppe ist $\Hom_K(L,\overline{K})$. Wegen Separabilität gilt $\#\Hom_K(L,\overline K)=[L:K]=n$. Sei \[P_x:=\det(T\cdot \Id - A_x), \]
dann gilt 
\begin{align*}
\Tr_K^L(x)&=-\text{ Koeffizient von } P_x \text{ bei } T^{n-1},\\
\N_K^L(x)&=(-1)^n \text{ Koeffizient von } P_x \text{ bei } T^0.
\end{align*}

\begin{Fakt} Sei $L/K$ endlich und separabel, dann gilt
 \begin{itemize}
  \item $P_x(T)=\prod_{\sigma\in\Hom_K(L,\overline{K})}(T-\sigma x)$
  \item $\Tr_K^L(x)=\sum_\sigma \sigma x$
  \item $\N_K^L(x)=\prod_\sigma \sigma x$
 \end{itemize}
\end{Fakt}

\begin{Beweis}
 Sei zuerst $L=K(x)$, dann ist \[f(T)=\prod_\sigma (T-\sigma x)\]
 das irreduzible Polynom von $x$ über $K$. Andererseits ist $P_x(x)=\det(x\Id-A_x)=0$.
 Beide Polynome sind unitär, also $f=P_x$. Das war (i). (ii) und (iii) folgen trivial.
 
 Der allgemeine Fall: $L/E/K$, $E=K(x)$. Wir zeigen $P_x(T)=\Irr(T,x,K)^{[L:E]}$.
 $L=\underbrace{E\oplus\ldots \oplus E}_{n\text{-mal}}$ als $K$-VR durch Wahl einer $E$-Basis von $L$. $A_x$ lässt diese $E$ invariant. Jede $K$-Einlagerung $\sigma:E\rightarrow \overline{K}$ hat genau $n$ Fortsetzungen zu $K$-Einlagerungen von $L$. Also
 \[\prod_{\tau\in \Hom_K(L,\overline K)}(T-\tau x)=\prod_{\sigma\in \Hom_K(E,\overline K)} (T-\sigma x)^n =\Irr(T,x,K)^n=P_x(T)\]
\end{Beweis}

\begin{Fakt}[Turmsätze für Spur und Norm]
 Seien $M/L/K$ endlich und separabel, dann gilt:
 \begin{enumerate}
  \item $\Tr_K^M=\Tr_K^L \circ \Tr_L^M$
  \item $\N_K^M=N_K^L \circ N_L^M$
 \end{enumerate}
\end{Fakt}

\begin{Fakt}
 Sei $L/K$ endlich und separabel, dann ist die \highl{Spurform}
 \[ \Tr_K^L: L\times L\rightarrow K: \, (x,y)\mapsto \Tr_K^L(xy) \]
 eine nichtausgeartete, symmetrische $K$-Bilinearform.
\end{Fakt}

\begin{Satz}[Satz vom primitiven Element]
 Sei $L/K$ endlich und separabel. Dann gibt es ein $x\in L$ mit $L=K(x)$. In diesem Fall heißt die Erweiterung \highl{einfach} und $x$ \highl{primitives Element}.
\end{Satz}

\begin{Beweis}[vom vorherigen Fakt]
 $L=K(\theta)$. Dann ist $1,\theta,\theta^2,\ldots,\theta^{n-1}$ ($n=[L:K]$) $K$-Basis von $L$. Der Spurform entspricht die Matrix 
 \[ M=\left(\Tr_K^L(\theta^{i+j})\right)_{0\leq i,j\leq n-1}.\]
 Seien $\sigma_1,\ldots, \sigma_n:L\rightarrow \overline{K}$ die $K$-Einbettungen von $L$ (Anzahl = $n$ wegen separabel). Sei $\theta_i:=\sigma_i\theta$ und 
 \[ N= \begin{pmatrix}
        1 & 1 &\hdots & 1\\
        \theta_1 & \theta_2 &\hdots & \theta_n\\
        \vdots & \vdots & & \vdots\\
        \theta_1^{n-1} & \theta_2^{n-1} & \hdots& \theta_n^{n-1}
       \end{pmatrix}.
\]
Dann gilt $M=N\cdot N^T$, also $\det M=(\det N)^2$. $\det N\neq 0$, wegen $\theta_i$ alle verschieden (VANDERMONDE).
\end{Beweis}

\begin{Folgerung}
 Sei $L/K$ endlich und separabel, $w_1,\ldots, w_n$ $K$-Basis von $L$. Die \highl{Diskriminante} von $w_1,\ldots,w_n$ ist definiert als
 \[ d(w_1,\ldots,w_n)=\det(\sigma_i\omega_j)^2=\det(\Tr(w_iw_j)). \]

 Sie ist stets $\neq 0$
 \end{Folgerung}
 
 \begin{Satz}[\glqq Satz 1\grqq: Additive Struktur der Ringe ganzer algebraischer Zahlen]
  Sei $K$ algebraischer ZK vom Grad $n$. Dann ist $\cO_K$ freie abelsche Gruppe vom Rang $n$. D.h. es existieren $w_1,\ldots,w_n\in \cO_K$, s.d. jede Zahl aus $\cO_K$ sich eindeutig als $\IZ$-Linearkombinationen der $w_i$ schreiben lässt.
 \end{Satz}
 
 \begin{Definition}
  Solche $w_1,\ldots,w_n$ heißen \highl{Ganzheitsbasen}.
 \end{Definition}
 
 \begin{Beweis}
  Sei $A=\cO_K$, $\alpha_1,\ldots,\alpha_n$ eine $\IQ$-Basis von $K$. O.B.d.A. $\alpha_i\in A$ (Mult. mit pos. nat. Zahl). Sei $\alpha_1',\ldots,\alpha_n'$ duale Basis bzgl. Spurform (d.h. $\Tr_\IQ^K(\alpha_i\alpha_j')=\delta_{ij}$). Wir wählen ein $c\in \IN\oN$, s.d. $c\alpha_1',\ldots,c\alpha_n'$ ganz sind (d.h. aus $A$). Sei $x\in A$, dann sind alle $cx\alpha_i'\in A$.
Es gilt $x=m_1\alpha_1 + \ldots + m_n \alpha_n$, $m_j\in\IQ$. Dann ist 
\begin{align*}
 \Tr(cx\alpha_i') & = \sum_j \Tr(cm_j \alpha_j \alpha_i')\\
 &= \sum_j cm_j \Tr(\alpha_j,\alpha_i')=cm_i.
\end{align*}
Dabei ist $\Tr(cx\alpha_i')\in\IQ$ und ganz $/\IZ$. Also $cm_i\in\IZ$. D.h. $x$ liegt in der endlich erzeugten abelschen Gruppe
\[ B=\IZ c\inv \alpha_1 + \ldots + \IZ c\inv \alpha_n.\]
Additive Untergruppen in $K$ sind torsionsfrei, also ist obige Gruppe $B$ frei. $A$ liegt in freier abelscher Gruppe $B$ vom Rang $n=[K:\IQ]$.
Also ist auch $A$ frei von endlichem Rang $\leq n$. $A$ enthält freie abelsche Gruppe vom Rang $n$, nämlich $\IZ\alpha_1+\ldots + \IZ\alpha_n$. Also hat auch $A$ den Rang $n$.
 \end{Beweis}

\begin{Beispiel}
 Sei $K=\IQ(\zeta_p)$ mit $\zeta_p=e^{2\pi i/p}, p\in \IP, p>2$ (der sogenannte $p$-te \highl{Kreisteilungskörper}).
 
 $\Irr(T,\zeta_p,\IQ)=\frac{T^p-1}{T-1}=T^{p-1}+T^{p-2}+\ldots+T+1$, denn mit Substitution $T=S+1$ folgt $\frac{(S+1)^p-1}{S}=S^{p-1}+\binom{p}{1}s^{p-2}+\ldots+\binom{p}{p-1}$ ist EISENSTEIN-Polynom, also irreduzibel $/\IQ$. Somit $[\IQ(\zeta_p):\IQ]=p-1$. $K/\IQ$ ist normal, die GALOIS-Gruppe ist kanonisch so 
 zu $(\IZ/p\IZ)\kreuz$ isomorph, via $a\mapsto \sigma_a$, $\sigma_a\zeta_p=\zeta_p^a$. Wir zeigen nun, dass $1,\zeta_p,\ldots,\zeta_p^{p-2}$ eine Ganzheitsbasis ist. Äquivalent ist $\cO_K=\IZ[\zeta_p]$.
 Jedenfalls ist das eine $\IQ$-Basis von $K$, die $\zeta_p^a$ sind alle ganz, also $\IZ[\zeta_p]\subset \cO_K$. Der Index ist endlich, da beide abelsche Gruppen den Rang $p-1$ haben\footnote{$A\subset \IZ^n$, $A$ frei von $\rk A=n$ \folge $(\IZ^n:A)<\infty$: Sei $a_1,\ldots,a_n$ $\IZ$-Basis von $A$. Dann $|\IZ/A|=|\frac{\IZ\oplus\ldots\oplus\IZ}{a_1\IZ\oplus\ldots\oplus a_n\IZ}|=|\IZ/a_1\IZ\oplus\ldots\oplus\IZ/a_n\IZ|=|\IZ/a_1\IZ|+\ldots+|\IZ/a_n\IZ|<\infty$.}  .
 
 Sei $\pi=1-\zeta_p$, wir zeigen: $\pi$ ist irreduzibel in $\cO_K$:
 \[ 1+T+\ldots + T^{p-1}=\prod_{a=1}^{p-1} (T-\zeta_p^a)\]
 
 $T=1:$ $p=\prod (1-\zeta_p^a)$, $\N_\IQ^K(\pi)=\prod(1-\zeta_p^a)=p$, $\N_\IQ^K(1-\zeta_p^a)=p$ für alle $a\in (\IZ/p\IZ)\kreuz$. Wäre $1-\zeta_p^r=\alpha\cdot \beta$ \folge $\N(\alpha) \N(\beta)=p$
\end{Beispiel}

\begin{Lemma}[\glqq Lemma 1\grqq{}]
 $\eps\in \cO_K$ ist Einheit\gdw $\N_\IQ^K(\eps)=\pm 1$
\end{Lemma}

\begin{Beweis}
$\eps\cdot \eta = 1$ \folge $\underbrace{\N(\eps)}_{\in\IZ} \underbrace{\N(\eta)}_{\in\IZ}=1$ \folge $\N(\eta)=\pm 1$.
 Ist $\N(\eps)=\pm 1= \prod_\sigma \sigma\eps = \eps \cdot \prod_{\sigma\neq \Id} \sigma\eps.$ Der zweite Faktor beseteht aus ganzen Zahlen, ist also aus $\cO_K$, somit ist $\eps$ eine Einheit.
\end{Beweis}

 $p=\N(\alpha)\N(\beta)$ impliziert $\N(\alpha)=\pm 1$ oder $\N(\beta)=\pm 1$. Also $\alpha$ oder $\beta$ Einheit. Also sind alle $\pi_r$ irreduzibel.
 \[ \pi_r=1-\zeta_p^r=\underbrace{(1-\zeta_p)}_\pi(1+\zeta_p + \ldots + \zeta_p^{r-1}).\]
 Also ist $\pi_r=\pi\cdot \eps_r$, $\eps_r=1+\zeta_p + \ldots + \zeta_p^{r-1}$ ist Einheit. Die $\eps_r$ heißen \highl{Kreiseinheiten}.
 Es folgt $p=\eps_1\eps_2\ldots \eps_{p-1} \pi^{p-1}$.

\begin{Lemma}[\glqq Lemma 2\grqq{}]
 Ist $c\in\IZ$ in $\cO_K$ durch $\pi$ teilbar, so ist $c$ in $\IZ$ durch $p$ teilbar ($\pi=1-\zeta_p$).
\end{Lemma}

\begin{Beweis}
 $c=\pi x$, $x\in \cO_K$.
 
 $\N(c)=c^{p-1}=\N(\pi)\N(x)=p \underbrace{\N(x)}_{\in\IZ}$. Also gilt $p\mid c^{p-1}$ \folge $p\mid c$.
\end{Beweis}
 
 Sei nun $x\in \cO_K$, $x=a_0+a_1\zeta_p+ \ldots + a_{p-2} \zeta_p^{p-2}$ mit $a_j\in\IQ$.
 
 \[\Tr_\IQ^K (\zeta_p^r)=\begin{cases}
                          -1 & r \not\equiv 0 \mod{p}\\
                          p-1 & r \equiv 0 \mod{p}
                         \end{cases}
 \]
 
 Also ist $\Tr(\zeta_p x) = \sum_{j=0}^{p-2} a_j \Tr(\zeta_p^{j+1})=-(a_0+a_1+\ldots + a_{p-2})$.




 

