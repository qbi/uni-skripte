%!TEX root = Algebraische_Zahlentheorie-Skriptum.tex 

\renewcommand{\lecdate}{07.01.2015}

\begin{Beweis}
 Die Implikation $\Leftarrow$ ist trivial: \[ |x_n-x|_1\rightarrow 0 \folge |x_n-x|_1^c\rightarrow 0\]
 
 \Hinrichtung Sei $|x|_1<1$, dann ist $(x^n)$ Nullfolge für $|\cdot|_1$, also auch für $|\cdot|_2$: $|x|^n_2\rightarrow 0$ \folge $|x|_2<1$. Mithin $|x|_1<1$ \gdw $|x_2|<1$. 
 
 Wir fixieren $y\in K\kreuz$ mit $|y|_1>1$ ($|\cdot|_1$ ist nicht trivial). Für jedes $x\in K\kreuz$ ist dann $|x|_1=|y|_1^t$ mit $t=\frac{\log |x|_1}{\log |y|_1}\in\IR$. Wir wählen Folge rationaler Zahlen $\alpha_i=\frac{a_i}{b_i}$, $b_i>0$ mit $\alpha_i\textdownarrow t$. Dann gilt $|x|_1=|y|_1^t<|y|_1^{a_i/b_i}$ \folge $|x^{b_i}/y^{a_i}|_1<1$ \folge $|x^{b_i}/y^{a_i}|_2<1$. \folge $|x|_2<|y|_2^{a_i/b_i}$. Für $i\rightarrow \infty$ folgt $|x|_2\leq |y|_2^t$. Analog gibt $\beta_i\uparrow t$ die Abschätzung: $|x|_2\geq |y|_2^t$. Somit $|x|_2=|y|_2^t$. Es folgt $\log |x|_1=t\log |y|_1$ und $\log |x|_2=t\log |y|_2$.
 \[\rightarrow \frac{\log |x|_1}{\log |x|_2}=\frac{\log |y|_1}{\log |y|_2}=c \hspace{.5cm}\forall x\in K\kreuz \]
 
 Nach Voraussetzung war $|y|_1>1$, also $|1/y|_1<1$. \folge $|1/y|_2<1$ \folge $|y|_2>1$\folge $c>0$.
\end{Beweis}

\paragraph{Interludium: Das 2015 - Paket}
\begin{enumerate}
 \item $K=\IQ(\sqrt{-2015})$
 
 $-2015\equiv 1\mod{4}$, also $d_K=-2015=-5\cdot 13\cdot 31$
 
 MINKOVSKI-Konstante: $M_K=\frac{2!}{2}\frac{4}{\pi}\sqrt{2015}<29$ (Taschenrechner)
 
 Primzahlen $<29$: verzweigt: $5,13$. $5\cO_K=\Fp_5^2$, $13\cO_K=\Fp_{13}^2$
 
 $-2015\equiv 1\mod{8}$ \folge $2$ zerlegt: $2\cO_K=\Fp_2\cdot\overline{\Fp_2}$
 
 $\left(\frac{-2015}{3}\right)=\left(\frac{-5}{3}\right)=\left(\frac{1}{3}\right)=1$ \folge 3 zerlegt.
 
 $\left(\frac{-2015}{23}\right)=\left(\frac{285}{23}\right)=\left(\frac{55}{23}\right)=\left(\frac{9}{23}\right)=1$ \folge 23 zerlegt.
 
 usw.
 
 \begin{tabular}{l|l}
  zerlegt & $2,3,7,11,17,23$\\
  träge & $19$\\
  verzweigt & $5,13$
 \end{tabular}

 Also: $\Cl_K$ wird erzeugt von $\Fp_l$, $l=2,3,5,7,11,13,17,23$.
 
 Ganzheitsbasis ist $1, \omega=\frac{1+\sqrt{-2015}}{2}$. $\N(a+b\omega)=\left(a+\frac{b}{2}\right)^2 + 2015\cdot \frac{b^2}{4}=a^2+ab+504b^2$, $a,b\in\IZ$.
 
 Sei $b=1$. \begin{tabular}{>{$}c<{$}|>{$}c<{$}}
             a & a^2+a+504 \\\hline
             0 & 504 = 2^3\cdot 3^2\cdot 7 \folge (\omega)=\Fp_2^3\Fp_3^2\Fp_7\\
             1 & 506 = 2\cdot 11\cdot 23\\
             2 & 510 = 2\cdot 3\cdot 5 \cdot 17\\
             & \vdots\\
             23 & 1056=2^5\cdot 3\cdot 11
            \end{tabular}

 Also fallen Erzeuger weg: \begin{tabular}{>{$}l<{$}l}
                            a=1 & $\Fp_{23}$ weg\\
                            a=2 & $\Fp_{17}$ weg\\
                            a=6 & $546 = 2\cdot 3\cdot 7\cdot 13$, $\Fp_{13}$ weg\\
                            a=7 & $560 = 2^4\cdot 5\cdot 7$, $\Fp_7$ weg\\
                            a=9 & $594 = 2\cdot 3^3\cdot 11$, $\Fp_{11}$ weg\\
                           \end{tabular}

 Also wird $\Cl_K$ erzeugt von $\Fp_2,\Fp_3,\Fp_5$. \begin{tabular}{>{$}l<{$}l}
                            a=23 & $\Fp_{3} \sim \Fp_2^5\Fp_11$ in $\Cl_K$\\
                           \end{tabular}
  
  Lücke:[
  $\N(17+\omega)=810=2\cdot 3^4\cdot 5$. $(17+\omega)=\Fp_2\Fp_3^4\Fp_5$ \folge $\Cl_K$ erzeugt von $\Fp_3,\Fp_5$, also $\Cl_K=\langle\Fp_3\rangle\times\langle\Fp_5\rangle.$
  
  $\Fp_5$ ist kein HIR: $\IN\Fp_5=5=a^2+ab+504b^2$ \folge $20=(2a+b)^2+2016b^2$ \lightning. $N(8+\omega)=576=3^2\cdot 2^6$ \folge $\Fp_3$ hat in $\Cl_K$ Ordnung 13 oder 26. Für Ordnung 13 folgt $4\cdot 13^{13}=(2a+b)^2+2016b^2$. Das geht nicht. Somit $\Cl_K\cong C_{26}\times C_2$, $h_K=52$ ]
  
  
  Also $\Cl_K$ erzeugt von $\Fp_2$, $\Fp_5$. $\Fp_5^2\sim 1$, aber $\Fp_5$ nicht: $\Fp_5=(a+b\omega)$
  
  \folge $5=\IN\Fp_5=a^2+ab+504b^2$. \folge $b=0$, $5=a^2$ \lightning. Also $\Fp_5$ kein Hauptideal.
  
  $a=17$: $810=2\cdot 3^4\cdot 5$ \folge $\Fp_2\cdot\Fp_3^4\cdot \Fp_5 \sim 1$.
  
  $a=8$: $576=2^6\cdot 3^2$
  
  \folge $\Fp_2^6\cdot \Fp_3^{\pm 2} \sim 1$ \folge $\Fp_2^{24}\Fp_3^{\pm 8}\sim 1$.
  
  Für $+8$: $\Fp_2^{24}\sim \Fp_3^{-8}$, $\Fp_3^{-8}\sim\Fp_2^2\sim\Fp_5^2\sim\Fp_2^2$. Damit $\Fp_2^{24}\sim\Fp_2^2$ \folge $\Fp_2^{22}\sim 1$.
  
  Für $-8$: $\Fp_2^{24}\sim \Fp_2^{-2}$ \folge $\Fp_2^{26}\sim 1$.
  
  Annahme $\Fp_2^{22}\sim 1$ \folge $2^{22}=a^2+ab+504 b^2$ \folge $2^{24}=(2a+b)^2+2015 b^2$.
  
  $2^{24}/2015\approx 8326,17$. Wurzel $<91$. Ergebnis negativ: Mithin $\Fp_2^{26}\sim 1$.
  
  $\Fp_2^2$ ist kein Hauptideal: $4=a^2+ab+504b^2$, $16=(2a+b)^2+2015b^2$. $\Fp_2^2=(2)=\Fp_2\overline{\Fp_2}$\folge $\Fp_2=\overline{\Fp_2}$ \lightning
  
  $\Fp_2^{13}\sim 1$ \gdw $\Fp_2^{13}=(a+b\omega)\folge 2^{15}=(2a+b)^2+2015b^2$. $2^{15}/2015<16,27$ \folge $b<5$ - keine Lösungen.
  
  Somit hat $\Fp_2$ ind $\Cl_K$ die Ordnung $26$. Angenommen $\Fp_5\sim\Fp_2^a$ \folge $a=13$. $5\cdot 2^{15}=(2a+b)^2+2015b^2$, $b<9$ - keine Lösungen.
  
  Ergebnis: $\Cl_K=\langle\Fp_2\rangle\times\langle\Fp_5\rangle=C_{26}\times C_2$.
  
  ÜA: $2^{28}=(2a+b)^2+2014 b^2$ hat $\IZ$-Lösungen. Welche?
  
\item Fundamentaleinheit von $\IQ(\sqrt{2015})$

$K=\IQ(\sqrt{2015})$, $\cO_K=\IZ+\IZ\omega$, $\omega=\sqrt{2015}$, $d_K=4\cdot 2015$. 

Kettenbruch von $omega$: $44<\omega<45$: $44^2=1936$, $45^2=2025$.

Also $\omega=44+(\omega-44)$.

$\frac{1}{\omega-44}=\frac{\omega+44}{79}=1+\frac{\omega-35}{79}$

$\frac{79}{\omega-35}=\frac{79(\omega+35)}{790}=7+\frac{\omega-35}{10}$

$\frac{10}{\omega-35}=\frac{10(\omega+35)}{790}=1+\frac{\omega-44}{79}$

$\frac{79}{\omega-44}=\frac{79(\omega+44)}{79}=\omega+44=88+(\omega-44)$

Ergebnis: $\sqrt{2015}=[44,\overline{1,7,1,88}]$

\begin{tabular}{>{$}l<{$}|*{5}{>{$}c<{$}}}
 n & 0 & 1 & 2 & 3 & 4\\\hline
 a_n &  & 44 & 1 & 7 & 1\\\hline
 p_n & 1 & 44 & 45 & 359 & 404\\\hline
 q_n & 0 & 1 & 1 & 8 & 9\\\hline
 p_n^2-2015q_n^2 & & -79 & 10 & -79 & 1
\end{tabular}

\folge $\eps_K=404+9\cdot \sqrt{2015}$, $\N\eps_K=+1$.
\item Die Idealklassengruppe von $\IQ(\sqrt{2015})$

$d_K=4\cdot 2015=2^2\cdot 5\cdot 13\cdot 31$. $M_K=\frac{2!}{2^2}\cdot 2\sqrt 2015<45$

Also verzweigt: $2,5,13,31$.

$\left(\frac{2015\cdot 4}{3}\right)=\left(\frac{2}{3}\right)=-1$ \folge $3$ träge.

 \begin{tabular}{l|l}
  zerlegt & $17,19$\\
  träge & $3,7,11,23,29,37,41,43$\\
 \end{tabular}
 
 $\Cl_K$ wird erzeugt von $\Fp_l$, $l=2,5,13,17,19,31$.
 
 $\N(a+b\omega)=a^2-2015b^2$. Sei $b=1$:
 
 \begin{tabular}{>{$}c<{$}|>{$}c<{$}}
             a & a^2-2015\\\hline
             45 & 10=2\cdot 5\\
             48 & 289 = 17^2\\
             39 & -494 = -2\cdot 13\cdot 19\\
             37 & -646=-2\cdot 17\cdot 19\\
             31 & -1054=-2\cdot 17\cdot 31
            \end{tabular}
\end{enumerate}

 \begin{tabular}{>{$}l<{$}l}
                            a=48 & $\Fp_{17}^2\sim 1$\\
                            a=45 & $\Fp_{2}\sim\Fp_5$ \folge $\Fp_5$ weg\\
                            a=39 & $\Fp_{19}$ weg\\
                            a=37 & $\Fp_{17}$ weg\\
                            a=31 & $\Fp_{31}$ weg\\
                           \end{tabular}

  \folge $\Cl_K$ wird erzeugt von $\Fp_2$, $\Fp_{13}$, beide der Ordnung $\leq 2$.
  
  Mit Geschlechtertheorie folgt sofort: $\Cl_K$ hat Ordnung $4$.
  
  Von Hand: 
  
  Man darf annehmen: $1<a+b\omega<\eps_K$ ($a,b$ positiv). 
  
  $(a+b\omega)(a-b\omega)=\pm 2$. $-\frac{2}{a+b\omega}<a-b\omega<\frac{2}{a+b\omega}<1$.
  
  \folge $-1<b\omega-a<1$ \folge $0<2b\omega <1+\eps_K$ \folge $0<b<\frac{1+\eps_K}{\omega}=\frac{405+\omega}{\omega}=\frac{405}{\omega}+1<19$ \folge keine $\IZ$-Lösungen.
  
  Analog $\Fp_3 \not\sim 1$. Fazit: $\Cl_K = \langle \Fp_2\rangle \times \langle \Fp_{13}\rangle \cong C_2\times C_2$.