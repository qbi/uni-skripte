\renewcommand{\lecdate}{10.02.2015}

\begin{Fakt}
 Sei $L/K$ vollverzweigt und zahm, dann ex. $\pi_L\in\cO_L$, $\pi_K\in\cO_K$, s.d. $\pi_L$ Wurzel eines Polynoms $X^e-\pi_K$ ist, $e=[L:K]$. Umgekehrt erzeugt jede Wurzel von $X^e-\pi_K$ mit $\ggT(e,p)=1$ eine voll- und zahm verzweigte Erweiterung. 
\end{Fakt}

\begin{Beweis}
 Die 2. Aussage hatten wir gerade.
 
 Sei $[L:K]=e=e(L/K)$, $\ggT(e,p)=1$.
 Wähle $\pi_K\in\Fm_K\setminus\Fm_K^2$ und $\alpha\in\cO_L$ mit $|\alpha|^e=|\pi_K|$, also $\Fm_L=(\alpha)$, $\alpha=\pi_L$. Dann ist $L=K(\alpha)$, siehe oben. Wir wollen $\pi_K$ abändern zu $\pi_K'$, s.d. eine Wurzel von $X^e-\pi_K'$ den Körper $L$ über $K$ erzeugt.
 
 Es gilt $\alpha^e=\pi_K\cdot u$, $u\in\cO_L\kreuz$. Wegen $l=k$ ($l=\cO_L/Fm_L$, $k=\cO_K/\Fm_K$) gilt $u\equiv u_0 \mod{\Fm_L}$, $u_0\in\cO_K\kreuz$. Setze $\pi=u_0\cdot \pi_K$, dann ist $|\pi|=|\pi_K|$. Sei $f(X)=X^e-\pi$ und $\alpha_1,\ldots,\alpha_e$ die Wurzeln von $f$. Sie sind verschieden, denn $f'(X)=eX^{e-1}$. Es gilt \[|f(\alpha)|=\prod_{j=1}^e |\alpha-\alpha_j|, \]
 sowie $|\alpha_j|^e=|\pi|$, also $|\alpha_j|=|\alpha|=|\pi_L|<1$ für alle $j$. Andererseits ist
 \[ \alpha^e=u\cdot \pi_K=(u_0+x)\pi_K\]
$x\in\Fm_L=\pi+\pi y$, $y=xu_0\inv\in\Fm_L$. Somit $|\alpha^e-\pi|=|\pi y|<|\pi|$, also $|f(\alpha)|<|\pi|$ \folge $\prod_{j=1}^e |\alpha-\alpha_j|<|\pi|=|\pi_L|^e$. Weiter gilt $|\alpha-\alpha_j|\leq |\pi_L|$, wegen $\alpha,\alpha_j\in\Fm_L$. Es ex. also ein $j$, s.d. $|\alpha-\alpha_j|<|\pi_L|=|\alpha_j|$. Nun gilt $|f'(\alpha_j)|=\prod_{i\neq j} |\alpha_i-\alpha_j|=|\alpha_j|^{e-1}$ und $|\alpha_i-\alpha_j|\leq |\alpha_j|=|\alpha|$, also $|\alpha_i-\alpha_j|=|\alpha_j|$.

Somit $|\alpha-\alpha_j|<|\alpha_i-\alpha_j|$ für alle $i\neq j$. Nach KRASNERs Lemma folgt $K(\alpha_j)\subset K(\alpha)$. Wegen $[K(\alpha_j):K]=[K(\alpha):K]=e$ folgt $K(\alpha_j)=K(\alpha)$.
\end{Beweis}

\begin{Fakt}
 \begin{enumerate}
  \item $E/L$, $L/K$ zahm verzweigt \folge $E/K$ zahm verzweigt.
  \item $L/K$ zahm verzweigt, $E/K$ endlich (in $\overline\IQ_p$) \folge $EL/E$ zahm verzweigt.
  \item $E/K$, $F/K$ zahm verzweigt \folge $EF/K$ zahm verzweigt.
 \end{enumerate}
\end{Fakt}

\begin{Beweis}
 \begin{enumerate}
  \item klar
  \item $L/K$ zahm verzweigt \folge $\exists$ $F/K$, s.d. $F/K$ unverzweigt, $L/F$ voll- und zahm verzweigt. Also $L=F(\alpha)$, $\alpha^e=\pi_F$, $e=e(L/K)$ prim zu $p$. Dann ist $LE=E(\alpha)$ und $\alpha$ Wurzel eines Polynoms $X^e-\pi_F$, $p\nmid e$. Wir zeigen, dass solche Erweiterungen stets zahm verzweigt sind. Der Zerfällungskörper eines solchen Polynoms ist zweistufig metabelsch (Gruppe selbst und Faktorgruppe abelsch): 1. Schritt: Adjunktion mit der $e$-ten Einheitswurzel (hat abelsche GALOIS-Gruppe), 2. Schritt: Hat Grad, der $e$ teilt: $\Gal(\text{2. Schritt})\hookrightarrow \mu_e$. 
  \item folgt formal aus (i) und (ii)
 \end{enumerate}
\end{Beweis}

\begin{Folgerung}
 Jede endliche Erweiterung von lokalen ZK $L/K$ zerlegt sich kanonisch in 3 Teile: \[ L/K^t/K^{nr}/K\]
 mit \begin{itemize}
      \item $K^{nr}/K$ maximal unverzweigt
      \item $K^t/K^{nr}$ zahm und vollverzweigt
      \item $L/K^t$ wild und vollverzweigt (Grad $=p^n$)
     \end{itemize}
\end{Folgerung}

\begin{Beweis}
 $K^t$=Kompositum aller zahm verzweigten Erweiterungen in $L/K$.
\end{Beweis}

\begin{Lemma}
 Sei $L/K$ Galerw. lokaler ZK, $\sigma\in\Gal(L/K)$, $E$ Zwischenkörper, Dann gilt \[ e(\sigma E/K)=e(E/K),\] \[f(\sigma E/K)=f(E/K). \]
\end{Lemma}

\begin{Beweis}
 Sei $\Fm_E=(\pi_E)$, dann ist $(\sigma \pi_E)=\sigma\Fm_E$. $\pi_E^{e(E/K)}=u \pi_E$ \folge $(\sigma \pi_E)^{e(E/K)}=v\pi_K$, $u,v\in\cO_E\kreuz$. Es folgt $e(E/K)=e(\sigma E/K)$. Die Formel für $f$ folgt aus $e\cdot f=[E:K]$ und $[\sigma E:K]=[E:K]$.
\end{Beweis}

\begin{Folgerung}
 Ist $L/K$ Galoiserweiterung, so auch $K^t/K$.
\end{Folgerung}

\subsection{Galoistheorie der lokalen ZK}

Sei $L/K$ Galoiserweiterung lokaler ZK, dann lässt $G=\Gal(L/K)$ den Ring $\cO_L$ invariant, wie auch das maximale Ideal $\Fm_L$. Also induziert jedes $\sigma\in G$ einen Automorphismus von $l/k$, $l=\cO_L/\Fm_L$. Also haben wir einen HM $\Gal(L/K)\rightarrow \Gal(l/k)$.

Sei $M/K=K^{nr}/K$. Sie ist normal und der HM $\Gal(M/K)\rightarrow\Gal(m,k)$ ist Iso. Es gilt $l=m$.

\begin{Definition}
 Der Kern dieses Homomorphismus heißt \highl{Trägheitsgruppe} (inertia group). Das ist die Galoisgruppe des vollverzweigten Schritts.
\end{Definition}

\begin{Fakt}
 Sei $L/K$ wie oben, $K^t/K$ die max. zahmverzweigte Erweiterung in $L/K$. Der Kern des HM $\Gal(L/K) \rightarrow \Gal(K^t/K)$ heißt \highl{1. Verzweigungsgruppe}. Dies ist eine $p$-Gruppe. Also ist die $Gal(L/K)$ aufläsbar.
\end{Fakt}
