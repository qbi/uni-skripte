\renewcommand{\lecdate}{20.01.2015}

\begin{Fakt}
$\IZ_p$ ist kompakt, $\IQ_p$ ist lokalkompakt, aber nicht kompakt.
\end{Fakt}

\begin{Beweis}
\[\IQ_p=\bigcup_{n=1}^\infty p^{-n}\IZ_p\]
und $\IZ_p$ ist offen in $\IQ_p$. $\IQ_p\setminus\IZ_p=\{ x\in\IQ_p : |x|_p\geq p\}$ abgeschlossen. Es gibt eine endliche Teilüberdeckung. Sei $(x_n)$ Folge aus $\IZ_p$, $\exists a_0\in\{ 0,1,\ldots,p-1\}$ s.d. unendlich viele der $x_n$ mit $a_0$ beginnen. Unter denen gibt es unendlich viele, die mit $a_0+a_1p$ beginnen usw. \folge es ex. eine Teilfolge $(x_{\mu(n)})$, s.d.
\[ |x_{\mu(n)}-(a_0+a_1p+\ldots+a_np^n|_n<\frac{1}{p^n}.\]
Sei $x=\sum_{j=0}^\infty a_jp^j$, dann gilt $|x_{\mu(n)}-x|_p\leq \frac{1}{p^n}$ \folge konvergente Teilfolge gefunden.
\end{Beweis}

\begin{Bemerkung}
\begin{enumerate}
\item $\IZ_p$ ist offen und abgeschlossen in $\IQ_p$ (clopen):
\[ \IZ_p=\{ x\in\IQ_p : |x|_p\leq 1\}=\{ x\in\IQ_p : |x|_p< p\}\]
Jeder topologische Raum ist disjunkte Vereinigung seiner Zusammenhangskomponenten. $\IR$ ist zerlegt, $\IQ$ ist total unzusammenhängend, d.h. jeder Punkt ist seine Zusammenhangskomponente:
Sei $\alpha\in E\subset\IQ$ zusammenhängend. Ist $\beta\in E\setminus\{\alpha\}$, so existiert $\gamma\in\IR\setminus\IQ$ zwischen $\alpha$ und $\beta$, es gilt 
\[ E=\underbrace{(E\cap \{ x\in\IQ : x<\gamma\})}_{\text{offen}} \sqcup \underbrace{(E\cap \{ x\in\IQ : x>\gamma\})}_{\text{offen}}.\]
\folge $E=\{\alpha\}$.

$\IZ_p$ ist auch total unzusammenhängend:
$\alpha\in\IZ_p$ habe zusammenhängende Obermenge $E\subset\IZ_p$. Enthält diese $\beta\neq\alpha$, so gilt
\[ \alpha=\sum a_jp^j, \beta=\sum b_jp^j,\hspace{2cm} a_0=b_0,\ldots,a_{r-1}=b_{r-1}, a_r\neq b_r.\]
Nun ist \[ \IZ_p=\bigsqcup_{0\leq a<p^r}(a+p^r\IZ_p)\]
disjunkte Zerlegung in clopen sets, $\alpha, \beta$ liegen in verschiedenen. Also $E=\{\alpha\}$.
\item $\IQ_p$ ist wie $\IR$ vollständig und lokalkompakt. Das sind Voraussetzungen für Analysis und co. Es gibt in allem Analogien für $p$-adische Körper: $p$-adische Mannigfaltigkeiten, $p$-adische DGLn, $p$-adische LIE-Algebren, $p$-adische LIE-Gruppen, $p$-adische Funktionalanalysis, $p$-adische transzendente Funktionen. $p$-adische Funktionentheorie ist schwieriger, als über $\IC$: $[\IC:\IR]=2$, aber $[\overline{\IQ_p}:\IQ_p]=\infty$. Überdies ist $\overline{\IQ_p}$ nicht vollständig. $\tilde{\overline{\IQ_p}}:=\IC_p$ ist algebraisch abgeschlossen und vollständig (sog. TATE-Körper).

Über $\IQ_p$ werden Träume wahr: $\sum a_n$ konvergiert \gdw $a_n\rightarrow 0$.
\end{enumerate}
\end{Bemerkung}

\begin{Beispiel}
Was ist $\sqrt{2}$?

In $\IQ_2$ gibt es keine Zahl mit Quadrat $2$: $x^2=2$ \folge $|x|_2^2=\frac{1}{2}$ \folge $|x|_2=\frac{1}{\sqrt{2}}$ \lightning.

Sei $p>2$, wenn $x^2=2$ ist - mit $x\in\IQ_p$, so folgt $y^2\equiv 2 \mod{p}$ hat Lösung. D.h. $2$ ist Quadrat in $\IF_p$ \gdw $\left(\frac{2}{p}\right)=1$ \gdw $p\equiv \pm 1\mod{8}$. Also gibt es für $p\equiv  \pm 5\mod{8}$ keine Zahl mit Quadrat $2$ in $\IQ_p$. 

Sei $p\equiv \pm 1\mod{8}$, $\alpha_0\in\IZ_p$ so, dass $\alpha_0^2\equiv 2\mod{p}$ gilt. Setze $\alpha_{n+1}:=\alpha_n-\frac{\alpha_n^2-2}{2}$. Das ist korrekt definierte Folge: $|\alpha_0|_p=1$, $|\alpha_0^2-2|_p<1$, $|\alpha_0^2+2|_p=|\alpha_0^2-2+4|_p=1$. Induktion: $|\alpha_{n+1}|_p=\left|\frac{\alpha_n^2+2}{2\alpha_n}\right|_p$
\[ \alpha_{n+1}^2-2=\frac{(\alpha_n^2+2)^2}{4\alpha_n^2}-2=\frac{(\alpha_n^2-2)^2}{4} \folge |\alpha_{n+1}^2-2|_p<1\]
\folge $|\alpha_{n+1}^2+2|_p=|\alpha_{n+1}^2-2+4|_p=1$.

Es folgt $|\alpha_{n+1}-\alpha_n|=|\alpha_n^2-2|$ und $|\alpha_{n+1}^2-2|_p=|\alpha_n^2-2|_p^2$ \folge $|\alpha_n^2-2| \rightarrow 0$.

Damit ist $(\alpha_n)$ CAUCHY-Folge, also konvergent gegen $\alpha$ say. Aus der Rekursionsformel folgt $\alpha=\alpha-\frac{\alpha^2-2}{2}$ \folge $\alpha^2=2$.

Konkret für $p=7$: \[ \alpha = a_0 + a_1\cdot 7 + a_2\cdot 49 + a_3 \cdot 243 + \ldots \hspace{2cm} a_j\in\{ 0,1,\ldots,6\}\]
$a_0^2\equiv 2\mod{7}$ \folge $a_0=3$ oder $a_0=4$. Wir wählen $a_0=3$. $\alpha_1=3-\frac{9-2}{6}=\frac{11}{6}$. Dann $\alpha_1^2-2=\frac{121-72}{36}=\frac{49}{36}$.
\[ \alpha_2=\frac{11}{6}-\frac{49/36}{11/3}=\frac{11}{6}-\frac{49}{132}=\frac{242}{132}-\frac{49}{132}=\frac{195}{132}.\]
Damit $\alpha^2-2=\frac{2401}{(132)^2}=\frac{7^4}{2^4\cdot 3^2\cdot 11}$
\end{Beispiel}

\begin{Beispiel}
$y'=y$, Ansatz: $y=\sum_{j=0}^\infty a_jx^j$, damit $y'=\sum_{j=1}^\infty ja_jx^{j-1}$

\folge $a_j=(j+1)a_{j+1}$ \folge $a_j=\frac{a_0}{j!}$. Also $y=c\cdot \exp(x)=c\sum_{n=0}^\infty\frac{x^n}{n!}$. $\ord_p(n!)=\frac{n-s(n)}{p-1}$ ($s(n)=$ Quersumme von $n$ in $p$-adischer Darstellung).

Archimedische STIRLING-Formel:
\begin{align*}
\log(n!)&=n\log(n)-n+\frac{1}{2}\log(2\pi n)+\frac{\theta_n}{12n}, \hspace{2cm} 0<\theta_n<1\\
n!&=e^{n\log n - n + \ldots}\\
&= n^n e^{-n}\sqrt{2\pi n} e^{\theta_n/12 n}
\end{align*}
Konvergenzradius: $\limsup_n (\sqrt[n]{|n!|_p})^{-1}$ \folge konvergiert für $p>2$ auf $p\IZ_p$.
\end{Beispiel}

\begin{Beispiel}
Die additive Gruppe von $\IQ_p$ ist LCA (locally compact abellion). Solche Gruppen besitzen ein verschiebungsinvariantes BOREL-Maß, ein sog. HAAR-Maß. Das HAAR-Maß auf $\IQ_p$ ist
\[\lambda(a+p^n\IZ_p)=\frac{1}{p^n} \]
mit Normierung $\lambda(\IZ_p)=1$.
\end{Beispiel}