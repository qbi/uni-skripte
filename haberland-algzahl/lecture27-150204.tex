\renewcommand{\lecdate}{04.02.2015}

\begin{Bemerkung}[Variante von HENSELs Lemma]
 Sei $K$ vollständig, bzgl. diskreter nichtarchimedischer Bewertung $|\cdot|$, $\cO$ Bewertungsring, $\Fm$ maximales Ideal, $k=\cO/\Fm$, $f\in\cO[X]$ unitäres Polynom, $\overline{f}=\overline{g}\overline{h}$ in $k[X]$ mit $\overline{g}, \overline{h}$ teilerfremd. Dann existieren $g,h\in\cO[X]$ unitär mit $g\mapsto \overline{g}$, $h\mapsto\overline h$ und $f=gh$.
 
 \begin{Beweis}
  O.B.d.A. seien $\overline{g}, \overline{h}$ unitär. Seien $g_0,h_0\in\cO[X]$ Liftungen von $\overline{g}$, $\overline{h}$, unitär vom selben Grad. Also $f-g_0h_0\in\Fm[X]$.
  
  Ansatz: \[g=g_0+a_1\pi+a_2\pi^2+\ldots\] \[h=h_0+b_1\pi+b_2\pi^2+\ldots \]
  mit $\Fm=(\pi)$ und $a_j,b_j\in\cO[X]$, sowie $\deg a_i<\deg \overline{g}$, $\deg b_i\leq \deg \overline{h}$.
  Beide Reihen konvergieren, $g$ ist unitär wg. $\deg a_i<\deg \overline{g}$. Außerdem gilt $g\mapsto \overline g, h\mapsto \overline h$.
  
  Sei $g_n=\sum_{i=0}^n a_i\pi^i$, $h_n=\sum_{i=0}^nb_i\pi^i$.
  
  Induktionsanfang: $f\equiv g_0h_0 \mod{\pi}$
  
  Seien $a_i,b_i$ gefunden s.d. $f\equiv g_nh_n\mod {\pi^{n+1}}$.
  Wir suchen Polynome $a_{n+1}, b_{n+1}\in\cO[X]$, s.d. $g_{n+1}=g_n+a_{n+1}\pi^{n+1}$, $h_{n+1}=h_n+b_{n+1}\pi^{n+1}$.
  \begin{align*}
   f&\equiv g_{n+1}h_{n+1} \mod{\pi^{n+2}}\\
   &\equiv g_nh_n+(g_nb_{n+1}+a_{n+1}b_n)\pi^{n+1}\mod{\pi^{n+2}}
  \end{align*}
  $f-g_nh_n\equiv 0\mod{\pi^{n+1}}$, also ist $f_{n+1}:= \frac{f-g_nh_n}{\pi^{n+1}}\in\cO[X]$ und wir haben zu sichern:
  \[\overline{f}_{n+1}=\overline{g}\overline b_{n+1} + \overline{a}_{n+1}\overline{h}\in k[X] \]
  Nach Voraussetzung ist $\ggT(\overline g, \overline h)=1$ \folge $\exists \overline c, \overline d \in k[X]$ mit $1=\overline g\overline c + \overline h \overline d$
  \folge $\overline f_{n+1}=\overline g\overline c_1+\overline h \overline{d}_1=\overline g(\overline c_1+\overline r\overline h)+\overline h(\overline d_1-\overline r\overline g)$ für $\overline r\in k[X]$.
  
  Also kann man erreichen, dass der Faktor bei $\overline h$ Grad $<\deg \overline g$ hat.
 \end{Beweis}
 
 Was hat das mit HENSELs Lemma zu tun? Sei $\overline f=(X-\overline \lambda)\overline h(X)$, $\overline \lambda\in k$ \folge $\overline f(\overline \lambda)=0$. $\ggT(X-\overline \lambda,\overline h)=1$ \folge $\overline f'(\overline \lambda)\neq 0 \folge |f(\lambda)|<1$, $|f'(\lambda)|=1$ ($\lambda$ Liftung von $\overline \lambda$ nach $\cO$).
\end{Bemerkung}

\begin{Folgerung}
 Unverzweigte Erweiterungen sind normal.
\end{Folgerung}

\begin{Beweis}
 $K(\zeta_n)=K(\mu_n)$.
\end{Beweis}

\begin{Fakt}
 \begin{enumerate}
  \item $M/L$, $L/K$ unverzweigt \folge $M/L$ unverzweigt
  \item $L/K$, $M/K$ unverzweigt (in $\overline\IQ_p$) \folge $LM/K$ unverzweigt
  \item $L/K$ unverzweigt, $E/K$ endlich (in $\overline\IQ_p$) \folge $LE/E$ unverzweigt
 \end{enumerate}
\end{Fakt}

\begin{Beweis}
 \begin{enumerate}
  \item[(iii)] $[L:L\cap E]=[LE:E]$
  
  Analoges Bild für die Restklassenkörper \folge $f(le/e)=f(l/(l\cap e))$. Aus $e(L/K)=1$ folgt $e(L/(L\cap E))=1$, somit $e(LE/E)=e(L/(L\cap E))=1$.
  \item[(i)] folgt aus der Multiplikativität von $E$
  \item[(ii)] folgt aus vorherigem Fakt: $L=K(\zeta_m)$, $M=K(\zeta_n), LM=K(\zeta_g)$, $g=\kgV(m,n)$ (prim zu $p$).
 \end{enumerate}
\end{Beweis}

\begin{Folgerung}
 In jeder endlichen Erweiterung $L/K$ gibt es maximale unverzweigte Erweiterungen $L/K^{nr}/K$ (nr - non ramified).
\end{Folgerung}

\begin{Folgerung}
 Die endlichen unverzweigten Erweiterungen in $\overline K/K$ sind in kanonischer Bijektion zu den endlichen Erweiterungen in $\overline k/k$.
\end{Folgerung}

\begin{Beweis}
 $\IF_{q^n}=\IF_q(\mu_{q^n-1})$, $K_n=K(\mu_{q^n-1})$.
\end{Beweis}

\begin{Fakt}
 Sei $L/K$ endliche unverzweigte Erweiterung. Dann ist der kanonische Homomorphismus
 \[ \Gal(L/K) \rightarrow \Gal(l/k)\]
 ein Isomorphismus.
\end{Fakt}

\begin{Beweis}
 Sei $\sigma:L\rightarrow L$ ein $K$-Automorphismus. $\sigma\cO_L=\cO_L$, $\sigma\Fm_L=\Fm_L$, wg. $|\sigma x|=|x|$. Also induziert $\sigma$ einen $k$-Automorphismus von $l$. Das liefert den Homomorphismus
 \[ \Gal(L/K) \rightarrow \Gal(l/k)\]
 (Man muss sich die Existenz beider Gruppen klar machen, insbesondere die Separabilität von $l/k$ (endl. Erweiterung endlicher Körper ist separabel).)
 Für geeignetes $n$ prim zu $p$ ($K\supset \IQ_p$) ist $L=K(\zeta_n)$, $l=k(\overline\zeta_n)$. Die Abbildung $\mu_n(L)\rightarrow\mu_n(l)$ ist ein Isomorphismus. Ist also $\overline \sigma=Id$ auf $\mu_n(l)$, so auch $\sigma=Id$ auf $\mu_n(L)$. Also ist obiger Homomorphismus injektiv. Beide Gruppen haben dieselbe Ordnung $[L:K]=[l:k]$.
\end{Beweis}

\subsection{Zahm verzweigte Erweiterungen} %2.6
\begin{Definition}
 Seien $L/K$ lokale ZK (also endliche Erweiterungen von $\IQ_p$)
 \begin{enumerate}
  \item $L/K$ heißt \highl{vollverzweigt} \gdw $f(L/K)=1$
  \item $L/K$ heißt \highl{zahm verzweigt} \gdw $p\nmid e(L/K)$
  \item $L/K$ heißt \highl{wild verzweigt} \gdw $p\mid e(L/K)$
 \end{enumerate}
\end{Definition}

\begin{Fakt}
 Sei $L/K$ vollverzweigt und zahm, $\Fm_L=(\pi_L)$, dann ist $\pi_L$ Wurzel eines EISENSTEIN-Polynoms vom Grad $e=e(L/K)$: \[ X^l+a_{l-1}X^{l-1}+\ldots+a_0\]
 wobei alle $a_j\in\Fm_K$ und $a_0\notin \Fm_K^2$. Es gilt $L=K(\pi_L)$. Umgekehrt erzeugt jede Wurzel eines solchen Polynoms eine vollverzweigte Erweiterung vom Grad $e$.
\end{Fakt}

\begin{Beweis}
 Alle Konjugierten von $\pi_L$ (= Wurzeln von $\Irr(X,\pi_L,K)$) haben diesselbe Bewertung in $\overline \IQ_p$ (unter Automorphismen ändert sich die Bewertung nicht).
 Also sind die Koeffizienten von $\Irr(X,\pi_L,K)$ alle aus $\Fm_K$. Der Absolutterm $a_0$ ist bis auf Vorzeichen das Produkt aller Konjugierter von $\pi_L$. Andererseits ist $\pi_K=u\cdot \pi_L^e$, $\pi_L$ hat höchstens $e$ Konjugierte, also 
 $|a_0|=|\pi_L|^{e'}$, $a_0=v\cdot \pi_K^r$ \folge $|a_0|=|\pi_K|^r=|\pi_L|^{er}$ \folge $e'=er$, $e'\leq e$ \folge $r=1$, $e'=e$.
 Also $a_0\in\Fm\setminus \Fm_K^2$. Weiter folgt $[K(\pi_L):K]=e$ \folge $K(\pi_L)=L$.
 
 Sei nun $\alpha$ Wurzel von $f$, dann ist $[K(\alpha):K]=e$. Weiter gilt $|a_0|=|\alpha|^e$ \folge $\alpha=\pi_L$ und die Erweiterung $K(\alpha)/K$ ist voll verzweigt.
\end{Beweis}
