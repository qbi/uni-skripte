% Standard-Präambel.
% Diese Datei lädt alle Pakete und Optionen, die nicht für mathematische Zwecke gebraucht werden.

% Einige Inspiration habe ich mir hier von Jens Kubiziel et al. (uni-skripte.lug-jena.de) geholt.

% LuaLaTeX gehört die Zukunft.
\usepackage{luatextra}
\directlua{dofile("../../Shared/luafunctions.lua")}

% Deutsche Sprache.
\usepackage{index}
\usepackage[ngerman]{babel}
\usepackage[ngerman=ngerman-x-latest]{hyphsubst}
\usepackage[babel,german=quotes]{csquotes}

% Seitenlayout.
\usepackage{parskip}
\usepackage[a4paper,twoside]{geometry}

\usepackage{fancyhdr}

\fancyhead{}
\fancyfoot{}
\fancyhead[LE]{\rightmark}
\fancyhead[RO]{\rightmark}
\fancyfoot[LE,RO]{\thepage}

\pagestyle{fancy}

% Weitere Pakete.
%\usepackage{enumerate}
\usepackage{paralist}
\usepackage[hidelinks]{hyperref}
\usepackage{longtable}
\usepackage{multicol}
\usepackage{etoolbox}
\usepackage{url}

\newif\ifnormalLT
\normalLTtrue

\makeatletter
\patchcmd {\LT@start}
          {\vfil \break}
          {\ifnormalLT \vfil \break \fi}
          {\typeout{Patching longtable succeeded!}}
          {\typeout{Patching longtable failed!}\ERROR}
\patchcmd {\LT@start}
          {\penalty \z@}
          {\ifnormalLT \penalty \z@ \fi}
          {\typeout{Patching longtable succeeded!}}
          {\typeout{Patching longtable failed!}\ERROR}
\makeatother

\setdefaultenum{(i)}{}{}{}

% Grafikpakete.
\usepackage{pgf,tikz}
\usetikzlibrary{shapes,snakes,trees,positioning,arrows}
\usepackage{xcolor}
\usepackage{graphicx}
\usepackage{colortbl}

% Hervorhebung für wichtige Begriffe, insbesondere Definitionen.
% Sofort verknüpft mit dem Index.
\newcommand*{\highl}[2][]{\textbf{\boldmath{#2}}%
	\ifthenelse{\equal{#1}{}}{\index{#2}}{\index{#1}}%
}
\newcommand*{\nenne}[2][]{#2%
	\ifthenelse{\equal{#1}{}}{\index{#2}}{\index{#1}}%
}

\newcommand{\gqq}[1]{\glqq{}#1\grqq{}}
\newcommand{\gq}[1]{\glq{}#1\grq{}}

\makeindex