\renewcommand{\lecdate}{21.01.2015}

\begin{Satz}[\glqq Satz 2\grqq{} HENSELSs Lemma]
 Sei $K$ vollständig in der nichtarchimedischen Bewertung $|\cdot|$, $\cO=\{x\in K: |x|\leq 1\}$, $f\in\cO[X]$, $x_0\in\cO$ approximative Nullstelle:
 \[ |f(x_0)|<|f'(x_0)^2|.\]
 Dann besitzt $f$ in $\cO$ genau eine Wurzel $x\in\cO$ mit $|x-x_0|<\frac{|f(x_0)|}{|f'(x_0)|}$.
 
 Die Folge $(x_n)$ mit $x_{n+1}=x_n-\frac{f(x_n)}{f'(x_n)}$ konvergiert gegen $x$.
\end{Satz}

\begin{Beweis}
 Setze $c_0=|f(x_0)|/|f'(x_0)|<f'(x_0)\leq 1$, $c:=|f(x_0)|/|f'(x_0)|^2<1$. Wir zeigen induktiv
 \begin{enumerate}
  \item $|x_n|\leq 1$ (also $x_n\in\cO$)
  \item $|x_n-x_0|\leq c_0$
  \item $|f'(x_n)|=|f'(x_0)|$
  \item $|f(x_n)|/|f'(x_n)|^2\leq c^{2^n}$.
 \end{enumerate}
Das ist klar für $n=0$.
 
 Der Induktionsschritt auf $n+1$:
 \begin{enumerate}
  \item Aus (iv) wissen wir nach Voraussetzung: $|x_{n+1}-x_n|=\frac{|f(x_n)|}{|f'(x_n)|}\leq c^{2^n}|f'(x_n)|<1$. Also $|x_{n+1}|\leq \max(|x_n|,|x_{n+1}-x_n|)$.
  \item $|x_{n+1}-x_0|\leq \max(|x_{n+1}-x_n|, |x_n-x_0|)$, $|x_n-x_0|\leq c_0$
  \[ |x_{n+1}-x_n|\leq c^{2^n}|f'(x_0)|=c^{2^n-1}c_0<c_0\]
  \folge $|x_{n+1}-x_0|\leq c_0$
  \item $f'(x_{n+1})=f'(x_n)+\alpha\frac{f(x_n)}{f'(x_n)}$, $\alpha\in\cO$, also
  \[ \frac{f'(x_{n+1})}{f'(x_n)}=1+\alpha\frac{f(x_n)}{f'(x_n)^2}.\]
  $|\alpha \frac{f(x_n)}{f'(x_n)^2}|\leq c^{2^n}<1$ \folge $\frac{f'(x_{n+1}}{f'(x_n)}=1$
  \item $f(x_{n+1})=f(x_n)-f'(x_n)\frac{f(x_n)}{f'(x_n)}+\beta (\frac{f(x_n)}{f'(x_n)})^2$, $\beta\in\cO$ (TAYLOR-Entwicklung)
  \folge $\left| \frac{f(x_{n+1})}{f'(x_{n+1})}^2\right|\leq \left| \frac{f(x_n)}{f'(x_n)^2}\right|^2\leq (c^{2^n})^2=c^{2^{n+1}}$
 \end{enumerate}

 Es folgt $x_n\rightarrow x\in\cO$. Aus der Rekursionsformel folgt $x=x-\frac{f(x)}{f'(x)}$\folge $f(x)=0$. Aus $(ii)$ folgt $|x-x_0|\leq c_0$. Angenommen $f$ hat noch eine Wurzel $y\in\cO$ mit $|y-x_0|\leq c_0$. Dann folgt $0=f(y)=f(x)+f'(x)(x-y)+\gamma (y-x)^2$, $\gamma\in\cO$. Also $|f'(x)|\leq |y-x|\leq c_0$. Aus (iii) folgt $|f'(x)|=|f'(x_0)|$. $c_0=|f(x_0)|/|f'(x_0)|$ \folge $|f'(x_0)|^2\leq |f(x_0)|$ \lightning
\end{Beweis}

\subsection{Lokale Körper (mit Chrakteristik $0$)}

\begin{Definition}
 Eine Abbildung $\nu:K\rightarrow \IR\cup\{\infty\}$ heißt \highl{Exponentialbewertung} genau dann, wenn
 \begin{enumerate}
  \item $\nu(x)=\infty\gdw x=0$,
  \item $\nu(xy)=\nu(x)+\nu(y)$,
  \item $\nu(x+y)\geq \min(\nu(x),\nu(y))$.
 \end{enumerate}
\end{Definition}

\begin{Bemerkung}
 \begin{enumerate}
  \item Durch $|x|=c^{\nu(x)}$, $0<c<1$, entsteht eine Bewertung von $K$. Diese ist nichtarchimedisch.
  \item Die triviale Exponentenbewertung $\nu(x)=0$ für alle $x\in K\kreuz$ wird ausgeschlossen.
 \end{enumerate}
\end{Bemerkung}

 \begin{Beispiel}
  \begin{enumerate}
   \item $K=\IQ$, $p\in\IP$, $\nu_p(x)=\ord_p(x)$. Sie induziert die $p$-adische Bewertung $|\cdot|_p$. Mann nimmt gern $c=\frac{1}{p}$.
   \item Milde Verallgemeinerung: $K$ alg. ZK, $\Fp\subset\cO_K$ maximales Ideal. $\nu_\Fp(x)=\ord_\Fp(x)$ ist Exponentialbewertung und führt zu Bewertung $|\cdot|_\Fp$ auf $K$. Liegt $\Fp$ über der Primzahl $p$, so ist $\IQ_p\subset K_\Fp$.
  \end{enumerate}
 \end{Beispiel}

\begin{Definition}
 Eine Exponentialbewertung heißt \highl{diskret} \gdw $\nu(K\kreuz)$ ist bezüglich Addition diskrete Untergruppe von $\IR$, d.h. $\nu(K\kreuz)\cong \IZ$.
\end{Definition}

\begin{Beispiel}
 Die $\nu_\Fp$ sind diskret.
\end{Beispiel}

\begin{Fakt}
 Sei $\nu$ eine diskrete Exponentenbewertung von $K$, $\cO_\nu:=\{ x\in K: \nu(x)\geq 0\}$, $\Fm_\nu:=\{ x\in K: \nu(x)>0\}$.
 
 Dann ist $\cO_\nu$ ein diskreter Bewertungsring. $\Fm_\nu$ sein maximales Ideal, es ist ein Hauptideal. Der Körper $\cO_\nu/\Fm_\nu$ heißt \highl{Restklassenkörper}.
\end{Fakt}

\begin{Beweis}
 $\cO_\nu$ ist offensichtlich ein Ring, er ist integer (denn: in einem Körer), kommutativ, mit $1$. Es gilt für $x\in K\kreuz$: $x\in\cO_\nu$ oder $x\inv\in\cO_\nu$. Weiter ist $\Fm_\nu$ ein Ideal. Ist $x\in\cO_\nu\setminus\Fm_\nu$, so ist $\nu(x)=0$ \folge $\nu(x\inv)=0$ \folge $x\inv \cO_\nu$. Also ist $\cO_\nu$ lokaler Ring mit $\Fm_\nu$ als einzigem maximalen Ideal.
 Sei $\pi\in\Fm_\nu$ mit maximaler Exponentenbewertung ($\nu$ ist diskret). Dann ist $\nu(K\kreuz)=\nu(\pi)\IZ$. Also ist jedes $x\in K\kreuz$ darstellbar als $x=\pi^{\nu(x)}\cdot u$, $u\in\cO_\nu\kreuz$. Insbesondere ist $(\pi)=\Fm_\nu$.
\end{Beweis}

\begin{Definition}
 Ein exponentenbewerteter Körper der Chrakteristik $=0$ heißt \highl{lokaler Körper} genau dann, wenn
 \begin{enumerate}
  \item Die Exponentialbewertung ist diskret,
  \item $K$ ist vollständig (bzgl. der induzierten Bewertung),
  \item der Restklassenkörper ist endlich.
 \end{enumerate}
\end{Definition}

\begin{Fakt}
 Lokale Körper sind lokalkompakt.
\end{Fakt}

\begin{Beweis}
 Wir zeigen, dass $\cO_\nu$ kompakt ist. Dann folgt die Aussage: $K=\bigcup_{n=0}^\infty \pi^{-n}\cO_\nu$ ($K$ ist lokalkompakt als Vereinigung abzählbarvieler kompakter Mengen).
 
 Sei dazu $(x_n)$ Folge aus $\cO_\nu$. Dann ex. Teilfolge, deren Restklassen in $\cO_\nu/\Fm_\nu$ gleich sind. Wegen $\Fm_\nu^n/\Fm_\nu^{n+1}\cong \cO_\nu/\Fm_\nu$ (Isomorphismus = Multiplikation mit $\pi^n$) lässt sich das iterieren. \folge Wir finden Teilfolge $(y_n)$ mit $y_n\equiv y_{n+1} \mod{\Fm_\nu^n}$. Das ist Fundamentalfolge, also konvergent. Damit erhalten wir Folgenkompaktheit und damit Kompaktheit (metrischer Raum).
\end{Beweis}

\begin{Bemerkung}
 Ist $K$ in der Bewertung $|\cdot|$ lokalkompakt, so ist $K$ vollständig (ÜA).
\end{Bemerkung}

\begin{Beispiel}
 Die $\IQ_p$ sind lokale Körper: $\IZ_p/p\IZ_p=\IF_p$.
\end{Beispiel}

\begin{Fakt}
 Sei $K$ lokalkompakt in der Bewertung $|\cdot|$, $L/K$ endliche Erweiterung, dann besitzt $L$ höchstens eine Bewertung, welche $|\cdot|$ fortsetzt.
\end{Fakt}

\begin{Beweis}
 $K$ ist vollständig:
 
 $\overline\ID(0,r)$ ist Kompakt für $r<<1$, also auch $\overline\ID(x,r)$, für alle $x\in K$. Ist $(x_n)$ CAUCHY-Folge, so liegen alle $x_n$ in $\overline{\ID}(x_{n_0},\eps)$ für $n\geq n_0$. Es gibt also konvergente Teilfolge. Eine CAUCHY-Folge mit konvergenter Teilfolge ist konvergent.
 
 Sei $|\cdot|_L$ Fortsetzung von $|\cdot|$ auf $L$. Wir konstruieren eine Art Norm auf $L$:
 Sei $\omega_1,\ldots,\omega_n$ $K$-Basis von $L$, für $x\in L$ gilt $x=\sum \alpha_i\omega_i$, $\alpha_i\in K$. Wir setzen $\|x\|:=\max_{i=1}^n(|\alpha_i|)$.
 Eigenschaften:
 \begin{enumerate}
  \item $\|x\|=0 \gdw x=0$,
  \item $\|x+y\|\leq \|x\|+\|y\|$,
  \item $\|\alpha x\|=|\alpha|\|x\|$
 \end{enumerate}
 $K^n\rightarrow L: (\alpha_1,\ldots,\alpha_n)\mapsto \sum_{j=1}^n\alpha_j\omega_j$ ist Isometrie, falls links die Norm $|(\alpha_1,\ldots,\alpha_n)|:=\max\{|\alpha_i|\}$ ist. Mit $K$ ist auch $K^n$ lokalkompakt. Also ist $S=\{ x\in L : \|x\|=1\}$ kompakt: $yS$ liegt in präkompakter Umgebung der $0$ für geeignete $y$. Wir vergleichen $|\cdot|_L$ und $\|\cdot\|$. $|x|_L\leq\sum|\alpha_i||\omega_i|_L\leq c\|x\|$ für ein $c>0$. Wir zeigen $\exists c'>0$, s.d. $\|x\|\leq c' |x|_L$.

 Indirekt: Sei $(x_n)$ Folge aus $S$ mit $|x_n|_L\rightarrow 0$. Dann existiert eine Teilfolge $(y_n)$ mit $y_n\rightarrow y\in S$ in $\|\cdot\|$, d.h. $\| y_n-y\|\rightarrow 0$. Es folgt 
 \[ |y|_L\leq |y_n-y|_L+|y_n|_L\leq c\| y_n-y\|+|y_n|_L\rightarrow 0.\]
 Somit $y=0$ \lightning. Also $\exists \eps>0 \forall x\in S: |x|_L\geq \eps$. D.h. $\|x\|=1 \folge |x|_L\geq \eps$ \folge $\|x\| \leq c' |x|_L$ für alle $x\in S$.
 Sei $x\in L\kreuz$, $x=\sum \alpha_i\omega_i$, $\|x\|=|\alpha_j|>0$ \folge $\|\alpha_j\inv x\|=1$.
 \[ \|x\|=|\alpha_j|\|\alpha_j\inv x\|\leq c'|\alpha_j| |\alpha_j\inv x|_L=c'|x|_L.\]
 
 Sind $|\cdot|_L$ und $|\cdot|_L'$ zwei Fortsetzungen von $|\cdot|$ auf $L$, so haben wir
 \[ |x|_L\leq c|x|_L' \text{ und } |x|_L'\leq c'|x|_L.\]
 Ersetze $x$ durch $x^N$, $N\rightarrow\infty$ \folge $|x|_L=|x|_L'$
\end{Beweis}
