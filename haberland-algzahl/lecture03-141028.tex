%!TEX root = Algebraische_Zahlentheorie.tex

\renewcommand{\lecdate}{28.10.14}

\subsection{Ganze algebraische Zahlen}

\begin{Definition}
 Sei $A$ ein Ring (kommutativ, mit 1, integer) und $L\supset A$ Körper. Ein Element $x\in L$ heißt \highl{ganz} in $A$ (kurz : ganz $/A$) genau dann, wenn ein \highl{unitäres Polynom}\footnote{d.h. $\exists a_1,\ldots,a_n\in A$, $n\geq 1$, s.d. $f(x)=x^n+a_1x^{n-1}+\ldots+a_0$} $f\in A[X]$ mit $f(x)=0$. 
\end{Definition}

\begin{Bemerkung}
 $x$ ganz $/A$ \folge $x$ algebraisch $/\QK(A)$.
\end{Bemerkung}

\begin{Beispiel}
 \begin{enumerate}
  \item $A=\IZ$, $L=\IQ$. $x\in\IZ$ ist ganz $/\IZ$. Wurzel von $T-x$. Sei $x\in\IQ$ ganz $/\IZ$, also
  \[x^n+a_1x^{n-1}+\ldots +a_n=0,\, a_j\in\IZ.\]
  Sei nun $x=\frac{r}{s}$, $r,s\in\IZ$, $s\neq 0$, $\ggT(r,s)=1$. Dann
  \[ r^n+a_1r^{n-1}s+\ldots+a_ns^n=0.\]
  $s$ teilt sicher die Summe $a_1r^{n-1}s+\ldots+a_ns^n$, also auch $r^n$. Sei $p\mid s$, $p$ prim \folge $p\mid r^n$\folge $p\mid r$. Ein Widerspruch. Also $s=\pm 1$ und damit $x\in\IZ$.
  \item $A=\IZ$, $L=\IQ(i)$. Sei $x=a+bi\in\IZ[i]$. Wir bestimmen $c,d\in\IZ$, s.d. $x^2+cx+d=0$ gilt.
  Dann $a^2-b^2+ac+d=0$ und $2ab+bc=0$. Also genügen $c=-2a$ und $d=a^2+b^2$.
  Sei $x\in\IQ(i)$ ganz $/\IZ$, also $x^n+a_1x^{n-1}+\ldots+a_0=0$, $a_j\in\IZ$. Weiter wie für $\IZ$.
  \item $A=\IZ$, $L=\IQ(\sqrt{-5})$ liefert $x$ ganz $/\IZ$ genau dann, wenn $x\in\IZ[\sqrt{-5}]$.
 \end{enumerate}
\end{Beispiel}

\begin{Fakt}
 Seien $L\supset A$ wie oben, dann ist $x\in L$ ganz $/A$ genau dann, wenn ein endlich erzeugter $A$-Modul $M\subset L$, $M\neq 0$ mit $xM\subset M$ existiert.
\end{Fakt}

\begin{Beweis}
 Sei $x$ ganz $/A$, also $x^n+a_1x^{n-1}+\ldots+a_n=0$, $a_j\in A$. Bilde $M=A+Ax+\ldots+ Ax^{n-1}\subset L$. Sei umgekehrt $M$ ein endlich erzeugter $A$-Modul, also $M=Av_1+\ldots+ Av_n$.
 Es folgt \begin{align*}
           xv_1&= a_{11}v_1+\ldots+ a_{1n}v_n\\
           & \hspace{1cm}\vdots\\
           xv_n&= a_{n1}v_1+\ldots+ a_{nn}v_n
          \end{align*}
also ist die Determinante
\[\det\begin{pmatrix}
       a_{11}-x & a_{12} & \hdots & a_{1n}\\
       a_{21} & a_{22}-x & \hdots & a_{1n}\\
       \vdots & \vdots & \ddots & \vdots\\
       a_{n1} & a_{n2} & \hdots & a_{nn}-x
      \end{pmatrix}=0.
\]
Das ist Ganzheitsgleichung für $x$.
\end{Beweis}

\begin{Definition}
 Sei $A\subset L$ wie oben, die Menge $B$ der $x\in L$, welche ganz $/A$ sind, heißt \highl{ganzer Abschluss} von $A$ in $L$.
 
 $A$ heißt \highl{ganzabgeschlossen} in $L$, falls $B=A$ ist. $A$ heißt ganzabgeschlossen, falls $A$ ganzabgeschlossen im QK ist.
\end{Definition}

\begin{Beispiel}
 \begin{enumerate}
  \item $\IZ$ ist ganzabg.
  \item $\IZ[i]$ ist der ganze algebraische Abschluss von $\IZ$ in $\IQ(i)$.
  \item $\IZ[\sqrt{5}]=\{a+b\sqrt{5} : a,b\in\IZ\}$ ist nicht ganzabg.
  
  Für $x=\frac{1+\sqrt{5}}{2}$ ist $x^2-x-1=0$, also $x$ ganz in $\IZ$, also erst recht ganz $/\IZ[\sqrt{5}]$.
 \end{enumerate}
\end{Beispiel}

\begin{Definition}
 $A\subset B$ Ringe (kommutativ, mit 1, integer). $B$ heißt \highl{ganz} $/A$ genau dann, wenn alle Elemente aus $B$ ganz $/A$ sind.
\end{Definition}

\begin{Fakt}
 Sei $A\subset L$ wie oben und $B$ der ganze Abschluss von $A$ in $L$. Dann ist $B$ ein Ring in $L$, der $A$ umfasst.
\end{Fakt}

\begin{Beweis}
 Seien $x,y\in B$, $M,N\subset L$ endlich erzeute $A$-Moduln mit $xM\subset M$ und $yN\subset N$ ($M\neq 0 \neq N$).
 Dann ist $MN\subset L$ ein endlich erzeugter $A$-Modul $\neq 0$ und $(x+y)MN\subset MN$, also $xyMN\subset MN$.
\end{Beweis}

\begin{Bemerkung}
 $M=Av_1+\ldots + Av_m$. $N=Aw_1+\ldots+Aw_n$
 
 \folge $MN=\sum Av_iw_j$
\end{Bemerkung}

\begin{Fakt}
 Seien $A\subset B\subset C$ Ringe, $B$ ganz $/A$, $C$ ganz $/B$, dann ist $C$ ganz $/A$.
\end{Fakt}

\begin{Beweis}
 Sei $x\in C$, $x^n+b_1x^{n-1}+\ldots+b_n=0$ Ganzheitsgleichung. $b_j\in B$. Sei $B_0=A[b_1,\ldots,b_n]$, das ist endlich erzeugte $A$-Algebra. Wir zeigen durch Induktion über $n$, dass $B_0$ sogar endlich ist (d.h. endlich erzeugt als $A$-Modul). $A[b_1]$ ist endlich erzeugter $A$-Modul nach vorletzten Fakt.
 Nach IV ist $A[b_1,\ldots,b_{n-1}]$ endlich erzeugter $A$-Modul. $b_n$ ist ganz $/A$, also erst recht ganz $/A[b_1,\ldots,b_{n-1}]$. Somit ist $B_0=A[b_1,\ldots,b_n]$ endlich erzeugt als $A[b_1,\ldots,b_{n-1}]$-Modul. Das war der Induktionsschritt.
 
 Nun zeigen wir $x$ ganz $/A$. Mit $B_0$ ist auch $B_0[x]$ endlich erzeugter $A$-Modul, da $x$ ganz $/B_0$ ist. Weiter gilt für $M=B_0[x]$: $xM\subset M$, also ist $x$ ganz $/A$.
\end{Beweis}

\begin{Definition}
 Sei $K$ algebraischer Zahlenkörper, d.h. endliche Erweiterung von $\IQ$. Der \highl{Ring der ganzen Zahlen} $\cO_K$ ist definiert als der ganze Abschluss von $\IZ$ in $K$:
 \[\cO_K=\{ x\in K: x \mbox{ ganz } /\IZ\} \]
\end{Definition}

\begin{Beispiel}
 Sei $K=\IQ(e^{2\pi i/n})$. Dann $\cO_K=\IZ[e^{2\pi i/n}].$
\end{Beispiel}

\begin{Bemerkung}
 \begin{itemize}
  \item Nach dieser Definition hat man lange gesucht.
  \item $\cO_K$ ist kommutativ, mit 1, integer, enthält $\IZ$ und ist ganzabgeschlossen. Der QK ist $K$:
  
  Sei $x\in K$ \folge $\exists \alpha_1,\ldots,\alpha_n\in\IQ$:  \[ x^n+\alpha_1x^{n-1}+\ldots + \alpha_n=0\]
  Dann existieren $a_0,\ldots,a_n\in \IZ$ mit $a_0\neq 0$, s.d. \[ a_0x^n+ a_1x^{n-1}+\ldots+a_n=0.\]
  
  Multipliziert man beide Seiten mit $a_0^{n-1}$ folgt $a_0x$ ganz $/\IZ$, also aus $\cO_K$. Mithin $x=\frac{a_0x}{a_0}$ mit $a_0x,a_0\in \cO_K$.
 \end{itemize}
\end{Bemerkung}

\begin{Beispiel}[Quadratische Zahlkörper]
 $[K:\IQ]=2$ \gdw $K=\IQ(\sqrt{d})$, $d\in \IZ$, $d\neq 0,1$ sqf (squarefree). Solche $K$ sind paarweise verschieden, denn ist $\IQ(\sqrt d)=\IQ(\sqrt e)$, folgt $\sqrt d = \alpha + \beta \sqrt{e}$ und $\sqrt e= \gamma + \delta \sqrt d$ mit $\alpha,\beta,\gamma,\delta\in\IQ$. Es folgt $d=\alpha^2+2\alpha\beta\sqrt e +\beta^2e $ \folge $\alpha\beta=0$. Ist $\beta=0$, so ist $\sqrt{d}=\alpha$ $\lightning$.
 Ist $\alpha=0$ folgt $\sqrt d=\beta\sqrt e$, also $d=\beta^2 e$.
\end{Beispiel}

Für $\IQ(\sqrt d)/\IQ$ ist $\Gal=\{ \Id,\sigma\}$. $\sigma(a+b\sqrt{d})=a-b\sqrt d$. $\N(x)=x\sigma(x)=a^2-db^2$. $\Tr(x)=x+\sigma(x)=2a$.
Wir zeigen $x\in \IQ(\sqrt{d})$ ganz $/\IZ$ (d.h. aus $\cO_K$) genau dann, wenn $\Tr(x),\N(x)\in\IZ$.

\begin{Beweis}
 Sei $x$ ganz $/\IZ$, d.h. $x$ ist Wurzel eines unitären $\IZ$-Polynoms $f\in \IZ[T]$. Für $x\in \IQ$ ist dann $x\in\IZ$ (siehe oben). Sei $x\notin\IQ$, $p(T)=\Irr(T,x,\IQ)=T^2-\Tr(x)T+\N(x).$ Dieses Polynom teilt $f$ in $\IQ(T)$. Nach GAUSS' Lemma, hat $p$ Koeffizienten in $\IZ$. Die andere Implikation ist trivial. 
\end{Beweis}

Somit $a+b\sqrt d \in \cO_K$ genau dann, wenn $2a\in\IZ$ und $a^2-db^2\in\IZ$. Also $a=\frac{r}{2}$, $r\in\IZ$, $\frac{r^2}{4}-db^2\in\IZ$. 

$b=\frac{s}{2}, s\in\IZ$, $\frac{r^2-ds^2}{4}\in\IZ$ genau dann, wenn $r^2\equiv ds^2 \mod{4}$. Ist $d\equiv 1 \mod{4}$, so gilt die Kongruenz genau dann, wenn $r\equiv s \mod{2}$.
Ist $d\equiv 2,3 \mod{4}$, so gilt die Kongruenz genau dann, wenn $r\equiv s \equiv 0 \mod{2}$.
Damit haben wir folgenden Fakt:

\begin{Fakt}
Sei $K=\IQ(\sqrt{d})$, $d\in \IZ$, $d\neq 0,1$ sqf. Dann ist
\[\cO_K=\begin{cases}\{ a+b\sqrt{d}, a,b\in\IZ\} & d\equiv 2,3 \mod{4}\\ \{ \frac{a+b\sqrt d}{2}, a,b\in\IZ, a\equiv b\mod{2}\}=\{ a+b\frac{1+\sqrt{d}}{2}\} & d\equiv 1 \mod{4}\end{cases}.\]
\end{Fakt}

\begin{Fakt}
 Faktorielle Ringe sind ganzabgeschlossen.
\end{Fakt}

\begin{Beweis}
 Sei $K$ der QK von $A$, $A$ faktoriell. Sei $x=\frac{r}{s}\in K$, $\ggT(r,s)=1$, $r,s\in A$.
 
 $x$ ganz $/A$: $\left(\frac{r}{s}\right)^n+a_1\left(\frac{r}{s}\right)^{n-1}+\ldots+a_n=0$, $a_j\in A$.
 \[r^n + a_1 r^{n-1} s+ \ldots + a_ns^n=0 \]
 
 \folge $s\mid r^n$ \folge $s\in A\kreuz$ \folge $x\in A$.
\end{Beweis}










