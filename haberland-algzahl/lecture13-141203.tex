\renewcommand{\lecdate}{03.12.2014}

D.h. im Bild von $\N_K^L$ liegt $-\eps_L$, also hat das Bild Index $1$ oder $2$.

Es gibt keine Einheit in $E_K$ mit Norm $-1$ in $E_L$:
\begin{align*}
 \N_L^K(a+b\omega^2+\omega(c+d\omega^2))&=((a+b\omega^2)+\omega(c+d\omega^2))\\
 &= ((a+b\omega^2)-\omega(c+d\omega^2))\\
 &=(a+b\omega^2)^2-\omega^2(c+d\omega^2)^2\\
 &=a^2+2ab\omega^2+2b^2-c^2\omega^2-4cd-2d^2\omega^2\\
 &=(a^2+2b^2-4cd)+\omega^2(2ab-c^2-2d^2)\\
 &\overset{!}{=}-1
\end{align*}
\begin{align*}
 a^2+2b^2-4cd&=-1 && \folge a \text{ ungerade}\\
 2ab-c^2-2d^2&=0 && \folge c \text{ gerade}
\end{align*}

Also $a^2+2b^2-4cd\equiv 1+0,2 \mod{8}\not\equiv -1\mod{8}$

Es folgt: $\N_L^K(E_K)=\langle -\eps_L\rangle$. Sei $\eps_K:=1+\omega$.
\begin{align*}
\N\frac{1+\omega}{1-\omega}&=\N\frac{(1+\omega)^2}{1-\omega^2}\\ 
&=\frac{1}{(1-\omega^2)^2}\N(1+\omega)^2\\
&=\frac{1}{(1-\sqrt{2})^2}(1-\sqrt{2})^2=1 
\end{align*}

\begin{align*}
 \frac{1+\omega}{1-\omega}&=\frac{(1+\omega)^2}{1-\sqrt 2}\\
 &=\frac{(1+\omega)^2(1+\sqrt{2})}{-1}\\
 &=-(1+2\omega+\omega^2+\omega^2+2\omega^3+2)\\
 &=-(3+2\omega+2\omega^2+2\omega^3)=\eta_K\\
\end{align*}

Dann ist $\eps_K=1+\omega$, $\eta_K=3+2\omega+2\omega^2+2\omega^3$ Fundamentalsystem von Einheiten: Dazu genügt es zu zeigen, dass $\eta_K$ den Kern von $N_L^K$ erzeugt (bis auf Vorzeichen): Sei $\theta\in E_K$ beliebig, dann ist $\N_L^K=(-\eps_L)^m$ \folge $\N_L^K(\theta)=\N_L^K(\eps_K^m)$ \folge $\N(\theta\eps_K^{-m})=1$ \folge $\theta\eps_K^{-m}=\eta_K^n$ \folge $\theta=\eps_K^m\eta_K^n$.

Sei $\eta\in E_K$ mit $\N_L^K(\eta)=1$. Man findet Potenz $\eta_K$ und Vorzeichen, s.d. $\pm\eta\eta_K^m$ zwischen $1$ und $\eta_K$ liegt. Sei dies jetzt das neue $\eta:=\pm\eta\eta_K^m$.

$\eta=\alpha+\omega\beta$, $\alpha=a+b\sqrt{2}$, $\beta=c+d\sqrt 2$. Also \[1\leq \alpha+\beta\omega<\eta_K \folge \frac{1}{\eta_K}<\frac{1}{\alpha+\beta\omega}\leq 1.\]
Nun ist $\frac{1}{\alpha+\beta\omega}=\alpha-\beta\omega$, wegen $\N\eta=1$. \folge $\frac{1}{\eta_K}<\alpha-\beta\omega\leq 1$.
\begin{equation}
 1+\frac{1}{\eta_K}<2\alpha<1+\eta_K
\end{equation}

Analog $-1\leq \beta\omega-\alpha<-\frac{1}{\eta_K}$ \folge
\begin{equation}
 0\leq 2\beta\omega < \eta_K-\frac{1}{\eta_K}
\end{equation}

Nun ist $1=\N_\IQ^K(\eta)=\eta$, $\sigma^2\eta=|\sigma\eta|^2$

$\sigma: \omega\mapsto i\omega$, $\sigma^2: \omega\mapsto -\omega$. $\sigma^2\eta=\frac{1}{\eta}\omega$, wegen $\eta\sigma^2\eta=\N_L^K(\eta)=1$. \folge $|\sigma\eta|^2=1$

$\sigma\eta=(a-b\sqrt{2})+i\omega(c-d\sqrt{2})$, $|\sigma\eta|^2=(a-b\sqrt{2})^2+\sqrt 2(c-d\sqrt 2)^2=1$ \folge $-1\leq a-b\sqrt{2}\leq 1$ (3), \folge $-\frac{1}{\sqrt 2} \leq c-d\sqrt{2}\leq\frac{1}{\sqrt 2}$ (4)

(1): $\frac{1}{2}(1+\frac{1}{\eta_K})<a+b\sqrt 2< \frac{1}{2}(1+\eta_K)$

(2): $0\leq c+d\sqrt 2 < \frac{1}{2\omega} (\eta_K-\frac{1}{\eta_K})$.

(1)+(3): $\frac{1}{2} (1+\frac{1}{\eta_K})-1<2a<\frac{1}{2}(1+\eta_K)+1$ 

\folge $a=1$, $b=0$, $c-d=0$ (via Taschenrechner)

\subsection{Primideale in Erweiterungen}
Zunerst etwas kommutative Algebra:

\begin{Definition}
 Sei $A$ integer Ring, $S\subset A$ heißt \highl{multiplikativ}, falls
 \begin{enumerate}
  \item $0\neq S$, $1\in S$
  \item $x,y\in S$ \folge $xy\in S$
 \end{enumerate}
\end{Definition}

\begin{Beispiel}
 Es gibt Kleinste: $\{1\}$ und größte: $A\oN$. Wichtigstes Beispiel: $S=A\setminus\Fp$, für $\Fp$ prim.
\end{Beispiel}

\begin{Definition}
 Die \highl{Lokalisierung} von $A$ nach $S$ ist 
 \[ S\inv A:=\{a/s : a\in A, s\in S\}\subset K=QK(A).\]
 
 Speziell: $S=A\setminus \Fp$, $\Fp$ prim, dann schreibt man $A_\Fp$ und nennt $A_\Fp$ die Lokalisierung von $A$ an der Stelle $\Fp$.
\end{Definition}

\begin{Bemerkung}
 $S\inv A$ ist wieder ein Ring mit $1$, integer. Man hat kanonischen Ringhom. $A\rightarrow S\inv A$: $a\mapsto a/1$.
\end{Bemerkung}

\begin{Beispiel}
 \begin{enumerate}
  \item $\Fp=(0)=\{0\}$, also $S=A\oN$, dann ist $A_{(0)}=K=QK(A)$.
  \item $A=\IZ$, $\Fp=(p)$, $p\in\IP$.
  \[\IZ(p)=\{\tfrac{a}{b}\in\IQ : \ggT(a,b)=1, b>0, p\nmid b \} \]
 \end{enumerate}
\end{Beispiel}

\begin{Definition}
 Ein \highl{lokaler Ring} ist ein kommutativer Ring mit $1$, welcher genau ein maximales Ideal besitzt.
\end{Definition}

\begin{Fakt}
 Sei $\Fp\subset A$ Primideal. Dann ist $A_\Fp$ lokaler Ring. Sein maximales Ideal ist $m_\Fp=\Fp A_\Fp$. Der Homomorphismus $A\rightarrow A_\Fp$ induziert injektiven Homomorphismus
 $A/\Fp \rightarrow A_\Fp/m_\Fp$. Danei ist $A_\Fp/m_\Fp$ der Quotientenkörper von $A/\Fp$. Ist insbesondere $\Fp$ maximal, so ist $A_\Fp/m_\Fp=A/\Fp$.
\end{Fakt}

\begin{Beweis}
 Zeuerst: $m_\Fp\cap A=\Fp$: Die Inklusion $\Fp\subset m_\Fp\cap A$ ist trivial. Sei $x\in A\cap m_\Fp$, also $x=a/s$, $a\in\Fp$, $s\in A\setminus\Fp$. Somit ist $sx\in\Fp$, aber $s\notin\Fp$ \folge $x\in\Fp$.
 
 Sei $y\in A_\Fp\setminus m_\Fp$, also $y=x/s$, $x\in A\setminus\Fp$, $s\in A\setminus\Fp$. Also ist $y$ Einheit in $A_\Fp$ und somit $m_\Fp$. Weiter ist jedes Ideal $\neq (1)$ in $m_\Fp$ enthalten. Also ist $A_\Fp$ lokaler Ring.
 
 Der Kern von $A\rightarrow A_\Fp\rightarrow A_\Fp/m_\Fp$ ist $A\cap m_\Fp=\Fp$, also ist $A/\Fp\rightarrow A_\Fp/m_\Fp$ injektiv (Homomorphiesatz). Jedes Element aus $A_\Fp/m_\Fp$ hat die Form als $+m_\Fp=(a+\m_\Fp)(s+m_\Fp)\inv$, also ist dies der QK von $A/\Fp$.
\end{Beweis}

\begin{Fakt}
 Sei $A$ ein DED-Ring, $S\subset A$ multiplikativ. Dann ist auch $S\inv A$ ein DED-Ring. Die Abbildung
 \[ \Id(A)\rightarrow \Id(S\inv A): \Fa\mapsto S\inv \Fa\]
 verursacht einen Isomorphismus zwischen $\Id(S\inv A)$ und der Untergruppe von $\Id(A)$, welche von den Primidealen $\Fp\subset A$ erzeugt wird, die $S$ nicht schneiden ($\Fp\cap S=\emptyset$).
\end{Fakt}

\begin{Beweis}
 \begin{enumerate}
  \item $S\inv A$ ist NOETHERsch
  
  Sei dazu $\Fa_s$ ein Ideal in $S\inv A$ und $\Fa:=\Fa_s\cap A$. Dann gilt $\Fa_s=S\inv\Fa$: $S\inv\Fa\subset\Fa_s$ ist trivial, sei also $x\in\Fa_s$, $x=y/s$, $y\in A$, $s\in S$. Dann folgt $y=xs\in\Fa_s\cap A=\Fa$ \folge $x\in S\inv\Fa$.
  
  Ist $\Fa=(x_1,\ldots,x_n)\subset A$ Ideal, so ist $S\inv \Fa=(x_1,\ldots,x_n)\subset S\inv A$.
  \item $S\inv A$ ist ganzabgeschlossen: Sei $K$ der QK von $A$, $x\in K$ ganz über $S\inv A$, d.h.
  \[ x^n+\frac{a_1}{s_1}x^{n-1}+\ldots+\frac{a_n}{s_n}=0, a_j\in A, s_j\in S.\]
  Sei $s=s_1\cdot\ldots\cdot s_n$, Multiplikation mit $s^n$ zeigt: $sx$ ganz über $A$, also aus $A$. Mithin $x\in S\inv A$.
  \item Jedes Primideal $\neq (0)$ ist maximal. 
  
  Sei $\Fp_S\subset S\inv A$ prim und $\neq(0)$. Dann ist auch $\Fp:=\Fp_S\cap A$ Primideal (wh. $A/\Fp \rightarrow S\inv A/S\inv\Fp$ und $S\inv\Fp=\Fp_S$). Also ist $\Fp$ maximal in $A$, denn $\Fp\neq (0)$. Sei 
  $\Fp_S\subset m_S\subset S\inv A$, dann folgt durch Schnitt mit $A$: $\Fp\subset m_S\cap A\subset A$.
  Ist $m_S\cap A=A$, so folgt $1\in m_S$ \folge $m_S=S\inv A$ \lightning (maximale Ideale sind nie der ganze Ring nach Def.).
  Also $m_S\cap A=\Fp$ und es folgt
  \[S\inv(m_S\cap A)=m_S=S\inv\Fp=\Fp_S. \]
  Somit ist $\Fp$ maximal.
  \item $\Id(A)\rightarrow\Id(S\inv A): \Fa\mapsto S\inv\Fa$ ist surjektiv: $\Fa_S=S\inv(\Fa_S\cap A)$. Wir hatten das für ganze Ideale gesehen. Es veralgemeinert sich sofort auf gebrochene Ideale wg. $S\inv(\Fa\Fb)=S\inv\Fa\cdot S\inv\Fb$ (Übungsaufgabe).
  
  Der Kern besteht aus den gebrochenene Idealen $\Fa\subset A$, für welche $S\inv\Fa(=\Fa\cdot S\inv A)=S\inv A$ gilt. Das ist genau dann der Fall, wenn $\exists a\in\Fa, s\in S: a/s=1$ \gdw $a=s$ \gdw $\Fa\cap S\neq \emptyset$.
 \end{enumerate}
\end{Beweis}

\begin{Bemerkung}
 \begin{enumerate}
  \item Lokalisierung eliminiert Ideale.
  \item Die maximalen Ideale von $S\inv A$ sind in Bijektion zu denjenigen $A$, welche $S$ nicht schneiden via \[m_S \mapsto m_S\cap A \] und umgekehrt \[m\mapsto S\inv m. \]
 \end{enumerate}

\end{Bemerkung}

