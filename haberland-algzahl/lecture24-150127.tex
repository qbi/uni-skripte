\renewcommand{\lecdate}{27.01.2015}

\begin{Bemerkung}
 \begin{enumerate}
  \item Das eigentliche Problem ist die Existenz der Fortsetzung.
  \item Die Fortsetzung ist äquivalent zu $\|\cdot\|$, falls sie existiert.
  \item $L$ ist in der Fortsetzung lokalkompakt, also vollständig.
 \end{enumerate}
\end{Bemerkung}

\begin{Fakt}
 Sei $L/K$ endlich, $L$ bewertet, $\hat K$ sei lokalkompakt. Dann ist  $\hat L=L\cdot \hat K$\footnote{Dieses Produkt ist nur definiert, falls ein Körper existiert, der beide Faktoren enthält (in diesem Fall leistet dies $\hat L$) und es ist der kleinste Körper, welcher beide Körper enthält.} und es gilt $[\hat L:\hat K]\leq [L:K]$.
\end{Fakt}

\begin{Beweis}
 $L\hat K\subset \hat L$ und $L\hat K/\hat K$ ist endliche Erweiterung. Nach obigem Fakt ist $L\hat K$ vollständig, also abgeschlossen in $\hat L$. Nun ist $L\subset L\hat K$ dicht in $\hat L$, also $L\hat K=\hat L$.
 $[L\hat K:\hat K]\leq [L:K]$ ist Algebra I.
\end{Beweis}

\begin{Fakt}
 Jede endliche Erweiterung $K/\IQ_p$ besitzt Fortsetzung von $|\cdot|_p$ auf $K$.
\end{Fakt}

\begin{Beweis}
 $\IZ_p$ ist dBR, einziges maximales Ideal ist $(p)=p\IZ_p$, alle anderen Ideale sind $(p^m)$, $m\geq 2$ und $(0)$. Somit ist $\IZ_p$ NOETHERsch, ganzabgeschlossen (da faktoriell), jedes Primideal $\neq 0$ ist maximal. Also $\IZ_p$ DED-Ring. Sei $\cO$ der ganze Abschluss von $\IZ_p$ in $K$, dann ist auch $\cO$ DED-Ring, $\cO$ hat nur endlich viele Primideale. Also ist $\cO$ ein HI-Ring, somit 
 \[ p\cO = (\pi_1^{e_1})\cdot\ldots\cdot(\pi_m^{e_m}),\hspace{.5cm} e_j\geq 1\]
 mit maximalen Idealen $(\pi_1),\ldots,(\pi_m)$. Nach chinesischem Restsatz induzieren sie nichtäquivalente Bewertungsfortsetzungen auf $K$ via $\nu_i(x)=\frac{1}{e_i}\ord_{\pi_i}(x)$.
 Nach Eindeutigkeitsaussage ist $m=1$, also $p\cO=\pi^e\cO$ \folge $p=u \pi^e$ mit $u\in \cO\kreuz$. 
 $\nu_\pi(x):=\frac{1}{e}\ord_\pi(x)$ induziert die Bewertungsfortsetzung
 \[ |x|_\pi = p^{-\nu_\pi(x)/e}.\]
 Tatsächlich: $x\in\IQ_p\kreuz$ \folge $x=vp^m$, $v\in\IZ_p\kreuz$.
 \[ |x|_\pi=p^{-\nu_pi(x)/e}=p^{-em/e}=p^{-m}=|x|_p\]
\end{Beweis}

\begin{Folgerung}
 $|x|_\pi=|\N_{\IQ_p}^K(x)|_p^{1/n}$, $n=[K:\IQ_p]$
\end{Folgerung}

\begin{Beweis}
 Sei $L/\IQ_p$ normale Hülle von $K/\IQ_p$, $G=\Gal(L/\IQ_p)$. Mit $x\mapsto |x|$ ist auch $x\mapsto |\sigma x|$ für $\sigma\in G$ eine Bewertung von $L$, also $|\sigma x|=|x|$. Es folgt $|\N_{\IQ_p}^L(x)|=\prod_{\sigma\in G} |\sigma x|=|x|^{[L:\IQ_p]}$. Ist $x\in K$, so folgt 
 \[ \prod_{\sigma\in G} \sigma x = (\N_{\IQ_p}^K(x))^{[L:K]}.\]
\end{Beweis}

\begin{Bemerkung}
 \begin{enumerate}
  \item Es ist nicht offensichtlich, dass $x\mapsto |\N_{\IQ_p}^K(x)|^{1/n}$ die Dreiecksungleichung erfüllt.
  \item $|\cdot|_p$ auf $\IQ_p$ setzt sich eindeutig auf $\overline\IQ_p$ fort.
 \end{enumerate}
\end{Bemerkung}

\begin{Fakt}[KRASNERs Lemma]
 Sei $\alpha\in\overline\IQ_p$ fixiert. Ist dann $\beta\in\overline{\IQ_p}$ genügend nahe $\alpha$ (d.h. $|\alpha-\beta|$ klein), so folgt $\IQ_p(\alpha)\subset \IQ_p(\beta)$\footnote{Davon gibt es natürlich kein Äquivalent in $\IQ$. \glqq Wieder ein kleines Wunder der $p$-adik.\grqq}.
\end{Fakt}

\begin{Beweis}
 Seien $\alpha=\alpha_1,\ldots,\alpha_n$ die Konjugierten von $\alpha$, also die Wurzeln des irreduziblen Polynoms von $\alpha$ über $\IQ_p$. Es gelte $|\alpha-\beta|<|\alpha-\alpha_i|$, $i=2,\ldots,n$.
 Sei $K$ der Zerfällungskörper von $\Irr(T,\alpha,\IQ_p)$, d.h. $K=\IQ_p(\alpha_1,\ldots,\alpha_n)$. Wir betrachten $K(\beta)/\IQ_p(\beta)$, das ist Gal-Erweiterung. Sei $\sigma\in \Gal(K(\beta)/\IQ_p(\beta))$, dann gilt
 \[ |\sigma\alpha-\sigma\beta|=|\sigma\alpha-\beta|=|\alpha-\beta|,\]
 also $|\sigma\alpha-\alpha|\leq \max\{|\sigma\alpha-\beta|, |\beta-\alpha|\}=|\alpha-\beta|<|\alpha-\alpha_i|$. Es folgt $\sigma\alpha=\alpha$ ($\sigma\alpha=\alpha_j$), d.h. $\alpha\in\IQ_p(\beta)\folge \IQ_p(\alpha)\subset\IQ_p(\beta)$.
\end{Beweis}

\begin{Folgerung}
 Sind $\alpha,\beta\in\overline\IQ_p$ genügend nahe, so gilt \[ \IQ_p(\alpha)=\IQ_p(\beta).\]
\end{Folgerung}

\begin{Satz}[\glqq Satz 3\grqq]
 Jede endliche Erweiterung von $\IQ_p$ ist der Form $K_\Fp=K\cdot \IQ_p$ für einen geeigneten ZL $K$ und eine $\Fp\mid p$ von $\cO_K$.
\end{Satz}

\begin{Beweis}
 Sei $E$ endliche Erweiterung von $\IQ_p$, $E=\IQ_p(\alpha)$ (Satz vom primitiven Element). $f=\Irr(X,\alpha,\IQ_p)=$ unitäres Polynom über $\IQ_p$. Wir wählen $g\in \IQ[X]$ unitär vom selben Grad wie $f$, so dass die Koeffizienten von $f$ und $g$ nahe sind bzgl. $|\cdot|_p$. Wir zeigen, dass dann auch die Wurzeln von $f$ und $g$ nahe sind. D.h. Stetigkeit der Wurzeln, als Funktion der Koeffizienten.
 
 Zuerst: $disc(f)\neq 0$, also auch $disc(g)\neq 0$. Somit hat auch $g$ keine mehrfachen Wurzeln.
 Seien $\alpha=\alpha_1,\alpha_2\ldots,\alpha_n$ die Wurzeln von $f$ und $\eta>0$ so, dass $|\alpha_i-\alpha_j|\geq \eta$ für alle $i\neq j$.
 
 Seien $\beta_1,\ldots,\beta_n$ die Wurzeln von $g$. Aus $g(\beta_i)=0$ folgt  \[ |\beta_i|^n\leq C\max(1,|\beta_i|,\ldots,|\beta_i|^{n-1}).\]
 Es gilt also $|\beta_i|\leq C$ und damit $|\beta_i-\beta_j|\leq C$. Es gilt $\prod_{i\neq j} |\beta_i-\beta_j|^2$ ist nahe $\prod_{i\neq j}|\alpha_i-\alpha_j|^2=C_1$. Also $\exists \eta'>0$, s.d. $|\beta_i-\beta_j|\geq \eta'$ für $i\neq j$.
 
 Nun ist $|g(\alpha_1)|=|g(\alpha_1)-f(\alpha_1)|$ klein, also ist auch $\prod_{i=1}^n|\alpha_1-\beta_i|$ klein. Alle Faktoren sind beschränkt, also muss wenigstens ein Faktor klein sein, say $|\alpha_1-\beta_1|$ klein. Da die $\beta_i$ Mindestabstand haben, ist nur $|\alpha_1-\beta_1|$ klein.
 
 KRASNERs Lemma zeigt: $\IQ_p(\alpha_1)=\IQ_p(\beta_1)$. Wir fanden also $K/\IQ$ s.d. $K\IQ_p=E$ für vorgelegtes $E/IQ_p$. Sei $\cO_K$ der Ganzheitsring in $K$. $\cO_E=\{ x\in E : |x|\leq 1\}$. Dann ist $\cO_K\subset\cO_E$: 
 \begin{align*}
  & x^n+a_1x^{n-1}+\ldots+a_n=0, a_j\in\IZ\\
  \folge & |x|^n\leq \max(1,|x|,\ldots,|x|^{n-1})\\
  \folge & |x| \leq 1
 \end{align*}

 Sei $\Fm_E=\{ x\in \cO_E : |x|<1\}$. Das ist maximales Ideal in $\cO_E$. $\Fp:=\cO_K\cap \Fm_E$ ist Primideal und $\neq (0)$, wg. $p\in\Fp$
 Für $m\in\IZ$ gilt $|m|_\Fp<1$ \gdw $|m|_p<1$, also auch auf $\IQ$.
 \folge $|\cdot|_\Fp$ und $|\cdot|_p$ stimmen auf $\IQ$ überein \folge auf $\IQ_p$ \folge $K_\Fp=E$.
 
\end{Beweis}
