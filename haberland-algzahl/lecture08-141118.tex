%!TEX root = Algebraische_Zahlentheorie.tex

\renewcommand{\lecdate}{18.11.2014}

\begin{Fakt}
 Sei $K$ ein algebraischer ZK und $\Fa,\Fb$ Ideale in $\cO_K$ $\neq (0)$. Dann gilt $\IN(ab)=\IN(a)\IN(b)$.
\end{Fakt}

\begin{Beweis}
 g.z.z. $\IN(\Fp^e)=(\IN\Fp)^e$
 
 Wir betrachten $\cO_K\supset \Fp\supset\Fp^2\supset\ldots\supset \Fp^e$. Sicher sind dabei alle Inklusionen echt.
 
 $\cO_K/\Fp$ ist endlicher Körper, $\Fp^i/\Fp^{i+1}$ ist VR über $O_k/\Fp$. Wir zeigen er hat Dimension $1$. Sei dazu $x\in\Fp^i\setminus\Fp^{i+1}$, $b:=(x)+\Fp^{i+1}$ ist Ideal in $\cO_K$. Es gilt $\Fp^i\supset b\supsetneq \Fp^{i+1}$. Multiplikation mit $\Fp^{-i}$ gibt $\cO_K\supset b\Fp^{-i}\supsetneq\Fp$ maximal \folge $\cO_K=b\Fp^{-i}$ \folge $b=\Fp^i$\folge $x \mod{\Fp^i}$ ist Basis.
 
\end{Beweis}

\begin{Folgerung}
 $\IN$ setzt sich zu HM $\IN:\Id_K\rightarrow \IR_+^*$ fort, durch \[ \Fa=\prod\Fp^{\nu_\Fp} \mapsto \IN\Fa=\prod (\IN\Fp)^{\nu_\Fp}\]
\end{Folgerung}

\begin{Bemerkung}
 In \S 1.4 hatten wir Diskriminanten für Ideale definiert: Ist $\alpha_1,\ldots,\alpha_n$ eine $\IZ$-Basis von $\Fa$ ($n=[K:\IQ]$), so ist
 \[d(\Fa)=\det\nolimits^2(\sigma_i\alpha_j)=\det\Tr(\alpha_i\alpha_j). \]
 $d$ ist unabhängig von der Basiswahl. Für $\Fa\supset\Fb$ gilt $d(\Fb)=(\Fa:\Fb)^2d(\Fa)$
\end{Bemerkung}

\begin{Fakt}
 Sei $\alpha\in \cO_K$, $\alpha\neq 0$, dann ist \[ \IN((\alpha))=|\N_\IQ^K(\alpha)|.\]
\end{Fakt}

\begin{Beweis}
 $d((\alpha))=(\cO_K:(\alpha))^2d_K$. Ist $\omega_1,\ldots,\omega_n\in \cO_K$ Ganzheitsbasis von $\cO_K$, so ist $\alpha \omega_1,\ldots,\alpha \omega_n$ $\IZ$-Basis von $(\alpha)=\alpha \cO_K$. Somit \[d((\alpha))=\det\nolimits^2(\sigma_i(\alpha\omega_j))=\det\nolimits^2(\sigma_i(\alpha)\sigma_i(w_j))=\prod_i(\sigma_i\alpha)^2\det\nolimits^2(\sigma_i \omega_j)=(\N_\IQ^K\alpha)^2d_k.\]
 \end{Beweis}
 
 \begin{Satz}[\glqq Satz 4\grqq: Endlichkeit der Klassengruppe]
  Für jeden alg. ZK $K$ ist die Idealklassengruppe $\Cl_K$ endlich.
 \end{Satz}
 
 \begin{Beweis}
  Sei $n=[K:\IQ]$, $\omega_1,\ldots,\omega_n$ Ganzheitsbasis. Sei weiter $\Fa\subset \cO_K$ Ideal.
  \[ S:= \left\lbrace \sum_{i=1}^n a_i\omega_i : 0\leq a_i\leq (\IN \Fa)^{1/n}+1, a_i\in\IZ\right\rbrace\]
  
  Dann ist $\card S > \IN\Fa:$
  $\card S\geq ([(\IN\Fa)^{1/n}+1]+1)^n> \IN\Fa$, wg. $[x+1]>x$. Nach DIRICHLET-Schubfachschluss ex. also $\alpha,\beta\in S$, $\alpha\neq \beta$ mit $\alpha-\beta=\xi\in\Fa$. Das heißt $(\xi)=\Fa\cdot\Fb$ mit ganzem Ideal $\Fb$. Es gilt $|\N_{\IQ}^K(\xi)|=\prod_{\sigma}|c_1\sigma\omega_1+\ldots+c_n\sigma\omega_n|$
  mit $c_i\in\IZ, 0\leq |c_i|\leq (\IN\Fa)^{1/n}+1$. Man sieht $\exists C>0$, s.d. $|\N(\xi)|\leq C\IN(\Fa)$, daher hängt $C$ nur von $\omega_1,\ldots,\omega_n$ ab, nicht von $\Fa$. Wegen $(\xi)=\Fa\Fb$ folgt: $\IN\Fb\leq C$, $\Fb$ liegt in der selben Idealklasse wie $\Fa\inv$.
  
  Wir haben gezeigt: In jeder Idealklasse gibt es ganze Ideale mit Absolutnorm $\leq C$.
  Nun zeigen wir: Es gibt in $\cO_K$ nur endlich viele ganze Ideale mit Absolutnorm $\leq C$. Es g.z.z., dass es nur endlich viele solcher Primideale gibt. $\IN\Fp$ ist Potenz von $p$ (Denn $p$ = Charakteristik von $\cO_K/\Fp$). 
  
  Es gibt nur endlich viele Primzahlen $\leq C$.
  Zu jeder Primzahl $p$ ex. nur endlich viele Primideale $\Fp$ mit $\IN\Fp=p^e$: $\Char(\cO_K/\Fp)=p\folge p=0$ in $\cO_K/\Fp$ \folge $p\in\Fp$\folge $\Fp\mid (p)$ wie jedes Ideal hat $(p)=p\cO_K$ nur endlich viele Primteiler.
  \end{Beweis}
  
  \subsection{MINKOWSKI-Theorie}
   
   Sei $K$ algebraischer ZK, $S=\Hom(K,\IC)$. $\# S=[K:\IQ]$, $\sigma: K\rightarrow\IR$ heißt \highl{reelle Einlagerung}, sonst $\sigma(K)\not\subset \IR$ \highl[komplexe Einlagerung]{komplexe}. Letztere treten in Paaren $\sigma,\overline{\sigma}$ auf.
   Sei $r=r(K)$ die Anzahl der reellen, $2s=2s(K)$ die Anzahl der komplexen Einlagerungen. Es gilt $r+2s=n$.
   
   \begin{Definition}
    Der \highl{MINKOWSKI-Raum} $K_{\IR}$ ist definiert als der $\IR$-Vektorraum $\IR^r\oplus \IC^s$. Die \emph{kanonische Einlagerung} $j:K\rightarrow K_{\IR}$ ist definiert als
    \[ j(\alpha)=(\sigma_1\alpha,\ldots,\sigma_r\alpha,\tau_1\alpha,\ldots,\tau_s\alpha) \]
    mit $\sigma_1,\ldots,\sigma_r$ die reellen $\sigma\in S$, die $\tau_j$ jeweils Repräsentanten der Paare komplex konjugierter Einlagerungen.
   \end{Definition}

   \begin{Bemerkung}
   Nicht sehr kanonisch\footnote{\glqq als alternativlos anzunehmen\grqq}: $r!s!2^s$ Auswahlmöglichkeiten. Kanonisch: $K_\IR = K\otimes_{\IQ} \IR$, $j(\alpha)=\alpha\otimes 1$ (Tensorprodukt). 
   \end{Bemerkung}
   
   \begin{Beispiel}
    \begin{enumerate}
     \item $K=\IQ(\sqrt d), d>1$ sqf.
     
     $r=2,s=0$, $j:K\rightarrow\IR^2$: $\alpha+\beta\sqrt d \mapsto (\alpha+\beta\sqrt d, \alpha-\beta\sqrt d)$
     \item $K=\IQ(\sqrt d)$, $d<0$ sqf.
     
     $r=0,s=1$, $j:K\rightarrow\IC$: $\alpha+\beta\sqrt d \mapsto \alpha+\beta\sqrt d$
     \item $K=\IQ(\zeta_p)$, $p>2$ prim, $\zeta_p=e^{2\pi i/p}$
     
     $r=0$, $s=\frac{p-1}{2}$, $j: \zeta_p \mapsto (\zeta_p^a)_{a\in A}$ mit $A = \{ a_1,\ldots,a_{(p-1)/2}\}\subset (\IZ/p\IZ)\kreuz$ jeweils eines aus $\{a,-a \}$
    \end{enumerate}
   \end{Beispiel}
   
   \begin{Definition}[Skalarprodukt für $K_\IR$]
    \[ \langle x,y\rangle:= \sum_{i=1}^r x_{\sigma_i}y_{\sigma_i} + \sum_{j=1}^s(x_{\tau_j}\overline{y}_{\tau_j}+\overline{x}_{\tau_j}y_{\tau_j}).\]
    \highl[Skalarprodukt für $K_\IR$]{}
   \end{Definition}

   \begin{Bemerkung}
    \begin{enumerate}
     \item $\IC$ ist reeller Vektorraum der Dimension $2$, $\langle u,v\rangle=u\overline v+ \overline uv=2\R(u\overline v)$ ist $\IR$-Skalarprodukt.
     \item Für $\alpha,\beta\in K$ gilt
     \[ \langle j(\alpha),j(\beta)\rangle=\sum_{\sigma\in\Hom(K,\IC)}\sigma(\alpha)\overline{\sigma(\beta)}.\]
     \item ON-Basis von $K_\IR$ ist 
     \begin{align*}
      e_i&=(0,\ldots,0,1,0,\ldots,0) && 1\leq i\leq r,\\
      f_j&=(0,\ldots,0,\tfrac{1}{\sqrt 2},0,\ldots,0) && r+1\leq j\leq r+s,\\
      g_j&=(0,\ldots,0,\tfrac{i}{\sqrt 2},0,\ldots,0) && r+1\leq j\leq r+s
     \end{align*}
      \item Wir haben noch $\IC\rightarrow\IR^2: x+iy\mapsto (x,y)$, also \[f:K_\IR\rightarrow\IR^n:\, x\mapsto (x_{\sigma_1},\ldots,x_{\sigma_r},\R x_{\tau_1},\I x_{\tau_1},\ldots,\R x_{\tau_s},\I x_{\tau_s})\]
   
      ist ein Isomorphismus von $\IR$-VR. $\IR^n$ trägt Standardskalarprodukt und hat Standardbasis $e_1,\ldots,e_n$ (Achtung: kleine Überladung). Dabei ist $f(e_i)=e_i$ für $1\leq i\leq r$, $f(f_j)=\frac{1}{\sqrt{2}}e_{r+2j-1}$,
      $f(g_i)=\frac{1}{\sqrt{2}} e_{r+2j}$. $f$ ist also keine Isometrie.
   
   \item Hat man EUKLIDischen VR $V$, so trägt $V$ genau ein verschiebungsinvariantes Maß, welches auf \[Q=\left\{\sum t_ie_i: 0\leq t_i \leq 1\right\},\hspace{1cm} e_1,\ldots,e_n \text{ ON-Basis}\] 
   den Wert $1$ hat.
   
   Wir haben also auf $K_\IR$ ein Maß $\mu$ und auf $\IR^n$ das LEBESGUE-Maß $\lambda$. Für den Einheitsquader $Q$ ist $\mu(Q)=2^s\lambda(f(Q))$:
   \begin{align*}
    \lambda(f(Q))&=\lambda\left(\left\{\sum t_ic_ie_i : c_i=1 \mbox{ für }1\leq i\leq r, c_i=\tfrac{1}{\sqrt 2}\mbox{ für } r+1\leq i\leq n\right\}\right)\\
    &=\left(\frac{1}{\sqrt 2}\right)^{2s}=s^{-2}
   \end{align*}
    Also gilt für alle messbaren Mengen $\Omega\subset K_\IR$: $\mu(\Omega)=2^s\lambda(f(\Omega))$.
    
    \item Ist $V$ ein EUKLIDischer VR, $v_1,\ldots,v_n\in V$, $n=\dim V$, \[Q(v_1,\ldots,v_n)=\left\{\sum_i t_i v_i: 0\leq t_i\leq 1 \right\}= \text{ \highl{Parallelotop},}\] so ist $\mu(Q)=\sqrt{\det(\langle v_i,v_j\rangle)}$.
    \end{enumerate}
   \end{Bemerkung}

   Interludium:   
   \begin{Definition}
    Sei $V$ endlich dimensionaler $\IR$-VR. Eine Untergruppe $\Gamma\subset V$ heißt \highl{Gitter} genau dann, wenn $\Gamma$ diskret und cokompakt ist (d.h. $V/\Gamma$ ist kompakt).
   \end{Definition}

   \begin{Beispiel}
   \begin{enumerate}
    \item $\IZ^n\subset\IR^n$ ist Gitter, denn $\IR^n/\IZ^n=(\IR/\IZ)^n=(S_1)^n$.
    \item $\IQ^n\subset\IR^n$ ist kein Gitter, da nicht diskret.
    \item $\IZ^m\times 0 \subset \IR^n$ ist für $m<n$ kein Gitter, da nicht cokompakt.
   \end{enumerate}
   \end{Beispiel}

   \begin{Fakt}
    Sei $\Gamma\subset V$ ein Gitter. Dann existiert eine Basis $\omega_1,\ldots,\omega_n$ von $V$ mit $\Gamma= \IZ\omega_1 + \ldots + \IZ\omega_n$. Umgekehrt ist jedes solche $\Gamma$ ein Gitter.
   \end{Fakt}
   
   \begin{Beweis}
    Sei $\Gamma=\IZ\omega_1+ \ldots +\IZ\omega_n$ für Basis $\omega_1,\ldots,\omega_n$ von $V$.
     $U=\{ \sum\lambda_i\omega_i : -\frac{1}{2}< \lambda_i < \frac{1}{2}\}$ ist offene Umgebung von $0$ und enthält keine weiteren Punkte aus $\Gamma$.
    Sei $F=\{ \sum t_i\omega_i: 0\leq t_i\leq 1\}$ - Fundamentalmasche. $F$ ist kompakt und wird surjektiv auf $V/\Gamma$ abgebildet. Also ist $V/\Gamma$ kompakt. 
    
    Sei nun $\Gamma$ Gitter in $V$. $\Gamma$ spannt Teilraum $V_0=\Span\Gamma$ auf. Sei $V_1$ komplementärer Raum, dann gilt: $V/\Gamma=V_0/\Gamma\oplus V_1$ ist nur kompakt für $V_1=\{0\}$ (denn der einzige kompakte Vektorraum ist $\{0\}$). Also erzeugt $\Gamma$ den Raum $V$ und enthält somit Basis $\omega_1,\ldots,\omega_n$ von $V$. Sei $\Gamma_0=\IZ\omega_1+\ldots+\IZ\omega_n$ Gitter. Die Gruppe $\Gamma/\Gamma_0\subset V/\Gamma_0$ ist diskret, also abgeschlossen, also kompakt und daher endlich\footnote{$X$ kompakt genau dann, wenn jede offene Überdeckung enthält endliche Teilüberdeckung.}.
    
    \folge $(\Gamma:\Gamma_0)=N<\infty$ \folge $N\Gamma\subset \Gamma_0$ \folge $\Gamma_0\subset\Gamma\subset\frac{1}{N}\Gamma_0$\folge $\Gamma$ freie abelsche Gruppe vom Rang $n$.
   \end{Beweis}

