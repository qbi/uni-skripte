\renewcommand{\lecdate}{10.12.2014}

\begin{Beweis}[mittels Lokalisierung]
 Sei $S=A\setminus \Fp$, dann ist $S\inv A$ ein diskreter Bewertungsring, $S\inv B$ ist ein DED-Ring mit nur endlich vielen Ringidealen: Es kommen nur $\FP\cap S\neq \emptyset$ in Frage. \folge $\FP\cap A\subset\Fp$ \folge $\FP\cap A=\Fp$, also $\FP\mid\Fp$. Also ist $S\inv B$ HIR (nach chinesischem Restsatz). $S\inv B$ ist der ganze Abschluss von $S\inv A$ in $L$: Sei dazu $x\in B$, $s\in S$ \folge $x/s\in L$ ganz über $S\inv A$: 
 \[ x^n+a_1x^{n-1}+\ldots +a_n=0\hspace{1cm} a_j\in A\]
 \folge $\left(\frac{x}{s}\right)^n+\frac{a_1}{s}\left(\frac{x}{s}\right)^{n-1}+\ldots+\frac{a_n}{s_n}=0$ ist Ganzheitsgleichung für $x/s$ über $S\inv A$. Sei $y\in L$ ganz über $S\inv A$, also
 \[ y^n+\frac{a_1}{s_1}y^{n-1}+\ldots+\frac{a_n}{s_n}=0.\]
 Multiplikation mit $s^n$, $s=s_1\cdot\ldots\cdot s_n$ zeigt
 \[ (sy)^n+a_1'(sy)^{n-1}+\ldots+a_n'=0\hspace{1cm} a_j'\in A.\]
 Also ist $sy$ ganz über $A$ \folge $sy\in B$ \folge $y\in S\inv B$.
 
 Wir haben die Aufgabe nun vereinfacht: $A$ dBR, $B$ endlich erzeugter $A$-Modul, ist HIR und torsionsfrei als $A$-Modul, da $B$ in einem Körper liegt. Algebra 1: Dann ist $B$ freier $A$-Modul. Sei $b_1,\ldots,b_n$ eine Basis von $B$ als $A$-Modul. Wir zeigen, dass dann die Bilder $\overline{b_1},\ldots,\overline{b_n}$ in $B/\Fp B$ linear unabhängig über $A/\Fp$ sind. Sei $\sum \overline{a}_j\overline{b}_j=0$, $\overline{a}_j\in A/\Fp$. Nun folgt $\sum a_j b_j\in\Fp B$ ($a_j$ Urbild von $\overline a_j$ in $A$). \folge $\sum a_jb_j=\sum c_j b_j$, $c_j\in\Fp$ \folge $a_j=c_j$ $\forall j$ \folge $\overline{a_j}=0$.
 Somit \[\dim_{A/\Fp} B/\Fp B=\rk_A B = [L:K]. \]
 (Letztes, weil eine $A$-Basis von $B$ auch eine $K$-Basis von $L$ ist.)
 
 Sei $\Fp B= \FP_1^{e+1}\cdot\ldots\cdot \FP_g^{e_g}=\prod_{\FP\mid \Fp} \FP^{e(\FP/\Fp)}$. Wir haben Isomorphismus von $A/\Fp$-VR \[ B/\Fp B \rightarrow \bigoplus_{\FP\mid \Fp} B/\FP^{e(\FP/\Fp)}.\]
 Wir zeigen $\FP^r/\FP^{r+1}\cong B/\FP$ als $A/\Fp$=VR: Sei $\FP=(\Pi)$, $B \rightarrow \FP^r/\FP^{r+1}$: $x\mapsto x\Pi^r+\FP^{r+1}$. Das ist surjektiver Homomorphismus von $A$-Moduln. Der Kern ist $(\Pi)=\FP$. 
 \[ [L:K]=\dim_{A/\Fp} B/\Fp B = \sum_{\FP\mid \Fp} \dim_{A/\Fp} B/\FP^{e(\FP/\Fp)}=\sum_{\FP\mid\Fp} e(\FP/\Fp) f(\FP/\Fp).\]
\end{Beweis}

\begin{Bemerkung}
 \begin{enumerate}
  \item Es fehlen noch die Formeln (Übungsaufgaben)
  \begin{itemize}
   \item $e(S\inv\FP/S\inv\Fp)=e(\FP/\Fp)$
   \item $f(S\inv\FP/S\inv\Fp)=f(\FP/\Fp)$
  \end{itemize}
  \item 
  \begin{itemize}
   \item $S\inv(\Fa\Fb)=S\inv\Fa \cdot S\inv \Fb$
   \item $A/\Fp=S\inv A/S\inv\Fp$, $B/\Fp=S\inv B/S\inv \FP$
  \end{itemize}
  \item Im Fall $K/\IQ$ ist alles viel einfacher: $p\in \IP$, $p\cO_K=\Fp_1^{e_1}\cdot\ldots\cdot \Fp_g^{e_g}$. $\cO_K/\Fp\cO_K$ hat $\IF_p$-Dimension $[K:\IQ]$, $\cO_K/\Fp\cO_K=\bigoplus \cO_K/\Fp_i^{e_i}$.
  $\card \cO_K/\Fp^{e_\Fp}=(\cO_K:\Fp^{e_\Fp})=\IN(\Fp^{e_\Fp})=(\IN\Fp)^{e_\Fp}$. Also $[K:\IQ]=\sum_{\Fp\mid p} e(\Fp/p) f(\Fp/p)$.
  \item Der Satz zeigt: über $\Fp$ liegen in $B$ höchstens $[L:K]$ maximale Ideale.
 \end{enumerate}
\end{Bemerkung}


\begin{Fakt}[Multiplikativität von $e$ und $f$]
 Seien $K\subset L\subset M$ endliche separable Erweiterungen, $A$ DED-Ring in $K$, $B$ sein ganzer Abschluss mit $C=$ ganzer Abschluss in $M$. Sei $Q$ Primideal in $C$ über $\FP=$ Primideal in $B$ über $\Fp=$ Primideal in $A$. Dann gilt
 \begin{enumerate}
  \item $e(Q/\Fp)=e(Q/\FP)e(\FP/\Fp)$
  \item $f(Q/\Fp)=f(Q/\FP)f(\FP/\Fp)$
 \end{enumerate}
\end{Fakt}

\begin{Beweis}
 Klar.
\end{Beweis}


\begin{Fakt}
 Sei $L/K$ GALOIS-Erweiterung mit GALOISgruppe $G$. $A\subset K$ DED-Ring, $B$ ganzer Abschluss in $L$. Sei $\Fp\subset A$ maximales Ideal. Dann operiert $G$ transitiv auf $\FP/\Fp$ und es gilt $e(\FP/\Fp), f(\FP/\Fp)$ sind gleich für alle $\FP/\Fp$. Also
 \[ \Fp B = \left( \prod_{\FP\mid\Fp} \FP\right)^e\]
\end{Fakt}

\begin{Beweis}
 Seien $\FP_1,\FP_2$ zwei Primteiler von $\Fp$ und $\sigma \FP_1\neq\FP_2$ für alle $\sigma\in G=\Gal(L/K)$. Dann folgt $\sigma\FP_1\neq \tau\FP_2$ für alle $\sigma,\tau\in G$. Also ex. $x\in B$ mit $x\equiv 0 \mod{\sigma\FP_1}$ für alle $\sigma\in G$ und $x\equiv 1 \mod{\tau\FP_2}$ für alle $\tau\in G$ (chinesischer Restsatz). $\N_K^L(x)=\prod_{\sigma\in G}\sigma x$ liegt in $A$, da ganz und in $K$. Also $N(x)\in A\cap \FP_1=\Fp$. Andererseits ist $\sigma  x\notin \FP_2$ für alle $\sigma\in G$. Somit auch $\N(x)\notin\FP_2$ \folge $\N(x)\notin \FP_2\cap A=\Fp$ \lightning.
 
 Die GALOISgruppe operiert also transitiv auf den $\FP/\Fp$. $\sigma\in G$ verursacht Isomorphismus $B/\FP \rightarrow B/\sigma\FP$: $x+\FP\mapsto \sigma x+\sigma\FP$, also $f(\sigma\FP/\Fp)=f(\FP/\Fp)$. $\Fp B=\prod \FP^{e_\FP}=\prod (\sigma\FP)^{e_\FP}$ \folge $e_\FP$ ist dasselbe für alle $\FP\mid \Fp$.
\end{Beweis}

\begin{Fakt}
 Sei $K$ ein algebraischer ZK, dann ist $p\in \IP$ in $K$ verzweigt \gdw $p$ teilt die Diskriminante $d_K$.
\end{Fakt}

\begin{Beispiel}
 $K=\IQ(i)$, nur $p=2$ ist verzweigt: $2=-i(1+i)^2$, $d_K=\det^2\begin{pmatrix}
                                                                 1&i\\1&-i
                                                                \end{pmatrix}=-4$.
\end{Beispiel}

\begin{Beweis}
 $A=\cO_K/p\cO_K$ ist endliche $\IF_p$-Algebra\footnote{$K$ Körper, $A$ ist $K$-Algebra :\gdw $A$ ist $K$-Vektorraum und $A$ hat Multiplikation, die sich mit der VR-Struktur verträgt, d.h. bilinear ist}, kommutativ mit $1$. $p$ ist genau dann verzweigt, wenn $A$ nilpotente Elemente enthält.
 Ist $p\cO_K=\Fp_1\cdot\ldots\cdot\Fp_g$ mit paarweise verschiedenen Faktoren, so ist
 \[\cO_K/p\cO_K = \bigoplus_i \cO_K/\Fp_i=\bigoplus_i \IF_{p^f_i}. \]
 Ist $p$ verzeigt, so sitzt in $A$ Teilalgebra $\cO_K/\Fp^e\cO_K$, $e\geq 2$. Das Element $\overline{x}$ für $x\in\Fp\setminus\Fp^2$ ist nilpotent.
 
 Wir definieren eine Spur auf $A$: $a\in A$, $\Tr_{\IF_p}^A(a)=\Tr_{\IF_p}^A(a:A\rightarrow A, x\mapsto ax)$. Es gilt $\Tr(a)\in \IF_p$. Für $x\in\cO_K$ und $\overline{x}\in A$ Bild von $x$ gilt $\Tr_\IQ^K(x)\equiv\Tr_{\IF_p}^A(\overline{x})\mod{p} (*)$:
 
 Ist $\omega_1,\ldots,\omega_n$ Ganzheitsbasis von $\cO_K$, so ist $\overline\omega_1,\ldots,\overline\omega_n$ $\IF_p$-Basis von $A$: sie erzeugen $A$ als $\IF_p$-Basis und $\dim_{\IF_p} A=\rk_\IZ \cO_K = n$.
 Sei $x\omega_i=\sum m_{ij}\omega_j$, $m_{ij}\in\IZ$, dann ist $\Tr_\IQ^K(\overline x)=\sum m_{ii}$ und $\Tr_{\IF_p}^A(\overline x)=\sum\overline m_{ii}$. Weiter hat $A$ eine Zerlegung \[ A=\bigoplus_{\Fp\mid p} A_\Fp,\]
 wobei $p\cO_K=\prod_{\Fp\mid p} \Fp^{e(\Fp/p)}$, $A_\Fp=\cO_K/\Fp^{e(\Fp/p)}$ (chinesischer Restsatz). Dabei sind die $A_\Fp$ invariant unter Multiplikation. Deshalb gilt $\Tr_{\IF_p}^A(a) = \sum_{\Fp\mid p} \Tr_{\IF_p}^{A_\Fp}(a)$. Sei $a_1,\ldots,a_n$ eine $\IF_p$-Basis von $A$, $d_A$ die Diskriminante: $d_A(a_1,\ldots,a_n):=\det(\Tr_{\IF_p}^A(\omega_i\omega_j))$. Dann gilt für andere Basis $b_1,\ldots,b_n$: $d_A(b_1,\ldots,b_n)=c^2d_A(a_1,\ldots,a_n)$, für ein $c\in\IF_p\kreuz$. Aus $(*)$ folgt: Ist $a_1,\ldots,a_n$ Ganzheitsbasis, so ist $d_A(\overline a_1,\ldots,\overline a_n=0$ in $\IF_p$ \gdw $d_k\equiv 0\mod{p}$. Es gilt $d_{A\oplus B}=d_A\cdot d_B$:
 
 $a_1,\ldots,a_k$ Basis von $A$, $b_1,\ldots,b_l$ Basis von $B$. Dann ist $\Tr()=\begin{pmatrix}
                                                                                  \Tr(a_ia_j) & 0\\
                                                                                  0 & \Tr(b_rb_s)
                                                                                 \end{pmatrix} $ ($0$, weil Spur von $(a_i,0)\cdot (0,b_j)=(0,0)$) .
                                                                                 
  Es folgt \[ d_A=\prod_{\Fp\mid p} d_{A\Fp}. \] Sei $A_\Fp=\cO_K/\Fp^e$, $e\geq 2$. $A_\Fp$ hat $A_\Fp$-invariante Teilräume $A_\Fp\supset \Fp/\Fp^e\supset\ldots\supset \Fp^{e-1}/\Fp^e$. Wähle $\IF_p$-Basis von $\Fp^{e-1}/\Fp^{e}$, ergänze Basis zu Basis von $\Fp^{e-2}/\Fp^e$
  
  In der obigen Basis sind also in der Matrix zu $x$ die Diagonalelemente alle gleich $0$. Also ist $\Tr_{\IF_p}^{A_\Fp}(x)=0$. Die Matrix zu $d_{A_\Fp}$ sieht also so aus:\[ \begin{pmatrix}
                                                                                                                                                                                \textasteriskcentered & 0\\
                                                                                                                                                                                0 & 0
                                                                                                                                                                               \end{pmatrix}.
\]
Es folgt $d_{A_\Fp}=0$, also $p\mid d_K$. Ist $p$ unverzweigt, so ist $A_\Fp=\cO_K/\Fp$ für alle $\Fp\mid p$ eine endlich Erweiterung von $\IF_p$, also $d_A\neq 0$, denn $\IF_{p^n}/\IF_p$ stets separabel:
\[ T^{p^n}-T=f(T), f'(T)=-1\]
\folge  Spurform ist nichtausgerartet \folge $\det(\Tr(a_ia_j))\neq 0$ für $\IF_p$-Basen $a_1,\ldots,a_n$.
  \end{Beweis}
