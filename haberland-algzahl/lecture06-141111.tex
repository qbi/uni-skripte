%!TEX root = Algebraische_Zahlentheorie.tex

\renewcommand{\lecdate}{11.11.14}

\subsection{Die Ringe ganzer algebraischer Zahlen II - mult. Struktur}%1.5

\begin{Satz}[\glqq Satz 2\grqq{}]
 Sei $K$ ein algebraischer ZK, $\cO_K$ der Ring der ganzen algebraischen Zahlen. Dann ist $\cO_K$ NOETHERsch, ganzabgeschlossen und jedes Primideal $\neq (0)$ ist maximal.
\end{Satz}

\begin{Beweis}
 \begin{enumerate}
  \item Ganzabgeschlossen ist klar: $\cO_K$ ist der ganze Abschluss von $\IZ$ in $K$ und ganz sein ist transitiv.
  \item Sei $\Fa\subset \cO_K$ Ideal, dann ist $\Fa$ additive Untergruppe in $\cO_K$, frei von endlichem Rang, also auch $\Fa$ frei von endlichem Rang. Somit ist $\Fa$ als $\cO_K$-Modul endlich erzeugt. Das liefert NOETHERsch.
  \item Sei $\Fp\subset \cO_K$ Primideal $\neq(0)$. Dann enthält $\Fp$ auch orthodoxe ganze Zahlen $\neq 0$. Sei $x\in\Fp\setminus\{0\}$, $\N(x)=\prod\sigma x$ ist ganz und aus $\IQ$, also aus $\IZ$. Somit enthält $\Fp$ positive natürliche Zahlen. Sei $m>0$ die kleinste, dann gilt $(m)=m\cO_K \subset \Fp \subset \cO_K$. Nun gilt \[ (\cO_K:m\cO_K)=m^{[K:\IQ]}.\]
  Also hat $m\cO_K$ endlichen Index in $\cO_K$, also auch $\Fp$. Somit ist $\cO_K/\Fp$ endlicher, integer Ring, also ein Körper.
 \end{enumerate}
\end{Beweis}

\begin{Bemerkung}
 Das sind geometrische Eigenschaften.
 \begin{itemize}
  \item ganzabgeschlossen bedeutet: ohne Singularität
  \item Primideal $\neq (0)$ sind maximal bedeutet Dimension von $\cO_K$ ist $1$
  \item \glqq Schemata\grqq{} (GROTHENDIEK)
 \end{itemize}
\end{Bemerkung}

\begin{Definition}
 Ein integer, kommutativer Ring mit $1$ heißt \highl{DEDEKIND-Ring}, wenn er NOETHERsch, ganzabgeschlossen und der Dimension $1$ ist, also jedes Primideal $\neq (0)$ ist maximal.
\end{Definition}

\begin{Satz}[\glqq Satz 3\grqq: Primidealzerlegung in DEDEKIND-Ringen]
 Jedes Ideal $\neq (0),(1)$ in einem DED-Ring
 besitzt eine Zerlegung in Primideale: \[\Fa=\Fp_1\cdot\ldots\cdot \Fp_n. \]
 Diese ist bis auf Reihenfolge eindeutig bestimmt.
\end{Satz}

\begin{Beweis}[VAN DER WAERDEN]
 \begin{enumerate}
  \item Sei $\Fa\neq (0)$ Ideal in DED-Ring $A$. Wir zeigen: $\exists \Fp_1,\ldots,\Fp_r$ maximal, s.d. $\Fp_1\ldots\Fp_r\subset\Fa$.
  
  Indirekt. Sei $\varphi$ die Menge aller Ideale $\neq (0)$, für welche das nicht gilt. $A$ NOETHERsch \folge $\varphi$ besitzt maximale Elemente, sei $\Fa$ ein solches. $\Fa$ selbst ist natürlich nicht prim. Also ex. $x,y\in A$ mit $x\notin \Fa, y\notin\Fa$, aber $xy\in\Fa$. Seien $\Fa_1=(\Fa,x)$, $\Fa_2=(\Fa,y)$, dann ist $\Fa_1\supsetneq \Fa$ und $\Fa_2\supsetneq \Fa$ und $\Fa_1\Fa_2\subset\Fa$. Wegen der Maximalität von $\Fa$ enthalten $\Fa_1$ und $\Fa_2$ Produkte von maximalen Idealen, also auch $\Fa$ selbst. $\lightning$
  \item Sei $\Fp$ maximales Ideal in $A$. Wir definieren:
  \[ \Fp\inv = \{ x\in K : x\Fp\subset A\},\hspace{1cm} K=\QK(A).\]
  Es gilt $A\subset \Fp\inv$, $\Fp\inv$ ist abgeschlossen unter Addition und $y\Fp\inv\subset\Fp\inv$ für alle $y\in A$. D.h. $\Fp\inv$ hat alle Eigenschaften eines $A$-Ideals, außer der in $A$ zu liegen. Wir zeigen $\Fp\inv\neq A$.
  Sei dazu $x\in\Fp$, $x\neq 0$ und $r$ die kleinste positive natürliche Zahl, s.d. $\exists \Fp_1,\ldots,\Fp_r$ mit $\Fp_1\cdot\ldots\cdot \Fp_r\subset (x)$.
  Dann muss wenigstens eines der $\Fp_i$ in $\Fp$ liegen: Gibt es $x_i\in\Fp_i\setminus\Fp$ für alle $i$, so gilt $x_1\ldots x_r\in\Fp_1\ldots\Fp_r$, aber $x_1\ldots x_r\notin \Fp$. Ohne Einschränkung $\Fp_1\subset \Fp$. Es folgt $\Fp_1=\Fp.$ Weiter ist $\Fp_2\ldots\Fp_r\not\subset (x)$, wegen $r$ minimal. Also existiert ein $y\in\Fp_2\ldots\Fp_r$ mit $y\notin (x)$. Aber $y\Fp\subset (x)$ \folge $x\inv y \Fp\subset A$ \folge $x\inv y\in \Fp\inv$. Nun ist $x\inv y\notin A$, denn wäre $x\inv y=a\in A$, so folgt $y=ax\in (x)$ $\lightning$. Also $\Fp\inv\supsetneq A$.
  \item Wir haben $\Fp\subset \Fp\Fp\inv \subset A$ (wobei $\Fp\Fp\inv=\{\sum x y : x\in \Fp, y\in\Fp\inv\}$)
  
  Da $\Fp\Fp\inv$ ein $A$-Ideal ist, folgt $\Fp\Fp\inv=\Fp$ oder  $\Fp\Fp\inv=A$. Wir zeigen letzteres. Indirekt: Sei $\Fp\Fp\inv=\Fp$. Also überführt jedes $x\in\Fp\inv$ den endlich erzeugten $A$-Modul $\Fp$ in sich selbst. $x\Fp\subset \Fp$. Also ist $x$ ganz /$A$, also $x\in A$. $\lightning$.

  \item Sei $\varphi$ die Menge aller Ideale $\neq (0),(1)$ ohne Primidealzerlegung und sei $\Fa\in\varphi$ maximal. Dann liegt $\Fa$ im maximalen Ideal $\Fp$. Also
  \[\Fa\subset \Fa\Fp\inv \subset \Fp\Fp\inv=A. \]
  $\Fa\Fp\inv$ ist echt größer also $\Fa$, da sonst $\Fp\inv$ nur ganze Elemente enthielte. Weiter ist $\Fa\Fp\inv$ ein $A$-Ideal. Mithin besitzt $\Fa\Fp\inv$ die Zerlegung
  \[\Fa\Fp\inv = \Fp_1\ldots\Fp_m \]
  in maximale Ideale. Es folgt $\Fa=\Fa\Fp\Fp\inv= \Fp\Fp_1\ldots\Fp_m\lightning$
  \item Eindeutigkeit. Ist $\Fa\Fb\subset\Fp$, so folgt $\Fa\subset \Fp$ oder $\Fb\subset\Fp$.
  
  Ist $x\in\Fa\setminus\Fp$, $y\in\Fb\setminus\Fp$ \folge $xy\in\Fp$ \folge $x\in\Fp$ oder $y\in\Fp$ $\lightning$.
  
  Ist $\Fp_1\ldots\Fp_r=\Fq_1\ldots\Fq_s$, so folgt $\Fp_1\ldots\Fp_r\subset \Fq_1$ \folge $\Fp_1\subset\Fq_1$ \folge $\Fp_1=\Fq_1$.
  Multiplikation mit $\Fp\inv$ gibt $\Fp_2\ldots\Fp_r=\Fq_2\ldots\Fq_s$. Nun IV benutzbar. IA: $\Fp_1\ldots\Fp_m=A$\folge $m=0$.
 \end{enumerate}
\end{Beweis}

Wir klären, was $\Fp\inv$ ist:

\begin{Definition}
 Sei $A$ DEDEKIND-Ring, $K$ sein QK. Ein \highl{gebrochenes Ideal} von $A$ ist ein $A$-Modul $\Fa\subset K$, s.d. ein $c\in A\oN$ existiert
mit $c\Fa\subset A$. Man schließt $(0)$ aus.
 \end{Definition}

\begin{Bemerkung}
 $c\Fa$ ist dann gewöhnliches Ideal. Alle Ideale sind gebrochene Ideale: $c=1$. Gebrochene Ideale sind endlich erzeugte $A$-Moduln.

 Mantra: Die Nenner der Elemente aus gebrochene Ideal sind beschränkt.
\end{Bemerkung}

\begin{Beispiel}
 \begin{enumerate}
  \item $\{\frac{a}{p}: a\in\IZ \}$ ist gebrochenes Ideal von $\IQ$. Aber $\{ \frac{a}{p^m}: a\in\IZ, m\in \IN\}$ ist keins.
  \item Ist $A$ DED-Ring + HIR, dann sind die gebrochenene Ideale der Form $(x)=Ax$, $x\in K\kreuz$.
  \item Gebrochene Hauptideale: Jedes $x\in K\kreuz$ erzeugt gebrochenes Ideal $(x)=Ax$.
  \item Das $\Fp\inv$ aus dem Beweis ist ein gebrochenes Ideal. Es ist $A$-Modul (klar) und für $c\in\Fp\oN$ gilt $c\Fp\inv\subset A$.
 \end{enumerate}
\end{Beispiel}

\begin{Bemerkung}
 Es gelten für gebrochene Ideale die selben Rechenregeln, wie für die üblichen Ideale.
\end{Bemerkung}

\begin{Fakt}
 Sei $A$ DEDEKIND-Ring, $K$ der QK, $\Fa\subset K$ gebrochenes Ideal, dann ist auch
 \[ \Fb=\{x\in K: x\Fa\subset A \}\]
 ein gebrochenes Ideal und $\Fa\Fb=(1)=A$. D.h. die gebrochenen Ideale von $A$ bilden bzgl. Multiplikation eine abelsche Gruppe $\Id_A$. Man schreibt $\Fb=\Fa\inv$.
\end{Fakt}

\begin{Beweis}
 Sei zuerst $\Fa$ gewöhnliches Ideal in $A$. Dann gilt $\Fa=\Fp_1\ldots\Fp_r$. Setze $\Fc=\Fp_1\inv\ldots\Fp_r\inv$. Es folgt $\Fa\Fc=A$, also $\Fc\subset\Fb$. Sei $x\in\Fb$ \folge $x\Fa\subset A$ \folge $x\Fa\Fc\subset \Fc$ \folge $xA\subset \Fc$ \folge $x\in \Fc$ \folge $\Fb\subset\Fc$.
 
 $\Fc$ ist gebrochenes Ideal, da die $\Fp_i\inv$ es sind und Produkte gebrochener Ideale wieder gebrochene Ideale sind:
 \[ c\Fa\subset A, d\Fb\subset A \folge (cd)\Fa\Fb\subset A.\]
 
 Sei nun $\Fa$ gebrochenes Ideal, $c\in A\oN$, s.d. $c\Fa\subset A$. $(c\Fa)\inv=c\inv\Fa\inv$, genauer zeigen wir: $c(c\Fa)\inv$ ist reziprok zu $\Fa$: $c(c\Fa)\inv\Fa=A$. Das ist äquivalent zu $(c\Fa)\inv(c\Fa)=A$, und das ist richtig. Somit $\Fa\inv=c(c\Fa)\inv$, $(c\Fa)\inv$ ist gebrochenes Ideal, also auch $c(c\Fa)\inv$.
\end{Beweis}

\begin{Folgerung}
 $\Id_A$ ist freie abelsche Gruppe, jedes $\Fa\in\Id_A$ hat eindeutige Darstellung \[\Fa=\prod_{\Fp}\Fp^{\ord_{\Fp}(\Fa)}\]
 $\ord_{\Fp}(\Fa)\in \IZ$, fast alle $=0$ (alle bis auf endl. viele).
\end{Folgerung}
