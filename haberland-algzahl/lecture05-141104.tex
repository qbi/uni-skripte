%!TEX root = Algebraische_Zahlentheorie.tex

\renewcommand{\lecdate}{04.11.14}

\begin{align*}
\Tr(\zeta_p^{-r} x) &= (p-1)a_r-\sum_{j\neq r} a_j && \text{mit } 1\leq r\leq p-2 \mbox{ oder } r=0\\
&= pa_r - \sum_{j=0}^{p-1} a_j
\end{align*}

Nun ist $\zeta_p^{-r}x-\zeta_px\in \cO_K$, also $\Tr(\zeta_p^{-r}x-\zeta_px)\in \IZ$ und folglich $pa_r\in\IZ$. Somit
\[px=b_0+b_1\zeta_p+\ldots + b_{p-2}\zeta_p^{p-2}\hspace{1cm} b_j\in\IZ.\]

Nun Substitution $\zeta_p=1-\pi$:
\[px=c_0+c_1\pi+\ldots+ c_{p-2}\pi^{p-2}\hspace{1cm} c_j\in\IZ. \]

$\pi$ teilt $c_0$ in $\cO_K$ \folge[L2] $p\mid c_0$ in $\IZ$.
\[ \folge px-pd_0=c_1\pi+\ldots + c_{p-2}\pi^{p-2}\hspace{1cm}\mbox{mit } d_0\in\IZ\]
$\pi\mid c_1$ in $\cO_K$\folge $p\mid c_1$ in $\IZ$ \folge $c_1=pd_1$ usw. bis 
\[ px=p(d_0+d_1\pi+\ldots +d_{p-2}\pi^{p-2})\hspace{1cm} d_j\in\IZ.\]

Rücksubstitution: $\pi=1-\zeta_p$ \folge $x\in\IZ[\zeta_p]$. Wir halten das Ergebnis im folgenden Fakt fest.

\begin{Fakt}
Sei $2<p\in\IP$, $\zeta_p=e^{2\pi i/p}$ und $K=\IQ(\zeta_p)$. Dann $\cO_K=\IZ[\zeta_p]$. Insbesondere ist $1,\zeta_p,\ldots,\zeta_p^{p-2}$ Ganzheitsbasis.
\end{Fakt}

\begin{Bemerkung}
 Mit etwas mehr Aufwand:
 
 $m\geq 2$, $K=\IQ(\zeta_m)$, $\zeta_m=e^{2\pi i/m}$, dann ist $\cO_K=\IZ[\zeta_m]$. (Siehe NEUKIRCH, AlgZT, p. 62ff)
\end{Bemerkung}

\begin{Fakt}
 Sei $K$ algebraischer Zahlenkörper, $\omega_1,\ldots,\omega_n$ Ganzheitsbasis, dann heißt
 \[ d_K:= \det(\sigma_i\omega_j)^2=\det(\Tr(\omega_i\omega_j))\]
 die \highl{Diskriminante} von $K$.
 Sie ist unabhängig von der Wahl der Ganzheitsbasis.
\end{Fakt}

\begin{Beweis}
 Sei $\omega_1',\ldots,\omega_n'$ eine andere Ganzheitsbasis, also $\omega_i'=\sum_j a_{ij} \omega_j$, $a_{ij}\in\IZ$ und $\omega_i=\sum_j b_{ij}\omega_j'$, $b_{ij}\in\IZ$.
 
 $A=(a_{ij})$, $B=(b_{ij})$ \folge $AB=BA=I$ \folge $\det A=\det B =\pm 1$.
 
 Weiter ist $d(\omega_1',\ldots,\omega_n')=(\det A)^2 \cdot d(\omega_1,\ldots,\omega_n)$.
\end{Beweis}


\begin{Bemerkung}
 Man hat für beliebige $\omega_1,\ldots,\omega_n\in K$ die \highl{Diskriminante} \[d(\omega_1,\ldots,\omega_n)=\det(\sigma_i\omega_j)^2=\det(\Tr(\omega_i\omega_j)).\]
 Diese ist $\neq 0$ genau dann, wenn die $\omega_i$ $\IQ$-linear unabhängig sind. Wie oben sieht man: $d$ ist nur von $\IZ\omega_1 + \ldots + \IZ\omega_n$ abhängig.
\end{Bemerkung}

\begin{Fakt}
 Sind $M\subset N\subset K$ zwei additive Untergruppen vom Rang $n=[K:\IQ]$, so gilt $d_M=(N:M)^2 d_N$.
\end{Fakt}

\begin{Beweis}
 Sei $\omega_1,\ldots, \omega_n$ Basis von $N$, $\eta_1,\ldots,\eta_n$ Basis von $M$, $A=(a_{ij})$ sei definiert durch
 \[\eta_i=\sum_j a_{ij} \omega_j,\,\, a_{ij}\in\IZ. \]

 Dann ist $d(\eta_1,\ldots,\eta_n)=\det(A)^2 d(\omega_1,\ldots,\omega_n)$. Wir zeigen $(N:M)=|\det(A)|$.
 
 \paragraph{Elementare Zeilenoperationen:} \begin{itemize}
                                            \item Vertauschung zweier Zeilen
                                            \item Addieren eines Vielfachen einer Zeile zu einer anderen
                                           \end{itemize}
Das entspricht Linksmultiplikation mit $\IZ$-Matrix der $\det$ $\pm 1$. Analog kann man Spaltenoperationen mit Rechtsmultiplikationen von $\IZ$-Matrizen realisieren. 
 $A$ besitzt betragsmäßig kleinstes Element $\neq 0$. Wir machen es zu $a_{11}$. Division mit Rest: Man kann in 1. Zeile und 1. Spalte alle Einträge betragsmäßig $< |a_{11}|$ machen: iterieren bis in 1. Zeile und 1. Spalte nur noch $a_{11}$ und Nullen stehen. Nun so weiter: $\exists B,C\in \GL(n,\IZ)$, s.d. 
 \[B A C= \begin{pmatrix}
           a_{11} & &\\
           & \ddots & \\
           & & a_{nn}
          \end{pmatrix}=\diag(d_1,\ldots,d_n)=D.
 \]
 Multiplikation mit $B$ und $C$ entspricht Wahl anderer Basen in $M$ und $N$. Also $d_M=(\det D)^2 d_N$. $(N:M)=|\prod d_i|$ ist klar.
 \[ \left(\IZ^n: \bigoplus_{i=1}^n d_i\IZ\right)=\prod_{i=1}^n(\IZ: d_i\IZ)=\prod |d_i|\]
 \end{Beweis}

\begin{Folgerung}
 Sind $\omega_1,\ldots,\omega_n\in \cO_K$ und $d(\omega_1,\ldots,\omega_n)$ ist quadratfrei, so ist dies Ganzheitsbasis.
\end{Folgerung}

\begin{Beispiel}
 \begin{enumerate}
  \item $K=\IQ(\sqrt{d})$, $d\neq 0,1$, sqf, $d\in \IZ$.
  \begin{itemize}
   \item $d\equiv 1\mod{4}$: $d_K=\det^2\begin{pmatrix}
                                                1 & \frac{1+\sqrt d}{2}\\
                                                1 & \frac{1-\sqrt d}{2}
                                               \end{pmatrix}
=(-\sqrt{d})^2=d$
\item $d\equiv 2,3\mod{4}$: $d_K=\det^2 \begin{pmatrix}
                                                1 & \sqrt d\\
                                                1 & -\sqrt d
                                               \end{pmatrix}
=4d$
  \end{itemize}
\begin{tabular}{>{$}c<{$}|*{5}{>{$}c<{$}}}
 d &-2& -1& 2& 3& 5\\\hline
 d_K &-8 &-4 &8 &12 &5
\end{tabular}
\item  $K=\IQ(\zeta_p)$, $p>2$ prim, $\zeta_p=e^{2\pi i/p}$

$d_K=\det(\Tr_{\IQ}^K(\zeta_p^{i+j}))_{1\leq i,j\leq p-1}$, denn mit $1,\zeta_p,\ldots,\zeta_p^{p-2}$ ist auch $\zeta_p,\zeta_p^2,\ldots,\zeta_p^{p-1}$ eine Ganzheitsbasis.

\[\Tr(\zeta_p^r)=\begin{cases}
                 -1 & r\not\equiv 0 \mod{p}\\
                 p-1 & r\equiv 0\mod{p}
                \end{cases}\]

                Also \begin{align*} d_K&=\det\begin{pmatrix}
                             -1 & -1 & \hdots & p-1\\
                             \vdots & \vdots & & \vdots\\
                             -1 & p-1 & \hdots & -1\\
                             p-1 & -1 & \hdots & -1
                            \end{pmatrix} = (-1)^{p(p-1)/2}\det \begin{pmatrix}
                             p-1 & \hdots & -1\\
                             \vdots & \ddots & \vdots\\
                             -1 &  \hdots & p-1
                            \end{pmatrix}\\
                            &=(-1)^{(p-1)/2}\det(pI-A) \hspace{5mm}\mbox{ mit } A=\begin{pmatrix}
                                                                       1 & \hdots & 1\\
                                                                       \vdots& & \vdots\\
                                                                       1 & \hdots & 1
                                                                      \end{pmatrix}
\end{align*}
$A$ hat $(p-2)$-fache Eigenwert $0$, wegen Rang 1. Der letzte Eigenwert von $A$ ist $p-1$, denn $A\cdot (1,\ldots,1)^T=(p-1)(1,\ldots,1)^T$. Somit ist $\det (\lambda I-A)=\lambda^{p-2}(\lambda-p+1)$. Nun $\lambda=p$ einsetzen: $d_K=(-1)^{\frac{p-1}{2}}p^{p-2}$.
\item DEDEKIND

$f=T^3-T^2-2T-8$
\begin{enumerate}
 \item $f$ ist irreduzibel $/\IQ$: Angenommen $f(\alpha)=0$ für ein $\alpha\in\IQ$ \folge $\alpha\in\IZ$ \folge $\alpha\mid 8$ \folge $\alpha=\pm 1, \pm 2, \pm 4, \pm 8$. Aber das sind alles keine Nullstellen von $f$.
 
 Für jede Nullstelle $\alpha\in\IC$ von $f$ gilt $K=\IQ(\alpha)$ hat Grad $3$ über $\IQ$.
 \item Die Diskriminante von $\IZ[\alpha]\subset \cO_K$.
 \[ \IZ[\alpha] = \IZ \oplus \IZ \alpha \oplus \IZ \alpha^2\]
 \[ d(1,\alpha,\alpha^2)=\det\begin{pmatrix}
                              \Tr 1 & \Tr \alpha & \Tr \alpha^2\\
                              \Tr \alpha & \Tr \alpha^2 & \Tr \alpha^3\\
                              \Tr \alpha^2 & \Tr \alpha^3 & \Tr \alpha^4
                             \end{pmatrix}
                             =\begin{vmatrix}
                               3 & 1 & 5\\
                               1 & 5 & 31\\
                               5 & 31 & 49
                              \end{vmatrix}=-4\cdot \underbrace{503}_{\in\IP}\]

\begin{align*}
 \Tr(1)&=3\\
 \Tr(\alpha)&=\alpha+\alpha'+\alpha''=-\text{Koeffizient an }T^2=1 \\
 \Tr(\alpha^2)&=\alpha^2 + (\alpha')^2+(\alpha'')^2=(\alpha+\alpha'+\alpha'')^2-2(\alpha\alpha'+\alpha\alpha''+\alpha'\alpha'')\\
 &=1-2(-2)=5\\
 \Tr(\alpha^3)&=\Tr(\alpha^2+2\alpha+8)=5+2+24=31\\
 \alpha^4&=\alpha^3+2\alpha^2+8\alpha=3\alpha^2+10\alpha+8\\
 \Tr(\alpha^4)&= 15+10+24=49
\end{align*}

\item $\beta=\frac{\alpha+\alpha^2}{2}$ ist ganz.

\begin{align*}
\beta^2&=\frac{1}{4} (\alpha^2+2\alpha^3+\alpha^4)= \frac{1}{4}(\alpha^2+2\alpha^2+4a+16+3\alpha^2+10\alpha+8)\\
&= \frac{3}{2}\alpha^2+\frac{7}{2}\alpha+6 = 3\cdot \frac{\alpha+\alpha^2}{2}+2\cdot \alpha + 6\\
&= 3\beta + 2\alpha+6\\
 \alpha\beta &= \frac{1}{2}(\alpha^3+\alpha^2)=\frac{1}{2} (2\alpha^2+ 2\alpha + 8)=\alpha^2+\alpha+4\\
 &=2\beta+4
\end{align*}

Also gilt $\beta M\subset M$ für $M=\IZ+\IZ\alpha + \IZ\beta$.
Es folgt: Index ist $2$ und $\cO_K$ hat Ganzheitsbasis $1,\alpha,\frac{\alpha+\alpha^2}{2}$. $d_K=-503$.
\item 2 ist ein außerwesentlicher Diskriminantenteiler. Für alle $\gamma\in \cO_K\setminus \IZ$ hat $\IZ[\gamma]$ einen durch 2 teilbaren Index in $\cO_K$. Insbesondere ist $\IZ[\gamma]\neq \cO_K$.

Sei dazu $\gamma=a+b\alpha+ c\beta:$ $a,b,c\in\IZ$, $b,c$ nicht beide $=0$. Wir wissen $\beta^2=3\beta+2\alpha+6$, $\alpha\beta=2\beta+4$, $\alpha^2=2\beta-\alpha$.

Übergangsmatrix zwischen $1,\alpha,\beta$ und $1,\gamma,\gamma^2$:
\begin{align*}
1&=1\\ 
\gamma&=a+b\alpha+c\beta\\
\gamma^2&=a^2+2ab\alpha+ac\beta+b^2\alpha^2+2bc\alpha\beta+c^2\beta^2\\
&=(a^2+8bc+6c^2)+(2ab-b^2+2c^2)\alpha + (2ac+2b^2+4bc+3c^2)\beta
\end{align*}

$\det \begin{pmatrix}
       1 & a & a^2+8bc+6c^2\\
       0 & b & 2ab-b^2+2c^2\\
       0 & c & 2ac+2b^2+4bc+3c^2
      \end{pmatrix}\equiv
      \det \begin{pmatrix}
       1 & a & a^2\\
       0 & b & b^2\\
       0 & c & c^2
      \end{pmatrix}\equiv bc(c-b) \equiv 0 \mod{2}
$
\end{enumerate}
\end{enumerate}
\end{Beispiel}




