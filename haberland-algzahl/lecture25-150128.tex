\renewcommand{\lecdate}{28.01.2015}

\begin{Definition}
 Seien $L/K/\IQ_p$ endliche Erweiterungen. $\Fm_K=(\pi_K)$, $\Fm_L=(\pi_L)$. Dann ist $\pi_K=u\cdot \pi_L^e$ mit $u\in\cO_L\kreuz$ (oder äquivalent: $|\pi_K|=|\pi_L|^e$).
 
 Dann heißt $e=e(L/K)$ \highl{Verzweigungsindex}. Der \highl{Restklassengrad} $f=f(L/K)$ ist der Grad der Restklassenkörpererweiterung \[ f=[\cO_L/\Fm_L : \cO_K/\Fm_K].\]
\end{Definition}

\begin{Bemerkung}
 Die beiden Turmformeln sind klar.
\begin{enumerate}
 \item $e(M/K)=e(M/L)e(L/K)$
 \item $f(M/K)=f(M/L)f(L/K)$
\end{enumerate}
 \end{Bemerkung}

 \begin{Satz}[\glqq Satz 4\grqq]
  Seien $L/K/\IQ_p$ endliche Erweiterungen. Dann gilt
  \[ [L:K]=e(L/K)f(L/K).\]
  Insbesondere sind $\cO_L/\Fm_L$ und $\cO_K/\Fm_K$ endlich.
 \end{Satz}
 
 \begin{Beweis}
  Seien $\alpha_1,\ldots,\alpha_f\in\cO_L$, s.d. $\overline{\alpha_1},\ldots,\overline{\alpha_f}\in\cO_L/\Fm_L$ eine Basis des $\cO_K/\Fm_K$-Vektorraums $\cO_L/\Fm_L$ sind. Wir zeigen, dass die Vektoren 
  $\alpha_i\pi_L^j$, $1\leq i\leq f$, $0\leq j<e$ $K$-linear unabhängig sind.
  
  Sei $\sum \lambda_{ij}\alpha_i\pi_L^j=0$, $\lambda_{ij}\in K$. O.B.d.A. $\lambda_{ij}\in\cO_K$ und nicht alle aus $\Fm_K$. Sei $\mu_j:=\sum_{i=1}^f\lambda_{ij}\alpha_i\in\cO_L$, diese sind nicht alle $=0$:
  \[ \sum \overline\lambda_{ij}\overline{\alpha_i}=0 \folge \overline\lambda_{ij}=0 \lightning\]
  
  Sei $\mu_r\neq 0$. Wir teilen durch die höchste Potenz $s$ von $\pi_K$, welche alle $\pi_{ir}$ teilt. Dann ist wenigstens ein Koeffizient eine Einheit. Deshalb ist die Reduktion modulo $\Fm_L$ ungleich $0$.
  Es folgt 
  \[ \mu_r=\pi_K^s\cdot u_r, \text{ mit } u_r\in\cO_L\kreuz \text{ wegen }\overline{u_r}\neq 0\text{)}.\]
  Also $\mu_r=v_r\pi_L^{es}$, $v_r\in\cO_L\kreuz$.
  
  In $\sum_{j=0}^{e-1}\mu_j\pi_L^j$ müssen 2 Summanden $\neq 0$ denselben Betrag haben, also dieselbe Potenz von $\pi_L$ enthalten. Es gibt also $0\leq a<b\leq e-1$ mit $|\mu_a\pi_L^a|=|\mu_b\pi_L^b|$. Nun ist 
  $|\mu_a|=|\pi_L|^{es}$, $|\mu_b|=|\pi_L|^{et}$. Das gibt $e(s-t)=b-a$ \folge $a=b$ \lightning
  
  Wir haben gezeigt $e\cdot f\leq [L:K]$.
  
  Setze $M:=\sum_{i,j}\cO_K \alpha_i\pi_L^j\subset \cO_L$ und $N:=\sum_{j}\cO_K\cdot \alpha_i\subset \cO_L$.
  Dann ist $M=N+\pi_LN+\ldots+\pi_L^{e-1}N$. Wir zeigen $\cO_L=N+\pi_L\cO_L$. Sei dazu $\alpha\in\cO_L$, es folgt 
  \[ \alpha\equiv \sum \lambda_i\alpha_i \mod{\pi_L}\]
  für gewisse $\lambda_i\in\cO_K$. \folge $\alpha-\sum\lambda_i\alpha_i\in\pi_L\cO_L$.
  
  Iterieren: $\cO_L=N+\pi_L(N+\cO_L)=N+\pi_LN+\pi_L^2\cO_L$ usw.
  Es folgt $\cO_L=M+\pi_L^e\cO_L=M+\pi_K\cO_L$.
  
  Iterieren: $\cO_L=M+\pi_K(M+\pi_K\cO_L)=M+\pi_KM+\pi_K^2\cO_L$ ($\pi_KM\subset M$). Also $\cO_L=M+\pi_K^2\cO_L$ usw. \folge $\cO_L=M+\pi_K^N\cO_L$ für alle $N\geq 1$.
  
  Das zeigt: $M$ ist dicht in $\cO_L$. $M$ ist stetiges Bild eines Kompaktums, also selbst kompakt und damit abgeschlossen in $\cO_L$. Das zeigt $M=\cO_L$.
  Es folgt: $\cO_L$ ist freier $\cO_K$-Modul mit Basis $\alpha_i\pi_L^j$, also vom Rang $e\cdot f$.
  Dies ist dann automatisch eine $K$-Basis von $L$.
 \end{Beweis}

\begin{Folgerung}[aus dem Beweis]
 $\cO_L$ ist freier $\cO_K$-Modul vom Ranf $[L:K]$.
\end{Folgerung}

\begin{Bemerkung}
 In Kapitel $\S$ I haben wir ähnliches durch Lokalisieren erreicht. Für Ringe ganzer Zahlen $\cO_L/\cO_K$. Für Zahlkörper $L/K$ ist das falsch.
\end{Bemerkung}

\subsection{Bewertung von Zahlkörpern}
\begin{Satz}[\glqq Satz 5\grqq]
 Sei $K$ algebraischer ZK.
 \begin{enumerate}
  \item Jede Bewertung von $K$ setzt eine $p$-adische oder die archimedische Bewertung von $\IQ$ fort.
  \item Die Fortsetzungen von $|\cdot|_\infty$ sind in Bijektion zu den Einlagerungen $\sigma:K\rightarrow \IR$ ($r$ Stück)
  und den Paaren $\tau,\overline{\tau}:K\rightarrow\IC$ ($s$ Stück).
  \item Die Fortsetzungen von $|\cdot|_p$ sind in Bijektion zu den Primidealen $\Fp\subset\cO_K$ mit $\Fp\mid p$ ($\gdw \Fp\cap\IZ=(p)$)
  \item Die Bijektion in (ii) ist $|x|_\rho:=|\rho(x)|$.
  \item Die Bijektion in (iii) ist $|x|_\Fp:=p^{-\ord_\Fp(x)/e(\Fp/p)}$.
 \end{enumerate}
\end{Satz}

\begin{Beweis}
 \begin{itemize}
  \item Sei $|\cdot|$ nichttriviale Bewertung von $K$. Wir zeigen, dass die Einschränkung auf $\IQ$ auch nichttrivial ist. Sei dazu $\omega_1,\ldots,\omega_n$ $\IQ$-Basis von $K$. Sei $|\alpha|=1$ für alle $\alpha\in\IQ\kreuz$. Dann folgt $|x|\leq C$ für alle $x\in K\kreuz$ (Standardargument: Folge von Potenzen betrachten). Daraus folgt $|x|=1$ für alle $x\in K\kreuz$. Der Satz von OSTROWSKI zeigt (i).
  \item Jede der Einlagerungen $\rho:K\rightarrow \IR$ oder $\IC$ definiert Fortsetzung von $|\cdot|_\infty$. Wir zeigen, dass sie (abgesehen von komplex konjugierten) inäquivalent sind. $j:K\rightarrow K_\IR$ hat dichtes Bild, da $j(\cO_K)$ Gitter ist: $j(K)=\bigcup_{m\in\IN\oN} \frac{1}{m} j(\cO_K)$.
  
  Sei $|\sigma x|=|\tau x|^s$ für alle $x\in K$, $x\in \IQ$ zeigt $s=1$. Sind $\sigma,\tau$ verschieden (auch nicht komplex konj.), so ex. Folgt $(x_n)$ aus $K$ mit $\sigma x_n\rightarrow 0$ und $\tau x_n\rightarrow 1$ (wegen der Dichtigkeit).
  
  Also $|\sigma x_n|\rightarrow 0$, $|\tau x_n|\rightarrow 1$ \lightning.
  
  Sei nun $|\cdot|$ eine Bewertung von $K$, welche $|\cdot|_\infty$ fortsetzt und sei $\hat K$ die Komplettierung. Dann haben wir $\hat\IQ=\IR\subset \hat K$, $\hat K=\IR\cdot K$ mit $[\hat K: \IR]$ endlich.
  
  $\IR$ und $\IC$ besitzen nur eine Fortsetzung, welche $|\cdot|_\infty$ von $\IQ$ fortsetzen. Für $\IR$ ist das klar, da $\IQ\subset \IR$ dicht ist. Für $\IC$ so: Alle Normen als $\IR$-VR sind äquivalent, d.h. $\|\cdot\|_1\leq C\cdot \|\cdot\|_2$ und $\|\cdot\|_2\leq C'\cdot\|\cdot\|_1$, also auch alle Bewertungen, die $|\cdot|_\infty$ auf $\IR$ fortsetzen. Somit sind sie gleich. Also kommt jede solche Bewertung von $K$ von einer Einlagerung $\rho:K\rightarrow \IR$ oder $\IC$
  \item Sei $|\cdot|$ nichtarchimedische Bewertung von alg. ZK $K$. $|\cdot|$ setze $|\cdot|_p$ auf $\IQ$ fort. Sei
  \[ \cO:=\{ x\in K : |x|\leq 1\},\]
  \[ \Fm:=\{ x\in K : |x|<1\}.\]
  Dann ist $\Fm$ maximales Ideal: Jedes Element aus $\cO\setminus\Fm$ ist Einheit. $\cO_K\cap\Fm$ ist nichttriviales Primideal in $\cO_K$: $p\in\cO_K\cap\Fm$. Also ist $\cO_K\cap \Fm=\Fp$ maximales Ideal in $\cO_K$.
  Wir haben $K\subset K_\Fp = $ Vervollständigung von $K$ in $|\cdot|_\Fp$, \[\cO_K\subset S_\Fp\inv\cO_K \subset \cI\subset \cO_\Fp,\] 
  $S_\Fp=\cO_K\setminus\Fp$ Lokalisierung,
  \[ \Fp\subset S\inv\Fp \subset \Fm \subset \Fm_\Fp.\]
  Klar ist $S\inv\cO_K\subset \cO$. Sei $x\notin S\inv\cO_K$, dann folgt $\ord_\Fp(x)<0$ \folge $\ord_\Fp(x\inv)>0$ \folge $x\inv S\inv\Fp\subset S\inv\Fm=\Fm$ \folge $|x\inv|<1$ \folge $|x|>1$ \folge $x\notin \cO$. Somit $S\inv\cO_K=\cO$.
  
  $S\inv\cO_K$ ist dBR, sein maximales Ideal ist also HI: $S\inv\Fp=(\pi)$. $K$ ist der QK von $S\inv\cO_K$ (Kapitel I). Es folgt $|\cdot|=|\cdot|_\Fp$:
  $\Fm=S\inv\Fp=(\pi)$, alsi ist $\pi$ für $|\cdot|$ ein Element maximaler Bewertung $<1$. Jedes $x\in K\kreuz$ hat die Form $x=u\cdot \pi^m$, $m\in\IZ$, $u$ Einheit.
  \folge $|x|=|\pi|^m$. $p=\nu\pi^e$ mit $e=e(\Fm/p)$ \folge $|p|=\frac{1}{p}=|\pi|^e$ \folge $|\pi|=\frac{1}{p^{1/e}}$ \folge $|x|=\frac{1}{p^{m/e}}$, $m=\ord_\Fp(x)$.
  \end{itemize}
\end{Beweis}





