\renewcommand{\lecdate}{16.12.2014}

  \begin{Folgerung}
   Sei $L/K$ endliche Erweiterung von ZK, dann in $\cO_K$ nur endlich viele Primideale verzweigt in $\cO_L$.
  \end{Folgerung}

  \begin{Beweis}
   Mult. des Verzw.index.
  \end{Beweis}

  \begin{Folgerung}
   In $K/\IQ$, $K\neq \IQ$ gibt es stets verzweigte Primzahlen.
  \end{Folgerung}
  
  \begin{Beweis}
   $|d_K|>1$.
  \end{Beweis}
  
  \begin{Beispiel}[E. ARTIN]
   $f=T^5-T+1\in\IZ[T]$. Diskriminante: $d(T^5+a^T+b)=A\cdot a^5+B\cdot b^4$.
   
   Gesucht: $A,B$. Wähle $a=-1$, $b=0$. Wurzeln sind $0, \pm 1, \pm i$. Dann ist $d$\footnote{Produkt aller Differenzen von Wurzeln zum Quadrat}$=(1-i)^2 2^2 (1+i)^2 (-1+i)^2 (2i)^2(-1-i)^21^2i^2(-1)^2(-i)^2=-2^8=-A$ \folge $A=2^8$. Wähle $a=0$, $b=-1$. Wurzeln sind $\zeta_5^r=e^{2\pi ir/5}$. Damit $d=\det(\sum_{k=0}^5 \zeta_5^{ik}\zeta_5^{kj})=\det\begin{pmatrix}
                                                      5 & 0 & 0 & 0 & 0\\ 0 & 0 & 0 & 0 & 5\\  0 & 0 & 0 & 5 & 0 \\ 0 & 0 & 5 & 0 & 0\\ 0 & 5 & 0 & 0 & 0                                                                                                                                                                                                                                                                                                          
                                                                                                                                                                                                                                                                                                                                                               \end{pmatrix}=5^5$
                                                                                                                                                                                                                                                                                                                                                               
$d(T^5+aT+b)=2^8a^5+5^5b^4$, $d(T^5-T+1)=5^5-2^8=3125-256=19\cdot 151$.
\begin{enumerate}
 \item $f$ ist irred. modulo $5$.
 
 $f$ hat keine Wurzeln in $\IF_5$, aber auch keine in $\IF_25$: Sei $\omega^2=2$, $\omega\in\overline{\IF_5}$, dann ist $\IF_25=\{ \alpha+\omega\beta : \alpha,\beta\in\IF_5\}$. Sei $\theta=\alpha+\omega\beta$ Wurzel von $f$, $\theta^5=a^5+\beta^5\omega^5=\alpha+4\beta\omega$, also 
 $0=\theta^5-\theta=\alpha+4\beta\omega-\alpha-\beta\omega+1=1+3\beta\omega$ $\lightning$ (denn $3\beta\omega=-1$ impliziert $\omega\in\IF_5$ \lightning).
 
 Also ist $f$ auch irreduzibel in $\IZ[T]$. Sei $\theta\in\overline{\IQ}$ Nullstelle von $f$ und $K=\IQ(\theta)$, dann ist $[K:\IQ]=5$. Weiter ist $\cO_K=\IZ[\theta]$, da $disc(1,\theta,\theta^2,\theta^3,\theta^4)=disc(f)=19\cdot151$ quadratfrei ist.
 \item $f$ modulo $2$.
 
 $\overline{f}(T)=(T^2+T+1)(T^3+T^2+1)$. Die Faktoren sind irreduzibel, da ohne Nullen in $\IF_2$. Es gilt $\cO_K=\IZ[\theta]=\IZ[T]/(f)$ \folge $\cO_K/2\cO_K=\IF_2[T]/(\overline{f})=\IF_2[T]/(\overline g) \oplus \IF_2[T]/(\overline h)=\IF_4\oplus \IF_8$. \folge $2\cO_K=\Fp_2\cdot\Fp_2'$, $f(\Fp_2)=2$, $f(\Fp_2')=3$, $\IN\Fp_2=4$, $\IN\Fp_2'=8$.
 
 \item $f$ modulo $3$
 $\overline{f}\in\IF_3[T]$ ist irreduzibel: keine Wurzeln in $\IF_3$, auch keine in $\IF_9=\{ \alpha + \beta\omega: \alpha,\beta\in\IF_3\}$, $\omega\in\overline{\IF_3}$, $\omega^2=2$. Sei $\theta=\alpha+\beta\omega$ Wurzel, also $\theta^5-\theta+1=0$ \folge $\theta^6=\theta^2-\theta$.
 
 \begin{align*}
  \theta^6 = (\theta^2)^3&=(\alpha^2+2\alpha\beta\omega+2\beta^2)^3\\
  &= \alpha^2+ 2\beta^2+ 4\alpha\beta\omega\\
  \theta^2-\theta & = \alpha^2+2\beta^2+2\alpha\beta\omega-\alpha-\beta\omega\\
  \folge & 4\alpha\beta\omega = -\alpha+(2\alpha-1)\beta\omega\\
  \folge & a=0, 2\alpha\beta-\beta-4\alpha\beta=0 \folge \beta=0 \lightning
 \end{align*}

 Somit ist $\cO_K/3\cO_K=\IF_3[T]/(\overline{f})$ Körper $=\IF_{3^5}=\IF_{243}$ \folge $3\cO_K=\Fp_3$, $f(\Fp_3/3)=5$, $\IN\Fp_3=3^5=243$
 \item $f$ über $\IR$
 
 $f'(T)=5T^4-1$ hat genau $2$ reelle Nullstellen $\pm\frac{1}{\sqrt[4]{5}}$. Also liegen dort relative Extrema von $f$. $f''(T)=20T^3$. \folge bei $-\frac{1}{\sqrt[4]{5}}$ rel. Maximum und bei $\frac{1}{\sqrt[4]{5}}$ rel. Minimum. Schaut man sich deren Lage und das Verhalten im unendlichen an, wird klar: $f$ hat genau eine reelle Nullstelle. Also $\rk(f)=1$, $s(K)=2$. Die MINKOWSKI-Konstante ist also 
 \[ \frac{5!}{5^5} \left( \frac{4}{\pi}\right)^2\sqrt{19\cdot151 < 4}.\]
 Also wird $\Cl_K$ erzeugt von ganzen Idealen der Absolutnorm $<4$. Es gibt keine ganzen Ideale der Norm $2$ oder $3$ nach Punkt $2$ und $3$. Also ist $\Cl_K=1$ und $h_K=1$.
 
 Man kann zeigen: Die GALOIS-Gruppe dieses Polynoms ist $\Gal(f)=S_5$. Sei $E$ der Zerfällungskörper von $f$ über $\IQ$, dann liegt $\IQ(\sqrt{19\cdot 151})$ in $E$. ARTIN hat gezeigt: $E/IQ(\sqrt(19\cdot 151))$ ist unverzweigt. Das ist eine unverzweigte $A_5$ Erweiterung. 
\end{enumerate}

  \end{Beispiel}


