\renewcommand{\lecdate}{09.12.2014}

 \begin{Folgerung}
  Seu $A$ DED-Ring, $\Fp\subset A$, $\Fp\neq(0)$ Primideal, $S=A\setminus\Fp$. Dann ist $A_\Fp$ ein DED-Ring mit genau einem maximalen Ideal $\Fm_\Fp=\Fp\cdot A_\Fp$. $A_\Fp$ ist HIR, jedes gebrochene Ideal hat die Form $\Fm_\Fp^m$, $m\in\IZ$ (falls Exponent nicht negativ, dann echtes Ideal).
 \end{Folgerung}
 
 \begin{Beweis}
  Jedes Primideal $\Fq\neq\Fp$ schneidet $S$. Also $\Id(A_\Fp)=\langle m_\Fp\rangle$. Sei $\pi\in m_\Fp\setminus m_\Fp^2$, dann ist $m_\Fp^2\subsetneq (\pi)\subset m_\Fp$ \folge $m_\Fp=(\pi)$.
 \end{Beweis}
 
 \begin{Definition}
  Ein lokaler Ring, der HIR ist, heißt \highl{diskreter Bewertungsring}.
 \end{Definition}
 
 \begin{Beispiel}
 \begin{enumerate}
  \item $\IZ_{(p)}=\{ \frac{a}{b}: a\in\IZ, b\in\IN\oN, \ggT(a,b)=1, p\nmid b\}$.
  \item Diskrete Bewertungsringe sind nach den Körpern die einfachsten Ringe.
 \end{enumerate}
 \end{Beispiel}

 \begin{Bemerkung}
  Sei $O$ dBR, $\Fm=(\pi)$ sein maximales Ideal. Dann läss sich jedes $x\in K\kreuz$ ($K=QK(O)$) eindeutig schreiben als $x=\pi^m\cdot u$, $m\in\IZ$, $u\in O\kreuz$.
  Damit hat man 
  \[ \nu_\Fm : K\kreuz \rightarrow \IZ : x\mapsto m=\ord_\pi(x).\]
  Das ist eine sogenannte Exponentenbewertung auf $K$:
  \begin{enumerate}
   \item $\nu(xy)=\nu(x)+\nu(y)$
   \item $\nu(x+y)\geq \min(\nu(x),\nu(y))$
  \end{enumerate}
 \end{Bemerkung}

 \begin{Fakt}
  Sei $A$ ein DED-Ring, dann gilt in $K=QK(A)$: $A=\bigcap\limits_{\Fp}A_\Fp$. 
 \end{Fakt}
 
 \begin{Beweis}
  Klar ist $A\subset \bigcap A_\Fp$, sei also $0\neq x\in\bigcap A_\Fp$. Es gilt $(x)=\Fp_1^{m_1}\cdot\ldots\cdot\Fp_r^{m_r}$ für gewisse $m_j\in\IZ$ und $\Fp_i\neq\Fp_j$ für alle $i\neq j$. Also $xA_{\Fp_i}=\Fm_i^{m_i}$ \folge $m_i\geq 0$. Somit $x\in A$.
 \end{Beweis}

\begin{Fakt}
 Hat der DED=Ring $A$ nur endlich viele Primideale, so ist er ein HIR.
\end{Fakt}

\begin{Beweis}
 Seien $\Fp_1,\ldots,\Fp_N$ alle maximalen Ideale in $A$, $\pi\in\Fp_1\setminus\Fp_1^2$. Wähle $x\in A$ mit $x\equiv \pi\mod{\Fp_1^2}$, $x\equiv 1\mod{\Fp_j}$
 für alle $j\in 1,\ldots, N$.
 Dann ist $(x)=\Fp_1$.
 \end{Beweis}
 
 Nun wieder Arithmetik: Sei $L/K$ eine endliche Erweiterung von ZK, mit $\Fp\subset\cO_K$ maximal ist $\Fp\cO_L=\Fp_1^{e_1}\cdot\ldots\cdot\Fp_g^{e_g}$

Man studiert Zerlegungsverhalten von Primidealen in Erweiterungen alias Reziprozitätsgesetze.

\begin{Fakt}
 Sei $L/K$ endliche, separabel, $A\subset K$ DED-Ring mit QK $K$, $B$ der ganze Abschluss von $A$ in $L$. Dann ist auch $B$ ein DED-Ring.
\end{Fakt}

\begin{Beweis}
 $B$ ist ganzabgeschlossen, denn Ganzheit ist transitiv. 
 
 Sei $\FP\neq (0)$ Primideal in $B$. Dann ist $\FP\cap A = \Fp$ ein Primideal in $A$, denn $A/\Fp \rightarrow B/\FP$ ist injektiv.
 Wir zeigen: $\Fp\neq (0)$. Ist $x\in\FP\oN$, so gilt 
 \[ \Irr(T,x,K)=T^n+a_1T^{n-1}+\ldots+a_n, \hspace{1cm} a_j\in A.\]
 Dieses Polynomm ist irreduzibel über $K$, also ist $a_n\neq 0$ und $a_n\in\FP\cap A$: $a_n=-x^n-a_1x^{n-1}-\ldots-a_{n-1}x\in\FP$. $B/FP$ ist integer. Sei $M$ maximales Ideal in $B/\FP$, $\Fm=M\cap A/\Fp$. Dies ist nicht $A/\Fp$, da sonst $1\in M$. $A/\Fp$ ist Körper, also ist $\Fm=(0)$. Ist $M\neq(0)$, so haben wir also $\overline x\in M$, $\overline{x}\neq 0$, $x\in B$ Urbild. $x$ ist ganz über $A$, also
 $x^n+a_1x^{n-1}+\ldots+a_n=0$, $a_j\in A$. Also $\overline{x}^n+\overline{a_1}\overline{x}^{n-1}+\ldots+\overline{a_n}=0$, d.h. $\overline{x}$ ist algebraisch über $A/\Fp$. Bilde $\Irr(T,\overline{x},A/\Fp)=
 T^k+\alpha_1 T^{k-1}+\ldots + \alpha_k$, $\alpha_j\in A/\Fp$. Dann ist $\alpha_k\neq 0$, wegen Irreduzibilität. $\alpha_k$ liegt in $M$: $\alpha_k=-\overline{x}^k-\alpha_1\overline{x}^{k-1}-\ldots -\alpha_{k-1}\overline{x}\in M$. Also liegt $\alpha_k$ in $M\cap A/\Fp=(0)$ \lightning. Somit ist $\FP$ maximal.
 
 Nun: $B$ ist NOETHERsch. Sei $\omega_1,\ldots,\omega_n$ $K$-Basis von $L$, man darf annehmen $\omega_j\in B$. Sei $\omega_1',\ldots,\omega_n'$ duale Basis unter der Spurform $(\Tr_K^L(\omega_i\omega_j')=\delta_{ij}).$
 Man findet $c\in A$, $c\neq 0$, s.d. $c\omega_1',\ldots, c\omega_n'$ aus $B$ sind. Sei $x\in B$, $x=a_1\omega_1+\ldots +a_n\omega_n$. Dan ist $a_j=\Tr(x\omega_j')$, also $ca_j=\Tr(cx\omega_j')\in A$. Somit $B\subset Ac\inv\omega_1+\ldots + Ac\inv\omega_n$. $B$ ist also Untermodul eines endlich erzeugten $A$-Moduls. Ist $A$ NOETHERsch, so ist jeder Untermodul eines endlich erzeugten Moduls selbst endlich erzeugt (Algebra 1).
\end{Beweis}

\begin{Bemerkung}
 $B$ ist endlich erzeugter $A$-Modul, torsionsfrei, aber im Allgemeinen nicht frei.
\end{Bemerkung}

\begin{Definition}
 Sei $A$ ein DED-Ring, $\Fp\subset A$ maximal. Ein maximales Ideal $\FP\subset B$ \highl[liegt über - Relation]{liegt über} $\Fp$ genau dann, wenn $\FP\cap A = \Fp$.
\end{Definition}

\begin{Fakt}
 \begin{enumerate}
  \item $\FP\cap A=\Fp$ ist maximales Ideal in $A$.
  \item Für jedes $\Fp$ existieren $\FP$ mit $\Fp\cap A=\Fp$,
  \item deren Anzahl ist endlich.
 \end{enumerate}
\end{Fakt}

\begin{Beweis}
 \begin{enumerate}
  \item $\Fp=\FP\cap A$ ist jedenfalls Primideal: $A/\Fp \rightarrow B/\FP$ ist Einbettung. Ist $\Fp$ nicht maximal, so bleibt nur $\Fp=(0)$. Aber $\FP\neq (0)$ enthält Elemente aus $A\setminus 0$. $x\in\FP$, $\Irr(T,x,K)=T^n+\ldots+a_n$, $a_n\neq 0$, $a_n\in \FP\cap A$.
  \item Wir zeigen $\Fp B\neq B$. Sei dazu $\pi\in\Fp\setminus\Fp^2$, also $(pi)=\Fp\Fa$ mit $\Fp\nmid\Fa$. Also sind $\Fp$ und $\Fa$ teilerfremd. $\Fp\subsetneq \Fp+\Fa$, da sonst $\Fa\subset\Fp$ \folge $\Fp\mid \Fa$. Also $\Fp+\Fa=A$, somit existieren $p\in\Fp$, $a\in\Fa$ mit $p+a=1$. Dabei ist $a\in\Fp$, da sonst $1\in\Fp$ \lightning. Weiter gilt $a\Fp\subset\Fa\Fp=\pi A$. Angenommen $\Fp B= B$, dann folgt $aB=a\Fp B\subset \Fa\Fp B = \pi B$. Also $a=\pi b$, $b\in B$. Außerdem $a\in A$, $\pi\in A$, also $b\in K$. Es gilt $B\cap K=A$ ($A$ ganzabgeschlossen). Also ist $b\in A$. Aus $a=\pi b$ folgt $a\in \Fp$ \lightning.
  Damit ist $\Fp B\neq B$. Also existieren maximale Ideale $\FP$ mit $\Fp B\subset \FP\subsetneq B$. $\FP\cap A$ enthält $\Fp$ und ist Primideal in $A$, ist also gleich $\Fp$.
  \item Die Endlichkeitsaussage: $\Fp B= \FP_1^{e_1}\cdot\ldots\cdot \FP_g^{e_g}$. Sei $\FP\subset B$ maximal und $\FP\cap A=\Fp$. Es folgt $\Fp\subset\Fp$, also $\Fp B\subset \Fp$, also 
  $\FP_1^{e_1}\cdot\ldots\cdot \FP_g^{e_g}\subset \FP$ \folge $\FP \mid \FP_1^{e_1}\cdot\ldots\cdot \FP_g^{e_g}$ \folge $\FP=\FP_i$.
 \end{enumerate}
\end{Beweis}

\begin{Bemerkung}
 \begin{enumerate}
  \item In DED-Ringen gilt $\Fa\subset\Fb$ \gdw $\Fb\mid\Fa$, d.h. es existiert $\Fc$ ganz mit $\Fa=\Fb\Fc$ ($\Fc=\{ x\in K : x\Fb\subset \Fa\}\subset A$).
  \item $\FP\mid \Fp$ :\gdw $\FP\mid \Fp B$
 \end{enumerate}
\end{Bemerkung}

\begin{Definition}
 $\Fp B = \FP_1^{e_1}\cdot\ldots\cdot \FP_g^{e_g}$, $e(\FP/\Fp)$ heißt \highl{Verzweigungsindex}. Der Grad der Körpererweiterung $[B/\FP : A/\Fp]$ heißt \highl{Restklassengrad} $f(\FP/\Fp)$.
 \begin{itemize}
  \item $\Fp$ heißt \highl{verzweigt} in $L$ :\gdw $e(\FP/\Fp)=1$ für alle $\FP\mid\Fp$
  \item $\Fp$ heißt \highl{voll zerlegt} in $L$ :\gdw $\Fp$ unverzweigt und $f(\FP/\Fp)=1$ für alle $\FP\mid\Fp$
  \item $\Fp$ heißt \highl{träge} in $L$ :\gdw $\Fp$ ist unverzweigt in $L$ und hat in $L$ nur einen Primteiler, d.h. $\Fp B=\FP$ maximal.
  \end{itemize}
\end{Definition}

\begin{Bemerkung}
 Ist $A$ DED-Ring mit QK $K$, $L$ endlich, separable Erweiterung, $B$ der ganze Abschluss von $A$ in $L$, $\Fa\subset A$ Ideal $\neq (0)$. Dann gilt $\Fa B\cap A=\Fa$.
 
 Denn $\Fa\subset \Fa B \cap A$ ist klar, sei $\Fa=\Fp^r\cdot\Fq$, $\Fp\nmid\Fq$. $b:=\Fa B\cap B$, $\Fb=\Fp^sq'$, $\Fp\nmid\Fq$, dann ist $s\leq r$. Wir haben zu zeigen $r=s$. Lokalisieren mit $S=A\setminus\Fp$ liefert die Situation $A$ dBR, $\Fa=\Fp^r=(\pi^r)$, $\Fa B=\pi^r B=\{ \pi^r B : b\in B\}$, $\Fa B\cap A=\{ \pi^r b : b\in B\}\cap A$. $\pi^r b\in K$ \gdw $b\in K$, also $\Fa\Fb\cap A=\pi^r A$ \folge $r=s$. Folgelich ist $\Id(K)\rightarrow \Id(L)$ injektiv: $\Fa B=\Fb B$ \folge $\Fa B\cap A=\Fb B\cap A \folge \Fa=\Fb$.
\end{Bemerkung}

\begin{Satz}[\glqq Satz 7\grqq{}]
 Seien $L/K$, $A$, $B$ wie oben, $\Fp\subset A$ maximal. Dann gilt:
 \[ [L:K] = \sum_{\FP\mid\Fp} e(\FP/\Fp)f(\FP/\Fp)\]
\end{Satz}
