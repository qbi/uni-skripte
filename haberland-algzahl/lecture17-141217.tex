\renewcommand{\lecdate}{17.12.2014}

\subsection{Beispiele zum Zerlegungsverhalten}

Der Prototyp ist hierbei $\IQ(i)/\IQ$.

\begin{Satz}[KUMMER]
 Sei $L/K$ eine endliche Erweiterung von Zahlkörpern, $p(X)=\Irr(X,\theta,K)$ und $\cO_L=\cO_K[\theta]$\footnote{Wir haben gesehen, dass so ein Element nicht immer existiert. Das ist also eine gewisse Einschränkung.}, $\Fp\subset\cO_K$ Primideal $\neq (0)$, $\overline p(X)$ die Reduktion von $p(X)$ modulo $\Fp$, also $\overline{p}(X)\in\cO_K/\Fp[X]$. Sei $\overline{p}(X)=\overline{p_1}(X)^{e_1}\cdot\ldots\cdot \overline p_g^{e_g}$ die Zerlegung in irreduzible Polynome. Seien $p_i(X)$ Liftungen der $\overline p_i(X)$ zu unitären Polynomen in $\cO_K[X]$ (Koeffizienten bei höchster Potenz von $X$ ist 1). Dann sind
 \[ \FP_i=\Fp\cO_L + p_i(\theta)\cO_L\]
 die verschiedenen über $\Fp$ liegenden Primideale in $\cO_L$. Es gilt $e(\FP_i/\Fp)=e_i$, $f(\FP_i/\Fp)=\deg \overline{p}_i(X)$ und $\Fp\cO_L=\FP_1^{e_1}\cdot\ldots\cdot\FP_g^{e_g}$.
\end{Satz}

\begin{Beweis}
 $\cO_K[X]\rightarrow\cO_K/\Fp[X]$: $f\mapsto \overline{f}$ hat als Kern $\Fp\cO_K[X]$.
 
 $\cO_K[X]\rightarrow\cO_K/\Fp[X]\rightarrow \cO_K/\Fp[X]/(\overline{p}(X))$ hat als Kern $\Fp\cO_K[X]+(p(X))$: Dieses Ideal wird auf $0$ abgebildet, geht $f$ auf $0$, so ist $\overline{f}=\overline{p}\cdot\overline{q}$ \folge $f-pg\in\Fp\cO_K[X]$. Der Homomorphismus ist surjektiv.
 
 Man hat einen Isomorphismus $\cO_K[X]/(p(X))\rightarrow \cO_K[\theta]$: $f\mapsto f(\theta)$: Der Kern von $\cO_K[X]\rightarrow\cO_K[\theta]$ besteht aus den $f$ mit $f(\theta)=0$. Also $f=g\cdot p$ mit $g\in K[X]$. \folge $g\in\cO_K[X]$, da $p$ unitär ist (Polynomdivision). \folge $f\in (p(X))$. Unter diesem Isomorphismus geht $\Fp\cO_K[X]+(p(X))$ auf $\Fp\cO_K[\theta]$. Nach Voraussetzung ist $\cO_K[\theta]=\cO_L$. Also haben wir einen Isomorphismus $\cO_L/\Fp\cO_L \cong (\cO_K/\Fp[X])/(\overline{p}(X))$. Also
 \[ \cO_L/\Fp\cO_L = \cO_K[\theta]/\Fp\cO_K[\theta] \cong \cO_K[X]/(\Fp\cO_K+(p(X))) \cong (\cO_K/\Fp[X])/(\overline{p}(X))\]
 Chinesischer Restsatz: RHS ist isomorph zu $\bigoplus \cO_K/\Fp[X]/(\overline{p}_i(X)^{e_i})$, die Elementezahl $(\IN\Fp)^n$ mit $n=[L:K]=\deg p(X)$.  Ist $\overline{p}(X)=\prod_{j=1}^g \overline{p}_j(X)^{e_j}$, so sieht man rechts alle Primideale. Es sind genau die Hauptideale $(\overline p_j(X))$, ihre Bilder links sind die HI $(\overline p_j(\theta)=:\overline\FP_j)$. Der Körpergrad 
 \[ [(\cO_L/\Fp\cO_L)/\overline{\FP}_j:\cO_K/\Fp]=\deg \overline{p}_j(X),\]
 denn $(\cO_L/\Fp\cO_L)/\overline{\FP}_j=\cO_K/\Fp[X]/(\overline{p}_j(X))$. Sei $\FP_j$ das Urbild von $\overline\FP_j$ unter $\cO_L\rightarrow\cO_L/\Fp\cO_L$. Dann ist $\FP_j=\Fp\cO_L+p_j(\theta)\cO_L$.
 \begin{align*}
  \prod_{j=1}^g \FP_j^{e_j}&=\prod_j (\Fp\cO_L+p_j(\theta)\cO_L)^{e_j}\\
  &\subseteq \Fp\cO_L + \underbrace{\prod p_j(\theta)^{e_j}\cdot\cO_L}_{\equiv p(\theta)\equiv 0 \mod{\Fp\cO_L}}
 \end{align*}

 \folge $\prod \FP_j^{e_j}\subset\Fp\cO_L$.
 
 Also teilt $\Fp\cO_L$ das Ideal $\prod_{j=1}^g\FP_j^{e_j}$. Mithin $\Fp\cO_L=\prod \FP_j^{k_j}$, $k_j\leq e_j$.
 \[ (\cO_L/\Fp\cO_L)/\overline\FP_j=(\cO_L/\Fp\cO_L)/(\FP_j/\Fp\cO_L)=\cO_L/\FP_j\]
 Also $f(\FP_j/\Fp)=f_j=\deg\overline{p}_j(x)$.  $(\cO_L:\Fp\cO_L)=\IN(\Fp\cO_L)=\prod (\IN\FP_j)^{k_j}=(\IN\Fp)^{\sum f_jk_j}$
 
 Andererseits ist $\card k[X]/(\overline p(X))=\prod \card k[X]/(\overline p_j(X))^{e_j}$ ($k=\cO_K/\Fp$) $=(\IN\Fp)^{\sum e_jf_j}$ \folge $k_j=e_j$. 
\end{Beweis}

\begin{Bemerkung}
 $K=\IQ(\sqrt d)$, $d\in\IZ$, $d\neq 0,1$ sqf.
 
 Man kann zeigen:
 \begin{align*}
  p \text{ zerlegt} & \gdw \left(\frac{d_K}{p}\right)=1\\
  p \text{ träge} & \gdw \left(\frac{d_K}{p}\right)=-1\\
 \end{align*}
\end{Bemerkung}

\begin{Satz}[\glqq Satz 9\grqq{} QRL - Quadratisches Reziprozitätsgesetz]
 $p,l$ ungerade PZ, $p\neq l$, dann gilt: \[ \left(\frac{p}{l} \right)\left(\frac{l}{p}\right)=(-1)^{\frac{p-1}{2}\frac{l-1}{2}}\]
\end{Satz}

\begin{Folgerung}
 Das Zerlegungsverhalten von $p$ in $\cO_K$ hängt nur von $p$ modulo $4|d_K|$ ab. \folge der ARTIN-Führer ist $4|d_K|$.
\end{Folgerung}

\begin{Beispiel}[Kreisteilungskörper]
 Wir manchen es uns leichter und betrachten nur $K=\IQ(\zeta_p)$, $p>2$ Primzahl, $\zeta_p=e^{2\pi i/p}$. $K/\IQ$ ist GALOIS-Erweiterung, $\Gal(K/\IQ)=(\IZ/p\IZ)\kreuz$ via $a\mapsto \sigma_a$, $\sigma_a\zeta_p=\zeta_p^a$, $\cO_K=\IZ[\zeta_p]$, $d_K=(-1)^{\frac{p-1}{2}}p^{p-2}$. Also nur $p$ verzweigt.
\end{Beispiel}

\begin{Fakt}
 Sei $l\neq p$ Primzahl, dann gilt $l\cO_K=\cL_1\cdot\ldots\cdot\cL_g$ mit $f(\cL_i/l)=f$, $fg=\varphi(p)=p-1$. Dabei ist $f$ die Ordnung von $l$ in $\IF_p\kreuz$. Insbesondere ist $l$ voll zerlegt \gdw $l\equiv 1 \mod{p}$ und $l$ träge \gdw $l$ ist Primitivwurzel modulo $p$ (\gdw $l$ erzeugt $\IF_p\kreuz$). Somit hängt das Zerlegungsverhalten von $l$ in $\cO_K$ nur von $l$ (modulo $p$) ab.
 \end{Fakt}

\begin{Beweis}
 Das Zerlegungsverhalten von $l$ in $\cO_K$ liest man ab aus dem von $p(X)=\frac{X^p-1}{X-1}$ in $\IF_l[X]$ (Satz 8), resp. dem von $X^p-1$ in $\IF_l[X]$. $\overline p(X)=X^p-1$ ist separabel über $\IF_l$, wg. $\overline p'(X)=pX^{p-1}$. Der Zerfällungskörper von $\overline p$ über $\IF_l$ ist die kleinste Erweiterung $\IF_{l^f}$ von $\IF_l$, welche die $p$-ten Einheitswurzeln enthält. Da $\IF_{l^f}\kreuz$ zyklisch ist, geschieht das genau dann, wenn $e^f-1$ durch $p$ teilbar ist, wenn also $f$ die Ordnund von $l$ in $\IF_p\kreuz$ ist.
\end{Beweis}

\begin{Bemerkung}
 \begin{enumerate}
  \item Wie verhalten sich die Primzahlen auf Restklassen? Ist $a$ prim zu $m$ ($m\geq 2$), wieviele Primzahlen sind $\equiv a\mod{p}$.
  
  DIRICHLET: In jeder Restklasse liegen unendlich viele Primzahlen. Sogar Summe der Reziproken ist $\infty$. $\sum_{p\in\IP}\frac{1}{p}=\infty$. DIRICHLET: $\sum_{p\equiv a\mod{m}} \frac{1}{p}=\infty$.
  \item Sei $f\in\IZ[X]$ irreduzibel, betrachte $f_p\equiv f\mod{p}$, also $f_p\in\IF_p[X]$.
  
  Ist $\Gal(f)\subset S_n$ ($n=\deg f$) abelsch, so zeigt die Klassenkörpertheorie: $\exists a\in\IN\oN$ (ARTIN-Führer), s.d. die Anzahl der irred. Faktoren von $f_p$ nur von $p\mod a$ abhängt.
 \end{enumerate}
\end{Bemerkung}

