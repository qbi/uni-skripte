\renewcommand{\lecdate}{02.12.2014}

\begin{Beweis}
 Sei $S\subset \Hom(K,\IC)$ wie oben.
 \begin{enumerate}
  \item Fixiere $\rho_0\in S$, $x\in\cO_K$, $x\neq 0$.
  
  Wir zeigen: Es existiert $y\in\cO_K$, $y\neq 0$, s.d. 
  \begin{enumerate}
   \item $|\N_\IQ^K(y)|\leq \left(\frac{4}{\pi}\right)^s\sqrt{|d_K|}$
   \item $\log |\rho y|^{\N_\rho} < \log|\rho x|^{\N_\rho}$ für alle $\rho\neq\rho_0$.
  \end{enumerate}
  Sei dazu $\Omega\subset\IR^n$ ($n=r+2s=[K:\IQ]$) definiert durch $|x_\sigma|<c_\sigma$, $\sigma$ reell, $y_\tau^2+z_\tau^2\leq c_\tau^2$, $\tau\in S$ komplex. Und die $c$ so, dass $0<c_\rho<|\rho x|^{\N_\rho}$ für alle $\rho\neq\rho_0$ und $\prod_{\rho\in S} c_\rho = \left(\frac{4}{\pi}\right)^s\sqrt{|d_K|}$. Dann ist $\vol(\Omega)=2^r\pi^sc_1\ldots c_{r+s}=2^n\vol(j(\cO_K))$. Nach Satz oben: Es ex. $j(y)\neq 0$ in $\Omega$.
  \item Wir zeigen: Es existieren Einheiten $\eps\in E_K$ mit $\log|\rho\eps|^{\N_\rho}<0$ für alle $\rho\neq\rho_0$. Dazu starten wir mit $\alpha_1\in\cO_K$, $\alpha_1\neq 0$, nach (i) finden wir $\alpha_2,\ldots,\alpha_n\in\cO_K\oN$, s.d.
  \begin{enumerate}
   \item $\log|\rho\alpha_{j+1}|<\log|\rho\alpha_j| $ für alle $\rho\neq\rho_0$.
   \item $|\N_\IQ^K(\alpha_j)|\leq c$
  \end{enumerate}
  \folge $\IN((\alpha_j))\leq c$. Solche Ideale gibt es nur endlich viele, also existieren $k\neq l$, s.d. $(\alpha_k)=(\alpha_l)$. Es folgt $\alpha_k=\eps\alpha_l$ mit $\eps\in E_K$. Dann gilt \[\log|\rho\alpha_k|=\log|\rho\eps|+\log|\rho\alpha_l|. \] Für $k>l$ folgt $\log|\rho\eps|<0$ für $\rho_0\neq\rho$.
  \item Wir finden also Einheiten $\eps_rho$, $\rho\in S$, s.d. in $\lambda(\eps_\rho)$ alle Koordinaten negativ sind, bis auf die $\rho$-te.
 \end{enumerate}
  \end{Beweis}

  \begin{Lemma}
   Sei $A$ reelle $m\times m$-Matrix mit
   \begin{enumerate}
    \item $a_{kl}<0$ für alle $k\neq l$
    \item Alle Zeilensummen sind $=0$.
   \end{enumerate}
  Dann gilt $\rk A=m-1$
  \end{Lemma}

  \begin{Beweis}
    Es folgt $a_{kk}>0$. Wir zeigen: die ersten $m-1$ Spalten sind linear unabhängig. Sei $\sum_{i=1}^{m-1}\lambda_is_i=0$. Man darf annehmen: $\lambda_k=1$, $\lambda_j\leq 1$ für alle $j\neq k$.
    Dann folgt 
    \[0=\sum_{i=1}^{m-1} \lambda_i a_{ki} \geq \sum_{i=1}^{m-1} a_{ki} > \sum_{i=1}^m a_{ki}=0 \lightning.\]
  \end{Beweis}
  
  \begin{Fakt}
   Wir bilden die Matrix $M=(\log|\rho\eps_i|^{\N_\rho})_{i=1,2,\ldots,r+s-1}$, $\rho\in S$ für ein Fundamentalsystem von Einheiten $\eps_1,\ldots,\eps_{r+s-1}$. $M$ hat $r+s-1$ Zeilen, $r+s$ Spalten. Die Determinanten der Minoren vom Format $(r+s-1)\times (r+s-1)$ sind $\neq 0$ und unterscheiden sich nur um Vorzeichen. Ihr Betrag heißt \highl{Regulator} von $K$, geschrieben $R_K$.
  \end{Fakt}

  \begin{Beweis}
  Seien $v_1,\ldots,v_{n+1}\in\IR^n$, $\sum v_i=0$. 
  Dann ist \[\det(v_1,\ldots,\hat{v_i},\ldots,v_{n+1})=\pm\det(v_1,\ldots,\hat{v_j},\ldots,v_{n+1}),\] wegen $\det(v_1,\ldots,v_i+v_j,\ldots,v_{n+1})=0$.
  \end{Beweis}

 \subsection{Beispiele für Einheiten}
\subsubsection{Reell quadratische ZK}
$K=\IQ(\sqrt d)$, $d>1$, sqf, $\eps_K$ Fundamentaleinheit, es gilt $\N(\eps_K)=\pm 1$. Sei $d\equiv 2,3 \mod{4}$, dann produzieren die Einheiten die ganzzahligen Lösungen der \highl[PELLsche Gleichung]{PELLschen Gleichunen}: $a^2-db^2=\pm 1$:

Genauer: Ist $\N(\eps_K)=-1$, so erhält man die $\IZ$-Lösungen der Nicht-PELLschen Gleichungen $a^2-db^2=-1$ aus $\eps_K^m=a_m+b_m\sqrt{d}$, $m$ ungerade und die der PELLschen Gleichung $a^2-db^2=1$ aus $\eps_K^m$, $m$ gerade. Ist dagegen $\N(\eps_K)=+1$, so hat die Nicht-PELLsche Gleichung keine $\IZ$-Lösungen, $\eps_K^m$ produziert alle $\IZ$-Lösungen der PELLschen Gleichung.

Analog für $d\equiv 1\mod{4}$.

\begin{Bemerkung}
 Es gibt einen Kettenbruchalgorithmus zur Berechnung der Fundamentaleinheit, siehe z.B. BOREVICH, SHAFAREVICH.
\end{Bemerkung}

\subsubsection{Kreiseinheiten}
$K=\IQ(\zeta_p)$, $p>2$ prim, $\zeta_p=e^{2\pi i/p}$. $[K:\IQ]=p-1$, $r=0$, $s=\frac{p-1}{2}$, $\rk E_K=\frac{p-3}{2}$. Sei $\eps_r=\frac{1-\zeta_p^r}{1-\zeta_p}=1+\zeta_p+\ldots+\zeta_p^{r-1}$, $1\leq r\leq p-1$.
Dann ist $\eps_r\in\cO_K$, diese sind sogar Einheiten. $\eps_r\inv=\frac{1-\zeta_p}{1-\zeta_p^r}=\frac{1-\zeta_p^{rs}}{1-\zeta_p^r}=1+\zeta^r+\ldots +\zeta_p^{(s-1)r}\in\cO_K$, $rs\equiv 1\mod{p}$.

$\eps_1=1, \eps_{p-1}=\frac{1-\zeta_p^{p-1}}{1-\zeta_p}=1+\zeta_p+\ldots+\zeta_p^{p-1}=-\zeta_p^{p-1}$. Bleiben $\eps_2,\ldots,\eps_{p-2}$ -  das sind $p-3$ viele.
\[ \eps_{p-r} = \frac{1-\zeta_p^{p-r}}{1-\zeta_p}=\zeta_p^{p-r}\frac{(\zeta_p^r-1)}{1-\zeta_p}=-\zeta_p^{-r} \frac{1-\zeta_p^r}{1-\zeta_p}=-\zeta_p^{-r}\eps_r.\]

Bleiben also $\eps_2,\ldots,\eps_{\frac{p-1}{2}}$. Man kann zeigen: Diese erzeugen Untergruppe von endlichem Index in $E_K$.

\subsubsection{Hilfsresultat}
\begin{Fakt}
 Sei $f=X^n+a_1X^{n-1}+\ldots+a_n\in\IZ[X]$ ein EISENSTEIN-Polynom. Für $p\in\IP$: $p\mid a_j$ für alle $j$ und $p^2\nmid a_n$. Sei $\theta$ Wurzel von $f$, dann teilt $p$ nicht den Index $(\cO_K:\IZ[\theta])$ für $K=\IQ(\theta)$.
\end{Fakt}

\begin{Beweis}
 \begin{enumerate}
  \item Sei $x\in\IZ[\theta]$, also $x=b_0+b_1\theta+\ldots+b_{n-1}\theta^{n-1}$, $b_j\in\IZ$. Wir zeigen $\N(x)\equiv b_0^n \mod{p}$.
  \[ \N(x)=\prod_{\rho\in\Hom(K,\IC)} (b_0+b_1\rho(\theta)+\ldots+b_{n-1}\rho(\theta)^{n-1})\]
Sei $P=\prod_{j=1}^n(X_0+X_1Y_j+X_2Y_j^2+\ldots+X_{n-1}Y_j^{n-1})\in\IZ[X,Y]$. $P$ ist symmetrisch in $Y_1,\ldots,Y_n$, also auch die $P_{\alpha}(Y)$ in $P=\sum P_\alpha(Y)X^\alpha$, $\alpha=(\alpha_0,\ldots,\alpha_{n-1})$. $P_\alpha$ ist homogen vom Grad $\deg(\alpha)=\sum_{j=0}^{n-1}j\alpha_j$. Somit sind alle Polynome in den elementarsymmetrischen Polynomen von $Y_1,\ldots,Y_n$. Außer für $\alpha=(n,0,\ldots,0)$ haben sie konstanten Term $0$. Nun setze man $X_i=b_i$, $Y_j=\sigma_j\theta$, $\sigma_j\in\Hom(K,\IC)$. Es folgt $\N(x)\equiv b_0^n \mod{p}$ wegen $a_j\equiv 0\mod{p}$.
\item Sei nun $x\in\IZ[\theta]$ und $x=py$, $y\in\cO_K$. Dann ist $\N(x)\equiv 0\mod{p}$, also $b_0=pc_0$, $c_0\in\IN$ nach Punkt (i). Weiter ist
\[x-pc_0=\theta(b_1+b_2\theta+\ldots+b_{n-1}\theta^{n-2}). \]
\folge $\N(x-pc_0)=p^n\N(y-c_0)=\N(\theta)\N(b_1+b_2\theta+\ldots+b_{n-1}\theta^{n-2}).$ $\N(\theta)=\pm a_n$, also
\[ p^{n-1} \N(y-c_0)=\underbrace{\pm \left( \frac{a_n}{p}\right)}_{\in\IZ,\text{ prim zu } p}\N(b_1+b_2\theta+\ldots+b_{n-1}\theta^{n-2})\]
$\N(b_1+b_2\theta+\ldots+b_{n-1}\theta^{n-2})\equiv b_1^n \mod{p}$ wegen (i), also $p\mid b_1$, $b_1=c_1\cdot p$, $c_1\in\IZ$ usw.

Wir zeigten: $x\in\IZ[\theta]$, $x=py$, $y\in\cO_K$ \folge $y\in\IZ[\theta]$.

\item Hieraus folgt: $p$ teilt den Index $(\cO_K:\IZ[\theta])$ nicht. Indirekt: $p$ teilt den Index, d.h. in $\cO_K/\IZ[\theta]$ existiert eine Untergruppe $\overline A$ der Ordung $p$. Sei $A$ Urbild von $\overline A$ in $\cO_K$. Dann gilt $\cO_K\supset A\supset\IZ[\theta]$, $(A:\IZ[\theta])=p$. Es folgt, dass ein $y\in A\setminus\IZ[\theta]$ mit $py\in\IZ[\theta]$ existiert \lightning.
 \end{enumerate}

Sei $K=\IQ(\sqrt[4]{2})$, $[K:\IQ]=4$, $\omega=\sqrt[4]{2}>0$, $\Irr(T,\omega,\IQ)=T^4-2$ irreduzibel.

\begin{align*}
 d(1,\omega,\omega^2,\omega^3)&=\det\nolimits^2\begin{pmatrix}
                                                1 & 1 & 1& 1\
                                                \omega & i\omega & -\omega & -i\omega\\
                                                \omega^2 & -\omega^2 & \omega^2 & -\omega^2\\
                                                \omega^3 & -i\omega^3 & -\omega^3 & i\omega^3
                                               \end{pmatrix}\\
                               &=8\det\nolimits^2\begin{pmatrix}
						  1& 1& 1& 1\\
						  1 & i & -1 & -i\\
						  1 & -1 & 1 & -1\\
						  1 & -i & -1 & i
                                                \end{pmatrix}\\
                               &=32 \cdot \det\limits^2(-8i)= -2^{11}
\end{align*}
Nach dem Fakt von oben folgt: $1,\omega,\omega^2,\omega^3$ ist Ganzheitsbasis: $\cO_K=\IZ[\omega]$. $r=2$, $s=1$, $r+s-1=2$, $\rk E_K=2$.

Sei $L=\IQ(\sqrt 2)$, man hat $\N_L^K: E_K\rightarrow E_L$. Fundamentaleinheit in $E_L$ ist $\eps_L=1+\sqrt{2}$ (denn $a^2-2b^2=-1$ hat Lösung $(1,1)$).
$\eps\in E_K$, $\eps=\alpha+\omega\beta$, $\alpha,\beta\in\cO_L$, $\alpha=a+b\sqrt{2}$, $\beta=c+d\sqrt{2}$.
\[\N_L^K(\eps)=\alpha^2-\omega\beta^2=(a^2+2b^2-4cd)+\sqrt{2}(2ab-c^2-2d^2) \]
Es gilt $\N_L^K(1+\omega)=(1+\omega)(1-\omega)=1-\omega^2=1-\sqrt 2=-\frac{1}{1+\sqrt 2}=-\eps_L\inv$

\end{Beweis}

