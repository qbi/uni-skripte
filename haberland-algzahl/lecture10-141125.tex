\renewcommand{\lecdate}{25.11.2014}

\begin{Folgerung}[\glqq Folgerung 1\grqq]
 Für jeden ZK $K\neq \IQ$ ist $|d_K|>1$
\end{Folgerung}
\begin{Beweis}
 $\sqrt{|d_K|}\geq \frac{n^n}{n!}\left(\frac{\pi}{4}\right)^s$ \folge $|d_K|\geq \left(\frac{\pi}{4}\right)^{n} \frac{n^{2n}}{(n!)^2}=c_n$
 
 Für $n=2$ ist $c_n=\left(\frac{\pi}{2}\right)^2\frac{16}{4} = \frac{\pi^2}{4}>\frac{9}{4}>1$.
 
 Die Folge $(c_n)$ ist monoton wachsend:
 \begin{align*}
  \frac{c_{n+1}}{c_n}=\frac{\pi}{4} \frac{(n+1)^{2n+2}}{n^{2n}} \frac{1}{(n+1)^2}=\frac{\pi}{4} \left( 1+\frac{1}{n}\right)^{2n}
 \end{align*}
Ana I: Der Grenzwert ist $e^2$.
\end{Beweis}

\subsection{Beispiele von Idealklassengruppen}

\subsubsection{Imaginärquadratische ZK}
 
 $K=\IQ(\sqrt d)$, $d<0$ sqf
 
 $c_K=\frac{2!}{2^2}\frac{4}{\pi}=\frac{2}{\pi}<0.6332$
 
 $c_K\sqrt{|d_K|}<2 \gdw |d_K|< \pi^2$, also $\Cl_K=1$ für $d_K=-3,-4,-7,-8$ ($d=-3,-1,-7,-2$). 
 Man weiß, dass es nur noch die Diskriminanten $d_K=-11,-19,-43,-67,-163$ mit $h_K=1$ gibt (HEEGNER, Problem der 10. Diskriminante, Beweis schwierig).
 
 SIEGEL:\[ \lim_{-d\rightarrow\infty} \frac{\log h(\IQ(\sqrt d))}{\log|d_K|}=\frac{1}{2}\]
 
 \begin{Beispiel}
  $K=\IQ(\sqrt{-163})$, $d_K=-163$ (prim), $\cO_K=\IZ+\IZ\omega$ mit $\omega=\frac{1+\sqrt{-163}}{2}$. \[ c_K\sqrt{163}=\frac{2!}{2^2}\frac{4}{\pi} \sqrt{163} < 8.128\]
  
  $\Cl_K$ wird erzeugt von den Primidealen der Absolutnorm $<9$. Jedes Primideal enthält natürliche Zahlen $>0$. 
  Ist $m\in\Fp$, $m\in\IN\oN$, so ist auch jeder Primteiler von $m$ in $\Fp$, $\Fp$ enthält keine zwei Primzahlen $p,l\in\Fp$, $p\neq l$ \folge $1\in\Fp$ $\lightning$. Also liegt in $\Fp$ genau eine Primzahl $p\in\Fp$. Es folgt $p\cO_K\subset\Fp\subset \cO_K$. Ist also $\Fp\neq p\cO_K$, so folgt $\IN\Fp=p$, ist $\Fp=p\cO_K$, so folgt  $\IN\Fp=p^2$. Wir zeigen $p\cO_K$ ist prim für $p=2,3,5,7$, denn daraus folgt $\Cl_K=1$. 
  
  Wir betrachten $\cO_K/p\cO_K$.  $\omega=\frac{1+\sqrt{-163}}{2}$ \folge $\omega^2=\frac{1}{4}+\frac{1}{2}\sqrt{-163}-\frac{163}{4}=\omega-41$ 
  
  Ist $\overline{\omega}$ das Bild von $\omega$ in $\cO_K/p\cO_K$, so ist 
  \begin{align*}
   \overline\omega^2&=1+\overline\omega &p&=2,3,7\\
   \overline\omega^2&=-1+\overline\omega & p&=5
  \end{align*}
  
  Sei $\IF_p[T]\rightarrow \cO_K \rightarrow \cO_K/p\cO_K$: $T\mapsto \omega \mapsto \overline{\omega}$. Der Kern ist $(T^2-T-1)$ für $p=2,3,7$ und $(T^2-T+1)$ für $p=5$
  \begin{center}
  \begin{tabular}{l|ccccccc}
   $x$ & $0$ & $1$ & $2$ & $3$ & $4$& $5$& $6$\\\hline
   $x^2-x-1$ & $-1$ & $-1$ & $1$ & $5$ & $11$ & $19$ & $29$\\\hline
   $x^2-x+1$ & $1$ & $1$ & $3$ & $7$ & $13$ & & 
  \end{tabular}\end{center}

  Also sind alle vier Ringe $\cO_K/p\cO_K$ Körper, d.h. $p\cO_K$ ist maximales Ideal für $p=2,3,5,7$. Mithin $h_K=1$.
 \end{Beispiel}
 
 \begin{Beispiel}
  $K=\IQ(\sqrt{-41})$
  \begin{enumerate}
   \item $d_K=-4\cdot 41$, $\cO_K=\IZ+\IZ\omega$ mit $\omega=\sqrt{-41}$.
   \[ c_K\sqrt{-41}=\frac{2!}{2^2}\frac{4}{\pi}\sqrt{41\cdot 4}=\frac{4}{\pi} \sqrt{41} < 8,2\]
   \folge $\Cl_K$ wird erzeugt von den Primidealen der Absolutnorm $\leq 8$.
   \item Wir zeigen $2\cO_K=\Fp_2^2$ ($2$ ist verzweigt)
   
   Sei dazu $\Fp_2:=\IZ(1+\omega)+\IZ\cdot 2$. Wir zeigen $\Fp_2$ ist Ideal: $\omega(1+\omega)=\omega-41=\omega+1-21\cdot 2\in\Fp_2$ und $\omega\cdot 2=2(\omega+1)-1\cdot 2\in\Fp_2$. Also $\Fp_2=(1+\omega,2)$.
   \[ \IN\Fp_2=\left|\det\begin{pmatrix} 1&2\\1&0 \end{pmatrix}\right|=2\]
   Ein Kommutativer Ring mit $1$ und 2 Elementen ist ein Körper. Also $\Fp_2$ prim.
   
   $\Fp_2^2=((1+\omega)^2,2(1+\omega),4)$
   \[ (1+\omega)^2=1+2\omega-41=-40+2\omega=2(-20+\omega)\]
   
   Also $\Fp_2^2\subset 2\cO_K=(2)$. $\IN(\Fp_2^2)=(\IN\Fp_2)^2=4$, $\IN(2)=4$ \folge $\Fp_2^2=(2)=2\cO_K$
   
   \item Wir zeigen: $3,5,7$ sind zerlegt (split): $p\cO_K=\Fp\overline\Fp$, $\Fp\neq\overline\Fp$ und $\Fp$,$\overline{\Fp}$ prim.
   
   Sei $\Fp_3:=\IZ(1+\omega)+\IZ\cdot 3$. Das ist Ideal: $\omega(1+\omega)=\omega-41=(1+\omega)-14\cdot 3\in\Fp_3$ und $\omega\cdot 3=3(1+\omega)-1\cdot 3\in\Fp_3$.
   \[ \IN\Fp_3=\left|\det\begin{pmatrix} 1&3\\1&0 \end{pmatrix}\right|=3\]
   \folge $\Fp_3$ prim.
   
   $\overline\Fp_3=\IZ(1-\omega)+\IZ\cdot 3$. Angenommen $\overline\Fp_3=\Fp_3$, dann $1-\omega=a(1+\omega)+b\cdot 3$ mit $a,b\in \IZ$. Dann folgt $a=-1$ und $a+3b=1$, also $3b=2$ \lightning.
   Somit $\Fp_3\neq\overline\Fp_3$. $\Fp_3\cdot\overline\Fp_3=\IZ\cdot 3(1+\omega+\IZ\cdot 3(1-\omega) + \IZ\cdot 9 + \IZ\cdot 42 \subset 3\cdot \cO_K$.
   
   $\IN\Fp_3\overline\Fp_3=9=\IN(3)$ \folge $3\cdot \cO_K=\Fp_3\cdot\overline\Fp_3$
   
   $\Fp_5:=\IZ(2+\omega) + \IZ 5$ ist Ideal: $\omega(2+\omega)=2\omega-41=2\omega+2-9\cdot 5 \in\Fp_5$. $\omega\cdot 5 = 5(\omega+2)-2\cdot 5\in\Fp_5$.
   \[ \IN\Fp_5 = \left|\det\begin{pmatrix}
                       2 & 5\\ 1 & 0
                      \end{pmatrix}
\right|=5\]
\folge $\Fp_5$ prim.

Angenommen $\Fp_5=\overline\Fp_5$ \folge $2-\omega = a(2+\omega)+b\cdot 5$ mit $a,b\in\IZ$. Dann ist $2=2a+5b$ und $-1=a$, also $4=5b$ $\lightning$.
Wie oben folgt: $5\cdot \cO_K=\Fp_5\cdot\overline\Fp_5$.

Das $\Fp_5$ ist gut konstruiert. Wählt man nämlich $A=\IZ(1+\omega)+\IZ \cdot 5$, so ist das kein Ideal: $\omega(1+\omega)=\omega-41=(\omega+1)-42 \not\in A$.
Das Ideal $I=(1+\omega,5)$ ist $\cO_K$: $\omega(1+\omega)+9\cdot 5=3+(\omega+1)\in I$. Also ist $3\in I$ und $5\in I$, also $1\in I$.

$\Fp_7:= \IZ(1+\omega)+\IZ\cdot 7$ ist Ideal: $\omega(1+\omega)=\omega-41=\omega+1-6\cdot 7\in\Fp_7$, $\omega\cdot 7= 7\cdot(\omega+1)-7\in\Fp_7$.
\[ \IN\Fp_7=\left|\begin{pmatrix}
                   1 &7\\ 1&0
                  \end{pmatrix}
 \right|=7\]
 \folge $\Fp_7$ prim.
 
 Angenommen $\Fp_7=\overline\Fp_7$. Dann $1-\omega=a(1+\omega)+b\cdot 7$ für $a,b\in\IZ$. \folge $a=-1$, $1=-1+7b$ und $7b=2$ \lightning. \folge $7\cdot\cO_K=\Fp_7\cdot\overline\Fp_7$.
 
 Alle ganzen Ideale mit $\IN\leq 8$ sind: $(1),\Fp_2,\Fp_3,\overline\Fp_3,\Fp_2^2,\Fp_5,\overline\Fp_5,\Fp_2\Fp_3,\Fp_2\overline\Fp_3,\Fp_7,\overline\Fp_7,\Fp_2^3$. Also $h_K\leq 12$. $\Fp_2^2=(2)$ mit Bild $1$ in $\Cl_K$, $\Fp_2^3=(2)\cdot\Fp_2$ und $\Fp_2$ haben das selbe Bild in $\Cl_K$. Also $h_K\leq 10$.
 Diese $10$ Ideale werden surjektiv auf $\Cl_K$ abgebildet, $\Fp_3\cdot \overline\Fp_3=(3)$, also $\overline\Fp_3\cong \Fp_3\inv$ in $\Cl_K$, ebenso für $\Fp_5$, $\Fp_7$.
 Weiter $\Fp_2^2\cong 1$ in $\Cl_K$. $\N(1+\omega)=42=2\cdot 3\cdot 7$ \folge $(1+\omega)=\Fp_2\Fp_3\Fp_7$ oder $\Fp_2\overline\Fp_3\Fp_7$ oder $\Fp_2\Fp_3\overline{\Fp}_7$ oder $\Fp_2\overline\Fp_3\overline\Fp_7$. Also wird $\Cl_K$ erzeugt von $\Fp_3,\Fp_5,\Fp_7$. $\N(2+\omega)=45=3^2\cdot 5$, also $(2+\omega)=\Fp_3^2\cdot \Fp_5$ oder die anderen Möglichkeiten. \folge $\Cl_K$ wird erzeugt von $\Fp_3$ und $\Fp_7$. $\N(8+\omega)=64+41=105=3\cdot 5\cdot 7$ \folge $(8+\omega)=\Fp_3\Fp_5\Fp_7$ oder so ähnlich \folge $\Cl_K$ wird erzeugt von $\Fp_3$.
 
 Nun bestimmen wir die Ordnung von $\Fp_3$ in $\Cl_K$. Sie ist $\leq 10$. $\Fp_3^k=(a+b\omega)$ \folge $3^k=a^2+42b^2$. Ist umgekehrt $3^k=a^2+41b^2$, so folgt $(a+b\omega)=\Fp_3^r\overline\Fp_3^s$. Sind beide $r>0$, $s>0$, so steht rechts Faktor $\Fp_3\overline\Fp_3=(3)=3\cO_K$, also $3\mid a$, $3\mid b$. Sind $a,b$ nicht beide durch $3$ teilbar, so ist $(a+b\omega)=\Fp_3^k$ oder $=\overline\Fp_3$. Wir suchen das minimale $k\geq 1$, s.d. $3^k=a^2+41b^2$ 
 $\IZ$-Lösungen hat, wir wissen $k\leq 10$.
 \begin{align*}
  k&=1 & 3=a^2+41b^2 \lightning\\
  k&=2 & 9=a^2+41b^2 \lightning\\
  k&=3 & 27=a^2+41b^2 \lightning\\
  k&=4 & 81=a^2+41b^2 \lightning\\
 \end{align*}
 
 usw. $k=8$ liefert den ersten Treffer: $b=11$, $a=40$. $3^8=6561$, $a^2+41b^2=6561$.
 
 Also $(40+11\omega)=\Fp_3^8$ (oder $\overline\Fp_3^8$?\footnote{Tatsächlich ist letzteres der Fall, aber das benötigen wir nicht für unsere Aussage.}). Somit: Die Idealklassengruppe von $\IQ(\sqrt{-41})$ ist zyklisch der Ordnung $8$, erzeugt von $\Fp_3=(1+\omega,3)$

  \end{enumerate}

 \end{Beispiel}





