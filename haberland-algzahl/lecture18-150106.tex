\renewcommand{\lecdate}{06.01.2015}

\newpage
\section{Lokale Zahlkörper}

\subsection{Die reellen Zahlen}

Was ist $\sqrt 2$?
\begin{enumerate}
 \item $\emptyset, \{\emptyset\}, \{\emptyset, \{\emptyset\}\},\ldots$
 
 $\IN$ = Kardinalität der endlichen Mengen
 \item $\IZ$ $(a-b)=(a,b)\sim (c,d)=(c-d)$ \gdw $a+d=b+c$
 \item $\IQ$ QK
 \item Für $\IR$ brauchen wir einen Absolutbetrag
	\[ |\cdot |:\IQ\rightarrow \IQ_+=\{\alpha\in\IQ: \alpha\geq 0\} \]
	\[ |\alpha|=\begin{cases}
	             \alpha & \alpha\geq 0\\
	             -\alpha & \alpha<0
	            \end{cases}\]
	mit \begin{enumerate}
	     \item $|\alpha|=0 \gdw \alpha=0$
	     \item $|\alpha+\beta|\leq |\alpha|+|\beta|$
	     \item $|\alpha\beta|=|\alpha||\beta|$
	    \end{enumerate}
   Der Makel von $\IQ$: Der Körper ist in der induzierten Metrik $dist(\alpha,\beta)=|\alpha-\beta|$ nicht vollständig.

    Die Folge $x_0=2$, $x_{n+1}=x_n-\frac{x_n^2-2}{2x_n}$ ist CAUCHY-Folge ohne Grenzwert in $\IQ$.
    Sei $C$ der Ring aller Fundamentalfolgen (d.h. CAUCHY-Folgen) in $\IQ$. Das ist kommutativer Ring mit $1$, $\IQ$ bettet sich in $C$ ein, durch konstante Folgen. Sei $\Fm$ das Ideal der Nullfolgen. Dies ist maximales Ideal: $(\alpha_n)\notin\Fm$ \folge $\exists\eps>0\forall n_0\exists n\geq n_0: |\alpha_n|\geq \eps$.
    
    $\exists n_1\forall m,n\geq n_1: |\alpha_m-\alpha_n|<\frac{1}{2}\eps$. Sei \[\beta_n=\begin{cases}
                                                                                         0 & n<n_2\\
                                                                                         \frac{1}{\alpha_n} n\geq n_2
                                                                                        \end{cases}.\]
    Dann ist $(\beta_n)$ Fundamentalfolge: \[ |\beta_m-\beta_n|=\frac{\alpha_m-\alpha_n}{|\alpha_m||\alpha_n|}\leq\left(\frac{2}{\eps}\right)^2|\alpha_m-\alpha_n|\]
    und es gilt $(\alpha_n)(\beta_n)=(\alpha_n\beta_n)\equiv 1\mod{m}$. Es folgt: $C/\Fm$ ist ein Körper. Die Einbettung $\IQ\rightarrow C$ überlebt. Wir bezeichnen $C/\Fm$ mit $\IR$.
    
    Absolutbetrag auf $\IR$: \[|x|:=\begin{cases}
                                     x & x\geq 0\\
                                     -x & x<0
                                    \end{cases}
 \]
 Wobei $x>0$ \gdw $\alpha_n\geq\eta>0$ $\forall n\geq n_0$, $(\alpha_n)\in x$. Es gelten die 3 Eigenschaften wie oben. Also ist $dist(x,y)=|x-y|$ Metrik auf $\IR$
 
 \begin{Fakt}
  $\IR$ ist vollständig.
 \end{Fakt}

 \begin{Beweis}
  Sei $(x_n)$ Fundamentalfolge reeller Zahlen, also $\forall \eps>0 \exists n_0 \forall m,n\geq n_0: |x_m-x_n|<\eps$. Wir wählen $(\alpha_{nk})\in x_n$.
  
  $|x_m-x_n|<\eps$ heißt $|\alpha_{mk}-\alpha_{nk}|<\eps$ $\forall k<k_0$. Wähle für jedes $r\in\IN\oN$ einen Index $k_r\geq k$, s.d. $|\alpha_{rk_r}-\alpha_{rl}|<\eps_r$ $\forall l\geq k_r$ und $\eps_r\textdownarrow 0$ (z.B. $\eps=\frac{1}{r}$). Dann folgt:
  \begin{align*}
   |\alpha_{mk_m}-\alpha_{nk_n}|&\leq |\alpha_{mk_m}-\alpha_{ml}|+|\alpha_{ml}-\alpha_{nl}|+|\alpha_{nl}-\alpha_{nk_n}|\\
   &< \eps_m +\underbrace{|\alpha_{ml}-\alpha_{nl}|}_{<\eps}+\eps_n 
  \end{align*}
  für $l$ genügend groß. \folge $|\alpha_{mk_m}-\alpha_{nk_n}|<\eps_m+\eps+\eps_n$ \folge $(\alpha_{mk_m})$ ist Fundamentalfolge.

  Sie definiert reelle Zahl $x$. Wir zeigen $x_n\rightarrow x$. Das ist äquivalent zu $x_n-x\rightarrow 0$, d.h. $|x_n-x|<\eps$ $\forall n\geq n_0$ \gdw $|\alpha_{nl}-\alpha_{lk_l}|<\eps$ $\forall n\geq n_0, l\geq l_0=l_0(n)$.
  
  $|\alpha_{nl}-\alpha_{lk_l}|<\underbrace{|\alpha_{nl}-\alpha_{nk_n}|}_{<\eps/2}+\underbrace{|\alpha_{nk_n}-\alpha_{lk_l}|}_{<\eps_n}$
  für genügend große $n$.
 \end{Beweis}

 \begin{Bemerkung}
  $\IQ$ ist dicht in $\IR$ (ÜA). Man betrachte $(\alpha_n)\in x\in\IR$, $x_n=$ Konstante Folge $\alpha_n$. Dann gilt $x_n\rightarrow x$. 
 \end{Bemerkung}

 Was ist $\sqrt 2$? Antwort: Die Äquivalenzklasse, in welcher die Folge $(\alpha_n)$ liegt mit $\alpha_0=2$, $\alpha_{n+1}=\alpha_n-\frac{\alpha_n^2-2}{2\alpha_n}$:
 
 $\alpha_{n+1}=\frac{\alpha_n^2+2}{2\alpha_n}$ \folge $\alpha_n>0$.
 \[ \alpha_{n+1}^2-2=\left(\frac{\alpha_n^2+2}{2\alpha_n}-2\right)^2=\left( \frac{2-\alpha_n^2}{2\alpha_n}\right)^2=\left( \frac{\alpha_n^2-2}{2\alpha_n}\right)^2\]
 \folge $\alpha_n^2>2$, $\alpha_n\textdownarrow$
 
 $0<\alpha_{n+1}^2-2<\frac{(\alpha_n^2-2)^2}{16}$ \folge $\alpha_n\rightarrow \sqrt{2}$, also ist $(\alpha_n)$ Fundamentalfolge.
\end{enumerate}

\subsection{Bewertungen}

\begin{Definition}
 Sei $K$ ein Körper. Eine \highl{Bewertung} von $K$ ist eine Abbildung
 $|\cdot|: K\rightarrow \IR_+$ mit 
 \begin{enumerate}
  \item[(i)] $|x|=0 \gdw x=0$
  \item[(ii)] $|x+y|\leq |x|+|y|$
  \item[(iii)] $|xy|=|x||y|$.
 \end{enumerate}
\end{Definition}

\begin{Beispiel}
 \begin{enumerate}
  \item $\IQ$, $\IR$, $\IC$ mit ihren Absolutbeträgen.
  \item Jeder Körper trägt die triviale Bewertung
  \[ 
    |x|=\begin{cases}
         0 & x=0\\
         1 & x\in K\kreuz
        \end{cases}.
  \]
  Diese wird im Folgenden stets ausgeschlossen.
  \item $p\in\IP$, $0<c<1$, jedes $\alpha\in\IQ\kreuz$ lässt sich schreiben als $\alpha=p^{\nu_p(\alpha)}\cdot \beta$, $\beta$ prim zu $p$.
  
  Sei $|\alpha|_p=c^{\nu_p(\alpha)}$ - \highl{$p$-adische Bewertung}.
  
  Dann $\nu_p(\alpha\beta)=\nu_p(\alpha)+\nu_p(\beta)$ zeigt (iii). $\nu_p(\alpha+\beta)\geq \min(\nu_p(\alpha),\nu_p(\beta))$ zeigt sogar $|\alpha+\beta|_p\leq \max(|\alpha|_p,|\beta|_p)$, also folgt (ii) (\highl{ultrametrische Ungleichung}). $|\alpha|_p\neq 0$ für $\alpha\neq 0$ zeigt (i). Man nimmt gern $c=\frac{1}{p}$, also $|\alpha|_p=p^{-\nu_p(\alpha)}$.
 \end{enumerate}
\end{Beispiel}

\begin{Bemerkung}
 $dist(x,y)=|x-y|$ definiert eine Metrik auf bewertetem Körper.
\end{Bemerkung}

\begin{Definition}
 Zwei Bewertungen eines Körpers heißen äquivalent genau dann, wenn eine der folgenen Aussagen gilt:
 \begin{enumerate}
  \item Sie haben die selben offenen Mengen.
  \item Sie haben die selben abgeschlossenen Mengen.
  \item Sie haben die selben konvergenten Folgen \[|x_n-x|_1\rightarrow 0 \gdw |x_n-x|_2 \rightarrow 0. \]
 \end{enumerate}
\end{Definition}

\begin{Fakt}
 Diese drei Eigenschaften sind äquivalent.
\end{Fakt}

\begin{Beweis}
 (i) \gdw (ii) trivial.
 
 (i) \folge (iii): $|x_n-x|\rightarrow 0$ \folge $\{x_n: n\in\IN\}\cup \{x\}$ ist abgeschlossen bezüglich $|\cdot|_1$ \folge auch abgeschlossen bezüglich $|\cdot|_2$ \folge $x_n\rightarrow x$ bzgl. $|\cdot|_2$.
 
 (iii) \folge (i): Sei $E\subset K$ abgeschlossen bzgl $|\cdot|_1$, dann ist jeder Punkt aus $E$ isoliert oder Häufungspunkt bzgl $|\cdot|_1$. \folge das selbe bzgl. $|\cdot|_2$ \folge $E$ bzgl. $|\cdot|_2$ abgeschlossen.
\end{Beweis}

\begin{Fakt}
 Zwei Bewertungen $|\cdot|_1$, $|\cdot|_2$ eines Körper $K$ sind genau dann äquivalent, wenn ein $c\in\IR_+\kreuz$ mit $|\cdot|_2=|\cdot|_1^c$ existiert.
\end{Fakt}

