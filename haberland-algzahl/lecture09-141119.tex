\renewcommand{\lecdate}{19.11.2014}

\begin{Definition}
 Sei $V$ endlichdimensionaler VR über $\IR$, $\Omega\subset V$ heißt \highl{konvex} genau dann, wenn mit $v,w\in\Omega$ das Verbindungsintervall $[v,w]=\{tv+(1-t)w : t\in[0,1] \}$ in $\Omega$ liegt.
 
 $\Omega$ heißt \highl{zentralsymmetrisch} genau dann, wenn aus $v\in \Omega$ $-v\in\Omega$ folgt.
\end{Definition}

\begin{Fakt}[MINKOWSKIs Gitterpunktsatz]
 Sei $V$ EUKLIDischer VR (endl. dim.), $\Gamma\subset V$ Gitter, $\Omega\subset V$ messbar, konvex, zentralsymmetrisch.
Es gelte $\vol(\Omega)>2^n\vol(\Gamma)$, $n=\dim(V)$. Dann enthält $\Omega$ neben $0$ noch weitere Gitterpunkte.
\end{Fakt}


\begin{Bemerkung}
 $\vol(\Gamma)=\vol(F)$, $F$ Fundamentalmasche: $\Gamma=\IZ\omega_1+\ldots+\IZ\omega_n$, $F=\{\sum_{i=1}^n t_i\omega_i: 0\leq t_i\leq 1 \}$. $\vol(\Gamma)$ ist unabhängig von der Wahl der Basis von $\Gamma$:
 \[\vol(F)=\sqrt{|\det(\langle\omega_i,\omega_j \rangle)|} \]
 Für eine ON-Basis $e_1,\ldots,e_n$ von $V$ und $A$ mit $\omega_i=Ae_i$, ist  
 \[\vol(F)=\vol(A F(e_1,\ldots,e_n))=|\det A| \underbrace{\vol(F(e_1,\ldots,e_n))}_{=1},\] $AA^T=(\langle\omega_i,\omega_j\rangle)$.
\end{Bemerkung}

\begin{Beweis}
 Wir wählen die Fundamentalmasche vorsichtiger
 \[ F=\left\{\sum_{i=1}^nt_i\omega_i : 0\leq t_i<1 \right\}.\]
 Dann gilt: $V= \bigsqcup_{\omega\in\Gamma} F+\omega$ (disjunkte Vereinigung).
 Also folgt $\frac{1}{2}\Omega=\bigsqcup$ $(\frac{1}{2}\Omega\cap (F+\omega))$.
 
 \folge $2^{-n}\vol(\Omega)=\sum \vol(\frac{1}{2}\Omega\cap(F+\omega))=\sum_{\omega\in\Gamma} \vol((\frac{1}{2}\Omega-\omega)\cap F)$
 
Die LHS ist $>\vol(F)$, also können die Mengen $\frac{1}{2}\Omega-\omega$ nicht alle disjunkt sein. $\exists \omega_1,\omega_2\in\Gamma$ mit $\omega_1\neq\omega_2$, $\exists v_1,v_2\in\Omega$, s.d. $\frac{1}{2}v_1+\omega+1=\frac{1}{2}v_2+\omega_2$ gilt.

Somit ist $\frac{1}{2}(v_1-v_2)$ aus $\Gamma\setminus\{0\}$. Mit $v_2$ ist auch $-v_2\in\Omega$, also auch $\frac{1}{2}(v_1-v_2)\in \Omega$.
\end{Beweis}

\begin{Folgerung}
 Ist $\Omega$ kompakt, so genügt schon $\vol(\Omega)\geq 2^n\vol(\Gamma)$ für den vorherigen Fakt.
\end{Folgerung}

\begin{Beweis}
 $\vol((1+\eps)\Omega)>2^n\vol(\Gamma)\folge$ man findet $\omega_\eps\in\Gamma\oN$ in $(1+\eps)\Omega$. Wähle $\eps_n\textdownarrow 0$, $\omega_n=\omega_{\eps_n}\in\Gamma\oN$. $\omega_n\in(1+\eps)\Omega\subset 2\Omega$. Wg. $\Omega$ kompakt, besitzt $(\omega_n)$ konvergente Teilfolge. Da $\Gamma$ diskret ist, muss diese konstant ab bestimmtem Index sein: $\omega_n=\omega_\infty$ für alle $n>n_0$, $\omega_\infty\neq 0$, $\omega_\infty\in \bigcup_n(1+\eps_n)\Omega=\Omega$
\end{Beweis}


\begin{Bemerkung}
 Die Aussage ist bestmöglich:
 
 $V=\IR^n$, $\Omega=(-1,1)^n$, $\vol(\Omega)=2^n$, $\Gamma=\IZ^n$, $\vol(\Gamma)=1$. $\Omega$ enthält nur den Gitterpunkt $(0,\ldots,0)$. $\overline{\Omega}$ ist kompakt, zentralsymmetrisch, konvex und enthält $3^n-1$ weitere Gitterpunkte.
\end{Bemerkung}

\begin{Fakt}
 Sei $K$ alg. ZK, $\Fa\subset O_K$ Ideal $\neq(0)$. Dann ist $j(\Fa)$ ein Gitter in $K_\IR$ und es gilt \[ \vol(j(\Fa))=\IN\Fa\cdot \sqrt{|d_K|}.\]
\end{Fakt}

\begin{Beweis}
 $K_\IR=\IR^r \oplus \IC^s$ mit Skalarprodukt \[\langle x,y\rangle = \sum_{i=1}^r x_{\sigma_i}y_{\sigma_j}+ \sum_{j=1}^s(x_{\tau_j}\overline{y}_{\tau_j}+\overline{x}_{\tau_j}{y}_{\tau_j}) \]
 Es gilt $\langle j(\alpha),j(\beta)\rangle=\sum_{\sigma\in\Hom(K,\IC)}\sigma\alpha\overline{\sigma\beta}$. Sei $\omega_1,\ldots,\omega_n$ $\IZ$-Basis von $\Fa$, dann ist $j(\Fa)=\IZ j(\omega_1)+\ldots +\IZ j(\omega_n)$. Nach Definition der Diskriminante ist $d(\Fa)=\det^2(\sigma_i\omega_j)$. Andererseits ist $d(\Fa)=(\cO_K:\Fa)^2d(\cO_K)=(\IN\Fa)^2d_K$. Weiter ist
 \[ \vol(j(\Fa))=\sqrt{|\det(\langle j(\omega_k),j(\omega_l)\rangle)|}=|\det(\sigma_i\omega_j)|=\IN\Fa\sqrt{|d_K|}.\]
\end{Beweis}

\begin{Lemma}
 Sei $K$ alg. ZK, $\lambda>0$ reell. 
 \[ \Omega(\lambda):=\left\{ x\in K_\IR: \sum_{i=1}^r |x_{\sigma_i}| + 2\sum_{j=1}^s |x_{\tau_j}|\leq \lambda\right\}\]
 
 Dann gilt $\vol(\Omega(\lambda))=2^r\pi^s\frac{\lambda^n}{n!}$
\end{Lemma}

\begin{Beweis}
 Natürlich ist $\vol(\Omega(\lambda))=\lambda^n\vol(\Omega(1))$.
 \[f: K_\IR \rightarrow \IR^n: x_{\sigma_i}\mapsto x_{\sigma_i}, x_{\tau_j}\mapsto (\R x_{\tau_j},\I x_{\tau_j})\]
 
 Sei für $x\in \IR^n$: $x=(x_1,\ldots,x_r,y_1,z_1,\ldots,y_s,z_s)$
 
 \[f(\Omega(1))=\left\{ \sum |x_i| + 2\sum |\sqrt{y_i^2+z_i^2}|\leq 1\right\}\]
 Wir berechnen das LEBESGUE-Maß $\lambda$ (Achtung: leichte Überladung mit $\lambda\in\IR$) dieser Menge: Sei $y_j=r_j\cos\theta_j$, $z_j=r_j\sin\theta_j$.
 \[ \lambda(f(\Omega(1)))=\int_{\Delta}r_1\cdot\ldots\cdot r_s dx_1\ldots dx_r dr_1\ldots dr_s d\theta_1\ldots \theta_s\]
 mit 
 \[ \Delta=\left\{(x,r,\theta): \sum_{i=1}^r + 2\sum_{j=1}^sr_j\leq 1, r_j\geq 0, 0\leq\theta_j\leq 2\pi\right\}.\]
 Nimmt man nur die $x_i\geq 0$, so erhält man $2^r\lambda(f(\Omega(1)))$.
 
 Wir setzen noch $y_j=2r_j$, das gibt \[ \lambda(f(\Omega(1)))=2^r(2\pi)^s4^{-s} W_{r,s}(1)\]
 mit \[ W_{r,s}(\mu)=\int_E dx_1\ldots dx_r y_1\ldots y_s dy_1\ldots dy_s \hspace{1cm} \text{(FUBINI),}\]
 wobei \[ E=\left\{ (x,y)\in\IR_+^{r+s}: \sum_{i=1}^r+\sum_{j=1}^s y_j\leq \mu\right\}.\]
 
 Es gilt wieder $W_{r,s}(\mu)=\mu^nW_{r,s}(1)$, $n=r+2s$.
 \begin{align*}
  W_{r,s}(1)&=\int_0^1W_{r-1,s}(1-x_1) dx_1\\
  &=W_{r-1,s}(1) \int_0^1 (1-x_1)^{n-1} dx_1\\
  &=\frac{1}{n} W_{r-1,s}(1)
 \end{align*}
 Es folgt \[ W_{r,s}(1)=\frac{1}{n(n-1)\ldots(n-r+1)}W_{0,s}(1).\]
 Analog folgt:
 \begin{align*}
  W_{0,s}(1)&=\int_0^1 W_{0,s-1}(1-y_1)\cdot y_1 dy_1\\
  &= W_{0,s-1}(1) \int_0^1 y_1(1-y_1)^{2s-2} dy_1\\
  & \vdots\\
  &=\frac{1}{2s(2s-1)} W_{0,s-1}(1)
 \end{align*}
Insgesamt ergibt sich: $W_{r,s}(1)=\frac{1}{n!}$ (Beachte: $n-r=2s$) und damit 
\[ \lambda(f(\Omega(1)))=2^r4^{-s}(2\pi)^s\tfrac{1}{n!}.\]
Es folgt $\vol(\Omega(1))=2^s\lambda(f(\Omega(1)))=2^r\cdot\pi^s\cdot \frac{1}{n!}$.
\end{Beweis}


\begin{Bemerkung}
 $\Omega(1)$ ist kompakt, zentralsymmetrisch und konvex: Die ersten beiden Eigenschaften sieht man sofort. Konvex: Sind $x,y\in\Omega(1)$, $0\leq t\leq 1$, so ist 
 \[ |tx_0+(1-t)y_0|\leq t|x_0|+(1-t)|y_0|.\]
\end{Bemerkung}

\begin{Satz}[\glqq Satz 5\grqq: MINKOWSKI]
 Sei $K$ alg. Zahlenkörper, $n=r+2s=[K:\IQ]$. $c_K=\frac{n!}{n^n}\left(\frac{4}{\pi}\right)^s=$ \highl{MINKOWSKI-Konstante}.
 Dann gibt es in jeder Idealklasse aus $\Cl_K$ ganze Ideale $\Fa$ mit $\IN\Fa\leq c_K\sqrt{|d_K|}$.
\end{Satz}

\begin{Beweis}
 Sei $x\in \Cl_K$, $\Fb$ ganzes Ideal aus $x\inv$. Wähle $\lambda>0$ so groß, dass $\vol(\Omega(\lambda))=2^n\vol(j(\Fb)$ gilt. Das ist der Fall für $\lambda^n=2^n\IN\Fb\sqrt{|d_K|}n!2^{-r}\pi^{-s}$. Dann existiert $\beta\in\Fb$, $\beta\neq 0$ mit $j(\beta)\in\Omega(\lambda))$, also $\sum_{\sigma\in \Hom(K,\IC)} |\sigma\beta|\leq \lambda$.
 
 Es folgt $|N_\IQ^K|^{1/n}\leq \frac{1}{n}\sum_{\sigma\in\Hom(K,\IC)} |\sigma\beta| \leq \frac{\lambda}{n}$ (geometrisches Mittel $\leq$ arithmetisches Mittel).
 
 \folge $|N_\IQ^K(\beta)|\leq \frac{\lambda^n}{n^n}=\frac{n!}{n^n} 4^s\pi^{-s} \IN\Fb \sqrt{|d_K|}$. $(\beta)=\Fb\cdot\Fa$ für ein ganzes Ideal $\Fa$, wg, $\beta\in\Fb$. $\Fa$ liegt dann in der Klasse $x$ und $|N(\beta)|=\IN\Fa\cdot \IN\Fb$ \folge $\IN\Fa \leq \frac{n!}{n^n}\left( \frac{4}{\pi}\right)^s\sqrt{|d_K|}$
\end{Beweis}

