\documentclass[ngerman,halfparskip]{scrartcl}

\usepackage{babel} %Umlaute, neue deutsche Rechtschreibung
\usepackage[utf8]{inputenc} %Kodierung festlegen
\usepackage[T1]{fontenc} % font encoding festlegen
\usepackage{amsmath,amsfonts,amssymb,amsthm} %math. Symbole und Umgebungen
\usepackage{hyperref}
\usepackage{mathrsfs} 


\newtheorem{satz}{Satz}
\def\R{\mathbb R}
\def\C{\mathbb C}
\DeclareMathOperator{\Iso}{Iso}
\DeclareMathOperator{\Id}{Id}
\DeclareMathOperator{\kgV}{kgV}
\DeclareMathOperator{\ggT}{ggT}
\DeclareMathOperator{\Bild}{Bild}
\DeclareMathOperator{\Kern}{Kern}
\DeclareMathOperator{\Bahn}{Bahn}
\DeclareMathOperator{\Stab}{Stab}

\pagestyle{empty}

\begin{document}
\begin{satz}
Jede endliche Untergruppe von $\Iso(\R^2)$ ist eine Untergruppe von $D_n$.
\end{satz}

\begin{satz}
Jede Permutation kann man als Produkt (Verkettung) von disjunkten Zyklen darstellen. \\Bis auf Reihenfolge und $1$-Zyklen ist die Darstellung eindeutig. 
\end{satz}


\begin{satz}[Satz von Lagrange]
$G$ - endliche Gruppe, $H\leq G$, dann $|G/H|\cdot |H|= |G|$.
\end{satz}


\begin{satz}
$G$ - Gruppe; $H\triangleleft G$, dann $G/H$ Gruppe.
\end{satz}


\begin{satz}[Homomorphiesatz]
 Ein Gruppenhomomorphismus $f:G\rightarrow H$ induziert einen Isomorphismus $\overline f: G/\Kern_f \rightarrow \Bild_f$.
\end{satz}


\begin{satz}[1. Isomorphiesatz]
$ G$ - Gruppe; $H\leq G$; $K \triangleleft G$, dann
\begin{itemize}
\item $HK=\{hk\}$ ist Untergruppe von G.
\item $K\triangleleft HK$
\item $H\cap K \triangleleft  H$
\item $HK/K\cong H/(H\cap K)$
\end{itemize}
\end{satz}


\begin{satz}[2. Isomorphiesatz]
$G$ - Gruppe; $G\triangleright H; G\triangleright K; H\supseteq K$, dann
\begin{itemize}
\item $K\triangleleft H$
\item $ H/K\triangleleft G/K$
\item $(G/K)/(H/K)\cong G/H$
\end{itemize}
\end{satz}


\begin{satz}
 Die 5 platonischen Körper (Tetraeder, Würfel, Oktaeder, Dodekaeder, Ikosaeder) sind die einzigen regulären Polyeder im $\R^3$.
\end{satz}


\begin{satz}[Eulersche Polyederformel]
Bei einem Polyeder mit $E=\#\{$Ecken$\}; K=\#\{$Kanten$\}; F=\{$Flächen$\}$ gilt: $E-K+F=2$. 
\end{satz}




\begin{satz}
Sei $G\leq SO(3); \#G<\infty: \forall g\in G \det(f)>0 (=1)$, dann trifft einer der Fälle zu:
\begin{itemize}
\item $G=C_n$ (Drehungen um 0 in $\R^2$ um $\frac {2\pi}n$)
\item $G=D_n$ ($n>2$; Diedergruppe)
\item $G=A_4$ (Rotationsgruppe des Tetraeders)
\item $G=S_4$ (Rotationsgruppe des Würfels)
\item $G=A_5$ (Rotationsgruppe des Ikosaeders)
\end{itemize}
\end{satz}


\begin{satz}
$A_5$ ist einfach. 
\end{satz}




\begin{satz}[Satz von Möbius]
Sei $\alpha$ eine Abbildung $\alpha: \C \cup \{\infty\} \rightarrow \C \cup \{\infty\}$, die Kreise auf Kreise abbildet und bijektiv ist, dann ist $\alpha$ Möbiustransformation, oder $\overline{\alpha(z)}$ und $\alpha(\overline z)$ sind Möbiustransformationen.
\end{satz}



\begin{satz}
Lineare Algebra und analytische Geometrie ist nächstes Semester vorbei.
\end{satz}


\begin{satz}
Bei hyperbolischen Geraden im Einheitskreis mit $d(P,Q)=\ln [A,P,Q,B]$ gilt:
\begin{itemize}
\item $d(P,Q)=d(Q,P)$
\item $f\in PSl_2(\R): d(P,Q)=d(f(P),f(Q))$
\item Liegen $P,Q,R$ auf einer geraden ($Q$ zwischen $P$ und $R$), dann $d(P,R)=d(P,Q)+d(Q,R)$.
\item $d(P,R)\leq d(P,Q)+d(Q,R)$ (damit ist $d$ eine Metrik)
\end{itemize}
\end{satz}


\begin{satz}
In einem hyperbolischen Dreieck gilt $\alpha+\beta+\gamma<\pi$ und $\alpha, \beta, \gamma$ definieren ein Dreieck eindeutig (bis auf hyperbolische Bewegung und Konjugation). 
\end{satz}


\begin{satz}
Es existieren genau $7$ Wallpapergruppen bis auf Konjugation. (die 7 auf der Kopie)
\end{satz}



\begin{satz}
$G\subseteq \Iso_0(\R)$ kristallographische Gruppe ist konjugiert in $\R\cdot\Iso$ zu einer der 5 Gruppen auf dem Bild bis auf andere Winkel.
\end{satz}


\begin{satz}
 Ist $(P,G)$ Tiling, dann ist $G$ kristallographische Gruppe.
\end{satz}

\end{document}