% Einige zusätzliche Informationen für rubber
%  rubber erkennt nicht, dass die Datei weg kann, daher sagen wir es ihm
% rubber: clean $base.thm
%  rubber soll nach Änderungen an der Datei nochmal bauen
% rubber: watch $base.thm
% rubber: module          index
% rubber: index.tool      xindy
% rubber: index.language  german-din

\RequirePackage[l2tabu,orthodox]{nag}  % nag überprüft den Text auf veraltete
                   % Befehle oder solche, die man nicht in LaTeX verwenden
                   % soll -- l2tabu-Checker in LaTeX

\documentclass[ngerman,draft,parskip=half,twoside]{scrartcl}

\usepackage{ifthen}
\usepackage{xcolor}
\usepackage[draft=false,colorlinks,bookmarksnumbered,linkcolor=blue,breaklinks]{hyperref}

\usepackage[utf8]{inputenc}
\usepackage{babel}

\usepackage[T1]{fontenc}        % T1-Schriften notwendig für PDFs
\usepackage{lmodern}		% Latin Modern
\usepackage{textcomp}           % wird benötigt, damit der \textbullet
                                % für itemize in lmodern gefunden wird.

\usepackage[intlimits]{amsmath}
\usepackage[all,warning]{onlyamsmath}  % warnt bei Verwendung von nicht
                                       % amsmath-Umgebungen z.\,B. $$...$$
\usepackage{amssymb}     % wird für \R, \C,... gebraucht
\usepackage{fixmath}     % ISO-konforme griech. Buchstaben
\usepackage[euro]{isonums} % definiert Komma als Dezimaltrennzeichen

\usepackage[amsmath,thmmarks,hyperref]{ntheorem} % für die Theorem-Umgebungen
                                                 % (satz, defini, bemerk)
\usepackage{svn}         % Zum Auswerten und ordentlichen Darstellen der
                         % SVN-Schlüsselwörter (s. vor \begin{document})
                         % dafür muss in SVN noch das Flag svn:keywords
                         % auf "LastChangedRevision LastChangedDate"
                         % gesetzt werden
\usepackage{ellipsis}    % Korrektur für \dots
\usepackage{fixltx2e}
\usepackage[final,babel]{microtype} % Verbesserung der Typographie
\usepackage[babel,german=guillemets]{csquotes} % für Anführungszeichen
\usepackage{enumitem}
\usepackage{scrindex}
\usepackage{mathtools}   % Zur Definition von \abs und \norm
\usepackage{nicefrac}
\usepackage{booktabs}

% Damit auch die Zeichen im Mathemode in Überschriften fett sind
% <news:lzfyyvx3pt.fsf@tfkp12.physik.uni-erlangen.de>
\addtokomafont{sectioning}{\boldmath}

\newtheorem{thm}{Satz}[section]

% Hier die Definition, wie \autoref die Umgebungen nennen soll, die mit
% \newtheorem definiert wurden
\newcommand*{\thmautorefname}{Satz}
% Zwischen Unter- und Unterunterabschnitten sollte nicht unterschieden
% werden.
\renewcommand*{\subsectionautorefname}{Abschnitt}
\renewcommand*{\subsubsectionautorefname}{Abschnitt}

\pagestyle{headings}

% Um sicherzustellen, dass jeder Betrag/jede Norm links und rechts die
% Striche bekommt, sind diese Befehle da. Damit kann man nicht die
% rechten Striche vergessen und es wird etwas übersichtlicher. Aus
% mathtools.pdf, z. B. \abs[\big]{\abs{a}-\abs{b}} \leq \abs{a+b}
\DeclarePairedDelimiter{\abs}{\lvert}{\rvert}

% Um wichtige Begriffe im Text überall gleich vorzuheben (gleiches
% Markup), sollte dieser Befehl verwendet werden. Das Argument wird
% automatisch als Indexeintrag verwendet. Dieser kann aber auch als
% optionales Argument selbst bestimmt werden.
\newcommand*{\highl}[2][]{\textbf{\boldmath{#2}}%
  \ifthenelse{\equal{#1}{}}{\index{#2}}{\index{#1}}%
}

% Befehl für die Darstellung der Gliederungsüberschriften im Index
\newcommand*{\lettergroup}[1]{\minisec{#1}}

\newcommand*{\R}{\mathbb{R}}      % reelle Zahlen
\newcommand*{\C}{\mathbb{C}}      % komplexe Zahlen
\newcommand*{\N}{\mathbb{N}}      % natürliche Zahlen
\newcommand*{\Q}{\mathbb{Q}}      % gebrochene Zahlen
\newcommand*{\Z}{\mathbb{Z}}      % ganze Zahlen

\newcommand*{\Algeb}{\mathcal{A}}   % Algebra
\newcommand*{\BorelM}{\mathcal{B}}  % Borelmenge
\newcommand*{\PotM}{\mathcal{P}}    % Potenzmenge
\newcommand*{\E}{\mathbb{E}}        % Erwartungswert
\newcommand*{\V}{\mathbb{V}}        % Varianz
\newcommand*{\WKM}{\mathbb{P}}      % Wahrscheinlichkeitsmaß
\newcommand*{\NormVert}{\mathcal{N}} % Normalverteilung

\DeclareMathOperator{\cov}{cov}     % Kovarianz
\DeclareMathOperator{\card}{card}   % Kardinalität
\DeclareMathOperator{\vol}{vol}     % Volumen einer Menge

% http://www.tug.org/TUGboat/Articles/tb18-1/tb54becc.pdf
\newcommand*{\ez}{\mathrm{e}}               % eulersche Zahl
\newcommand*{\diff}[1]{%                    % Differentialoperator d (wie bei
                                            % \int ... dx)
  \mathop{\mathrm{\mathstrut d}}%
  \ifx#1(\else%
  \ifx#1[\else%
  \ifx#1\{\else%
    \!%
  \fi\fi\fi#1%
}

\SVN $LastChangedRevision$
\SVN $LastChangedDate$

\makeindex

\setlist[enumerate,1]{label=\arabic*.}
\setlist[enumerate,2]{label=\theenumi\arabic*}

\newlist{eigenschaften}{enumerate}{1}
\setlist[eigenschaften]{label=(\roman*)}

\begin{document}

\title{Eine Auswahl wichtiger Definitionen und Aussagen
 zur Vorlesung
  \enquote{Stochastik für Informatiker und Regelschullehrer}}
\date{WS 2008/09}
\author{Werner Linde}
\maketitle

\pdfbookmark[1]{Inhaltsverzeichnis}{inhaltsverzeichnis}
\setcounter{tocdepth}{2}
\tableofcontents

\clearpage
\section{Wahrscheinlichkeiten}
\subsection{Wahrscheinlichkeitsräume}

\subsubsection{Grundraum}

Der \highl{Grundraum} (meist mit $\Omega$ bezeichnet) ist eine Menge, die
mindestens alle bei einem stochastischen Versuch oder Vorgang auftretenden
Ergebnisse enthält. Die Teilmengen von $\Omega$ heißen
\highl[Ereignis]{Ereignisse}, die einpunktigen Teilmengen nennt man
\highl[Elementarereignis]{Elementarereignisse}.

\subsubsection{Eintreten eines Ereignisses}

Ein Ereignis $A\subseteq \Omega$ \highl{tritt ein}, wenn das beim Versuch oder
dem Vorgang beobachtete zufällige Ergebnis in der Menge $A$ liegt.

\subsubsection{\texorpdfstring{$\sigma$}{sigma}"=Algebra}

Auf dem Grundraum $\Omega$ wird ein System $\Algeb\subseteq \PotM(\Omega)$ von
Ereignissen ausgezeichnet, denen man in sinnvoller Weise die Wahrscheinlichkeit
ihres Eintretens zuordnen kann. Aus naheliegenden Gründen fordert man, dass
$\Algeb$ eine \highl[sigma-Algebra@$\sigma$-Algebra]{$\sigma$"=Algebra} bildet,
d.\,h.~$\Algeb$ erfüllt folgende Eigenschaften:
\begin{eigenschaften}
 \item $\emptyset\in\Algeb$.
 \item Aus $A\in\Algeb$ folgt $A^c\in \Algeb$.
 \item $A_1,A_2,\dotsc\in \Algeb$ impliziert $\bigcup_{j=1}^\infty A_j\in\Algeb$.
\end{eigenschaften}

Ist $\Omega$ höchstens abzählbar unendlich, so kann man als $\sigma$"=Algebra
stets die Potenzmenge $\PotM(\Omega)$ von $\Omega$ nehmen.
\subsubsection{Wahrscheinlichkeitsmaß} Ein \highl{Wahrscheinlichkeitsmaß} (oder
eine \highl{Wahrscheinlichkeitsverteilung}) $\WKM$ ist eine Abbildung von
$\Algeb$ nach $[0,1]$, die jedem Ereignis $A\in \Algeb$ die Wahrscheinlichkeit
seines Eintretens zuordnet und folgende Eigenschaften besitzt:
\begin{eigenschaften}
 \item Es gilt $\WKM(\emptyset)=0$ und $\WKM(\Omega)=1$.
 \item $\WKM$ ist $\sigma$"=additiv, d.\,h.~für disjunkte $A_j\in\Algeb$ folgt
  \begin{gather*}
    \WKM\Big(\bigcup_{j=1}^\infty A_j\Big)=\sum_{j=1}^\infty\WKM(A_j)\;.
  \end{gather*}
\end{eigenschaften}

\subsubsection{Wahrscheinlichkeitsraum}

Das Tripel $(\Omega,\Algeb,\WKM)$ heißt \highl{Wahrscheinlichkeitsraum}.
Zufällige Experimente werden durch geeignete Wahrscheinlichkeitsräume
beschrieben.

\subsubsection{Eigenschaften von Wahrscheinlichkeitsmaßen}
%Wahrscheinlichkeitmaße besitzen
%folgende wichtigen Eigenschaften:

\begin{eigenschaften}
 \item Jedes Wahrscheinlichkeitsmaß ist auch \highl{endlich additiv},
  d.\,h.~sind $A_1,\dotsc,A_n$ aus $\Algeb$ disjunkt, so folgt
  \begin{gather*}
    \WKM\Big(\bigcup_{j=1}^n A_j\Big)=\sum_{j=1}^n\WKM(A_j)\;.
  \end{gather*}

 \item Wahrscheinlichkeitsmaße sind \highl{monoton}, d.\,h.~gilt für
  $A,B\in\Algeb$ die Inklusion $A\subseteq B$, so impliziert dies
  $\WKM(A)\le\WKM(B)$.

 \item Für $A,B\in\Algeb$ mit $A\subseteq B$ folgt $\WKM(B\setminus
  A)=\WKM(B)-\WKM(A)$. Insbesondere ergibt sich hieraus
  $\WKM(A^c)=\WKM(\Omega\setminus A)=1-\WKM(A)$ für $A\in\Algeb$.

 \item Wahrscheinlichkeitsmaße sind \highl[stetig!von oben]{stetig von oben},
  d.\,h.~gilt für $A_j\in\Algeb$ die Aussage $A_1\supseteq A_2\supseteq\dotsb$,
  so folgt
  \begin{gather*}
    \WKM\Big(\bigcap_{j=1}^\infty A_j\Big)=\lim_{j\to\infty}\WKM(A_j)\;.
  \end{gather*}

 \item Wahrscheinlichkeitsmaße sind auch \highl[stetig!von unten]{stetig von
  unten}, d.\,h.~gilt für $A_j\in\Algeb$ die Aussage $A_1\subseteq
  A_2\subseteq\dotsb$, so folgt
  \begin{gather*}
    \WKM\Big(\bigcup_{j=1}^\infty A_j\Big)=\lim_{j\to\infty}\WKM(A_j)\;.
  \end{gather*}
\end{eigenschaften}

\subsection{Typen von Wahrscheinlichkeitsmaßen}
\subsubsection{Wahrscheinlichkeitsmaße auf höchstens abzählbar unendlichen Grundräumen}

Bei einem Experiment seien höchstens abzählbar unendlich viele
Versuchsergebnisse möglich. Dann kann man entweder
$\Omega=\{\omega_1,\dotsc,\omega_N\}$ oder aber
$\Omega=\{\omega_1,\omega_2,\dotsc\}$ wählen. Als $\sigma$"=Algebra nimmt man
in diesen Fällen stets die Potenzmenge $\PotM(\Omega)$. Setzt man
\begin{gather}
  \label{zu}
  p_i :=\WKM(\{\omega_i\})\,,\quad 1\le i\le N\quad\mbox{bzw.}\quad i=1,2,\dotsc,
\end{gather}
dann erhält man Zahlen mit den Eigenschaften
\begin{eigenschaften}
 \item $p_i\ge 0$ und
  \label{p1}

 \item $\sum_{i=1}^N p_i =1$ bzw. $\sum_{i=1}^\infty p_i =1$.
  \label{p2}
\end{eigenschaften}

Für eine Menge $A\subseteq \Omega$ folgt dann
\begin{gather}
  \label{diskret}
  \WKM(A):= \sum_{\{i\colon \omega_i\in A\}} p_i\;.
\end{gather}
Umgekehrt, gibt man eine Folge $(p_i)_{i\ge 1}$ reeller Zahlen mit \autoref{p1}
und \autoref{p2} vor, so wird durch \autoref{diskret} ein
Wahrscheinlichkeitsmaß $\WKM$ auf $\PotM(\Omega)$ definiert. Für endliche oder
abzählbar unendliche Grundräume $\Omega$ hat man also folgende Äquivalenz:
\begin{gather*}
  \{\WKM\colon \WKM\;\mbox{Wahrscheinlichkeitsmaß auf}\;\PotM(\Omega)\}
     \Longleftrightarrow
     \{(p_i)_{i\ge 1}\colon (p_i)_{i\ge 1}\;
       \mbox{erfüllen \eqref{p1} und \eqref{p2}}\}
\end{gather*}
Die Zuordnung erfolgt über \autoref{zu} bzw.~\autoref{diskret}.

\subsubsection{Diskrete Wahrscheinlichkeitsmaße}

Sei nunmehr $\Omega$ ein beliebiger Grundraum (nicht notwendig endlich oder
abzählbar unendlich). Ein Wahrscheinlichkeitsmaß $\WKM$ auf
$(\Omega,\PotM(\Omega))$ heißt \highl{diskret}, wenn es eine höchstens
abzählbar unendliche Teilmenge $D\subseteq \Omega$ mit $\WKM(D)=1$ gibt. Mit
$D=\{\omega_1,\omega_2,\dotsc\}$ gilt dann für $A\subseteq \Omega$ wie zuvor
\begin{gather*}
  \WKM(A):= \sum_{\{i\colon \omega_i\in A\}} p_i\;,
\end{gather*}
wobei $p_i:=\WKM(\{\omega_i\})$. Auf höchstens abzählbar unendlichen
Grundräumen ist somit \textbf{jedes} Wahrscheinlichkeitsmaß diskret.

\subsubsection{Wahrscheinlichkeitsdichten}
Eine stückweise stetige Funktion $p\colon\R\mapsto\R$ heißt
\highl{Wahrscheinlichkeitsdichte}, wenn
\begin{eigenschaften}
 \item $p(x)\ge 0$ für $x\in\R$ und

 \item $\int_{-\infty}^\infty p(x)\diff{x}=1$
\end{eigenschaften}
gelten.

\subsubsection{Borel-\texorpdfstring{$\sigma$}{sigma}"=Algebra}

Mit $\BorelM(\R)$ bezeichnet man die kleinste $\sigma$"=Algebra von Mengen aus
$\R$, die die halboffenen Intervalle enthält. Man nennt $\BorelM(\R)$ die
$\sigma$"=Algebra der \highl[Borelmenge]{Borelmengen}. Elemente von
$\BorelM(\R)$ sind z.\,B. alle offenen oder abgeschlossenen Mengen, deren
abzählbaren Vereinigungen und Durchschnitte usw.

\subsubsection{Stetige Wahrscheinlichkeitsmaße}

Gegeben sei eine Wahrscheinlichkeitsdichte $p$. Dann existiert ein eindeutig
bestimmtes Wahrscheinlichkeitsmaß $\WKM \colon\BorelM(\R)\mapsto [0,1]$ mit
\begin{gather*}
  \WKM([\alpha,\beta])=\WKM((\alpha,\beta])=\int_\alpha^\beta\,p(x)\diff{x}
\end{gather*}
für alle reelle Zahlen $\alpha<\beta$. Das so erzeugte Wahrscheinlichkeitsmaß
$\WKM$ heißt \highl{stetig} und $p$ nennt man die \highl{Dichte} von $\WKM$.
Stetige Wahrscheinlichkeitsmaße beschreiben Vorgänge, bei denen überabzählbar
viele reelle Zahlen als Ergebnis auftreten können (z.\,B. Lebenszeiten etc.).

\subsection{Die wichtigsten diskreten Wahrscheinlichkeitsverteilungen}

\subsubsection{Einpunktverteilung}

Gegeben sei ein $\omega_0\in \Omega$, fest aber beliebig. Dann wird durch
\begin{gather*}
  \delta_{\omega_0}(A)
     := \begin{cases}
          1 &\colon \omega_0\in A\\
          0 &\colon \omega_0\notin A
        \end{cases}
\end{gather*}
die \highl{Einpunktverteilung} in $\omega_0$ (oder das \highl{Diracsche
$\delta$-Maß} in $\omega_0$) definiert. Der Wahrscheinlichkeitsraum
$(\Omega,\PotM(\Omega),\delta_{\omega_0})$ beschreibt Vorgänge, bei denen mit
Wahrscheinlichkeit $1$ genau $\omega_0$ eintritt (deterministische Vorgänge).


\subsubsection{Gleichverteilung auf \texorpdfstring{$N$}{N} Punkten}

Gegeben seien $N$ Punkte $\omega_1,\dotsc,\omega_N\in\Omega$. Das Maß $\WKM$
auf $\PotM(\Omega)$ mit
\begin{gather*}
  \WKM:=\frac{1}{N}\sum_{i=1}^N \delta_{\omega_i}
\end{gather*}
heißt \highl{Gleichverteilung} auf $\{\omega_1,\dotsc,\omega_N\}$. Für ein
Ereignis $A$ gilt dann
\begin{gather*}
  \WKM(A)=\frac{\card\{ i\le N \colon \omega_i\in A\}}{N}=
     \frac{\mbox{Anzahl der günstigen Fälle für A}}{\mbox{Anzahl der möglichen Fälle}}\;.
\end{gather*}

\subsubsection{Binomialverteilung}

Sei $\Omega=\{0,\dotsc,n\}$ und sei $p\in[0,1]$ vorgegeben. Dann wird durch
\begin{gather*}
  B_{n,p}(\{k\}):=\binom{n}{k} p^k(1-p)^{n-k},\quad k=0,\dotsc,n\;,
\end{gather*}
ein Wahrscheinlichkeitsmaß $B_{n,p}$ auf $\PotM(\Omega)$ definiert. Man nennt
$B_{n,p}$ \highl{Binomialverteilung} mit den Parametern $n$ und $p$. Die Zahl
$B_{n,p}(\{k\})$ gibt die Wahrscheinlichkeit an, dass man bei $n$ unabhängigen
Versuchen genau $k$-mal Erfolg hat. Dabei ist die Erfolgswahrscheinlichkeit in
jedem einzelnen Versuch $p$, die für Misserfolg $1-p$.

\subsubsection{Hypergeometrische Verteilung} Gegeben seien Zahlen

$M,N,n\in\N_0$ mit $M,n\le N$. Dann wird durch
\begin{gather*}
  H_{N, M ,n}(\{m\}) :=\frac{\binom{M}{m}\binom{N-M}{n-m}}{\binom{N}{n}}
\end{gather*}
ein Wahrscheinlichkeitsmaß $H_{N,M,n}$ auf $\PotM(\{0,\dotsc,n\})$ definiert.
Man nennt $H_{N,M,n}$ \highl[Verteilung!hypergeometrische]{hypergeometrische
Verteilung} mit den Parametern $N,M$ und $n$. Sind in einer Lieferung von $N$
Geräten $M$ Stück defekt, so beschreibt $H_{N,M,n}(\{m\})$ die
Wahrscheinlichkeit, dass man in einer zufällig entnommenen Stichprobe vom
Umfang $n$ genau $m$ defekte Geräte beobachtet.

\subsubsection{Poissonverteilung}
Es sei $\Omega=\N_0=\{0,1,2,\dotsc\}$. Für eine Zahl $\lambda>0$ definiert man
die \highl{Poissonverteilung} mit Parameter $\lambda$ durch
\begin{gather*}
  P_\lambda(\{k\}):= \frac{\lambda^k}{k !}\,\ez^{-\lambda}\quad\mbox{für}\quad k\in\N_0\;.
\end{gather*}

Die Bedeutung der Poissonverteilung ergibt sich aus folgendem Satz:
\begin{thm}
  Gegeben sei eine Zahl $\lambda>0$. Für $n\in\N$ setze man
  $p_n:=\nicefrac{\lambda}{n}$. Dann folgt für alle $k\in\N_0$ stets
  \begin{gather*}
    \lim_{n\to\infty} B_{n,p_n}(\{k\})= P_\lambda(\{k\})\;.
  \end{gather*}
\end{thm}

Inhaltlich bedeutet dies: Führt man sehr viele unabhängige Versuche durch ($n$
Stück), bei denen jeweils nur mit sehr kleiner Wahrscheinlichkeit $p$ Erfolg
eintreten kann, so ist die Anzahl der insgesamt beobachteten Erfolge
approximativ gemäß $P_\lambda$ verteilt, wobei $\lambda= n\cdot p$.

\subsubsection{Geometrische Verteilung}
Bei einem einzelnen Versuch trete Erfolg wieder mit Wahrscheinlichkeit $p$ und
Misserfolg mit Wahrscheinlichkeit $1-p$ auf. Man führt nun so lange unabhängige
Versuche durch, bis man erstmals Erfolg beobachtet. Die Wahrscheinlichkeit,
dass dies im $(k+1)$-ten Versuch mit $k\in\N_0$ geschieht, wird durch die
\highl[Verteilung!geometrische]{geometrische Verteilung} mit Parameter
$p\in(0,1]$ beschrieben:
\begin{gather*}
  \WKM(\{k\}):= p\cdot(1-p)^k\,,\quad k\in\N_0\;.
\end{gather*}

\subsection{Die wichtigsten stetigen Wahrscheinlichkeitsverteilungen}

\subsubsection{Gleichverteilung auf einem Intervall}

Es sei $[a,b]$ ein endliches Intervall. Durch
\begin{gather*}
  p(x)
     := \begin{cases}
          \frac{1}{b-a} &\colon x\in [a,b]\\
          0 &\colon x\notin [a,b]
        \end{cases}
\end{gather*}
wird eine Wahrscheinlichkeitsdichte auf $\R$ definiert. Damit berechnet sich
die Wahrscheinlichkeit eines Intervalls $[\alpha,\beta]$ durch
\begin{gather}
  \label{gleich}
  \WKM([\alpha,\beta])
     =\int_\alpha^\beta p(x)\diff{x}
     = \frac{\mbox{Länge von}\;([\alpha,\beta]\cap[a,b])}{b-a}\;.
\end{gather}
Insbesondere ergibt sich im Fall $[\alpha,\beta]\subseteq [a,b]$ die Formel
\begin{gather*}
  \WKM([\alpha,\beta])= \frac{\beta-\alpha}{b-a}\;,
\end{gather*}
d.\,h.~die Wahrscheinlichkeit des Eintretens von $[\alpha,\beta]\subseteq[a,b]$
hängt nur von seiner Länge, nicht aber von seiner speziellen Lage innerhalb
$[a,b]$, ab. Das durch \autoref{gleich} erzeugte Wahrscheinlichkeitsmaß heißt
\highl{Gleichverteilung} auf dem Intervall $[a,b]$.

\subsubsection{Exponentialverteilung}

Gegeben sei eine Zahl $\lambda>0$. Man definiert die
\highl{Exponentialverteilung} $E_\lambda$ mit Parameter $\lambda>0$ durch ihre
Dichte
\begin{gather*}
  p(x)
     := \begin{cases}
          \lambda\,\ez^{-\lambda x} &\colon x>0\\
          0 &\colon x\le 0
        \end{cases}\;.
\end{gather*}
Für ein Intervall $[\alpha,\beta]\subseteq [0,\infty)$ berechnet sich damit die
Wahrscheinlichkeit seines Eintretens durch
\begin{gather*}
  E_\lambda([\alpha,\beta])=\lambda \int_\alpha^\beta \ez^{-\lambda x}\diff{x}
     = \ez^{-\lambda\alpha}- \ez^{-\lambda\beta}\;.
\end{gather*}

\subsubsection{Normalverteilung}

Gegeben seien Zahlen $\mu\in\R$ und $\sigma>0$. Die Funktion
\begin{gather*}
  p_{\mu,\sigma^2}(x):= \frac{1}{\sqrt{2\pi}\sigma}\,\ez^{-\nicefrac{(x-\mu)^2}{2\sigma^2}}\;,
     \quad x\in\R\;,
\end{gather*}
erzeugt ein Wahrscheinlichkeitsmaß $\NormVert(\mu,\sigma^2)$, das man
\highl{Normalverteilung} mit Mittelwert $\mu$ und Varianz $\sigma^2$ nennt. Es
gilt dann
\begin{gather*}
  \NormVert(\mu,\sigma^2)([\alpha,\beta])=\frac{1}{\sqrt{2\pi}\sigma}
     \int_\alpha^\beta \ez^{-\nicefrac{(x-\mu)^2}{2\sigma^2}}\diff{x}\;.
\end{gather*}
Im Fall $\mu=0$ und $\sigma=1$ erhält man die \highl{Standardnormalverteilung}
$\NormVert(0,1)$. Wahrscheinlichkeiten des Eintretens von Intervallen berechnen
sich in diesem Fall durch
\begin{gather*}
  \NormVert(0,1)([\alpha,\beta])=\frac{1}{\sqrt{2\pi}}
     \int_\alpha^\beta \ez^{-\nicefrac{x^2}{2}}\diff{x}\;.
\end{gather*}

\subsubsection{Gleichverteilung auf einer Menge im \texorpdfstring{$\R^n$}{Rn}}

Es sei $E\subseteq \R^n$ eine beschränkte und abgeschlossene Teilmenge, deren
$n$"=dimensionales Volumen $\vol_n(E)$ man berechnen kann. Man definiert die
\highl[Gleichverteilung!auf $E$]{Gleichverteilung auf $E$} durch den Ansatz
\begin{gather*}
  \WKM(A)=\frac{\vol_n(A\cap E)}{\vol_n(E)}\;.
\end{gather*}

Insbesondere ergibt sich für $A\subseteq E$ die Aussage
\begin{gather*}
  \WKM(A)=\frac{\vol_n(A)}{\vol_n(E)}\;,
\end{gather*}
d.\,h., wie im eindimensionalen Fall hängt die Wahrscheinlichkeit des
Eintretens einer Menge $A\subseteq E$ nur von deren Volumen ab, nicht aber von
deren Lage innerhalb $E$ noch von ihrer Gestalt.

\subsection{Verteilungsfunktion}
\subsubsection{Definition}

Für ein Wahrscheinlichkeitsmaß $\WKM$ auf $(\R,\BorelM(\R))$ wird die
\highl{Verteilungsfunktion} $F\colon\R\mapsto\R$ durch
\begin{gather}
  \label{F0}
  F(t):=\WKM((-\infty,t])\,,\quad t\in\R  \;,
\end{gather}
definiert.

\textit{Hinweis:}
Ist $\WKM$ ein diskretes Wahrscheinlichkeitsmaß auf $(\Omega,\PotM(\Omega))$
mit $\Omega\subseteq\R$, so modifiziert sich die Definition zu
\begin{gather*}
  F(t):=\WKM((-\infty,t]\cap\Omega)\,,\quad t\in\R  \;.
\end{gather*}

\subsubsection{Eigenschaften der Verteilungsfunktion}

\begin{thm}
  \label{VF}
  Die Verteilungsfunktion $F$ eines Wahrscheinlichkeitsmaßes besitzt folgende
  Eigenschaften:
  \begin{eigenschaften}
   \item $\lim_{t\to -\infty} F(t)=0$ und $\lim_{t\to\infty} F(t)=1$,

   \item die Funktion~$F$ ist nichtfallend und

   \item die Funktion~$F$ ist rechtsseitig stetig.
  \end{eigenschaften}
\end{thm}

\subsubsection{Weitere Eigenschaften von Verteilungsfunktionen}

\begin{enumerate}[label=(\alph*)]
 \item Für jedes halboffene Intervall $(\alpha,\beta]$ gilt
  \begin{gather*}
    \WKM((\alpha,\beta])= F(\alpha)-F(\beta)\;.
  \end{gather*}

 \item Die Funktion $F$ besitzt in einem Punkt $t_0\in\R$ genau dann einen
  Sprung der Höhe $h>0$ (man hat $F(t_0)-F(t_0-0)=h$), wenn $\WKM(\{t_0\})=h$
  gilt. Insbesondere hat die Verteilungsfunktion eines diskreten Maßes Sprünge
  in den Punkten, wo die Masse des Maßes konzentriert ist. Dazwischen ist sie
  konstant.

 \item Ist $F$ Verteilungsfunktion eines stetigen Wahrscheinlichkeitsmaßes
  $\WKM$ mit Dichte $p$, so berechnet sich $F$ aus
  \begin{gather*}
    F(t)=\int_{-\infty}^t\,p(x)\diff{x}\,,\quad t\in\R\;.
  \end{gather*}
  Insbesondere gilt für alle $t\in\R$, in denen $p$ stetig ist, die Gleichung
  \begin{gather*}
    F'(t)=\left(\frac{\diff{F}}{\diff{t}} \right)(t)= p(t)\;.
  \end{gather*}
\end{enumerate}

\subsection{Bedingte Verteilungen}

\subsubsection{Definition}

Es sei $(\Omega,\Algeb,\WKM)$ ein Wahrscheinlichkeitsraum. Dann wird für $B\in
\Algeb$ mit $\WKM(B)>0$ die \highl[Wahrscheinlichkeit!bedingte]{bedingte
Wahrscheinlichkeit} $\WKM(\,\cdot\,|B)$ (oder die Wahrscheinlichkeit von $A$
unter der Bedingung $B$) durch
\begin{gather}
  \label{Bed}
  \WKM(A|B):=\frac{\WKM(A\cap B)}{\WKM(B)}\quad\mbox{für}\;\;A\in\Algeb
\end{gather}
definiert. Sie gibt die Wahrscheinlichkeit dafür an, dass $A$ eintritt, unter
der Bedingung, dass $B$ bereits eingetreten ist. Häufig verwendet man
\autoref{Bed} auch in der Form
\begin{gather*}
  \WKM(A\cap B)= \WKM(B)\,\WKM(A|B)\;.
\end{gather*}

\subsubsection{Eigenschaften}

\begin{thm}
  Die Abbildung
  \begin{gather*}
    A\mapsto \WKM(A|B)
  \end{gather*}
  von $\Algeb$ nach $[0,1]$ ist ein Wahrscheinlichkeitsmaß mit den zusätzlichen
  Eigenschaften
  \begin{gather*}
    \WKM(B|B)=1\quad\mbox{und}\quad\WKM(B^c|B)=0\;.
  \end{gather*}
\end{thm}

\subsubsection{Formel über die totale Wahrscheinlichkeit}

\begin{thm}
  \label{total}
  Gegeben seien disjunkte Mengen $B_1,\dotsc,B_n$ in $\Algeb$ mit
  $\WKM(B_j)>0$. Dann gilt für $A\in \Algeb$ mit $A\subseteq\bigcup_{j=1}^n
  B_j$ die Aussage
  \begin{gather*}
    \WKM(A)=\sum_{j=1}^n\WKM(B_j)\cdot\WKM(A|B_j)\;.
  \end{gather*}
\end{thm}

\textit{Bemerkung:}
Insbesondere gilt der Satz im Fall $\bigcup_{j=1}^n B_j=\Omega$ für alle
$A\in\Algeb$.

\subsubsection{Formel von Bayes}
Zur Berechnung von a posteriori Wahrscheinlichkeiten ist die Formel von Bayes
wichtig. Sie besagt das folgende:
\begin{thm}
  Unter den Voraussetzungen aus \autoref{total} an $B_1,\dotsc,B_n$ und $A$
  folgt für $\WKM(A)>0$ die Identität
  \begin{gather}
    \label{Bayes}
    \WKM(B_k|A)=\frac{\WKM(B_k)\cdot\WKM(A|B_k)}{\sum_{j=1}^n
       \WKM(B_j)\cdot\WKM(A|B_j)}\quad\mbox{für}\;\;k=1,\dotsc,n\;.
  \end{gather}
\end{thm}

\textit{Bemerkung:}
Den Nenner in \autoref{Bayes} kann man (falls bekannt) durch $\WKM(A)$
ersetzen.

\subsection{Unabhängigkeit von Ereignissen}
\subsubsection{Unabhängigkeit von zwei Ereignissen}

Gegeben seien zwei Ereignisse $A,B$ aus einem Wahrscheinlichkeitsraum
$(\Omega,\Algeb,\WKM)$. Dann heißen $A$ und $B$ (stochastisch)
\highl{unabhängig}, wenn
\begin{gather*}
  \WKM(A\cap B)=\WKM(A)\,\WKM(B)
\end{gather*}
gilt.

\subsubsection{Eigenschaften}

Die $\emptyset$ und $\Omega$ sind von jeder Menge $A\in\Algeb$ unahängig. Sind
$A$ und $B$ unahängig, dann gilt dies auch für die Paare $A$ und $B^c$
bzw.~$A^c$ und $B^c$. \subsubsection{Unabhängigkeit von \texorpdfstring{$n$}{n}
Ereignissen} Die Ereignisse $A_1,\dotsc,A_n$ aus $\Algeb$ heißen (stochastisch)
\highl{unabhängig}, wenn für alle Teilmengen $I\subseteq\{1,\dotsc,n\}$ stets
\begin{gather}
  \label{unab1}
  \WKM\Bigl(\bigcap_{i\in I} A_i\Bigr)= \prod_{i\in I}\WKM(A_i)
\end{gather}
gilt. Man kann dies auch wie folgt formulieren: Für alle $m\ge 2$ und alle
$1\le i_1<\dotsb<i_m\le n$ hat man
\begin{gather}
  \label{unab2}
  \WKM(A_{i_1}\cap\dotsb\cap A_{i_m})= \WKM(A_{i_1})\dotsm\WKM(A_{i_m})\;.
\end{gather}
Die Ereignisse $A_1,\dotsc,A_n$ aus $\Algeb$ heißen
\highl[unabhängig!paarweise]{paarweise unabhängig}, wenn jeweils zwei
Ereignisse aus $A_1,\dotsc,A_n$ unabhängig sind, d.\,h.~\autoref{unab1} muss
nur für $\card(I)=2$ bzw. \autoref{unab2} nur für $m=2$ erfüllt sein.

\subsubsection{Eigenschaften}

Unabhängige Mengen $A_1,\dotsc,A_n$ sind auch paarweise unabhängig. Die
Umkehrung ist i.\,a.~falsch. Ebenso falsch ist, dass aus
\begin{gather*}
  \WKM(A_{1}\cap\dotsb\cap A_{n})= \WKM(A_{1})\dotsm\WKM(A_{n})
\end{gather*}
stets die Unabhängigkeit der $A_j$ folgt.

Sind $A_1,\dotsc, A_n$ unabhängig, so gilt dies auch für $(A_j)_{j\in J}$ mit
$J\subseteq \{1,\dotsc,n\}$.

\section{Zufallsvariable}
\subsection{Definition und Verteilungsgesetz}

\subsubsection{Das vollständige Urbild}

Für eine Abbildung $X\colon\Omega\mapsto\R$ und eine Teilmenge $B\subseteq\R$
wird das \highl[Urbild!vollständiges]{vollständige Urbild} von $B$ unter $X$
durch
\begin{gather*}
  X^{-1}(B):=\{\omega\in\Omega \colon X(\omega)\in B\}
\end{gather*}
definiert. Verkürzend schreibt man auch $X^{-1}(B)=\;\{X\in B\}$.

\subsubsection{Zufällige Größen}

Sei $\Omega$ eine Menge, die mit einer $\sigma$"=Algebra $\Algeb$ versehen ist.
Eine Abbildung $X\colon\Omega\mapsto\R$ heißt \highl[zufällige!Größe]{zufällige
Größe} oder \highl[Zufallsvariable!reellwertige]{reellwertige Zufallsvariable}
oder \highl[zufällige!reelle Zahl]{zufällige reelle Zahl}, wenn für jedes
$t\in\R$ die Menge $\{\omega\in\Omega\colon X(\omega)\le t\}$ zur
$\sigma$"=Algebra $\Algeb$ gehört.\\ \textit{Bemerkung:} In diesem Fall gilt
dann auch $X^{-1}(B)\in\Algeb$ für jede Borelmenge $B\subseteq \R$.

\subsubsection{Verteilungsgesetz einer zufälligen Größe}

Sei $(\Omega,\Algeb,\WKM)$ ein Wahrscheinlichkeitsraum. Für eine zufällige
Größe $X\colon\Omega\mapsto\R$ ist die Abbildung $\WKM_X \colon
\BorelM(\R)\mapsto [0,1]$ mit
\begin{gather*}
  \WKM_X(B)
     =\WKM\big(X^{-1}(B)\big)
     =\WKM\{\omega\in\Omega \colon X(\omega)\in B\}
     =\WKM(\{X\in B\})
     = \WKM(X\in B)
\end{gather*}
sinnvoll definiert.
\begin{thm}
  Die Abbildung $\WKM_X$ ist ein Wahrscheinlichkeitsmaß auf $(\R,\BorelM(\R))$.
\end{thm}
Man nennt $\WKM_X$ das \highl{Verteilungsgesetz} von $X$ (bzgl.~$\WKM$).

\subsubsection{Typen von zufälligen Größen}

Eine zufällige Größe $X$ heißt \highl{diskret}, wenn $\WKM_X$ ein diskretes
Wahrscheinlichkeitsmaß ist. Damit hat $\WKM_X$ die Gestalt
\begin{gather*}
  \WKM_X(B)=\sum_{\{i\colon x_i\in  B\}} p_i
\end{gather*}
mit geeigneten $x_i\in\R$ und $p_i\ge 0$. Die $x_i$ sind die möglichen Werte
von $X$, d.\,h.~es gilt $\WKM(X\in\{x_1,x_2,\dotsc\})=1$, und
\begin{gather*}
  p_i=\WKM\{\omega\in\Omega \colon X(\omega)=x_i\}\;.
\end{gather*}
Eine zufällige Größe $X$ heißt \highl{stetig}, falls $\WKM_X$ ein stetiges
Wahrscheinlichkeitsmaß ist. Das gilt genau dann, wenn mit einer
Wahrscheinlichkeitsdichte $p$ für alle $\alpha<\beta$ die Gleichung
\begin{gather*}
  \WKM_X([\alpha,\beta])
     =\WKM\{\omega\in\Omega \colon \alpha\le X(\omega)\le\beta\}
     =\int_\alpha^\beta\,p(x) \diff{x}
\end{gather*}
erfüllt ist. Die Funktion $p$ nennt man auch \highl{Verteilungsdichte} (oder
einfach \highl{Dichte}) von $X$.

\subsubsection{Speziell verteilte diskrete zufällige Größen}

Eine zufällige Größe $X$ heißt \highl{gleichverteilt} auf einer endlichen Menge
oder \highl{binomialverteilt} oder \highl{Poissonverteilt} etc., wenn $\WKM_X$
von diesem Typ ist. In allen diesen Fällen ist $X$ diskret. Zum Beispiel ist
$X$ gemäß $B_{n,p}$ verteilt (man schreibt auch $X\sim B_{n,p}$), falls für
$0\le k\le n$ stets
\begin{gather*}
  \WKM_X(\{k\})
     =\WKM\{\omega\in\Omega \colon X(\omega)=k\}
     = \binom{n}{k} p^k(1-p)^{n-k}
\end{gather*}
gilt. Analog ist $X$ gemäß $P_\lambda$ verteilt, sofern für $k\in\N_0$
\begin{gather*}
  \WKM_X(\{k\})
     =\WKM\{\omega\in\Omega \colon X(\omega)=k\}
     =\frac{\lambda^k}{k!}\ez^{-\lambda}\;.
\end{gather*}

\subsubsection{Speziell verteilte stetige zufällige Größen}

Eine zufällige Größe $X$ heißt \highl{gleichverteilt} auf einem Intervall, oder
\highl{exponentialverteilt} oder \highl{normalverteilt} etc., wenn $\WKM_X$ von
diesem Typ ist. Alle diese zufälligen Größen sind stetig. Zum Beispiel ist $X$
gleichverteilt auf $[a,b]$, falls für alle $\alpha<\beta$ stets
\begin{gather*}
  \WKM_X([\alpha,\beta])
     =\WKM\{\omega\in \Omega \colon \alpha\le X(\omega)\le \beta\}
     =\frac{\mbox{Länge von}\,([\alpha,\beta]\cap[a,b])}{b-a}
\end{gather*}
gilt. Oder $X$ ist $\NormVert(\mu,\sigma^2)$"=verteilt (man schreibt $X\sim
\NormVert(\mu,\sigma^2)$), sofern
\begin{gather*}
  \WKM_X([\alpha,\beta])
     =\WKM\{\omega\in \Omega \colon \alpha\le X(\omega)\le \beta\}
     = \frac{1}{\sqrt{2\pi}\sigma}
     \int_\alpha^\beta \ez^{-\nicefrac{(x-\mu)^2}{2\sigma^2}}\diff{x}\;.
\end{gather*}

\subsubsection{Identisch verteilte zufällige Größen}

Zwei zufällige Größen $X$ und $Y$ sind \highl[verteilt!identisch]{identisch
verteilt}, wenn $\WKM_X=\WKM_Y$ gilt, d.\,h.~für alle $B\in\BorelM(\R)$ hat man
\begin{gather*}
  \WKM\{\omega\in \Omega \colon X(\omega)\in B\}
     = \WKM\{\omega\in \Omega \colon Y(\omega)\in B\}\;.
\end{gather*}
Man schreibt dann $X\stackrel{d}{=}Y$.

\subsubsection{Verteilungsfunktion einer zufälligen Größe}
Die \highl{Verteilungsfunktion} $F_X$ einer zufälligen Größe ist die
Verteilungsfunktion ihres Verteilungsgesetzes, d.\,h., es gilt
\begin{gather*}
  F_X(t)
     =\WKM_X\big((-\infty,t]\big)
     =\WKM\{\omega\in\Omega \colon X(\omega)\le t\}\,,\quad t\in \R\;.
\end{gather*}
Für zwei zufällige Größen $X$ und $Y$ gilt genau dann $X\stackrel{d}{=}Y$, wenn
man $F_X=F_Y$ hat.

Die Funktion $F_X$ besitzt die Eigenschaften aus \autoref{VF}.

\subsection{Zufällige Vektoren und Unabhängigkeit zufälliger Größen}
\subsubsection{Zufällige Vektoren}

Sei $\Omega$ eine Menge mit einer $\sigma$"=Algebra $\Algeb$. Eine Abbildung
$\vec X \colon \Omega\mapsto\R^n$ heißt ($n$"=dimensionaler)
\highl[zufällige!Vektor@\nobreak-r Vektor]{zufälliger Vektor}, wenn seine
Koordinatenabbildungen $X_j \colon\Omega\mapsto \R$ alle zufällige Größen sind.
Dabei sind wie üblich die $X_j$ durch
\begin{gather*}
  \vec X(\omega)=(X_1(\omega),\dotsc,X_n(\omega))\;,\quad \omega\in\Omega\,,
\end{gather*}
definiert.

\subsubsection{Gemeinsames Verteilungsgesetz}

Sei $(\Omega,\Algeb,\WKM)$ ein Wahrscheinlichkeitsraum. Dann definiert man wie
im eindimensionalen Fall das Verteilungsgesetz $\WKM_{\vec X}$ von $\vec X$
durch
\begin{gather*}
  \WKM_{\vec X}(B)
     :=\WKM\big(\vec X^{-1}(B)\big)
     =\WKM\{\omega\in\Omega \colon (X_1(\omega),\dotsc,X_n(\omega))\in B\}\;.
\end{gather*}
Im Spezialfall $B=B_1\times\dotsb\times B_n$ für Borelmengen $B_j\subseteq\R$
folgt
\begin{gather*}
  \WKM_{\vec X}(B)=\WKM(X_1\in B_1,\dotsc, X_n\in B_n)\;.
\end{gather*}
Deshalb nennt man $\WKM_{\vec X}$ auch
\highl[Verteilungsgesetz!gemeinsames]{gemeinsames Verteilungsgesetz} der
zufälligen Größen $X_1,\dotsc,X_n$. \subsubsection{Randverteilungen} Für einen
zufälligen Vektor $\vec X$ nennt man die Verteilungsgesetze $\WKM_{X_j}$, $1\le
j\le n$, die \highl[Randverteilung]{Randverteilungen} von $\vec X$. Hierbei
sind wie zuvor die zufälligen Größen $X_j$ die zugehörigen
Koordinatenabbildungen.

\begin{thm}
  Die Randverteilungen berechnen sich aus der gemeinsamen Verteilung durch
  \begin{gather*}
    \WKM_{X_j}(B)
       = \WKM_{\vec X}(\R\times\dotsb\times \underbrace{B}_j
       \times\dotsb\times\R)\;,\quad B\in\BorelM(\R)\;.
  \end{gather*}
  Damit bestimmt die gemeinsame Verteilung die zugehörigen Randverteilungen.
\end{thm}

\textit{Bemerkung:}
Die Umkehrung der obigen Aussage ist i.\,a.~falsch, d.\,h.~es existieren
zufällige Vektoren $\vec X=(X_1,\dotsc,X_n)$ und $\vec Y=(Y_1,\dotsc, Y_n)$ mit
$\WKM_{X_j}=\WKM_{Y_j}$, $1\le j\le n$, aber mit $\WKM_{\vec X}\not=\WKM_{\vec
Y}$.

\subsubsection{Randverteilungen diskreter Vektoren}
\label{disk}

Wir betrachten hier nur den Fall $n=2$. Ein zufälliger $2$"=dimensionaler
Vektor hat die Gestalt $(X,Y)$ mit vorgegebenen zufälligen Größen $X$ und $Y$.
Weiterhin seien $X$ und $Y$ diskret und die Folgen $(x_i)_{i\ge 1}$
bzw.~$(y_j)_{j\ge 1}$ von reellen Zahlen bezeichnen die möglichen Werte von $X$
bzw.~$Y$. Dann nimmt der Vektor $(X,Y)$ die Werte $(x_i,y_j)_{i,j\ge 1}$ an und
für das Verteilungsgesetz von $\WKM_{(X,Y)}$, d.\,h.~die gemeinsame Verteilung
von $X$ und $Y$, gilt
\begin{gather*}
  \WKM_{(X,Y)}(B)
     =\sum_{\{(i,j)\colon (x_i,y_j)\in B\}} p_{ij}\,,\quad B\in\PotM(\R^2)\,,
\end{gather*}
wobei
\begin{gather*}
  p_{ij}= \WKM_{(X,Y)}(\{(x_i,y_j)\})=\WKM(X=x_i,Y=y_j)\;.
\end{gather*}

Für die Randverteilungen ergibt sich dann
\begin{gather*}
  \WKM_X(B)=\sum_{\{i\colon x_i\in B\}} q_i \qquad\mbox{und}\qquad
     \WKM_Y(B)=\sum_{\{j\colon y_j\in B\}} r_j\,,\quad B\in\PotM(\R)\,,
\end{gather*}
mit
\begin{gather*}
  q_i
     =\sum_{j=1}^\infty p_{ij}
     \qquad\mbox{und}\qquad
     r_j=\sum_{i=1}^\infty p_{ij}\;\;\;.
\end{gather*}

\subsubsection{Randverteilungen stetiger Vektoren}
\label{stet}

Zur besseren Übersichtlichkeit betrachten wir auch hier nur den Fall $n=2$. Der
$2$"=dimensionale Vektor $(X,Y)$ sei wie oben definiert. Diesmal nehmen wir
aber an, dass $\WKM_{(X,Y)}$ eine Dichte hat, es also eine Funktion $p
\colon\R^2\mapsto\R$ gibt, so dass für alle $\alpha<\beta$ und $\gamma<\delta$
stets
\begin{gather*}
  \WKM_{(X,Y)}\big([\alpha,\beta]\times[\gamma,\delta]\big)
     =\WKM\{\omega\in \Omega \colon
       \alpha\le X(\omega)\le\beta,\,\gamma\le Y(\omega)\le\delta\}
     =\int_\alpha^\beta\int_\gamma^\delta p(x,y)\diff{y} \diff{x}
\end{gather*}
gilt. Dann haben $X$ bzw.~$Y$ Verteilungsdichten $q$ und $r$ mit
\begin{gather*}
  q(x):=\int_{-\infty}^\infty p(x,y)\diff{y}\qquad\mbox{und}\qquad
     r(y):=\int_{-\infty}^\infty p(x,y)\diff{x}\;.
\end{gather*}

\subsubsection{Unabhängigkeit von zufälligen Größen}

Gegeben seien $n$ zufällige Größen $X_1,\dotsc,X_n$ auf $(\Omega,\Algeb,\WKM)$.
Gilt für beliebige Borelmengen $B_1,\dotsc,B_n\in\BorelM(\R)$ stets

\begin{gather}
  \label{unab}
  \WKM(X_1\in B_1,\dotsc,X_n\in B_n)=\WKM(X_1\in B_1)\dotsm\WKM(X_n\in B_n)\;,
\end{gather}
so heißen $X_1,\dotsc,X_n$ \highl{unabhängig}.

\textit{Bemerkung 1:}
Die Unabhängigkeit der $X_j$ ist äquivalent zu folgender Aussage: Für beliebige
Borelmengen $B_j\in\BorelM(\R)$ sind die Ereignisse
$\left(X_j^{-1}(B_j)\right)_{j=1}^n$ unabhängig. Das folgt aus der Tatsache,
dass man in \autoref{unab} für gewisse vorgegebene $B_j$ auch die reellen
Zahlen $\R$ einsetzen kann.

\textit{Bemerkung 2:}
Es reicht aus, wenn \autoref{unab} mit Intervallen $B_j$ der Form
$(-\infty,t_j]$ für alle $t_j\in\R$ gilt. Die zufälligen Größen
$X_1,\dotsc,X_n$ sind also dann und nur dann unabhängig, wenn für alle
$t_j\in\R$ stets
\begin{gather*}
  \WKM(X_1\le t_1,\dotsc,X_n\le t_n)=\WKM(X_1\le t_1)\dotsm\WKM(X_n\le t_n)
\end{gather*}
folgt.

\textit{Bemerkung 3:}
Aufgrund von \autoref{unab} ist die gemeinsame Verteilung von $X_1,\dotsc,X_n$
im Fall der Unabhängigkeit eindeutig durch ihre Randverteilungen $P_{X_j},$
$1\le j\le n$, bestimmt.

\subsubsection{Spezialfälle}

Besitzen $X$ und $Y$ die Eigenschaften aus \autoref{disk}, so sind $X$ und $Y$
dann und nur dann unabhängig, wenn
\begin{gather*}
  p_{ij}=q_i\cdot r_j\,,\quad 1\le i,j<\infty\;.
\end{gather*}
Im stetigen Fall (\autoref{stet}) sind $X$ und $Y$ genau dann unabhängig, wenn
\begin{gather*}
  p(x,y)=q(x)\cdot r(y)\,,\quad x,y\in\R\;.
\end{gather*}

\subsection{Rechnen mit zufälligen Größen}
\subsubsection{Transformationen}

Eine Abbildung $f\colon\R\mapsto\R$ heißt \highl{messbar}, wenn für jedes
$t\in\R$ die Menge $\{x\in \R \colon f(x)\le t\}$ eine Borelmenge ist. Stetige
Funktionen, Grenzwerte stetiger Funktionen oder auch monotone Funktionen
besitzen diese Eigenschaft.
\begin{thm}
  Sei $X$ eine zufällige Größe und sei $f\colon\R\mapsto\R$ messbar. Dann ist
  $Y:=f(X)$ ebenfalls eine zufällige Größe.
\end{thm}

\textit{Allgemeine Aufgabe:}
Man bestimme $\WKM_Y$ mit Hilfe von $\WKM_X$ und $f$. Folgendes Beispiel
illustriere die Situation: Sei $U$ gleichverteilt auf $[0,1]$, so ist mit
$f(s):=1-s$ auch $Y:=f(U)=1-U$ gleichverteilt auf $[0,1]$.

\subsubsection{Simulation stetiger zufälliger Größen}

Sei $X$ eine stetige zufällige Größe mit Verteilungsfunktion $F_X$. Wir nehmen
an, dass mit zwei Zahlen $-\infty\le a<b\le\infty$ die Verteilungsfunktion
$F_X(a)=0$, $F_X(b)=1$ erfülle und auf $(a,b)$ streng wachsend sei. Dann
existiert die inverse Funktion von $F_X$, die mit $F_X^{-1}$ bezeichnet wird,
und es gilt $F_X^{-1}\colon(0,1)\mapsto (a,b)$.
\begin{thm}
  Sei $U$ eine auf $[0,1]$ gleichverteilte zufällige Größe. Unter den obigen
  Voraussetzungen gilt dann für $Y:=F_X^{-1}(U)$ die Aussage $X\stackrel{d}{=}
  Y$.
\end{thm}

\textit{Anwendung:}
Sind $u_1,\dotsc,u_n$ unabhängig erzeugte reelle Zahlen, die gemäß der
Gleichverteilung aus $[0,1]$ gewählt wurden, so sind die Zahlen $x_j:=
F_X^{-1}(u_j)$ ebenfalls unabhängig und gemäß $\WKM_X$ verteilt.

\subsubsection{Lineare Transformationen}

Für reelle Zahlen $a\not=0$ und $b\in\R$ betrachte man die lineare
Transformation
\begin{gather*}
  Y:=a\,X+ b
\end{gather*}
einer zufälligen Größe $X$.

\begin{thm}
  Im Fall $a>0$ folgt
  \begin{gather*}
    F_Y(t)=F_X\left(\frac{t-b}{a}\right)\;.
  \end{gather*}
  Ist $a<0$, so ergibt sich
  \begin{gather*}
    F_Y(t)=1-\WKM\left(X<\frac{t-b}{a}\right)\;,
  \end{gather*}
  also
  \begin{gather*}
    F_Y(t)=1-F_X\left(\frac{t-b}{a}\right)
  \end{gather*}
  im Fall stetiger $X$.
\end{thm}

\textit{Folgerung:}
Besitzt $X$ die Verteilungsdichte $p$, so hat $Y=a\,X+b$ eine Dichte $q$, die
sich aus $p$ durch
\begin{gather*}
  q(t)=\frac{1}{\abs{a}}\;p\left(\frac{t-b}{a}\right)\,,\quad t\in\R\,,
\end{gather*}
ergibt.

\subsubsection{Addition zufälliger Größen}

Für zwei zufällige Größen $X$  und $Y$ wird ihre Summe $X+Y$ durch
\begin{gather*}
  (X+Y)(\omega):=X(\omega)+Y(\omega)\;,\qquad \omega\in\Omega\,,
\end{gather*}
definiert.

\begin{thm}
  Sind $X$ und $Y$ zufällige Größen, so gilt dies auch für $X+Y$.
\end{thm}
Das Verteilungsgesetz der Summe $X+Y$ kann man für unabhängige zufällige Größen
in einigen Fällen in einfacher Form angeben.

\begin{thm}
  Es seien $X$ und $Y$ unabhängige zufällige Größen.
  \begin{enumerate}
   \item Nehmen $X$ und $Y$ Werte in den ganzen Zahlen $\Z$ an, so folgt
    \begin{gather*}
      \WKM(X+Y=k)
         =\sum_{i=-\infty}^\infty\WKM(X=i)\,\WKM(Y=k-i)\;,\quad k\in \Z\;.
    \end{gather*}

   \item Besitzen $X$ und $Y$ Werte in $\N_0$, so ergibt sich
    \begin{gather*}
      \WKM(X+Y=k)=\sum_{i=0}^k\WKM(X=i)\,\WKM(Y=k-i)\;,\quad k\in \N_0\;.
    \end{gather*}
\end{enumerate}
\end{thm}

Im Fall stetiger zufälliger Größen gilt folgender Satz:
\begin{thm}
  Seien $X$ und $Y$ unabhängig mit Verteilungsdichten $p$ und $q$. Dann besitzt
  $X+Y$ die Verteilungsdichte $r$ mit
  \begin{gather*}
    r(x)
       =\int_{-\infty}^\infty p(x-y)\,q(y)\diff{y}
       = \int_{-\infty}^\infty p(y)\,q(x-y)\diff{y} \;.
  \end{gather*}
\end{thm}

Man nennt $r$ die \highl{Faltung} von $p$ und $q$ und schreibt $r=p*q$.

\subsubsection{Addition speziell verteilter zufälliger Größen}

\begin{thm}
  Im folgenden seien $X$ und $Y$ stets als unabhängig vorausgesetzt. Dann gilt:
  \begin{enumerate}[label=(\alph*)]
   \item Aus $X\sim B_{n,p}$ und $Y\sim B_{m,p}$ folgt $X+Y\sim B_{n+m,p}$.

   \item Aus $X\sim P_\lambda$ und $Y\sim P_\mu$ erhält man $X+Y\sim
    P_{\lambda+\mu}$.

   \item Aus $X\sim\NormVert(\mu_1,\sigma_1^2)$ und
    $Y\sim\NormVert(\mu_2,\sigma_2^2)$ folgt $X+Y\sim
    \NormVert(\mu_1+\mu_2,\sigma_1^2+\sigma_2^2)$.
  \end{enumerate}
\end{thm}

\subsection{Erwartungswert}
\subsubsection{Erwartungswert diskreter zufälliger Größen}

Eine zufällige Größe $X$ nehme Werte $x_1,x_2,\dotsc$ aus $[0,\infty)$ an. Dann
definiert man den \highl{Erwartungswert} von $X$ durch
\begin{gather*}
  \E X :=\sum_{i=1}^\infty x_i\,\WKM(X=x_i)\;.
\end{gather*}
Es gilt dann $0\le \E X \le \infty$.

Sind nunmehr die Werte von $X$ beliebige reelle Zahlen (nicht notwendig $\ge
0$), so sagt man, dass $X$ einen \highl[besitzt Erwartungswert]{Erwartungswert
besitzt}, wenn
\begin{gather*}
  \sum_{i=1}^\infty \abs{x_i}\,\WKM(X=x_i)<\infty\;.
\end{gather*}
In diesem Fall ist der Erwartungswert von $X$ mit
\begin{gather*}
  \E X :=\sum_{i=1}^\infty x_i\,\WKM(X=x_i)
\end{gather*}
eine wohldefinierte reelle Zahl.

\subsubsection{Erwartungswert stetiger zufälliger Größen}

Sei $p$ die Verteilungsdichte einer zufälligen Größe $X$. Dann \highl[besitzt
Erwartungswert]{besitzt $X$ einen Erwartungswert}, wenn
\begin{gather*}
  \int_{-\infty}^\infty \abs{x}\,p(x)\diff{x} <\infty\;,
\end{gather*}
und man definiert den \highl{Erwartungswert} von $X$ durch
\begin{gather*}
  \E X := \int_{-\infty}^\infty x\,p(x)\diff{x}\;.
\end{gather*}

\subsubsection{Beispiele zur Berechnung von Erwartungswerten}
\medskip

\begin{center}
  \begin{tabular}{@{}c@{\qquad}l@{}}
    \toprule
    \textbf{Verteilung von} $\mathbf{X}$
       & \multicolumn{1}{c@{}}{\textbf{Varianz von} $\mathbf{X}$}\\
    \midrule
    $X$ gleichverteilt auf $x_1,\dotsc,x_N$
       & $\E X=\frac{1}{N}\sum_{i=1}^N x_i$\\
    $X\sim B_{n,p}$
       & $\E X= n p$\\
    $X\sim P_\lambda$
       &$\E X= \lambda$\\
    $X$ geometrisch verteilt mit Parameter $p$
       & $\E X = \frac{1-p}{p}$\\
    $X$ gleichverteilt auf $[a,b]$
       & $\E X= \frac{a+b}{2}$\\
    $X\sim E_\lambda$
       & $\E X = \frac{1}{\lambda}$\\
    $X\sim \NormVert(\mu,\sigma^2)$
       &$ \E X = \mu$\\
    \bottomrule
  \end{tabular}
\end{center}

\subsubsection{Eigenschaften des Erwartungswertes}

\begin{thm}
  Der Erwartungswert einer zufälligen Größe hat folgende Eigenschaften:
  \begin{enumerate}
   \item Der Erwartungswert ist linear, d.\,h.~für alle $a,b\in\R$ und
    zufällige Größen $X$ und $Y$ gilt
    \begin{gather*}
      \E(a X+ b Y) = a\,\E X + b\,\E Y\;.
    \end{gather*}

   \item Sei $X$ diskret mit möglichen Werten $x_1,x_2,\dotsc$ aus $\R$. Dann
    existiert für eine Funktion $f \colon\R\mapsto\R$ der Erwartungswert $\E
    f(X)$ genau dann, wenn
    \begin{gather*}
      \sum_{i=1}^\infty \abs{f(x_i)}\,\WKM(X=x_i)<\infty\;,
    \end{gather*}
    und es gilt
    \begin{gather*}
      \E f(X)=\sum_{i=1}^\infty f(x_i)\,\WKM(X=x_i)\;.
    \end{gather*}

   \item Ist $X$ stetig mit Verteilungsdichte $p$, so existiert für eine
    messbare Abbildung $f \colon\R\mapsto\R$ genau dann der Erwartungswert von
    $f(X)$, wenn
    \begin{gather*}
      \int_{-\infty}^\infty \abs{f(x)}\,p(x)\diff{x}<\infty\;,
    \end{gather*}
    und man hat
    \begin{gather*}
      \E f(X)=\int_{-\infty}^\infty  f(x)\,p(x)\diff{x}\;.
    \end{gather*}

   \item Sind $X$ und $Y$ unabhängige zufällige Größen deren Erwartungswert
    existiert, so existiert auch der Erwartungswert von $X\cdot Y$, und es gilt
    \begin{gather*}
      \E(X\cdot Y)= \E X \cdot \E Y\;.
    \end{gather*}
  \end{enumerate}
\end{thm}

\subsection{Varianz und Kovarianz}
\subsubsection{Momente}

Sei $n\in\N$. Eine zufällige Größe $X$ besitzt ein \highl{$n$-tes Moment}, wenn
$\E\abs{X}^n<\infty$. Im diskreten Fall bedeutet dies
\begin{gather*}
  \sum_{i=1}^\infty \abs{x_i}^n\WKM(X=x_i)<\infty
\end{gather*}
und im stetigen
\begin{gather*}
  \int_{-\infty}^\infty \abs{x}^n\,p(x)\diff{x}<\infty\;.
\end{gather*}
Insbesondere hat $X$ ein erstes Moment, genau dann, wenn $\E X$ existiert.

\begin{thm}
  Sei $1\le m\le n$. Hat eine zufällige Größe $X$ ein $n$-tes Moment, so
  besitzt sie auch ein $m$-tes Moment. Insbesondere hat jede zufällige Größe
  mit zweitem Moment einen Erwartungswert.
\end{thm}

\subsubsection{Varianz}

Es sei $X$ eine zufällige Größe mit zweitem Moment. Sei $a:=\E X$. Dann
definiert man die \highl{Varianz} (oder \highl{Streuung}) von $X$ durch
\begin{gather*}
  \V X:= \E(X-a)^2\;.
\end{gather*}
Die Varianz gibt den mittleren quadratischen Abstand einer zufälligen Größe $X$
von ihrem Erwartungswert an. Sie ist ein Maß dafür, wie sehr die Werte von $X$
um $\E X$ schwanken.

\subsubsection{Eigenschaften der Varianz}

\begin{thm}
  Im folgenden seien $X$ und $Y$ zufällige Größen mit zweiten Momenten. Dann
  gelten die folgenden Aussagen:
  \begin{enumerate}
   \item Mit $a:=\E X$ berechnet sich die Varianz für diskrete zufällige Größen
    in der Form
    \begin{gather*}
      \V X = \sum_{i=1}^\infty (x_i-a)^2\,\WKM(X=x_i)\;,
    \end{gather*}
    und im stetigen Fall hat man
    \begin{gather*}
      \V X =\int_{-\infty}^\infty (x-a)^2\,p(x)\diff{x}\;.
    \end{gather*}

   \item Es besteht die Identität
    \begin{gather*}
      \V X = \E X^2 -(\E X)^2\;.
    \end{gather*}

   \item Für eine konstante zufällige Größe $X$ folgt $\V X=0$.

   \item Für $\alpha\in\R$ erhält man
    \begin{gather*}
      \V(\alpha\,X)=\alpha^2\,\V X\;.
    \end{gather*}

   \item Sind $X$ und $Y$ unabhängig, dann gilt
    \begin{gather*}
      \V(X+Y)=\V X + \V Y\;.
    \end{gather*}
  \end{enumerate}
\end{thm}

\subsubsection{Beispiele zur Berechnung von Varianzen}

\medskip

\begin{center}
  \begin{tabular}{@{}c@{\qquad}l@{}}
    \toprule
    \textbf{Verteilung von} $\mathbf{X}$
       & \multicolumn{1}{c@{}}{\textbf{Varianz von} $\mathbf{X}$}\\
    \midrule
    $X$ gleichverteilt auf $x_1,\dotsc,x_N$
       & $\V X=\frac{1}{N}\sum_{i=1}^N (x_i-\E X)^2$\\
    $X\sim B_{n,p}$
       & $\V X= n\,p\,(1-p)$\\
    $X\sim P_\lambda$
       &$\V X= \lambda$\\
    $X$ geometrisch verteilt mit Parameter $p$
       & $\V X = \frac{1-p}{p^2}$\\
    $X$ gleichverteilt auf $[a,b]$
       & $\V X= \frac{(b-a)^2}{12}$\\
    $X\sim E_\lambda$
       & $\V X = \frac{1}{\lambda^2}$\\
    $X\sim \NormVert(\mu,\sigma^2)$
       &$ \V X = \sigma^2$\\
    \bottomrule
  \end{tabular}
\end{center}

\subsubsection{Kovarianz}

Gegeben seien zwei zufällige Größen $X$ und $Y$ mit zweiten Momenten. Seien
$a:=\E X$ und $b:=\E Y$. Dann wird die \highl{Kovarianz} von $X$ und $Y$ durch
\begin{gather*}
  \cov(X,Y):= \E(X-a)(Y-b)
\end{gather*}
definiert.

\textbf{Eigenschaften:}

\begin{enumerate}
 \item Sind $X$ und $Y$ diskret mit möglichen Werten $x_1,x_2,\dotsc$
  bzw.~$y_1,y_2,\dotsc$ aus $\R$, so berechnet sich die Kovarianz aus der
  Formel
  \begin{gather*}
    \cov(X,Y)=\sum_{i,j=1}^\infty (x_i-a)(y_j-b)\,p_{ij}
  \end{gather*}
  wobei
  \begin{gather*}
    p_{ij}=\WKM(X=x_i, Y=y_j)\;.
  \end{gather*}

 \item Hat die Verteilung des zufälligen Vektors $(X,Y)$ eine Dichte $p
  \colon\R^2\mapsto\R$, so ergibt sich die Kovarianz von $X$ und $Y$ aus
  \begin{gather*}
    \cov(X,Y)=\int_{-\infty}^\infty \int_{-\infty}^\infty (x-a)(y-b)\,p(x,y)\diff{x}\diff{y}\;.
  \end{gather*}

 \item Sind $X$ und $Y$ unabhängig, so impliziert dies $\cov(X,Y)=0$,
  d.\,h.~$X$ und $Y$ sind \highl{unkorreliert}. Man beachte, dass aus der
  Unkorreliertheit i.\,a.~nicht die Unabhängigkeit folgt.

 \item Man hat
  \begin{gather}
    \label{cov}
    \abs{\cov(X,Y)}\le (\V X)^{\nicefrac{1}{2}}(\V Y)^{\nicefrac{1}{2}}\;.
  \end{gather}
\end{enumerate}

\subsubsection{Korrelationskoeffizient}

Für zwei zufällige Größen $X$ und $Y$ mit zweiten Momenten definiert man ihren
\highl[Korrelationskoeffizient]{Korrelationskoeffizienten} durch
\begin{gather*}
  \rho(X,Y)
     :=\frac{\cov(X,Y)}{(\V X)^{\nicefrac{1}{2}}(\V Y)^{\nicefrac{1}{2}}}\;.
\end{gather*}

Aus \autoref{cov} folgt
\begin{gather*}
  -1\le \rho(X,Y)\le 1\;.
\end{gather*}
Für unkorrelierte zufällige Größen gilt $\rho(X,Y)=0$. Der
Korrelationskoeffizient ist ein Maß für den Grad der Abhängigkeit von $X$ und
$Y$. Je näher $\rho(X,Y)$ an $1$ oder $-1$ liegt, desto größer ist die
Abhängigkeit zwischen $X$ und $Y$. Im stärksten Fall der Abhängigkeit von $X$
und $Y$, nämlich $Y=X$ bzw.~$Y=-X$, hat man $\rho(X,X)=1$ bzw.~$\rho(X,-X)=-1$.

\appendix
\clearpage
\addsec{Nutzungsbedingungen}

{\itshape
  Dieses Dokument wurde für die auf der Titelseite genannte Vorlesung erstellt
  und wird jetzt im Rahmen des Projekts
  \enquote{\href{http://uni-skripte.lug-jena.de/}{Vorlesungsskripte der
  Fakultät für Mathematik und Informatik}} weiter betreut. Das Dokument wurde
  nach bestem Wissen und Gewissen angefertigt. Dennoch garantiert weder der auf
  der Titelseite genannte Dozent, die Personen, die an dem Dokument mitgewirkt
  haben, noch die Mitglieder des Projekts für dessen Fehlerfreiheit. Für
  etwaige Fehler und dessen Folgen wird von keiner der genannten Personen eine
  Haftung übernommen. Es steht jeder Person frei, dieses Dokument zu lesen, zu
  verändern oder auf anderen Medien verfügbar zu machen, solange ein Verweis
  auf die Internetadresse des Projekts \url{http://uni-skripte.lug-jena.de/}
  enthalten ist.

  Diese Ausgabe trägt die Versionsnummer~\SVNLastChangedRevision{} und ist vom
  \SVNDate{}. Eine neue Ausgabe könnte auf der Webseite des Projekts verfügbar
  sein.

  Jeder ist dazu aufgerufen, Verbesserungen, Erweiterungen und
  Fehlerkorrekturen für das Skript einzureichen bzw. zu melden oder diese
  selbst einzupflegen -- einfach eine E"~Mail an die
  \href{mailto:uni-skripte@lug-jena.de}{Mailingliste
  \nolinkurl{<uni-skripte@lug-jena.de>}} senden. Weitere Informationen sind
  unter der oben genannten Internetadresse verfügbar.

  Hiermit möchten wir allen Personen, die an diesem Skript mitgewirkt haben,
  vielmals danken:
  \begin{itemize}
   \item Prof.\,Werner Linde (2008/09)
  \end{itemize}
}

\clearpage
\pdfbookmark[1]{Index}{index}
\printindex

\end{document}
