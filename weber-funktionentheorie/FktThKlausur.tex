\documentclass[ngerman,halfparskip]{scrartcl}

\usepackage{babel} %Umlaute, neue deutsche Rechtschreibung
\usepackage[utf8]{inputenc} %Kodierung festlegen
\usepackage[T1]{fontenc} % font encoding festlegen
\usepackage{amsmath,amsfonts,amssymb,amsthm} %math. Symbole und Umgebungen
\usepackage{hyperref}
\usepackage{mathrsfs} 
\usepackage{enumerate}

\newtheorem*{satz}{Satz}
\newtheorem*{satz*}{Satz}
\newtheorem*{folg}{Folgerung}
\theoremstyle{definition}
\newtheorem*{defin}{Definition}
\def\R{\mathbb R}
\def\C{\mathbb C}
\def\Z{\mathbb Z}
\def\N{\mathbb N}

\def\Res{\text{Res}}


\begin{document}
\huge Funktionentheorie \normalsize
\section{Holomorphe Funktionen}
\begin{defin}
$f:\Omega \to \C$ \textbf{holomorph} $\Leftrightarrow f$ in jedem Punkt von $\Omega$ stetig differenzierbar ($f\in C^1(\Omega)$)\\(differenzierbar $\Leftrightarrow$ stetig differenzierbar in $\C$)
\end{defin}

\begin{satz}
$f: z= x+iy\mapsto u(x,y)+iv(x,y)$ ist holomorph $\Leftrightarrow$

$u,v$ als reellwertige Funktionen auf $\Omega\subseteq \R^2 ~ C^1$-differenzierbar und $\frac{\partial u}{\partial x}=\frac{\partial v}{\partial y}\quad \frac{\partial v}{\partial x}=-\frac{\partial u}{\partial y}$ (Cauchy-Riemann-Differentialgleichungen)

\end{satz}

\begin{satz} $f$ holomorph in $\Omega$ (Gebiet), $f'=0$ in $\Omega$

$\Rightarrow f$ konstant
\end{satz}

\section*{Komplexe Kurvenintegrale}
\begin{defin}{Komplexe Kurvenintegrale} $f\in C(\Omega), \gamma C^1$-Kurve, $z:[a,b]\rightarrow C$ Parametrisierung.

$\int\limits_\gamma f(z)dz:=\int\limits_a^bf(z(t))\cdot z'(t)dt$ (Unabhängigkeit von Parametrisierung, wie im Reellen)

$\gamma= \gamma_1\oplus \ldots \oplus \gamma_n$ stückweise stetig differenzierbarer Weg (Kurve):

$\int\limits_\gamma f(z)dz=\sum\int\limits_{\gamma_i}$
\end{defin}

$|f(z)|\leq M$ auf $\gamma \Rightarrow |\int\limits_\gamma f(z)|\leq M L(\gamma)$

Integrabilitätsbedingung: $\frac{\partial f_j}{\partial x_k}=\frac{\partial f_k}{\partial x_j} \quad \forall j,k=1,\ldots, n$. 

\begin{folg} (Grundformeln der Funktionentheorie) 

$\int\limits_{C_r(z_0)}\frac{dz}{z-z_0}=2\pi i, \quad \int\limits_{C_r(z_0)}(z-z_0)^ndz=0; \qquad n\in\mathbb Z \backslash \{-1\}$

Speziell: $\int\limits_{C_r(0)}(z)^ndz=\begin{cases}2\pi i, \quad n=-1 \\ 0, \quad n\in \mathbb Z, n\neq -1\end{cases}$
\end{folg}

\begin{satz}(Stammfunktionen)
$f:\Omega\rightarrow \C$ stetig, Kurvenintegral sei wegunabhängig d.h. $\int\limits_\gamma f(z)dz$ nur von Anfangs- und Endpunkt abhängig.

$F(z):=\int\limits_{z_0}^z f(w)dw=\int\limits_\gamma f(w)dw, ~ \gamma$ Weg von $z_0$ nach $z$ ($z_0$ fest).

$\Rightarrow F$ holomorph und Stammfunktion von $f$, d.h. $F'=f$.
\end{satz}

\begin{folg} $\Omega$ sternförmiges Gebiet (oder einfaches Gebiet, d.h. $C^2$-diffeomorphes Bild eines Sternförmigen Gebietes), $f$ holomorph in $\Omega \Rightarrow$

$f$ besitzt Stammfunktion $F$ in $\Omega$ und $\int\limits _\gamma f(z)dz=0$ für jeden geschlossenen Weg in $\Omega$.

\end{folg}
\begin{defin}[komplexer Lograrithmus] $\Omega=\C \setminus \R_- =\{z=re^{i\varphi}: r>0, -\pi < \varphi < \pi \}$ geschlitzte Ebene (sternförmig). $f(z)=\frac 1z$ in $\Omega$ holomorph $\Rightarrow$ (F.3) es existiert Stammfunktion $F$ in $\Omega$ mit $F(1)=0$:

$F(z)[=\int\limits_\gamma \frac{dz}{z}=]=\int\limits_1^r\frac {dt}t +\int\limits _0^\varphi \frac{ire^{it}}{re^{it}}dt=$

$=\log r +i\varphi$ Stammfunktion zu $f$.

$\log z:=\int\limits _1^z\frac{dw}w=\log r+i\varphi$ für $z=re^{i\varphi}, ~r>0, -\pi<\varphi<\pi$ heißt Hauptzweig des Logarithmus (und ist Stammfunktion zu $\frac 1z$ mit $\log 1=0$)

$F$ heißt Zweig des Logarithmus, wenn $e^{F(z)}=z, ~z\in\Omega$.
\end{defin}

\begin{satz} Zu jedem einfachen Gebiet $\Omega$ mit $0\notin\Omega$ gibt es unendlich viele Zweige des Logarithmus. Sie unterscheiden sich um ganzzahlige Vielfache von $2\pi i$.
\end{satz}

\begin{defin}
$f_n:\Omega \rightarrow\C$ stetig, $f_n\rightarrow f$ gleichmäßig auf jeder kompakten Teilmenge von $\Omega$: \textbf{kompakte Konvergenz} auf $\Omega$
\end{defin}

\begin{satz}
$f_n$ (stetig)$\rightarrow f$ kompakt Konvergent auf $\Omega \Rightarrow$

$f$ stetig auf $\Omega$ und $\int\limits_\gamma f_n(z)dz \rightarrow \int\limits_\gamma f(z) dz$ (für jeden Weg $\gamma$)
\end{satz}

\section{Analytische Funktionen}
\begin{satz}
$f(z)=\sum\limits_{n=0}^\infty a_n(z-z_0)^n$ für $|z-z_0|<R$ (konvergent), $0<R\leq \infty \Rightarrow$

$f$ beliebig oft differenzierbar, Ableitungen durch gliedweise Differentiation.  (Konvergenzradius)
\end{satz}

\begin{defin}
$f:\Omega\rightarrow \C$ heißt \textbf{analytisch} $\Leftrightarrow$

$\forall z_0 \in\Omega \exists R > 0 : f$ in $K_r(z_0)$ (offene Kreis) Potenzreihenentwicklung

$f(z):=\sum\limits_{n=0}^\infty a_n (z-z_0)^n$
\end{defin}


$a_n=\frac{f^{(n)}(z_0)}{n!}~~(\forall n)$ eindeutig bestimmt (Koeffizienten)

\begin{satz*}[Identitätssatz für analytische Funktionen]
$f,g$ analytisch auf $\Omega, f(z_n)=g(z_n)$ für $z_n\rightarrow z_0\in\Omega, z_n\neq z_0\quad \Rightarrow
\qquad f\equiv g$.
\end{satz*}


\begin{satz}[Nullstellen analytischer Funktionen]
$f$ auf $\Omega$ analytische und nicht konstant $\Rightarrow$
\begin{enumerate}[a)]
\item $f(z_0)=0 \Rightarrow \exists k\in\mathbb N\setminus \{0\}$ und Umgebung von $z_0$ mit $f(z)=(z-z_0)^kg(z), \quad g(z)\neq 0$
\item $f$ an der Stelle $z_0$ Nullstelle $k$-ter Ordnung $\Leftrightarrow f(z_0)=f'(z_0)=\ldots=f^{(k-1)}(z_0)=0, \quad f^{(k)}(z_0)\neq 0$
\item In jeder komplexen Teilmenge von $\Omega$ hat $f$ höchstens endlich viele Nullstellen.
\end{enumerate}
\end{satz}

\begin{satz*}[Cauchyscher Integralsatz für einfache Gebiete]
$f$ holomorph in sternförmigem (oder allgemein einfachem) Gebiet $\Omega$, $\gamma$  geschlossener Weg $\Rightarrow$
\begin{gather*}\tag{$*$}
\int\limits_\gamma f(z)dz=0
\end{gather*}
\end{satz*}

Kreisring: kein einfaches Gebiet, denn: $\int\limits_{C_r(0)}\frac{dz}z=2\pi i \neq 0$

\paragraph{Bemerkung 2} $f$ in $\Omega=\C \setminus \{z_0\}$ holomorph.

$\int\limits_\gamma f(z)dz +\int\limits_{-C_r(z_0)}f(z)dz=\int\limits_\gamma-\int\limits_{C_r(z_0)}=\int\limits_{\partial M}=0$

$\Rightarrow \boxed{\int\limits _\gamma f(z)dz=\int\limits_{C_r(z_0)}f(z)dz}$

Hierbei: $\gamma$ umläuft $z_0$ einfach positiv, d.h. Strahl $\{z_0+te^{i\varphi}: t\geq 0\}$ trifft $z(t)$ (Parametrisierung von $\gamma$) in genau einem Punkt.



\begin{satz*}[Homologiesatz]
$\Omega$ Gebiet, $f$ in $\Omega\setminus \{z_0\} $ holomorph, $\gamma$ umlaufe $z_0$ einfach positiv $\Rightarrow$

$\int\limits_\gamma f(z)dz =\int\limits_{C_r(z_0)}f(z)dz$ falls Inneres von $\gamma$ und $C_r(z_0)$ in $\Omega$.
\end{satz*}

\begin{defin}
$\gamma_1,\gamma_2$ (geschlossene Wege) \textbf{homolog} in $\Omega \Leftrightarrow$

$\int\limits_{\gamma_1}fdz=\int\limits_{\gamma_2}fdz$ für jedes $f\in C^1(\Omega)$.

(z.B. in Homologiesatz: $\gamma$ und $C_r(z_0)$ homolog in $\Omega\setminus\{z_0\}$
\end{defin}

\section{Cauchysche Integralformel}
\begin{satz}[Cauchysche Integralformel für Kreise]
$f$ holomorph in $\Omega,\quad \overline{K_r(z_0)}\subseteq\Omega \Rightarrow$

$f(z)=\frac 1{2\pi i } \int \limits _{C_r(z_0)}\frac {f(w)}{w-z}dw, \qquad \forall z\in K_r(z_0)$
\end{satz}

\begin{satz}[Cauchysche Integralformel für einfach positiv umlaufende Wege] 
$\gamma$ geschlossener Weg, der jeden von ihm umschlossenen Punkt einfach postitiv umläuft, $\gamma$ und sein Inneres seien in $\Omega$ enthalten, $f$ in $\Omega$ holomorph $\Rightarrow$

$f(z)=\frac 1 {2\pi i} \int\limits_\gamma \frac {f(w)}{w-z}dw, \qquad \forall z$ im Inneren von $\gamma$.
 
\end{satz}

\begin{satz}[Potenzreihenentwicklung holomorpher Funktionen]
$f$ in $\Omega$ holomorph $\Rightarrow$
\begin{enumerate}[a)]
\item $f$ in $\Omega$ analytisch ($\Rightarrow f\in C^\infty(\Omega)$)
\item $f(z)=\sum\limits_{k=0}^\infty a_n (z-z_0)^n$ Potenzreihenentwicklung um $z_0 \in \Omega$

$\Rightarrow$ Konvergenzradius der Reihe mindestens $R=\text{dist} ~ (z_0,\partial\Omega)$.

$a_n=\frac {f^{(n)}(z_0)}{n!}=\frac 1{2\pi i} \int\limits _{C_r(z_0)} \frac{f(z)}{(z-z_0)^{n+1}}dz, \quad \forall r: 0<r<R.$

\end{enumerate}
(Cauchy-Formeln)  
\end{satz}

\begin{satz*}[Identitätssatz für holomorphe Funktionen]
$f,g$ in $\Omega$ holomorph, $z_n\rightarrow z_0 \in\Omega,~ z_n\neq z_0, ~f(z_n)=g(z_n) \Rightarrow f\equiv g$ in $\Omega$
\end{satz*}

\paragraph{Bemerkung}(Fortsetzung reeller analytischer Funktionen) 
Einzige holomorphe Fortsetzung von $e^x, ~x\in\R$ ins Komplexe: $e^z=e^{x+iy}=e^x(\cos y +i \sin y)=\sum\limits _0^\infty\frac{z^n}{n!}, \quad z\in\C$. Ebenso $\sin, \cos, \log$.


\section{Ganze Funktionen}%%Abschnitt 16.5
\begin{satz}
$f=\sum\limits_0^\infty a_n(z-z_0)^n$ für $|z-z_0|<R$.

$M(f,r):=\max\{|f(z)| :~ |z-z_0|=r\} $ für $0<r<R$

$\Rightarrow |a_n|\leq \frac{M(f,r)}{r^n}$ für $n=0,1,\ldots$


\end{satz}

\begin{defin}
$f$ auf ganz $\C $ holom.: $f$ \textbf{ganze Funktion}.

[$\Rightarrow f(z)=\sum\limits_0^\infty a_n z^n, \quad z\in\C \quad (r=\infty) $ (Satz 4.3)]
\end{defin}

\begin{satz*}[Satz von Liouville]
Jede beschränkte ganze Funktion ist konstant.
\end{satz*}

\begin{satz*}[Fundamentalsatz der Algebra] $p(z)$ nichtkonstantes Polynom mit kompl. Koeffizienten

$\Rightarrow p$ besitzt wenigstens eine Nullstelle in $\C$. 

($\Rightarrow$ Fundamentalsatz der Algebra)

\end{satz*}

\begin{satz*}[Satz von Morera]

$f$ in $\Omega$ stet. und $\int\limits_\gamma f(z)dz=0 $ für jed. geschlossenen, einfach gelagerten Weg $\gamma$ in $\Omega$ 

(ausreichend: für jeden geschlossenen Weg, der in einem sternförmigen Gebiet $\Omega'\subseteq\Omega$ liegt)

$\Rightarrow f$ holomorph.

\end{satz*}

\begin{satz*}[Satz über Vertauschung von Differentiation und Grenzübergang]
$f_n$ in $\Omega$ holom. u. komp. konv. gegen $f$ (d.h. $f_n(z)\rightarrow f(z)$ glm. auf jed. komp. Teilmenge von $\Omega$)

$\Rightarrow f$ holom. in $\Omega$ und

$f_n^{(k)}(z)\rightarrow f^{(k)}(z), \quad k\in\N$, im Sinne d. komp. Konv.
\end{satz*}

\paragraph{Zusammenfassung der Hauptsätze}$f$ stet. auf $\Omega$ Dann äquivalent:
\begin{enumerate}[a)]
\item $f$ holom. in $\Omega$
\item $f$ analyt. in $\Omega$
\item $f(x+iy)=u(x,y)+iv(x,y)$ mit $u,v:\Omega\rightarrow \R, \quad C^1$-Fktn. mit $u_x=v_y, \quad v_x=-u_y$
\item $\int\limits_\gamma f(z)dz=0$ für jed. geschlossenen einfach gelagerten Weg $\gamma $in $\Omega$ 
\end{enumerate}


\section{Laurent-Entwicklungen}
\begin{defin}
$f$ holom., $D(f)=\Omega\setminus \{z_o\} \quad z_0\in\Omega:$

$z_0$ heißt \textbf{isolierte Singularität}

$z_0$ \textbf{hebbare Singularität} $\Leftrightarrow \exists$ holom. Fkt. $g$ in $K_\varepsilon(z_0): ~ f(z)=g(z) $ in $K_\varepsilon(z_0)\setminus\{z_0\}$

$z_0$\textbf{Polstelle (Pol)} $\Leftrightarrow \lim\limits_{z\rightarrow z_0}|f(z)|=\infty.$

$z_0$ \textbf{wesentliche Singularität} $\Leftrightarrow$ weder hebare Sing. noch Polstelle
\end{defin}


\begin{defin}
Reihe $\sum\limits_{n=-\infty}^{\infty}a_n(z-z_0)^n$: \textbf{Laurentreihe}.

konvergiert an Stelle $z \Leftrightarrow r(z)=\sum\limits_0^\infty a_n (z-z_0)^n$ und $h(z)=\sum\limits_{n=-\infty}^{-1} a_n (z-z_0)^n:=\sum\limits_{n=1}^\infty a_{-n}\frac 1{(z-z_0)^n}$ konvergent

Dann: $\sum\limits_{-\infty}^\infty a_n(z-z_0)^n:=h(z)+r(z)$

$h(z):$ singulärer Teil, Hauptteil; $r(z)$ regulärer Teil, Nebenteil
\end{defin}

\begin{satz}$\sum\limits_{n=1}^\infty a_{-n}\frac 1{(z-z_0)^n}$
\begin{enumerate}[a)]
\item konvergiert für $z=z_1 \Rightarrow$ absolut konvergent $\forall z: |z-z_0|>|z_1-z_0|$
\item divergiert für $z=z_2 \Rightarrow$ divergent $\forall z: |z-z_0|<|z_2-z_0|$
\end{enumerate}
\end{satz}

\paragraph*{Folgerung 1} Hauptteil konvergiert für $z_1$, regulärer Teil konvergiert für $z_2$, $\rho:=|z_1-z_0|<R:=|z_2-z_0| \Rightarrow$

Laurentreihe konvergiert im Ringgebiet $\rho<|z-z_0|<R$

\begin{satz}
$\sum\limits_{n=-\infty}^\infty a_{n}{(z-z_0)^n}=f(z)$ (konvergent) für $\rho < |z-z_0|<R, \quad 0\leq \rho<R\leq\infty$.

$\Rightarrow$ gleichmäßig konvergent für $\rho+\varepsilon\leq|z-z_0|\leq r, \quad \varepsilon>0, \quad r<R$ (kompakte Kreisring) [$\Rightarrow f$ holomorph]

$a_n=\frac 1 {2\pi i} \int\limits_{C_r(z_0)}\frac {f(w)}{(w-z_0)^{n+1}}dw$ (für $\rho<r<R$).
\end{satz}

\begin{satz*}[Identitätssatz für Laurentreihe] $\sum\limits_{n=-\infty}^\infty a_{n}{(z-z_0)^n}=\sum\limits_{n=-\infty}^\infty b_{n}{(z-z_0)^n}$ für $\rho<|z-z_0|<R$

$\Rightarrow a_n=b_n \quad \forall n\in\Z$ (wegen Formel für $a_n$ Satz 2)

\end{satz*}

\begin{satz}
$f$ holomorph in Ringgebiet, $\Omega=\{z\in\C:\rho<|z-z_0|<R\}$ mit $0\leq\rho<R\leq\infty$.

$\Rightarrow f$ besitzt Reihenentwicklung

$f(z)=\sum\limits_{n=-\infty}^\infty a_{n}{(z-z_0)^n}$ in jeder kompakten Teilmenge von $\Omega$ gleichmäßig konvergent und $a_n=\frac 1 {2\pi i} \int\limits_{C_r(z_0)}\frac {f(w)}{(w-z_0)^{n+1}}dw$ ($\rho<r<R$) Cauchy-Formeln
\end{satz}

\begin{satz}
$f$: Laurentreihe wie oben. 

$M(f,r):=\max \{|f(z)|: ~|z-z_0|=r\}, \quad \rho<r<R$

$\Rightarrow |a_n|\leq \frac {M(f,r)}{r^n} \qquad \forall n\in\Z$

\end{satz} 


\paragraph{Beispiel} (Entwicklung des Kotangens) 

$\cot z=\frac {\cos z}{\sin z}, \quad z\neq n\pi, \quad n\in\Z,$ holomorph

($\sin z=\frac{e^{iz}-e^{-iz}}{2i}\neq 0,\quad \Im z\neq 0:$\\
$e^{iz}-e^{-iz}=0 \Leftrightarrow e^{2iz}=1 \Rightarrow \Im z=0$\\
$e^z=e^x\cdot \underbrace {e^{iy}}_{|\cdot|=1}=1 \Rightarrow e^x=1 \Rightarrow x=0$)

Laurententwicklung für $0<|z|<\pi:$

$z\cdot \cot z=\frac z{\sin z}\cdot \cos z:$ an Stelle $0$ hebbare Singularität.

[$\frac{\sin z}z=1-\frac{z^2}{3!}+\frac{z^4}{5!}-\ldots=:g(z), \quad g(0)\neq 0 \quad$
$g$ analytisch in $\C; \quad g\neq 0$ für $|z|<\pi \Rightarrow g_1=\frac 1g$ analytisch für $|z|<\pi$ ($\sin z = 0 \Leftrightarrow z=k\pi, \quad k\in\Z$)\\
$\Rightarrow z\cdot \cot z= \frac z{\sin z} \cdot \cos z= g_1(z)\cdot \cos z, \quad 0<|z|<\pi$]

$\Rightarrow z\cot z = \sum\limits_{n=0}^\infty a_nz^n, \qquad |z|<\pi$.

($\tan z= \sum\limits_1^\infty (-1)^{n-1}\frac{b_{2n}}{(2n)!}4^n(4^n-1)z^{2n-1}, \quad |z|<\frac \pi 2$\\
$z\cot z=\sum\limits_0^\infty(-1)^n\frac{B_{2n}}{(2n)!}4^nz^{2n}, \quad |z|<\pi$\\
$\sum_{i=0}^n \binom{n+1}{i}B_i=0, \quad n=1,2,\ldots ; \quad B_0=1$)

\begin{gather*} z\cos z =z-\frac{z^3}{2!}+\frac{z^5}{4!}-\frac{z^7}{6!}\pm \ldots =z\cot z \sin z \\
=(a_0z+a_1 z^2 +(a_2-\frac{a_0}{3!})z^3+(a_3-\frac{a_1}{3!})z^4+(a_4-\frac{a_2}{3!}+\frac{a_0}{5!})z^5+ (a_6-\frac{a_4}{3!}+\frac{a_2}{5!}-\frac{a_0}{7!})z^7+\ldots\\
\Rightarrow a_1=a_3=a_5=\ldots =0, \quad a_0=1,\quad a_2=-\frac 13, \quad a_4=-\frac 1{45},\quad a_6=-\frac 2{945}, \ldots\\
\cot z=\frac 1z-\frac 13z-\frac 1{45}z^3-\frac 2{945}z^5-\ldots
\end{gather*}
für $0<|z|<\pi$

\begin{satz*}[Satz von Riemann]
$f$ holomorph, beschränkt für $0<|z-z_0|<R$
$\Rightarrow z_0$ hebbare Singularität.
\end{satz*}

\begin{satz}[Charakterisierung von Polstellen]
$f$ holomorph, $0<|z-z_0|<R$. \\
$f$ Pol in $z_0 \Leftrightarrow f$ von der Form
\begin{gather*}
f(z)=\sum\limits_{n=-m}^\infty a_n(z-z_0)^n, \quad m\in\N\setminus\{0\}, \quad a_{-n}\neq 0.
\end{gather*}
($m$: Ordnung des Pols $z_0$)
\end{satz}

\begin{satz*}[Kriterien für Polstellen]~

\begin{enumerate}[a)]
\item isolierte Singularität $z_0$ von $f$ ist Pol $m$-ter Ordnung $\Leftrightarrow \lim\limits_{z\rightarrow z_0} (z-z_0)^mf(z)$ existiert und $\neq 0$.
\item $g$ holomorph in Umgebung von $z_0$ und $z_0$ Nullstelle $m$-ter Ordnung von $g$

$\Rightarrow \frac 1g$ in $z_0$ Pol $m$-ter Ordnung.

\end{enumerate}
\end{satz*}

\begin{satz*}[von Casorati-Weierstraß]
$z_0$ wesentliche Singularität von $f \Rightarrow$
\begin{enumerate}
\item $\forall a\in\C \exists (z_n): \quad z_n\rightarrow z_0 \land f(z_n)\rightarrow a$.
\item $\exists (w_n): \quad w_n\rightarrow z_0 \land |f(w_n)|\rightarrow\infty$.
\end{enumerate}
\end{satz*}

\begin{defin}
ganze Funktionen, kein Polynom: \textbf{ganz-transzendent}.
\end{defin}

\paragraph{Bemerkung 1} \begin{enumerate}[a)]
\item $f$ ganz-transzendent $\rightarrow f(\frac 1z)$ an der Stelle $0$ eine wesentliche Singularität.
\item $p$ Polynom von Grad $n\geq 1 \Rightarrow p(\frac 1z)$ an der Stelle $0$ ein Pol $n$-ter Ordnung.
\end{enumerate}

 \begin{satz*}[von Casorati-Weierstraß für ganz-transzendente Funktionen]
 $f$ ganz-transzendente Funktion $\Rightarrow$
 \begin{enumerate}
\item $\forall a\in\C\exists (z_n):\quad |z_n|\rightarrow \infty \quad \land \quad f(z_n)\rightarrow a.$
\item $\exists (w_n):\quad |w_n|\rightarrow \infty \land |f(w_n)|\rightarrow \infty$.
\end{enumerate}
\end{satz*}

\begin{satz*}[von Picard]
$f$ an der Stelle $z_0$ wesentliche Singularität $\Rightarrow$

$\exists a\in\C: \quad f$ nimmt in jeder Umgebung von $z_0$ alle Werte von $\C\setminus\{a\}$ an.
\end{satz*}




\section{Residuenkalkül}
\begin{defin}
$f$ holomorph in $\Omega \setminus \{z_0\}, ~z_0$ sei isolierte Singularität. 

$\Rightarrow f(z)=\sum\limits_{n=-\infty}^\infty a_n(z-z_0)^n,\quad 0<|z-z_0|<R.\quad R>0$.

$a_{-1}$: \textbf{Residuum} von $f$ an der Stelle $z_0$.

$\Res (f,z_0):=a_{-1}$.
\end{defin}

$\Rightarrow \Res (f,z_0)=a_{-1}=\frac 1{2\pi i }\int\limits_{C_r(z_0)}f(z)dz=\frac 1{2\pi i }\int\limits_{\gamma}f(z)dz$\hfill (*)

für $0<r<R$ und $\gamma$ Weg, der $z_0$ einfach positiv umläuft und mit seinem Inneren ganz in $\Omega$ liegt. (einfachste Form des Residuensatzes)\\
((*) folgt aus Formel für Koeffizienten von Laurentreihen)

\begin{satz}~\\
\begin{enumerate}[a)]
\item $\Res (f,z_0)=\lim\limits_{z\rightarrow z_0 }(z-z_0) f(z)$, falls GW existiert (d.h. falls $z_0$ hebbare Singularität oder Pol $1.$ Ordnung
\item $\Res (\alpha f+\beta g,z_0)=\alpha \Res(f,z_0)+\beta \Res (g,z_0), \quad \alpha,\beta\in\C$.
\item $f$ in $z_0$ Pol $1.$ Ordnung, $g$ holomorph in Umgebung von $z_0$

$\Rightarrow \Res (f\cdot g, z_0)=g(z_0)\Res (f,z_0)$.
\item $z_0$ einfache Nullstelle von $f \Rightarrow \Res (\frac 1z, z_0) =\frac 1 {f'(z_0)}$
\end{enumerate}
\end{satz}


\begin{satz*}[Residuensatz]
$f$ holomorph in $\Omega$ mit Ausnahme isolierter Singularitäten. Geschlossener Weg $\gamma$ in $\Omega$ treffe keine Singularitäten, dann:

$\int\limits_\gamma f(z) dz=2\pi i \sum\limits_{k=1}^N \Res (f,z_k)$

wobei $z_1,\ldots z_N$ die von $\gamma$ umschlossenen Singularitäten. 

(Allgemeiner und genauer: $\gamma$ sei Rand eines Gebietes $M\subseteq \Omega$, $M$ lasse sich in $M_1,\ldots,M_m$ zerlegen, $\overline M=\bigcup\limits _1^m\overline M_i,~M_i\cap M_j=\emptyset, i\neq j, ~\partial M_j$ umlaufe $M_j$ einfach positiv, in jedem $M_j$ höchstens eine Singularität.)
\end{satz*}

\begin{satz}
$p,q$ teilerfremde Polynome, $\text{Grad}(q)\geq \text{Grad}(p)+2,\quad f(z)=\pi\cot\pi z. \Rightarrow$

$\sum\limits_{n=-\infty}^\infty \frac{p(n)}{q(n)}=-\sum_{q(a)=0}\Res(\frac pq f,a)$. $\quad q\neq 0$ (Summation über Nst. von $q$)
\end{satz}

\begin{satz}
$f$ holomorph in $\C$ mit Ausnahme endlich vieler Singularitäten $z_1,\ldots,z_N$ die nicht auf der rellen Achse liegen. \\
$|f(z)|\leq \frac C{|z|^2}$ für $|z|\geq r, \Im z\geq 0$.

$\Rightarrow \int\limits_{-\infty}^\infty f(x)dx=2\pi i \sum\limits_{\Im z_k >0}\Res (f,z_k)$.

\end{satz}

\begin{satz}[Berechnung von Fourierintegralen $\int\limits_{-\infty}^\infty f(t)e^{ixt}dt$]
$f$ gleiche Vorraussetzungen wie in Satz 3, aber $|f(z)|\leq \frac C{|z|^2}, \quad |z|\geq r$

$g(z):=f(z)e^{ixz}$, dabei $x\in\R$ fest $\Rightarrow$

$$\int\limits_{-\infty}^\infty f(t)e^{ixt}dt=\begin{cases}
2\pi i \sum\limits_{\Im z_k>0}\Res(g,z_k) \text{ für } x>0\\
-2\pi i \sum\limits_{\Im z_k>0}\Res(g,z_k) \text{ für } x<0
\end{cases}$$
Zusatz: Falls statt $|f(z)|\leq \frac C{|z|^2}, \quad |z|\geq r$ ...
\end{satz}

\section{Konforme Abbildungen}

\begin{satz}
$f$ analytisch in Gebiet $\Omega$. Dann bleiben Winkel in allen Punkten erhalten in denen $f'(z)\neq 0$ 
\end{satz}
\begin{satz}
$f$ analytisch im Gebiet $\Omega \Rightarrow $\\
in jedem Punkt $z$ mit $f'(z)\neq 0$ besitzt Abbildung $f$ eine von der Richtugn unabhängige Verzerrung $|f'(z)|$. 

\end{satz}

\begin{defin}
$f$ im Gebiet $\Omega$ \textbf{konforme Abbildung} $\Leftrightarrow f$ besitzt an jeder Stelle $z\in\Omega$ die Eigenschaften \begin{enumerate}
\item Konstanz der Winkel (einschließlich Orientierung)
\item Konstanz der Verzerrung
\end{enumerate}
. 
\end{defin}


\begin{satz}
$f$ ist konforme Abbildung im Gebiet $\Omega \Leftrightarrow f$ analytisch in $\Omega$ und $f'(z)\neq 0$ in $\Omega$.
\end{satz}

\begin{defin}
$f$ konforme Abbildung zweiter Art in $\Omega \Leftrightarrow$ 
\begin{enumerate}
\item Konstanz des Betrags des Winkels, aber Orientierung ändert sich
\item Konstanz der Verzerrung
\end{enumerate}
\end{defin}
\begin{satz*}
$\Omega$ einfach zusammenhängendes Gebiet in $\C$, das mindestens 2 verschiedene Randpunkte bestitzt. $z_o\in\Omega, \quad 0\leq \varphi <2\pi \Rightarrow $ Es existiert genau eine konforme Abbildung $f(z)$ von $\Omega$ auf Einheitskreis $K_1(0)$ mit $f(z_0)=0$ und $f'(z_0)=|f'(z_0)|e^{i\varphi}$. 
\end{satz*}

\end{document}
