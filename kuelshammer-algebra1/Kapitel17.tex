\chapter{Endliche Schiefk�rper}
%%%%%%%%%%%%%%%%%%%
% Definition 17.1
%%%%%%%%%%%%%%%%%%%
\begin{defi}\label{defi:17.1}
Ein Ring $R\neq\left\{0\right\}$, in dem jedes Element $x\neq 0$ invertierbar ist, hei�t \highl{Schiefk�rper}.
\end{defi}

%%%%%%%%%%%%%%%%%%%
% Bemerkung 17.1
%%%%%%%%%%%%%%%%%%%
\begin{bem}\label{bem:17.1}
Ein Schiefk�rper ist also genau dann ein K�rper, wenn er kommutativ (bez�glich der Multiplikation) ist.
\end{bem}

%%%%%%%%%%%%%%%%%%%
% Satz 17.1
%%%%%%%%%%%%%%%%%%%
\begin{satz}[Wedderburn 1905]\label{satz:17.1}
Jeder endliche Schiefk�rper $D$ ist kommutativ, d.\,h. ein K�rper.
\end{satz}

%%%%%%%%%%%%%%%%%%%
% Beweis
%%%%%%%%%%%%%%%%%%%
\begin{beweis}[Witt]
F�r $x\in D$ hei�t $C_D(x)\coloneqq\left\{c\in D\colon cx=xc\right\}$ \highl{Schiefk�rper!Zentralisator}{Zentralisator} von $x$ in $D$.\\
Offenbar ist $C_D(x)$ ein "`Teilschiefk�rper"' von $D$.
Daher ist das Zentrum 
\[Z(D)\coloneqq\bigcap\limits_{x\in D}C_D(x)=\left\{z\in D\colon zx=xz \text{ f�r alle } x\in D\right\}\]
auch ein "`Teilschiefk�rper"' von $D$.
Da $Z(D)$ kommutativ ist, ist $Z(D)$ ein K�rper.
Daher ist $\abs{Z(D)}=:q$ eine Primzahlpotenz.\\
Offenbar ist $D$ ein $Z(D)$-Vektorraum, und f�r jedes $x\in D$ ist $C_D(x)$ ein $Z(D)$-Untervektorraum von $D$.

Wir setzen $n\coloneqq\dim_{Z(D)}D$ und $n_x\coloneqq\dim_{Z(D)}C_D(x)$ f�r $x\in D$.
Dann $\abs{D}=q^n$ und $\abs{C_D(x)}=q^{n_x}$ $(x\in D)$.\\
Seien $K_1,\ldots,K_t$ die Konjugationsklassen der multiplikativen
Gruppe $G\coloneqq U(D)=D\backslash\left\{0\right\}$, dabei sei $\abs{K_i}=1$ f�r $i=1,\ldots,s$ und $\abs{K_i}>1$ f�r $i=s+1,\ldots,t$.
Dann $K_1\cup\dotsc\cup K_s =
Z(G)=Z(D)\backslash\left\{0\right\}$.

F�r $i=1,\ldots,t$ sei $x_i\in K_i$.
Klassengleichung:
\[q^n-1=\abs{G}=\sum\limits_{i=1}^{t}\abs{K_i}=\abs{Z(G)}+\sum\limits_{i=s+1}^{t}\abs{G:C_G(x_i)}=q-1+\sum\limits_{i=s+1}^t\frac{q^n-1}{q^{n_i}-1}\]
mit $n_i\coloneqq n_{x_i}$ $(i=1,\ldots,t)$.
F�r $i=s+1,\ldots,t$ ist $q^{n_i}-1=\abs{C_G(x_i)}\Bigl|\abs{G}=q^n-1$, also:
\[q^{n_i}=\ggt(q^{n_i}-1,q^n-1)=q^{\ggt(n_i,n)}-1\]
nach �bungsaufgabe 5.
Folglich: $n_i\mid n$ (und $n_i<n$ f�r $i=s+1,\ldots,t$).
F�r $m\in\N$ sei $\Phi_m$ das $m$-te Kreisteilungspolynom (also $\Phi_m\in\Z\left[X\right]$).
Bekanntlich ist $X^m-1 = \prod\limits_{d\mid m}\Phi_d$, also $q^m-1=\prod\limits_{d\mid m}\Phi_d(q)$, wegen $\Phi_n(q)\mid q^n-1$ und $\Phi_n(q)\mid\frac{q^n-1}{q^{n_i}-1}$.
Insbesondere $\Phi_n(q)\mid q-1$, d.\,h. $\abs{\Phi_n(q)}\leq q-1$.\\
Wir nehmen an: $D$ nicht kommutativ, d.\,h. $n>1$.
Dann $\Phi_n(q) = \prod\limits_{j=1}^{\varphi(n)}(q-\zeta_j)$, wobei $\zeta_1,\ldots,\zeta_{\varphi(n)}$ die primitiven $n$-Einheitswurzeln in $\C$ sind.\\
Schreibe $\zeta_j=a_j+ib_j$ mit $a_j,b_j\in\R$ $(j=1,\ldots,\varphi(n))$.
Wegen $n>1$ ist $\zeta_j\neq 1$, d.\,h. $a_j<1$.
Daher:
\[\begin{array}{lcl}
\abs{q-\zeta_j}^2 & = & \abs{q-a_j-ib_j}^2\\
& = & (q-a_j)^2+b_j^2\\
& = & q^2-2a_jq+\underbrace{a_j^2+b_j^2}_{=\abs{\zeta_j}^2=1}\\
& = & q^2-2a_jq+1\\
& > & q^2-2q+1\\
& = & (q-1)^2,
\end{array}\]
d.\,h. $\abs{q-\zeta_j}>q-1$.\\
Also: $\abs{\Phi_n(q)}>q-1$. Widerspruch!
\end{beweis}

%%%%%%%%%%%%%%%%%%%
% Beispiel 17.1
%%%%%%%%%%%%%%%%%%%
\begin{bsp}\label{bsp:17.1}
Man zeigt leicht, dass:
\[\HQ\coloneqq \left\{\begin{pmatrix}a & b\\-\overline{b} & \overline{a}\end{pmatrix}\colon a,b\in \C\right\}\]
ein Schiefk�rper ist.
Die Elemente in $\HQ$ hei�en Quaternionen (\textsc{Hamilton}).
Offenbar ist $\HQ$ ein $\R$-Vektorraum der Dimension $4$.
Eine $\R$-Basis bilden die Elemente:
\[\underbrace{\begin{pmatrix}1 & 0\\ 0 & 1\end{pmatrix}}_{1}, 
\underbrace{\begin{pmatrix}i & 0\\ 0 & -i\end{pmatrix}}_{I},
\underbrace{\begin{pmatrix}0 & 1\\ -1 & 0\end{pmatrix}}_{J},
\underbrace{\begin{pmatrix}0 & i\\ i & 0\end{pmatrix}}_{K}\]
Dabei: $I^2 = J^2 = K^2 = -1$, $IJ = K = -JI$, $JK = I = -KJ$, $KI = J = -IK$.
\end{bsp}

%%%%%%%%%%%%%%%%%%%
% Bemerkung 17.2
%%%%%%%%%%%%%%%%%%%
\begin{bem}\label{bem:17.2}
Frobenius hat gezeigt, dass $\R, \C, \HQ$ die einzigen Schiefk�rper sind, die gleichzeitig endliche Dimension �ber $\R$ haben.
\end{bem}