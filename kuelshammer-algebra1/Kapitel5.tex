\chapter{Polynome}
$K$ K�rper

%%%%%%%%%%%%%%%%%%%
% Bemerkung
%%%%%%%%%%%%%%%%%%%
\begin{bem}[vergleiche lineare Algebra]
Ein \highl{Polynom} mit Koeffizienten in $K$ ist eine Folge $\varphi = \left(a_0, a_1, a_2, \ldots\right)$ von Elementen $a_i \in K$ mit $\left|\left\{i \in \N_0 \colon a_i \neq 0\right\}\right| < \infty$.\\
Diese Polynome bilden einen $K$-Vektorraum $P$ mit $\varphi + \psi := \left(a_0 + b_0, a_1 + b_1, \ldots\right)$, $r\varphi := \left(ra_0, ra_1, \ldots\right)$ f�r $\varphi = \left(a_0, a_1, \ldots\right)$, $\psi := \left(b_0, b_1, \ldots\right) \in P, r \in K$.\\
Setzt man $\varphi\psi = \left(c_0, c_1, c_2, \ldots\right)$ mit $c_0 = a_0b_0$, $c_1 = a_0b_1 + a_1b_0, \ldots, c_i := \sum\limits_{j + k = i}a_jb_k, \ldots$, so wird $P$ zu einem kommutativen Ring (\highl{Polynomring}) mit $r\left(\varphi\psi\right) = \left(r\varphi\right)\psi = \varphi\left(r\psi\right)$ f�r $r \in K, \ \varphi, \psi \in P$.\\
Nullelement in $P$ ist das Nullpolynom $(0, 0, \ldots)$, Einselement das Einspolynom $(1, 0, 0, \ldots)$.\\
Ferner hei�t $X := (0, 1, 0, \ldots)$ Unbestimmte oder Variable. F�r $\varphi = \left(a_0, a_1, a_2, \ldots\right) \in P$ ist $\varphi X = \left(0, a_0, a_1, \ldots\right)$, insbesondere ist $X^2 = (0, 0, 1, 0, \ldots)$, $X^3 = (0, 0, 0, 1, 0, \ldots)$ u.\,s.\,w.\\
Wegen $\left|\left\{i \in \N_0 \colon a_i \neq 0\right\}\right| < \infty$ ist also $\varphi = \sum\limits_{i = 0}^{\infty}a_iX^i$, dabei ist wie �blich $X^0$ das Einspolynom. Daher kann man jedes Element in $P$ in der Form $\varphi = \sum\limits_{i = 0}^{\infty}a_iX^i$ mit eindeutig bestimmten Koeffizienten $a_i \in K$ schreiben, von denen nur endlich viele von $0$ verschieden sind. Dies werden wir in Zukunft stets tun. F�r $\varphi = \sum\limits_{i = 0}^{\infty}a_iX^i$, $\psi = \sum\limits_{i = 0}^{\infty}b_iX^i \in P$ und $r \in K$ gilt dann:\\
\begin{itemize}
	\item $\varphi = \psi \Leftrightarrow a_i = b_i \text{ f�r alle } i$
	\item $\varphi + \psi = \sum\limits_{i = 0}^{\infty}\left(a_i + b_i\right)X^i$
	\item $r\varphi = \sum\limits_{i = 0}^{\infty}\left(ra_i\right)X^i$
	\item $\varphi\psi = \sum\limits_{i = 0}^{\infty}\left(\sum\limits_{j + k = i}a_jb_k\right)X^i$
\end{itemize}
Statt $P$ schreiben wir von jetzt ab $K[X]$.\\
Ist $0 \neq \varphi = \sum\limits_{i = }^{\infty}a_iX^i \in K[X]$, so hei�t $d = \deg\varphi = \max{\left\{i \in \N_0 \colon a_i \neq 0\right\}}$ Grad von $\varphi$. Dann ist $\varphi = \sum\limits_{i = 0}^da_iX^i$.\\
Das Nullpolynom erh�lt den Grad $-\infty$.\\
F�r $\varphi, \psi \in K[X]$ und $0 \neq r \in K$ gilt dann:
\begin{enumerate}[(i)]
	\item $\deg{\left(r\varphi\right)} = \deg{\left(\varphi\right)}$
	\item $\deg{\left(\varphi + \psi\right)} \leq \max{\left\{\deg\varphi, \deg\psi\right\}}$
	\item $\deg{\left(\varphi\right)} \neq \deg{\left(\psi\right)} \Rightarrow \deg{\left(\varphi + \psi\right)} = \max{\left\{\deg{\left(\varphi\right)}, \deg{\left(\psi\right)}\right\}}$
	\item $\deg{\left(\varphi\psi\right)} = \deg\varphi + \deg\psi$
\end{enumerate}
F�r $r, s \in K$ ist $rX^0 \pm sX^0 = \left(r + s\right)X^0$.\\
Daher k�nnen wir jeweils $r \in K$ mit $rX^0 \in K[X]$ identifizieren und so $K$ als Teilmenge von $K[X]$ auffassen. Dann nennt man die Elemente in $K$ auch die konstanten Polynome von $K[X]$. Ist $0 \neq \varphi = \sum\limits_{i = 0}^{\infty}a_iX^i \in K[X]$, $d = \deg{\left(\varphi\right)}$ und $a_d = 1$, so hei�t $\varphi$ normiert. Au�erdem betrachtet man das Nullpolynom als normiert. Wir f�r $\Z$ hat man:
\end{bem}

%%%%%%%%%%%%%%%%%%%
% Satz 5.1
%%%%%%%%%%%%%%%%%%%
\begin{satz}[Division mit Rest]\label{satz:5.1}
Seien $\varphi, \psi \in K[X]$ mit $\psi \neq 0$. Dann existiert eindeutig bestimmtes $\kappa, \rho \in K[X]$ mit $\varphi = \kappa\psi + \rho$ und $\deg\rho < \deg\varphi$.
\end{satz}

%%%%%%%%%%%%%%%%%%%
% Definition
%%%%%%%%%%%%%%%%%%%
\begin{defi}
$\kappa$ Quotient, $\rho$ Rest. Im Fall $\rho = 0$ schreibt man $\kappa = \frac{\varphi}{\psi}$
\end{defi}

%%%%%%%%%%%%%%%%%%%
% Beispiel
%%%%%%%%%%%%%%%%%%%
\begin{bsp}
$X^5 + X^4 + X^3 + X^2 + X + 1 = \left(X^3 - X^2 - 1\right)\underbrace{\left(X^2 + 2X + 3\right)}_{\kappa} + \underbrace{\left(4X^2 + 3X + 4\right)}_{\rho}$
\end{bsp}

%%%%%%%%%%%%%%%%%%%
% Bemerkung
%%%%%%%%%%%%%%%%%%%
\begin{bem}
Seien $\varphi, \psi \in K[X]$. Man nennt $\varphi$ Teiler von $\psi$ und schreibt $\varphi\mid\psi$, falls $\psi \equiv \varphi\omega$ f�r ein $\omega \in K[X]$ ist.\\
Dabei gilt wie in $\Z$:
\begin{enumerate}[(i)]
	\item $\varphi\mid 0$, $1\mid\varphi$, $\varphi\mid\varphi$
	\item $\varphi\mid\psi \wedge \psi\mid\omega \Rightarrow \varphi\mid\omega$
	\item $\varphi\mid\psi \Rightarrow c\varphi\mid d\psi \ \ (c, d \in \kappa\backslash\left\{0\right\})$
	\item $\varphi\mid\psi \wedge \psi\mid\varphi \Rightarrow \exists \ c \in \kappa \backslash \left\{0\right\} = \varphi = c\psi$
	\item $0\mid\varphi \Leftrightarrow \varphi = 0$
	\item $\varphi\mid\psi \wedge \varphi\mid\omega \Rightarrow \varphi\mid\alpha\psi + \beta\omega \ \ \left(\alpha, \beta \in K[X]\right)$
\end{enumerate}
Seien $\varphi_1, \ldots, \varphi_n \in K[X]$. Ein $\tau \in K[X]$ mit $\tau\mid\varphi_1, \ldots, \tau\mid\varphi_n$ hei�t \highl{Gemeinsamer Teiler}{gemeinsamer Teiler} von $\varphi_1, \ldots, \varphi_n$.\\
Mit $gT(\varphi_1, \ldots, \varphi_n)$ bezeichnen wir die Menge aller gemeinsamer Teiler von $\varphi_1, \ldots, \varphi_n$. Ein normiertes Polynom $\delta \in gT(\varphi_1, \ldots, \varphi_n)$ hei�t \highl{Gr��ter gemeinsamer Teiler}{gr��ter gemeinsamer Teiler} ($\ggT$), von $\varphi_1, \ldots, \varphi_n$, falls $\tau\mid\delta$ f�r alle $\tau \in gT(\varphi_1, \ldots, \varphi_n)$ gilt.\\
Wie in $\Z$ kann man zeigen, dass $\varphi_1, \ldots, \varphi_n$ stets genau einen $\ggT \delta$ haben. Wir schreiben $\delta = \ggT(\varphi_1, \ldots, \varphi_n)$. Wie in $\Z$ ist $\delta = \ggT(\varphi_1, \ggT(\varphi_2, \ldots, \varphi_n))$.\\
Daher gen�gt es meist, den $\ggT$ von zwei Polynomen zu berechnen. Dies geschieht win in $\Z$ mit dem erweiterten euklidischen Algorithmus:
\end{bem}

%%%%%%%%%%%%%%%%%%%
% Beispiel
%%%%%%%%%%%%%%%%%%%
\begin{bsp}
$\varphi = X^4 + X^3 + 2X^2 + X + 1, \psi = X^4 - X^3 + 2X^2 - X + 1$,\\
Erweiterter euklidischer Algorithmus:\\
\[ \begin{array}{cclc}
1 & 0 & X^4 - X^3 + 2X^2 + X + 1 & \\
0 & 1 & X^4 - X^3 + 2X^2 - X + 1 & 1\\
1 & -1 & 2X^3 + 2X & \frac{1}{2}X - \frac{1}{2}\\
-\frac{1}{2}X + \frac{1}{2} & \frac{1}{2}X + \frac{1}{2} & X^2 + 1 & \\
 & & 0 & 
\end{array}\]
Also $X^2 + 1 = \ggT(\varphi, \psi) = \left(-\frac{1}{2}X + \frac{1}{2}\right)\varphi + \left(\frac{1}{2}X + \frac{1}{2}\right)\psi$
\end{bsp}

%%%%%%%%%%%%%%%%%%%
% Satz 5.2
%%%%%%%%%%%%%%%%%%%
\begin{satz}\label{satz:5.2}
F�r $\varphi_1, \ldots, \varphi_n, \psi \in K[X]$ gilt: \[\ggT(\varphi_1, \ldots, \varphi_n)\mid\psi \Leftrightarrow \exists \ \xi_1, \ldots, \xi_n \in K[X] \colon \xi_1\varphi_1 + \ldots + \xi_n\varphi_n = \psi\]
\end{satz}

%%%%%%%%%%%%%%%%%%%
% Beweis
%%%%%%%%%%%%%%%%%%%
\begin{beweis}
Analog wie in $\Z$.
\end{beweis}

%%%%%%%%%%%%%%%%%%%
% Bemerkung
%%%%%%%%%%%%%%%%%%%
\begin{bem}
F�r $\varphi_1, \ldots, \varphi_n \in K[X]$ ist also genau dann $\ggT(\varphi_1, \ldots, \varphi_n) = 1$, wenn $\xi_1, \ldots, \xi_n \in K[X]$ existiert mit $\xi_1\varphi_1 + \ldots + \xi_n\varphi_n = 1$ Gegebenfalls hei�en $\varphi_1, \ldots, \varphi_n$ \highl{Teilerfremd}{teilerfremd}.
\end{bem}

%%%%%%%%%%%%%%%%%%%
% Definition
%%%%%%%%%%%%%%%%%%%
\begin{defi}
Ein normiertes Polynom $\pi \in K[X] \backslash K$ hei�t \highl{Irreduzibel}{irreduzibel}, falls $\pi$ keine Teiler $\tau$ mit $0 < \deg\tau < \deg\pi$ hat.
\end{defi}

%%%%%%%%%%%%%%%%%%%
% Beispiel
%%%%%%%%%%%%%%%%%%%
\begin{bsp}
\begin{enumerate}[(i)]
	\item Normierte Polynome vom Grad $1$ sind stets irreduzibel
	\item $X^2 + 1 \left(X - i\right)\left(X + i\right)$ ist irreduzibel in $\C[X]$, aber irreduzibel in $\R[X]$.
\end{enumerate}
\end{bsp}

%%%%%%%%%%%%%%%%%%%
% Satz 5.3
%%%%%%%%%%%%%%%%%%%
\begin{satz}[Eindeutige Primfaktorzerlegung]\label{satz:5.3}\index{Eindeutige Primfaktorzerlegung}
Zu jedem $\varphi \in K[X] \backslash K$ existiert bis auf die Reihenfolge eindeutig bestimmte irreduzible Polynome $\pi_1, \ldots, \pi_r \in K[X]$ und eine eindeutig bestimmte Konstante $c \in K$ mit $\varphi = c\pi_1 \cdot \ldots \cdot \pi_r$. Die Teiler von $\varphi$ sind dann die Polynome $d\pi_{i_1} \cdot \ldots \cdot \pi_{i_s}$ mit $d \in K \backslash \left\{0\right\}$ und $1 \leq i_1 < \ldots < i_s \leq r$.
\end{satz}

%%%%%%%%%%%%%%%%%%%
% Bemerkung
%%%%%%%%%%%%%%%%%%%
\begin{bem}
F�r $\varphi = \sum\limits_{i = 0}^{n}a_iX^i \in K[X]$ und $b \in K$ setzt man $\varphi(b) := \sum\limits_{i = 0}^{n}a_ib^i$ und sagt, dass $\varphi(b)$ durch Einsetzen von $b$ in $\varphi$ entsteht. Auf diese Weise definiert $\varphi$ eine \highl{Funktion} $K \mapsto K, b \mapsto \varphi(b)$.\\
Dabei k�nnen verschiedene $\varphi_1, \varphi_2 \in K[X]$ die gleiche Funktion definieren, z.\,B. gilt im Fall $K = \F_2, \varphi_1 = X^2 + X + 1, \varphi_2 = 1 \colon \varphi_1(0) = 1 = \varphi_2(0), \varphi_1(1) = 1 = \varphi_2(1)$, aber $\varphi_1 \neq \varphi_2$.\\
Daher ist wichtig zwischen Polynomen und den durch sie definierten Funktionen zu unterscheiden.\\
F�r $a, b \in K$ und $\varphi, \psi \in K[X]$ gilt offenbar:
\begin{enumerate}[(i)]
	\item $\left(\varphi \mystackrel{\pm}{\cdot} \psi\right)(b) = \varphi(b) \mystackrel{\pm}{\cdot} \psi(b)$
	\item $\left(a\varphi\right)(b) = a\varphi(b)$
\end{enumerate}
Ist $\varphi(b) = 0$, so hei�t $b$ \highl{Nullstelle} von $\varphi$.
\end{bem}

%%%%%%%%%%%%%%%%%%%
% Satz 5.4
%%%%%%%%%%%%%%%%%%%
\begin{satz}\label{satz:5.4}
F�r $\varphi \in K[X]$ und $b \in K$ ist $\varphi(b)$ der Rest bei der Division und $\varphi$ durch $X - b$. Insbesondere gilt $\varphi(b) = 0 \Leftrightarrow X - b\mid\varphi$.
\end{satz}

%%%%%%%%%%%%%%%%%%%
% Beweis
%%%%%%%%%%%%%%%%%%%
\begin{beweis}
Division mit Rest liefert $\kappa, \rho \in K[X]$ mit $\varphi = \left(X - b\right)\kappa + \rho$ und $\rho \in K$.\\
Einsetzen ergibt:\\
$\varphi(b) = \underbrace{(b - b)}_{= 0}\kappa(b) + \rho = \rho$. Daher gilt die erste Behauptung. Im Fall $\varphi(b) = 0$ ist also $\varphi = (X - b)\kappa$, d.\,h. $X - b\mid\varphi$. Umgekehrt gilt im Fall $\varphi = (X - b)\kappa \colon \varphi(b) = \underbrace{(b - b)}_{= 0}\kappa(b) = 0$.
\end{beweis}

%%%%%%%%%%%%%%%%%%%
% Satz 5.5
%%%%%%%%%%%%%%%%%%%
\begin{satz}\label{satz:5.5}
Jedes $\varphi \in K[X]$ mit $n := \deg\varphi \neq -\infty$ hat in $K$ h�chstens $n$ Nullstellen.
\end{satz}

%%%%%%%%%%%%%%%%%%%
% Beweis
%%%%%%%%%%%%%%%%%%%
\begin{beweis}
Seien $a_1, \ldots, a_m$ paarweise verschiedene Nullstellen von $\varphi$ in $K$. Dann sind $X - a_1, \ldots, X - a_m$ paarweise verschiedene irreduzible Polynome in $K[X]$. Nach \autoref{satz:5.4} ist $\varphi$ durch $X - a_1, \ldots, X - a_m$ teilbar, also auch durch $\underbrace{(X - a_1) \cdot \ldots \cdot (X - a_m)}_{\deg \ = m}$. Insbesondere ist $n = \deg\varphi \geq m$.
\end{beweis}

%%%%%%%%%%%%%%%%%%%
% Definition
%%%%%%%%%%%%%%%%%%%
\begin{defi}
F�r $\varphi = \sum\limits_{i = 0}^{\infty}a_iX^i \in K[X]$ definiert man die (\emph{formale})\highl{Ableitung} durch $\varphi' := \sum\limits_{i = 1}^{\infty}ia_iX^{i - 1} \in K[X]$.
\end{defi}

%%%%%%%%%%%%%%%%%%%
% Bemerkung
%%%%%%%%%%%%%%%%%%%
\begin{bem}
Dann gelten die folgende Rechenregeln (�bungs-Aufgabe):
\begin{enumerate}[(i)]
	\item $\left(\varphi + \psi\right)' = \varphi' + \psi'$
	\item $\left(\varphi\psi\right)' = \varphi'\psi + \varphi\psi'$
	\item $\left(\varphi \circ \psi\right)' = \left(\varphi' \circ \psi\right)\psi'$; dabei ist $\varphi \circ \psi$ f�r $\varphi = \sum\limits_{i = 0}^{\infty}a_iX^i$ definiert durch $\varphi \circ \psi := \sum\limits_{i = 0}^{\infty}a_i\psi^i$.
\end{enumerate}
\end{bem}