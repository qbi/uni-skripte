\chapter{Polynome und Codes}
\begin{itemize}
	\item Ziel der Codierungstheorie:
	\begin{itemize}
		\item sichere Daten�bertragung und -speicherung [Funkverkehr zu Satelliten, Mobilfunk, CD, DVD, ... ]
		\item automatische Erkennung und Korrektur von �bertragungsfehlern [Rauschen, Staub, Kratzer, Materialfehler, ...]
	\end{itemize}
	\item durch geeignete Codierung und Decodierung:
	\[\begin{array}{ccccccccc}
\text{Nachricht} & \rightarrow & \text{Codierer} & \rightarrow & \text{Kanal} & \rightarrow & \text{Decodierer} & \rightarrow & \text{Nachricht}\\
& & & & \uparrow & & & &\\
& & & & \text{Rauschen} & & & &
\end{array}\]
\end{itemize}

Dabei:
\begin{itemize}
	\item Nachricht = Vektor in $K^k$ $\left(K \text{ endlicher K�rper, } \left|K\right| = q\right)$ [oft $K = \F_2$, aber: bei CD, DVD: $K = \F_256$]
	\item Codierung: injektive Abbildung\\
	$\underbrace{E}_{\text{Encode}} \colon K^k \mapsto K^n \ \left(n > k\right)$  [oft: $E$ linear]
	\item Decodierung: $D \colon K^n \mapsto K^k$ mit $D \circ E = id_{K^k}$
	\item Hilfreich \addindex{Hamming!Abstand}{Hamming--Abstand} \randbem{Hamming: Pionier der Codierungstheorie} $d \colon K^n \times K^n \mapsto \R$ \randbem{$d$ Metrik}
	\[d\left(x, y\right) := \left|\left\{i \colon 1 \leq i \leq n, x_i \neq y_i\right\}\right| \text{ f�r } x = \left(x_1, \ldots, x_n\right), \left(y_1, \ldots, y_n\right) \in K^n\] misst Anzahl der Fehler:
	\begin{itemize}
		\item $d\left(x, y\right) \geq 0$, $d\left(x, y\right) = 0 \Leftrightarrow x = y$
		\item $d\left(x, y\right) = d\left(y, x\right)$
		\item $d\left(x, z\right) \leq d\left(x, y\right) + d\left(y, z\right)$ (Dreiecksungleichung)
	\end{itemize}
	\item \addindex{Hamming!Gewicht}{Hamming--Gewicht}: $w \colon K^n \mapsto \R$,
	\[w(x) = \left|\left\{i \colon 1 \leq i \leq n, x_i \neq 0\right\}\right| \text{ f�r } x = \left(x_1, \ldots, x_n\right) \in K^n\]
	Dann: $d\left(x, y\right) = w\left(x - y\right)$
\end{itemize}

%%%%%%%%%%%%%%%%%%%
% Definition 8.1
%%%%%%%%%%%%%%%%%%%
\begin{defi}{defi:8.1}
Ein (linearer) Code der L�nge $n$ ist ein Untervektorraum $C \subseteq K^n$. Man nennt $k := \dim C$ die Dimension von $C$. Au�erdem hei�t
\[\delta := \min{\left\{d\left(x, y\right) \colon x, y \in C, x \neq y\right\}} = \min{\left\{w(z) \colon 0 \neq z \in C\right\}}\]
\addindex{}{Minimalabstand} (\addindex{}{Minimalgewicht}) von $C$. Man spricht dann auch von einem $\left[n, k, \delta\right]$--Code.
\end{defi}

\emph{Idee:} $C$ ist das Bild von $E \colon K^k \mapsto K^n$. Ist $e \in E$ mit $2e < \delta$ so kann $C$ mindestens $e$ Fehler korrigieren. (Kreise von Radius $e$ um Codew�rter schneiden sich nicht.)

Ein "`guter"' Code hat folgende Eigenschaften:
\begin{itemize}
	\item gro�er Minimalabstand ($\Rightarrow$ gute Korrekturm�glichkeit)
	\item $\underbrace{\text{gro�e Dimension}}_{\text{m�glichst viele Codew�rter in } C}$ ($\Rightarrow$ kosteng�nstig, effizient)
\end{itemize}
Beide Ziele liegen in verschiedenen Richtungen (Kompromisse!).

%%%%%%%%%%%%%%%%%%%
% Bemerkung 8.1
%%%%%%%%%%%%%%%%%%%
\begin{bem}\label{bem:8.1}
Viele \randbem{Vorteil dieser Konstruktion, zu Codes als Untervektorraum: man muss nur $\varphi$ und $n$ speichern, keine Basis eines Untervektorraumes mit $k$ Elementen} in der Praxis auftretende Codes werden mit Hilfe von Polynomringen konstruiert. Dabei identifiziert man $K^n$ mit $K[X] / \left(X^n - 1\right)$ durch \[\left(a_0, a_1, \ldots, a_{n-1}\right) \leftrightarrow a_0 + a_1X + a_2X^2 + \ldots + a_{n-1}X^{n-1} + \left(X^n - 1\right)\]
F�r jeden Teiler $\varphi$ von $X^n - 1$ gilt: $\left(X^n - 1\right) \subseteq \left(\varphi\right)$.\\
Daher ist $C := \left(\varphi\right) / \left(X^n - 1\right) := \left\{\alpha\varphi + \left(X^n - 1\right) \colon \alpha \in K[X]\right\}$ ein Untervektorraum von $K[X] / \left(X^n - 1\right)$. Man nennt $C$ den \addindex{Zyklischer Code}{zyklischen Code} der L�nge $n$ �ber $K$ mit \addindex{}{Generatorpolynom} $\varphi$.
\end{bem}

%%%%%%%%%%%%%%%%%%%
% Satz 8.1
%%%%%%%%%%%%%%%%%%%
\begin{satz}\label{satz:8.1}
Sei $C := (\varphi) / \left(X^n - 1\right)$ wie oben und $l := \deg\varphi$, also $l \leq n$. Dann gilt: $\dim C = n - l$.
\end{satz}

%%%%%%%%%%%%%%%%%%%
% Beweis
%%%%%%%%%%%%%%%%%%%
\begin{beweis}
Schreibe $\varphi = a_0 + a_1X + \ldots + a_lX^l \ \left(a_0, \ldots, a_l \in K, a_l \neq 0\right)$.\\
Dann sind 
\[\begin{array}{rcllllll}
\varphi + \left(X^n - 1\right) & = & a_0 & + & a_1X + & & \ldots + a_lX^l & + \left(X^n - 1\right)\\
X\varphi + \left(X^n - 1\right) & = & & & a_0X + a_1X^2 + & & \ldots + a_lX^{l + 1} & + \left(X^n - 1\right)\\
& \vdots \\
X^{n - l - 1}\varphi + \left(X^n - 1\right) & = & & & & a_0X^{n - l - 1} + & \ldots + a_lX^{n - 1} & + \left(X^n - 1\right)
\end{array}\]
offenbar linear unabh�ngig in $C$.\\
Also: $\dim C \geq n - l$.\\
Wegen $\varphi\mid X^n - 1$ existiert ein $\psi \in K[X]$ mit $X^n - 1 = \varphi\psi$. Sei $\alpha \in (\varphi)$, etwas $\alpha = \beta\varphi$ mit $\beta \in K[X]$.\\
Division mit Rest: $\beta = \kappa\psi + \rho$ mit $\deg\rho < \deg\psi = n - l$.\\
Schreibe $\rho = r_0 + r_1X + \ldots + r_{n - l - 1}X^{n - l - 1} \ \left(r_i \in K\right)$.\\
Dann: \[
\begin{array}{lcl}
\alpha + \left(X^n - 1\right) & = & \beta\varphi + \left(X^n - 1\right) = \kappa\underbrace{\varphi\psi}_{= \left(X^n - 1\right)} + \rho\varphi + \left(X^n - 1\right)\\
& = & \rho\varphi + \left(X^n - 1\right)\\
& = & r_0\varphi + r_1X\varphi + \ldots + r_{n - l - 1}X^{n - l - 1}\varphi + \left(X^n - 1\right)
\end{array}\]
Daher bilden $\varphi + \left(X^n - 1\right), X\varphi + \left(X^n - 1\right), \ldots, X^{n - l - 1}\varphi + \left(X^n - 1\right)$ eine Basis von $C$.
\end{beweis}

%%%%%%%%%%%%%%%%%%%
% Beispiel 8.1
%%%%%%%%%%%%%%%%%%%
\begin{bsp}\label{bsp:8.1}
\begin{enumerate}[(a)]
	\item (\addindex{Code!Tern�rer Golay}{Tern�rer Golay Code})\\
	$q = 3, n = 11, \varphi = X^5 + X^4 - X^3 + X^2 - 1 \in \F_3[X], C := (\varphi) / \left(X^{11} - 1\right)$. [Man kann zeigen, dass $\varphi$ ein irreduzibler Teiler von $X^{11} - 1$ ist.]\\
	Dann hat $C$ L�nge $n = 11$ und Diemension $k = 6$.\\
	Indem \randbem{Minimalgewicht bei Polynomen: Anzahl von $0$ verschiedenen Koeffizienten.} man die $3^6 = 729$ Codew�rter berechnet (Computer!), sieht man, dass $C$ das Minimalgewicht $\delta = 5$ hat. (Das geht auch mit etwas Theorie.)\\
	Daher schneiden sich Kreise mit Radius $2$ um Codew�rter nicht. Jeder Kreis enth�lt: 
	\[1 + \underbrace{11 \cdot 2}_{\text{Vektoren mit Abstand } 1} + \underbrace{\binom{11}{2} \cdot 4}_{\text{Abstand } 2} = 243 = 3^5\]
	Vektoren. \randbem{Code kann 2 Fehler korrigieren.} Daher enthalten die Kreise vom Radius $2$ umd die $3^6$ Codew�rter, insgesamt genau $3^6 \cdot 3^5 = 3^{11}$ verschiedene Vektoren, �berdecken also ganz $\F_3^{11}$. Man spricht von einem \addindex{Code!perfekter}{perfekten Code}.
	\item (\addindex{Code!Bin�rer Golay}{Bin�rer Golay Code})\\
	$q = 2, n = 23, \varphi := X^{11} + X^{10} + X^6 + X^5 + X^4 + X^2 + 1 \in \F_2[X], C := (\varphi) / \left(X^n - 1\right)$. [Man kann zeigen, dass $\varphi$ ein irreduzibler Teiler von $X^{23} - 1$ ist.]\\
	Dann hat $C$ L�nge $n = 23$ und Dimension $k = 12$.\\
	Indem man die $2^{12} = 4096$ Codew�rter ausrechnet (Computer!), sieht man, dass $C$ Minimalgewicht $\delta = 7$ hat. (Das geht auch mit etwas Theorie.)\\
	Daher schneiden sich Kreise vom Radius $3$ um Codew�rter nicht. Jeder Kreise enth�lt:
	\[1 + 23 + \binom{23}{2} + \binom{23}{3} = 2048 = 2^{11}\]
	Vektoren. Diese $2^{12}$ Kreise enthalten insgesamt $2^{12} \cdot 2^{11} = 2^{23}$ Vektoren, �berdecken ganz $\F_2^{23}$. Man spricht wieder von einem perfekten Code.
\end{enumerate}
\end{bsp}

%%%%%%%%%%%%%%%%%%%
% Bemerkung 8.2
%%%%%%%%%%%%%%%%%%%
\begin{bem}\label{bem:8.2}
Die Golay-Codes sind sowohl f�r die Theorie als auch f�r die Praxis von gro�er Bedeutung. Man kann zeigen, dass sie im wesentlichen die beiden einzigen perfekten Codes sind, die mehr als einen Fehler korrigieren k�nnen.
\end{bem}