\chapter{Elementare Zahlentheorie}
%%%%%%%%%%%%%%%%%%%
% Satz 1.1
%%%%%%%%%%%%%%%%%%%
\index{Division mit Rest}
\begin{satz}[Division mit Rest]\label{satz:1.1}
Zu a $\in \Z$ und m $\in \N$ \randbem{$\N = [1, \infty[$} existiert stets eindeutig bestimmt q, r $\in \Z$ mit $a = qm + r$ und $0 \leq r < m$.
\end{satz}

%%%%%%%%%%%%%%%%%%%
% Definition 1.1
%%%%%%%%%%%%%%%%%%%
\begin{defi}
q Quotient, r Rest
\end{defi}

%%%%%%%%%%%%%%%%%%%
% Beweis
%%%%%%%%%%%%%%%%%%%
\begin{beweis}
Sei $R := \{a - qm \colon q \in \Z\}$. Dann: $R \cap \N_{0} \neq \varnothing$.\\
Denn: f�r $a \geq 0$ ist $a - 0m = a \in \N_{0}$, und f�r $a < 0$ ist $a - am = a(1 - m) \in \N_{0}$ ein kleinstes Element.\\
Sei $r := a - qm := min R \cap \N_{0} (q \in \Z)$.\\
Dann: $r \geq 0$, aber $r - m = a - (q + 1)m < 0$, d.h. $r < m$.\\
Also q, r wie gew�nscht.\\\\
Seien auch $q'$, $r' \in \Z$ mit $a = q'm + r'$ und $0 \leq r' < m$.\\
Dann: $r - r' = (q' - q)m$.\\
Wegen $\left|r - r'\right| < m$ folgt $q' - q = 0$ sonst $\left|(q' - q)m\right| \geq m$.\\
Also: $q' = q$, d.h. $r' = r$.
\end{beweis}

%%%%%%%%%%%%%%%%%%%
% Definition 1.2
%%%%%%%%%%%%%%%%%%%
\begin{defi}\label{defi:1.2}
Man nennt $a \in \Z$ \addindex{}{Teiler} von $b \in \Z$ und schreibt $a\mid b$, wenn ein $c \in \Z$ mit $b = ac$ existiert. Man sagt auch a teilt b, b ist Vielfaches von a, b ist durch a teilbar usw.
\end{defi}

%%%%%%%%%%%%%%%%%%%
% Satz 1.2
%%%%%%%%%%%%%%%%%%%
\begin{satz}\label{satz:1.2}
F�r alle a, b, c, x, y $\in \Z$ gilt:
\begin{enumerate}[(i)]
	\item $a\mid 0$, $1\mid a$, $a\mid a$
	\item $a\mid b \Rightarrow \pm a\mid\pm b$
	\item $a\mid b \wedge b\mid c \Rightarrow a\mid c$
	\item $a\mid b \wedge b\mid a \Rightarrow a = \pm b$
	\item $a\mid b \wedge a\mid c \Rightarrow a\mid bx + cy$
	\item $0\mid a \Rightarrow a = 0$
	\item $a\mid b \wedge a,b \in \N \Rightarrow a \leq b$
\end{enumerate}
\end{satz}

%%%%%%%%%%%%%%%%%%%
% Beweis
%%%%%%%%%%%%%%%%%%%
\begin{beweis}
Beweis nur f�r $(v)$ [Rest analog].\\
Nach Voraussetzung existiert $g,h \in \Z$ mit $b = ag$, $c = ah$. Folglich: \[bx + cy = agx + ahy = \underbrace{a(gx + hy)}_{\in \Z}\], d.h. $a\mid gx + hy$.
\end{beweis}

%%%%%%%%%%%%%%%%%%%
% Definition 1.3
%%%%%%%%%%%%%%%%%%%
\begin{defi}\label{defi:1.3}
Seien $a_{1}, \ldots, a_{n}, t \in \Z$ mit $t\mid a_{1}, \ldots, t\mid a_{n}$.\\
Dann hei�t $t$ \addindex{Gemeinsamer Teiler}{gemeinsamer Teiler} von $a_{1}, \ldots, a_{n}$.\\
Setze $gT(a_{1}, \ldots, a_{n}) = \left\{t \in \Z \colon t\ gemeinsamer\ Teiler\ von\ a_{1}, \ldots, a_{n}\right\}$.
\end{defi}

%%%%%%%%%%%%%%%%%%%
% Bemerkung 1.1
%%%%%%%%%%%%%%%%%%%
\index{Euklidischer Algorithmus}
\begin{bem}\label{bem:1.1}
Wegen $gT(a_{1}, \ldots, a_{n}) = gT(\pm a_{1}, \ldots, \pm a_{n})$ kann man $a_{1}, \ldots, a_{n} \in \N_{0}$ annehmen. Da man $a_{1}, \ldots, a_{n}$ auch permutieren darf, kann man $a_{1} \geq \ldots \geq a_{n}$ annehmen, sogar $a_{1} > \ldots > a_{n} > 0$. Division mit Rest liefert $q, r \in \Z$ mit $a_{1} = qa_{n} + r$, $0 \leq r < a_{n}$. Aus \autoref{satz:1.2} folgt leicht:
\[gT(a_{1}, \ldots, a_{n}) = gT(a_{1} - qa_{n}, a_{2}, \ldots, a_{n})\]
mit $r + a_{2} + \ldots + a_{n} < a_{1} + a_{2} + \ldots + a_{2}$.\\
Iteration ergibt schlie�lich:\\
\[gT(a_{1}, \ldots, a_{n}) = gT(b).\]
Dieses Verfahren hei�t euklidischer Algorithmus (\textsc{Euklid} 365--300).
\end{bem}

%%%%%%%%%%%%%%%%%%%
% Beispiel 1.1
%%%%%%%%%%%%%%%%%%%
\begin{bsp}
	\[\begin{array}{lll}
			gT(45, 27, 12) & = & gT(\overbrace{45 - 3 \times 12}^{9}, 27, 12)\\
			& = & gT(27, 12, 9)\\
			& = & gT(27 - 3 \cdot 9, 12, 9)\\
			& = & gT(12, 9)\\
			& = & gT(12 - 1 \cdot 9, 9)\\
			& = & gT(9, 3)\\
			& = & gT(9 - 3 \cdot 3, 3) = gT(3).\\
		\end{array}\]
\end{bsp}

%%%%%%%%%%%%%%%%%%%
% Definition 1.4
%%%%%%%%%%%%%%%%%%%
\index{Gr��ter gemeinsamer Teiler}
\begin{defi}\label{defi:1.4}
Seien $a_{1}, \ldots, a_{n} \in \Z$.\\
Ein $d \in \N_{0} \cap gT(a_{1}, \ldots, a_{n})$ hei�t \emph{gr��ter gemeinsamer Teiler} von $a_{1}, \ldots, a_{n}$ ($\ggT$), falls d durch jeden gemeinsamen Teiler von $a_{1}, \ldots, a_{n}$ teilbar ist.
\end{defi}

%%%%%%%%%%%%%%%%%%%
% Satz 1.3
%%%%%%%%%%%%%%%%%%%
\begin{satz}[Existenz ggT]\label{satz:1.3}
Zu $a_{1}, \ldots, a_{n} \in \Z$ existiert stets genau ein $\ggT \ d$.
\end{satz}

%%%%%%%%%%%%%%%%%%%
% Beweis
%%%%%%%%%%%%%%%%%%%
\begin{beweis}
Nach \autoref{bem:1.1} ist $gT(a_{1}, \ldots, a_{n}) = gT(b)$ f�r ein $b \in \N_{0}$. Dann ist $b$ ein $\ggT$ von $a_{1}, \ldots, a_{n}$. Ist $b' \in \N_{0}$ ein weiterer $\ggT$ von $a_{1}, \ldots, a_{n}$, so gilt:\[b\mid b'\ und\ b'\mid b,\] also $b = \pm b'$ nach \autoref{satz:1.2}. Wegen $b, b' \in \N_{0}$ folgt $b = b'$.
\end{beweis}

%%%%%%%%%%%%%%%%%%%
% Bemerkung 1.2
%%%%%%%%%%%%%%%%%%%
\begin{bem}
Man schreibt $d = \ggT(a_{1}, \ldots, a_{n})$ und zeigt leicht:\[d = \ggT(a_{1}, \ggT(a_{2}, \ldots, a_{n})).\] Daher gen�gt es meist, den $\ggT$ von zwei Zahlen zu berechnen.
\end{bem}

%%%%%%%%%%%%%%%%%%%
% Satz 1.4
%%%%%%%%%%%%%%%%%%%
\index{Erweiterter Euklidischer Algorithmus}
\begin{satz}[Erweiterter Euklidischer Algorithmus]\label{satz:1.4}
Seien $a, b \in \N$. Setze \[\begin{array}{lll}(x_{0}, y_{0}, z_{0}) & := & (1, 0, a),\\ (x_{1}, y_{1}, z_{1}) & := & (0, 1, b)\\\end{array}\] und $i := 1$.\\
Im Fall $z_{i} = 0$ brechen wir ab.\\
Im Fall $z_{i} \neq 0$ liefert Division mit Rest $q_{i}, r_{i} \in \Z$ mit $z_{i-1} = q_{i}z_{i} + r_{i}, 0 \leq r_{i} < z_{i}$.\\
Wir setzen \[(x_{i+1}, y_{i+1}, z_{i+1}) := (x_{i+1} - q_{i}x_{i}, y_{i-1} - q_{i}y_{i}, z_{i-1} - q_{i}z_{i}),\] erh�hen i um 1 und iterieren. Dieses Verfahren bricht ab, und am Ende ist \[\ggT(a, b) = z_{i-1} = x_{i-1}a + y_{i-1}b.\]
\end{satz}

%%%%%%%%%%%%%%%%%%%
% Beispiel 1.2
%%%%%%%%%%%%%%%%%%%
\begin{bsp}
$a = 143, b = 91$:\\
\[\begin{array}{llll}
x_{i} & y_{i} & z_{i} & q_{i}\\
1 & 0 & 143\\
0 & 1 & 91 & 1\\
1 & -1 & 52 & 1\\
-1 & 2 & 39 & 1\\
2 & -3 & 13 & 3\\
& & 0 \\
\end{array}\]
Also $\ggT(143, 91) = 13 = 2 \times 143 - 3 \times 91$!
\end{bsp}

%%%%%%%%%%%%%%%%%%%
% Beweis
%%%%%%%%%%%%%%%%%%%
\begin{beweis}
Wegen $b = z_{1} > z_{2} > \ldots > z_{i-1} > z_{i} = 0$ bricht das Verfahren ab.\\
Am Anfang $(i = 1)$ ist $\ggT(a, b) = \ggT(z_{i-1}, z_{i})$, und das bleibt w�hrend des Verfahrens erhalten. $\left[\text{beachte: } \ggT(z_{i-1}, z_{i}) = \ggT(z_{i}, \underbrace{z_{i+1}}_{z_{i-1} - q_{i}z_{i}})\right]$\\
Am Ende ist $z_{i} = 0$, also $\ggT(a, b) = \ggT(z_{i-1}, z_{i}) = z_{i-1}$.\\
Am Anfang $(i = 0, 1)$ ist $x_{i}a + y_{i}b = z_{i}$, und das bleibt w�hrend des Verfahrens erhalten:
\[\begin{array}{lcl}
x_{i+1}a + y_{i+1}b & = & (x_{i-1} - q_{i}x_{i})a + (y_{i-1} - q_{i}y_{i})b\\
& = & \underbrace{x_{i-1}a + y_{i}b}_{z_{i-1}} - q_{i}\underbrace{(x_{i}a + y_{i}b)}_{z_{i}}\\
& = & z_{i+1}
\end{array}\]
\end{beweis}

%%%%%%%%%%%%%%%%%%%
% Bemerkung
%%%%%%%%%%%%%%%%%%%
\begin{bem}
Der erweiterte euklidische Algorithmus liefert also nicht nur $d := \ggT(a, b)$, somdern auch $x, y \in \Z$ mit $d = ax + by$.\\
Ist man an $x, y$ nicht interessiert, so kann man bei der Iteration die $x_{i}, y_{i}$ weglassen.
\end{bem}

%%%%%%%%%%%%%%%%%%%
% Satz 1.5
%%%%%%%%%%%%%%%%%%%
\begin{satz}\label{satz:1.5}
F�r $a_{1}, \ldots a_{n}, b \in \Z$ gilt:\[\exists\ x_{1}, \ldots, x_{n} \in \Z \colon b = x_{1}a_{1} + \ldots + x_{n}a_{n} \Leftrightarrow \ggT(a_{1}, \ldots, a_{n})\mid b.\]
\end{satz}

%%%%%%%%%%%%%%%%%%%
% Beweis
%%%%%%%%%%%%%%%%%%%
\begin{beweis}
\begin{itemize}
\item["`$\Rightarrow$"'] Sei $b = x_{1}a_{1} + \ldots + x_{n}a_{n} (x_{1}, \ldots, x_{n} \in \Z).$\\
F�r $t \in gT(a_{1}, \ldots, a_{n})$ gilt dann $t\mid x_{1}a_{1} + \ldots + x_{n}a_{n} = b$. Insbesondere $\ggT(a_{1}, \ldots, a_{n})\mid b$.\\\\
\item["`$\Leftarrow$"'] Sei $d := \ggT(a_{1}, \ldots, a_{n}\mid b)$, etwa $b = cd$.\\
Es gen�gt zu zeigen, dass $y_{1}, \ldots, y_{n} \in \Z$ existieren mit $d = y_{1}a_{1} + \ldots + y_{n}a_{n}$, denn dann ist $b = cd = \underbrace{cy_{1}}_{x_{1}}a_{1} + \ldots + \underbrace{cy_{n}}_{x_{n}}a_{n}$.\\
Die Existenz von $y_{1}, \ldots, y_{n}$ zeigen wir induktiv.\\ 
Im Fall $n = 1$ ist $d = \pm a_{1}$.\\
Im Fall $n = 2$ folgt die Existenz von $y_{1}, y_{2}$ aus \autoref{satz:1.4}.\\
Im Fall $n > 2$ existieren nach Induktion $z_{1}, \ldots, z_{n} \in \Z$ mit \[\ggT(a_{2}, \ldots, a_{n}) = z_{2}a_{2} + \ldots + z_{n}a_{n}.\]
Au�erdem existieren $v_{1}, v_{2} \in \Z$ mit 
\[\begin{array}{lcl}
d & = & \ggT(a_{1}, \ggT(a_{2}, \ldots, a_{n}))\\ 
& \stackrel{\autoref{satz:1.5}}{=} & v_{1}a_{1} + v_{2}\ggT(a_{2}, \ldots, a_{n})\\
& = & v_{1}a_{1} + v_{2}z_{2}a_{2} + \ldots + v_{2}z_{n}a_{n}.
\end{array}\]
\end{itemize}
\end{beweis}

%%%%%%%%%%%%%%%%%%%
% Bemerkung
%%%%%%%%%%%%%%%%%%%
\begin{bem}
Daher gilt: \[\ggT(a_{1}, \ldots, a_{n}) = 1 \Leftrightarrow \exists\ x_{1}, \ldots, x_{n} \in \Z \colon x_{1}a_{1} + \ldots + x_{n}a_{n} = 1.\]
Gegebenenfalls hei�en $a_{1}, \ldots, a_{n}$ \addindex{Teilerfremd}{teilerfremd}.
\end{bem}

%%%%%%%%%%%%%%%%%%%
% Definition 1.5
%%%%%%%%%%%%%%%%%%%
\begin{defi}\label{defi:1.5}
Eine nat�rliche Zahl $p \neq 1$ hei�t Primzahl, falls $\pm 1, \pm p$ ihre einzigen Teiler sind. Setze \[\PZ = \left\{p \in \N \colon p \ Primzahl\right\}.\]
Ist $a \in \Z$ und $p \in \PZ$ mit $p\mid a$, so hei�t $p$ \addindex{}{Primteiler} (\addindex{}{Primfaktor}) von $a$.
\end{defi}

%%%%%%%%%%%%%%%%%%%
% Bemerkung
%%%%%%%%%%%%%%%%%%%
\begin{bem}
$\PZ = \left\{2, 3, 5, 7, 11, 13, \ldots\right\}$
\end{bem}

%%%%%%%%%%%%%%%%%%%
% Satz 1.6
%%%%%%%%%%%%%%%%%%%
\begin{satz}\label{satz:1.6}
Jede nat�rliche Zahl $a \neq 1$ hat einen Primteiler.
\end{satz}

%%%%%%%%%%%%%%%%%%%
% Beweis
%%%%%%%%%%%%%%%%%%%
\begin{beweis}
Wegen $1 \neq a\mid a$ ist $D := \left\{d \in \N \colon 1 \neq d\mid a\right\} \neq \varnothing.\\$
Folglich existiert $p := min \ D.$\\
W�re $p \notin \PZ$, so h�tte $p$ einen Teiler $t \in \N$ mit $1 \neq t \neq p$.\\
Dann w�re aber $t < p$ und $t \in D$ im Widerspruch zur Wahl von $p$. Also $p \in \PZ$!
\end{beweis}

%%%%%%%%%%%%%%%%%%%
% Satz 1.7
%%%%%%%%%%%%%%%%%%%
\begin{satz}[Euklid]\label{satz:1.7}
$\left|\PZ\right| = \infty$
\end{satz}

%%%%%%%%%%%%%%%%%%%
% Beweis
%%%%%%%%%%%%%%%%%%%
\begin{beweis}
Seien $p_1, \ldots, p_r \in \PZ$. Nach \autoref{defi:1.5} hat $m := p_1, \ldots, p_{r+1}$ einen Primfaktor $p_{r+1}$. Wegen $p_1 \nmid \ m, \ldots, p_r \nmid \ m$ ist $p_{r+1} \notin \left\{p_1, \ldots, p_r \right\}.$
\end{beweis}

%%%%%%%%%%%%%%%%%%%
% Bemerkung
%%%%%%%%%%%%%%%%%%%
\begin{bem}
Die gr��te bekannte Primzahl ist $2^{32582657}-1$ (04.09.2006). Sie hat fast 10 Millionen Dezimalstellen. (http://primes.utm.edu/largest.html)
\end{bem}

%%%%%%%%%%%%%%%%%%%
% Satz 1.8
%%%%%%%%%%%%%%%%%%%
\begin{satz}\label{satz:1.8}
F�r $a, b \in \Z$ ist jeder Primteiler $p$ von $ab$ ein Teiler von $a$ oder $b$.
\end{satz}

%%%%%%%%%%%%%%%%%%%
% Beweis
%%%%%%%%%%%%%%%%%%%
\begin{beweis}
Im Fall $p\mid a$ sind wir fertig.\\
Sei also $p \nmid a$.\\ 
Dann: $1 = \ggT(p, a) = xp + ya$ mit geeigneten $x, y \in \Z.$\\
Folglich: $p\mid xpb + yab = (xp + ya)b = 1b = b$
\end{beweis}

%%%%%%%%%%%%%%%%%%%
% Bemerkung
%%%%%%%%%%%%%%%%%%%
\begin{bem}
Induktiv folgt leicht: Sind $a_1, \ldots, a_r \in \Z$ und $p \in \PZ$ mit $p\mid a_1, \ldots, a_r$, so ist $p\mid a_i$ f�r ein $i \in \left\{1, \ldots, r\right\}.$
\end{bem}

%%%%%%%%%%%%%%%%%%%
% Satz 1.9
%%%%%%%%%%%%%%%%%%%
\index{Primfaktorzerlegung}
\begin{satz}[Eindeutige Primfaktorzerlegung]\label{satz:1.9}
Zu jeder nat�rlichen Zahl $m \neq \ 1$ existieren $p_1, \ldots, p_r \in \PZ$ mit $m = p_1 \ldots p_r.$ Sind auch $q_1, \ldots, q_s \in \PZ$ mit $m = q_1 \ldots q_s,$ so ist $r = s,$ und nach Umnummerierung gilt $p_i = q_i$ f�r $i = 1, \ldots, r.$
\end{satz}

%%%%%%%%%%%%%%%%%%%
% Beweis
%%%%%%%%%%%%%%%%%%%
\begin{beweis}
\emph{Existenz:} Nach \autoref{satz:1.6} hat $m$ einen Primteiler $p$. Im Fall $m = p$ sind wir fertig. Anderenfalls ist $\frac{m}{p} \in \N$ und $1 < \frac{m}{p} < m$. Induktiv k�nnen wir annehmen, dass $p_1, \ldots, p_{r-1} \in \PZ$ existiert mit $\frac{m}{p} = p_1 \ldots p_{r-1}$, d.h. $m = p_1 \ldots p_{r-1}p$.\\
\emph{Eindeutigkeit:} Aus $p_1 \ldots p_r = m = q_1 \ldots q_s$ folgt $p_r\mid q_1 \ldots q_s$, also $p_r\mid q_i$ f�r ein $i \in \left\{1, \ldots, s\right\}$, o.B.d.A. $p_r\mid q_s$. Wegen $q_s \in \PZ$ folgt $p_r = q_s$, d.h. $p_1 \ldots p_{r-1} = q_1 \ldots q_{s-1}$. Induktiv ist $r-1 = s-1$ und $p_i = q_i$ f�r $i = 1, \ldots, r-1$ (nach Umnummerierung).
\end{beweis}

%%%%%%%%%%%%%%%%%%%
% Bemerkung
%%%%%%%%%%%%%%%%%%%
\begin{bem}
\begin{enumerate}[(i)]
	\item Man nennt $m = p_1 \ldots p_r$ die Primfaktorzerlegung von $m$. Durch zusammenfassen mehrfach auftretender Faktoren kann man diese auch in der Form $m = q_1^{a_1} \ldots q_t^{a_t}$ mit paarweisen verschiedenen $q_1, \ldots, q_t \in \PZ$ und beliebigen $a_1, \ldots, a_t \in \N$ schreiben. Eine weitere Schreibweise ist $m = \prod\limits_{p \in \PZ}p^{b_p}$, wobei $b_p \in \N_0$ f�r alle $p \in \PZ$ und $\abs{\left\{p \in \PZ \colon b_p \neq 0\right\}} < \infty$ ist. So kann man auch $ 1 = \prod\limits_{p \in \PZ}p^0$ als Primfaktorzerlegung der 1 auffassen.
	
	\item Die Teiler von $m = \prod\limits_{p \in \PZ}p^{b_p}$ sind genau die Zahlen $\pm \prod\limits_{p \in \PZ}p^{t_p}$ mit $t_p \leq b_p$ f�r alle $p \in \PZ$.
	
	\item Die gemeinsamen Teiler von $m = \prod\limits_{p \in \PZ}p^{b_p}$ und $n = \prod\limits_{p \in \PZ}p^{c_p}$ sind also genau die Zahlen $\pm \prod\limits_{p \in \PZ}p^{t_p}$ mit $t_p \leq min \left\{b_p, c_p\right\}$ f�r alle $p \in \PZ$. Daher ist $\ggT(a, b) = \prod\limits_{p \in \PZ}p^{min\left\{b_p, c_p\right\}}$.
	
	\item Die Konstruktion gro�er Primzahlen und die Primfaktorzerlegung gro�er nat�rlicher Zahlen sind von erheblicher Bedeutung f�r die Kryptographie, d.h. die Lehre vom Ver- sowie Entschl�sseln geheimer Nachrichten.
\end{enumerate}
\end{bem}