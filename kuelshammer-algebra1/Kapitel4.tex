\chapter{Das RSA-Verfahren in der Kryptographie}
\highl{RSA}\emph{(Rivest, Shamir, Adlemon 1982)} Public Key Cryptography\\\\
Ausgangspunkt: Personen $A$, $B$, $C$,\ldots wollen miteinander �ber einen unsicheren Kanal kommunizieren.\\
Dazu soll jede Nachricht so verschl�sselt werden, dass nur der richtige Empf�nger sie entschl�sseln kann. Wie geht das?\\
Am Anfang w�hlt jeder Teilnehmer $T$ zwei gro�e Primzahlen $p_T, q_T$ (ca. 100 Stellen) und eine gro�e Zahl $r_T \in \N$, die zu $\varphi(p_Tq_T) = (p_T - 1)(q_T - 1)$ Teilerfremd ist. Mit dem erweitertem euklidischen Algorithmus berechnet er dann ein $s_T \in \N$ mit $r_Ts_T \equiv 1 \pmod{\varphi(p_Tq_T)}$.\\
Die Zahlen $p_T$, $q_T$, $s_T$ h�lt er geheim, die Zahlen $n_T loneqq p_Tq_T$ und $r_T$ kommen in ein allgemein zug�ngliches Telefonbuch.\\
\[\text{Schl�ssel von } T \colon p_T, q_T, s_T \text{ geheim; } n_T, r_T \text{ offen.}\]
Will Teilnehmer $A$ an Teilnehmer $B$ eine Nachricht $M$ schicken, so geht er folgenderma�en vor: \\
(Man kann annehmen: $M \in \N$ und $M < n_B$)\\
$A$ berechnet $M' \in \N$ mit $M' < n_B$ und $M' \equiv M^{r_B} \pmod{n_B}$. Dann schreibt er $M'$ an $B$\\
\[\begin{array}{ccccccccc}
\text{Absender} & \rightarrow & \text{Encoder} & \rightarrow & \text{Kanal} & \rightarrow & \text{Decoder} & \rightarrow & \text{Empf�nger}\\
& & & & \uparrow & & & & \\
& & & & \text{Feind} & & & & \\
\end{array}\]
Der Empf�nger $B$ kann $M'$ entschl�sseln, in dem er $(M')^{s_B}$ berechnet, denn es gilt: \[(M')^{s_B} \equiv M \pmod{n_B}\] Das sieht man folgenderma�en:\\
Wegen $r_Bs_B \equiv 1 \pmod{(p_B - 1)(q_B - 1)}$ existiert ein $t_B \in \N$ mit $r_Bs_B = 1 + t_B(p_B - 1)(q_B - 1)$. Also gilt: \[(M'^{s_B}) = (M^{r_B})^{s_B} = M^{1 + t_B(p_B - 1)(q_b - 1)} = M(M^{r_B - 1})^{t_B(q_B - 1)} \equiv M \pmod{p_B}\]
Im Fall $p_B\nmid M$ folgt aus dem \autoref{satz:3.8} (\textsc{Euler-Fermat}), und im Fall $p_B\mid M$ ist dies trivial.\\
Analog ist $(M')^{s_B} \equiv M \pmod{q_B}$, also auch $(M')^{s_B} \equiv M \pmod{p_Bq_B}$.\\
Nehmen wir an, ein "`Feind"' h�rt die verschl�sselte Nachricht $M'$ ab. Kann er sie entschl�sseln?\\
\emph{Theoretisch ja}: schlie�lich muss er "`nur"' $n_B$ in Primfaktoren zerlegen. Dann kennt er $p_B$ und $q_B$, kann also $s_B$ mit dem erweiterten euklidischen Algorithmus berechnen und $M'$ genauso wie $B$ entschl�sseln.\\
\emph{Praktisch nein}: denn es ist kein Verfahren bekannt, das "`gro�e"' Zahlen "`schnell"' in Primfaktoren zerlegt. Auch sind keine anderen Verfahren zur Entschl�sselung von $M'$ bekannt. Daher ist das RSA Verfahren in der Praxis gut etabliert. (\emph{http://www.rsasecurity.com})