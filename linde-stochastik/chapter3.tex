\chapter{Grenzwerts�tze}
\index{Grenzwerts�tze}	

\subsection*{Vorbemerkung}

	Gesetze der Wahrscheinlichkeitstheorie wirken bei einer gro�en Anzahl
	unabh�ngiger Versuche. Zum Beispiel Wahrscheinlichkeit, dass man in einem Spiel
	gewinnt:\\
	$\to$ Keinerlei Aussage �ber Ausgang eines Spiels\\
	$\to$ Sehr viele Spiele$\to$ es wirken Gesetze der Wahrscheinlichkeitstheorie\\
	
\subsection*{Mathematisches Modell}

	$X_1,\X_2,\ldots$ Folge unabh�ngiger identisch verteilter Zufallsgr��en\\
	$\p(\X_1\in B_1,\ldots,\X_n\in B_n)=\p(\X_1\in B_1)\cdot\ldots\cdot
		\p(\X_n\in B_n)\\
	\bigwedge\limits_n \p(X_1\in B_1)=\ldots=\p(\X_n\in B_n)$\\
	$\X_n$ Ergebnis im n-ten Versuch\\
	$a=\EX_1=\EX_2=\ldots=\EX_n$
	\[\frac{\X_1+\ldots+\X_n}{n}\to_{n\to\infty}a\]
	
	\begin{Satz}[Tschebyschefsche Ungleichung]
		\label{satz:tsch}
		Sei Y Zufallsgr��e mit $\p(\Y\geq 0)=1$. Dann gilt f�r $c>0$ stets
		\[\p(\Y\geq c)\leq\frac{\mathbb{E}\Y}{c}\]
	\end{Satz}
	
	\underline{Beweis:}\\
	(Y stetig)
	\[\mathbb{E}\Y=\int\limits_0^\infty tp(t)\;dt
		\geq\int\limits_c^\infty tp(t)\;dt
		\geq c\int\limits_c^\infty p(t)\;dt
		=c\p(\Y\geq c)\qquad\boxed{}\]
		
	\begin{Satz}
		Sei X Zufallsgr��e mit 2. Moment $(\mathbb{E}|\X|^2<\infty)$
		\[\p(|\X-\EX|\geq \varepsilon)\leq \frac{\mathbb{V}\X}{\varepsilon^2}
			\quad(\varepsilon>0)\]
	\end{Satz}
		
	\underline{Beweis:}\\
	\[\Y:=(\X-\EX)^2;\qquad c:=\varepsilon^2\]
	\[\p(\Y\geq c)=\p(|\X-\EX|)^2\geq\varepsilon^2)=\p(|\X-\EX|)\geq\varepsilon)\]
	\[\mathbb{E}\Y=\mathbb{V}\X\quad\to\quad
		\text{Behauptung nach \ref{satz:tsch}}\]
		
	\setcounter{Bsp}{0}
	\begin{Bsp}
		$\X\sim\mathcal{N}(a,\sigma^2)$
		\[\p(|\X-a|\geq 3\sigma)\leq \frac{\mathbb{V}\X}{9\sigma^2}=\frac{1}{9}\]
		$3\sigma$-Regel, mit gro�er Wahrscheinlichkeit liegt der beobachtete Wert
		in $[a-3\sigma,a+3\sigma]$
		\index{$3\sigma$-Regel}
	\end{Bsp}
	
\subsection*{Folgerung}

	$\X_1,\X_2,\ldots$unabh�ngig, identisch verteilt\\
	$S_n:=\X_1+\X_2+\ldots+\X_n\qquad a=\EX_1$
	\[\mathbb{E}S_n=\EX_1+\EX_2+\ldots+\EX_n=n\cdot a\]
	\[\mathbb{V}S_n=n\mathbb{V}\X_1=n\cdot\sigma^2\]
	\[\mathbb{E}(\frac{S_n}{n})=a;\qquad
		\mathbb{V}(\frac{S_n}{n})=\frac{1}{n}\mathbb{V}(S_n)=\frac{\sigma^2}{n}\]
	\[\to\p((\frac{S_n}{n}-a)\geq\varepsilon)\leq
		\frac{\mathbb{V}S_n}{\varepsilon^2}
		=\frac{1}{n}\frac{\sigma^2}{\epsilon^2}\to_{n\to\infty}0\]
		
\subsection*{Deutung}

	$\alpha>0,\varepsilon>0$ vorgegeben $\to$ es existiert ein 
	$n_0=n_0(\alpha,\varepsilon)$ mit $n\geq n_0 \to$ Wahrscheinlichkeit, ein
	$\omega\in\Omega$ mit
	\[|\frac{S_n}{n}-a|\geq\varepsilon\]
	ist $\leq \alpha$.\\
	(Schwache Gesetz der gro�en Zahlen!)
	
	\setcounter{Bsp}{0}
	\begin{Bsp}
		W�rfel n mal\\
		$\X_1,\X_2,\ldots$ Ergebnisse im 1.,2.,$\ldots$ Wurf\\
		$a=\EX_1=\frac{7}{2};\qquad\sigma^2=\mathbb{V}\X_1=\frac{35}{12}$\\
	
		$\to\p(|\frac{S_n}{n}-\frac{7}{2}|\geq\varepsilon)
		\leq\frac{35}{12n\varepsilon}$\\
		$n=10000;\quad\varepsilon=0.1\quad\to\quad
		\p(|\frac{S_n}{n}-\frac{7}{2}|\geq 0.1)\leq\frac{35}{12\cdot 100}\approx
		0.0291\\
		\to \p(3.4\leq\frac{S_n}{n}\leq 3.6)
		\geq1-\frac{35}{21\cdot 100}\approx 0.971\quad\to\quad
		\p(34000\leq\frac{S_n}{n}\leq 36000)\geq 0.971\ldots$
	\end{Bsp}
		
\subsubsection*{Anwendung}

	F�hren beliebig viele gleichartige Versuche durch:\\
	$\X_1,\X_2,\ldots$ Ergebnisse. $\B\subseteq\mathbb{R}$ (Borelmenge)
	\[\Y_j\omega):=\lb\begin{array}{l@{\quad:\quad}l}
		1 & \X_j(\omega)\in\B\\
		0 & \X_j(\omega)\not\in\B
	\end{array}\right.\]
	\[\Y_j(\omega)=1\Leftrightarrow \B \text{ im j-ten Versuch eingetreten}\]
	\[\Y_1+\ldots+\Y_n=\sharp\lb j\leq n:\X_j\in\B\rb\]
	\[r_n:=\frac{\Y_1,\ldots,\Y_n}{n}\quad\text{rel. H�ufigkeit des Eintretens
	von B beim n-ten Versuch}\]
	$\mathbb{E}\Y_j\stackrel{?}{=}0\p(\Y_j=0)+1\p(\Y_j=1)
	=\p(\X_j\in\B)=\p(\X_1\in\B)$\\
	Wahrscheinlichkeit des Eintretens von B in einem Versuch\\
	$\mathbb{V}\Y_j=p(1-p)\qquad p=\p(\X_1\in\B)\quad\to\quad
	\p(|r_n-p|\geq\varepsilon)\leq\frac{p(1-p)}{n\varepsilon^2}
	\to_{n\to\infty}0$\\
	Relative H�ufigkeiten konvergieren gegen die Wahrscheinlichkeiten des
	Eintretens\\
	
	\begin{Bsp}
		W�rfel n mal\\
		$r_n:=\frac{1}{n}\cdot\sharp\lb j\leq n:\text{ j-te
		Wurf}=6\rb\quad\to\quad\p(|r_n-\frac{1}{6}|\geq\varepsilon)
		\leq\frac{\frac{5}{36}}{n\varepsilon^2}$\\
		$n=10000,\quad\varepsilon=0.01\quad\to\quad
		\p(|r_n-\frac{ 1}{6}|\geq 0.01)\leq \frac{5}{36}=0.13\overline{8}$\\
		$1-\p(0.15\overline{6}\leq r_n\leq 0.17\overline{6}=$
	\end{Bsp}
	
\section*{Starkes Gesetz der gro�en Zahlen}

	$a=\EX_1,\X_1,\X_2,\ldots$ unabh�ngig, identisch verteilte Zufallsgr��en\\
	\[\p(\omega\in\Omega:\lim\limits_{n\to\infty}\frac{1}{n}
	\sum\limits_{j=1}^n\X_j(\omega)=a)=1\]
	$\bigwedge\limits_{\varepsilon>0}$ existiert ein zuf�lliges $n_0$, so dass f�r
	$n>n_0$ stets gilt:
	\[\left|\frac{1}{n}\sum\limits_{j=1}^n\X_j-a\right|\leq\varepsilon\]
		
	\begin{figure}[ht]
    	\centering\input{figures/gw01.pictex}
		\caption{Gesetz der gro�en Zahlen}
       	\label{fig:gw01}      
	\end{figure}		
		
\section*{Zentraler Grenzwertsatz}
\index{Zentraler Grenzwertsatz}

	$X_1,X_2,\ldots$ unabh�ngig identisch verteilte Zufallsgr��en,\\
	$a=\EX_1,\quad\sigma^2=\mathbb{V}\X_1,\quad S_n := \X_1+\ldots+\X_n\\
	\mathbb{E}S_n=n\cdot a\qquad \mathbb{V}S_n=n\cdot\sigma^2$
	
	\begin{figure}[ht]
    	\centering\input{figures/gw02.pictex}
       	\label{fig:gw02}      
	\end{figure}
	
	\[\mathbb{E}\left(\frac{S_n-na}{\sqrt{n}\sigma}\right)=0,\quad
	\mathbb{V}\left(\frac{S_n-na}{\sqrt{n}\sigma}\right)=1\]
		
	\begin{Theorem}[Zentraler Grenzwertsatz]
		$\X_1,\X_2,\ldots$ unabh�nig, identisch verteilte zuf�llige Gr��en\\
		$a=\EX_1,\quad\sigma^2=\mathbb{V}\X_1;\\
		\bigwedge\limits_{-\infty\leq\alpha<\beta\leq\infty}$ gilt:
		\[\p(\alpha\leq\frac{S_n-na}{\sigma\sqrt{n}}\leq\beta)\to_{n\to\infty}
		\frac{1}{\sqrt{2\pi}}\int\limits_\alpha^\beta
			e^{-\frac{x^2}{2}}\;dx\]
			d.h. $\frac{S_n-na}{\sigma\sqrt{n}}$ ist "`fast"' 
			$\mathcal{N}(0,1)$-verteilt\\
			$S_n$ ist "`fast"' $\mathcal{N}(na,n\sigma^2)$-verteilt\\
	\end{Theorem}
		
\subsection*{Spezialfall:}
	
	$\p(\X_j=0)=1-p\qquad \p(\X_j=1)=p\\
	a=p,\quad \sigma^2=1-p$
	\[\p(\alpha\leq\frac{S_n-np}{\sqrt{np(1-p)}}\leq\beta)\to_{n\to\infty}
	\frac{1}{\sqrt{2\pi}}\int\limits_\alpha^\beta
	e^{-\frac{x^2}{2}}\;dx\]
	\[S_n\sim B_{n,p}\]
	\[\sum\limits_{\lb k\leq n:\alpha\leq\frac{k-np}{\sqrt{np(1-p)}}\leq\beta\rb}
	\binom{n}{k}p^k(1-p)^{n-k}\qquad
	\text{\emph{Moivre-Laplace} (1740?)}\]
		
\subsection*{Anwendungsbeispiele}

	F�r "`gro�e"' n ersetze man die Verteilung von $\frac{S_n-na}{\sqrt{n}\sigma}$
	durch die $\mathcal{N}(0,1)$-Verteilung bzw. die Verteilung von $S_n$ durch
	$\mathcal{N}(na,n\sigma^2)$
	\begin{enumerate}
		\item{W�rfel n mal,$a=\frac{7}{2},\quad\sigma^2=\frac{35}{12}$\\
			\[\p(\alpha\leq\frac{S_n-na}{\sqrt{n}\sigma}\leq\beta)
			=\p(\alpha\leq\frac{S_n-na}{\sqrt{\frac{35n}{12}}}\leq\beta)
			\approx\mathcal{N}(0,1)([\alpha,\beta])\]
			$n=10000,\quad\alpha=-2,\quad\beta=-2$
			\[\p(35000-200\sqrt{\frac{35}{12}}\leq S_n\leq 35000+200
			\sqrt{\frac{35}{12}})\approx\frac{1}{\sqrt{2\pi}}\int\limits_{-2}^2
			e^{-\frac{x^2}{2}}\;dx\approx 0.9545\]}
		 \item{Rechnungen bei Banken\\
		 	Gewinn oder Verlust der Banken seien gleichverteilt in
			$[-0.005,0.005]$\\
			$a=0,\quad\sigma^2=\frac{(b-a)^2}{12}=\frac{10^{-4}}{12}$\\
			$10^6$ Rundungen
			\[\p(\alpha\leq\frac{S_n}{\sqrt{\frac{10^6\cdot 10^{-4}}{12}}}\leq
			\beta)\approx\frac{1}{\sqrt{2\pi}}\int\limits_\alpha^\beta
			e^{-\frac{x^2}{2}}\;dx\]
			$\p(\alpha\frac{10}{\sqrt{12}}\leq S_n\leq\beta\frac{10}{\sqrt{12}}$\\
			$\p($Ein Euro oder mehr Verlust$)\to\beta=-\frac{\sqrt{12}}{10},
			\quad \alpha=-\infty\\
			\to \approx 0.364517\ldots$\\
			2 Euros $\approx 0.149349$\\
			$\vdots$\\
			10 Euros $\approx 0.00026603$\\}
		\item{\underline{Zuf�llige Irrfahrt}\\
			$S_n$ - Ort nach n Schritten, d.h. M�gliche Werte:
			$-n,-n+2,\ldots,n-2,n$\\
			$p=\frac{1}{2},\quad\mathbb{E}S_n=0,\quad\mathbb{V}S_n=\frac{n}{4}$
			\[\p(\alpha\leq\frac{S_n}{\sqrt{\frac{n}{2}}}\leq\beta)
			\approx\frac{1}{\sqrt{2\pi}}\int\limits_\alpha^\beta
			e^{-\frac{x^2}{2}}\;dx=\p(2\sqrt{n}\alpha\leq S_n\leq
			2\sqrt{n}\beta)\]
			\[\alpha=-1,\beta=1\to\p(-2\sqrt{n}\leq S_n\leq2\sqrt{n})
			\approx 0.6826\]
			\[\alpha=-2,\beta=2\to\p(-4\sqrt{n}\leq S_n\leq4\sqrt{n})
			\approx 0.9545\]}
	\end{enumerate}
